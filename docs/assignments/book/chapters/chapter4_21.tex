% =================================================================
% فایل chapter4_21.tex (اصلاح نهایی: شماره‌گذاری اجباری ۲۱.۱)
% =================================================================

\clearpage 

% -----------------------------------------------------------------
% تنظیمات حیاتی برای اصلاح شماره‌گذاری
% این دستورات باعث می‌شوند تمام زیربخش‌ها با ۲۱ شروع شوند.
% -----------------------------------------------------------------
\setcounter{section}{21} % شماره بخش اصلی را ۲۱ فرض می‌کنیم
\setcounter{subsection}{0} % شمارنده زیربخش را صفر می‌کنیم
% دستور زیر فرمت شماره‌گذاری را "اجبار" می‌کند که به صورت (۲۱.۱) باشد
\renewcommand{\thesubsection}{21.\arabic{subsection}} 

% -----------------------------------------------------------------
% تیتر اصلی (دستی - برای جلوگیری از شماره‌های اضافی)
% -----------------------------------------------------------------
\vspace*{0.5cm} 
\begin{flushright} 
	{\huge \textbf{۲۱. \hspace{0.2cm} طراحی نرم‌افزار حساس به ارزش}}
	\addcontentsline{toc}{section}{۲۱. طراحی نرم‌افزار حساس به ارزش}
\end{flushright}

% -----------------------------------------------------------------
% نام نویسنده (وسط‌چین با فاصله مناسب)
% -----------------------------------------------------------------
\vspace{1.5cm} 

\begin{center}
	\Large \textbf{پائولان کورنهوف} \\
	\vspace{0.3cm} 
	\large \lr{(Paulan Korenhof)}
\end{center}

\vspace{1.5cm} 

% -----------------------------------------------------------------
% فهرست مطالب بخش (استایل ساده و شیک)
% -----------------------------------------------------------------
\noindent
\textbf{\large محتویات بخش}
\vspace{0.5cm}

\noindent
\textbf{۲۱.۱ مقدمه} \dotfill \pageref{sec:21-intro} \par
\vspace{0.3cm}

\noindent
\textbf{۲۱.۲ خوب، بد، و «هرگز خنثی»} \dotfill \pageref{sec:21-good-bad} \par
\vspace{0.1cm}
\noindent \hspace{0.8cm} ۲۱.۲.۱ عدم خنثی بودن \dotfill \par
\noindent \hspace{0.8cm} ۲۱.۲.۲ تأثیر در سطح خرد \dotfill \par
\noindent \hspace{0.8cm} ۲۱.۲.۳ تأثیر در سطح کلان \dotfill \par
\noindent \hspace{0.8cm} ۲۱.۲.۴ در مجموع \dotfill \par
\vspace{0.3cm}

\noindent
\textbf{۲۱.۳ به‌کارگیریِ «هرگز خنثی»} \dotfill \pageref{sec:21-employing} \par
\vspace{0.1cm}
\noindent \hspace{0.8cm} ۲۱.۳.۱ چالشی برای طراحان \dotfill \par
\noindent \hspace{0.8cm} ۲۱.۳.۲ طراحی حساس به ارزش \lr{(VSD)} \dotfill \par
\noindent \hspace{0.8cm} ۲۱.۳.۳ ارزش‌ها \dotfill \par
\noindent \hspace{0.8cm} ۲۱.۳.۴ ارزش‌های قانونی و طراحی \dotfill \par
\vspace{0.3cm}

\noindent
\textbf{منابع} \dotfill \pageref{sec:21-references} \par

\vspace{1cm}

% -----------------------------------------------------------------
% کادر اهداف یادگیری
% -----------------------------------------------------------------
\noindent
\fcolorbox{black}{gray!10}{%
	\begin{minipage}{\dimexpr\linewidth-2\fboxsep-2\fboxrule}
		\vspace{0.2cm}
		\begin{center}
			\textbf{\large اهداف یادگیری}
		\end{center}
		\vspace{0.2cm}
		\begin{itemize}
			\item درک اینکه چرا فناوری یک ابزار خنثی نیست.
			\item توانایی تشخیص تأثیر غیرخنثیِ یک فناوری خاص.
			\item تفکر درباره چگونگی تعبیه ارزش‌های اخلاقی در طراحی نرم‌افزار.
		\end{itemize}
		\vspace{0.2cm}
	\end{minipage}
}
\vspace{1cm}

% =================================================================
% شروع متن اصلی (مقدمه) - با شماره‌گذاری صحیح ۲۱.۱
% =================================================================
\clearpage

% چون در بالا دستور renewcommand دادیم، این دستور الان خروجی «۲۱.۱ مقدمه» می‌دهد.
\subsection{مقدمه} 
\label{sec:21-intro}

نرم‌افزار تقریباً در همه جنبه‌های زندگی روزمره ما دخیل است: در دسکتاپ‌ها، لپ‌تاپ‌ها و گوشی‌های هوشمند ما وجود دارد، اما ما همچنین شاهد پیاده‌سازی آن در طیف فزاینده‌ای از اقلام پرمصرف مانند خودروها، دوچرخه‌ها، مسواک‌های برقی، اجاق‌گازها، دستگاه‌های تناسب اندام و اسباب‌بازی‌ها هستیم. ما از نرم‌افزار برای پرداخت، ارتباط، خرید آنلاین، برنامه‌ریزی مسیر، برنامه‌ریزی حمل‌ونقل عمومی، تماشای سریال و فیلم، تصمیم‌گیری در مورد اینکه کدام بیمه را بگیریم، به کدام حزب سیاسی رأی دهیم، سفارش غذا، چک کردن سلامتی‌مان و غیره استفاده می‌کنیم.

و این فقط ما به عنوان شهروندان خصوصی نیستیم که با کمال میل از نرم‌افزار برای بسیاری از امور در فعالیت‌های روزانه خود استفاده می‌کنیم. نهادهای دولتی و کسب‌وکارها نیز برای انجام بسیاری از فرآیندهای خود به‌شدت به نرم‌افزار متکی هستند. آن‌ها گاهی اوقات حتی فرآیندهای تصمیم‌گیری خاصی را کاملاً خودکار می‌کنند، مانند تصمیم‌گیری در مورد اینکه آیا کسی باید برای سرعت غیرمجاز جریمه شود، آیا به کسی باید وام یا کارت اعتباری داده شود، یا اینکه آیا کسی یک متقاضی شغلی امیدوارکننده است یا خیر.

در حالی که استفاده از نرم‌افزار به عنوان ابزاری برای کمک به ما در انواع وظایف مزایای زیادی دارد، اما یک نکته مهم (یا یک «گیر») وجود دارد. آن نکته در مورد برنامه‌های نرم‌افزاری این است که مانند تمام فناوری‌ها، آن‌ها **ذاتاً خنثی نیستند**: هر فناوری دارای یک سوگیری خاص است، شیوه‌ای خاص که در آن احتمالاً بر اعمال ما، انتخاب‌های ما، ادراک ما و نحوه تفسیر ما از جهان اطرافمان تأثیر می‌گذارد. در همین حال، تأثیر نرم‌افزار بر زندگی افراد می‌تواند بسیار زیاد باشد، به‌ویژه که به‌طور فزاینده‌ای عناصر بیشتری از زندگی ما به این برنامه‌ها وابسته و با آن‌ها درهم‌تنیده می‌شوند.

هدف این فصل جلب توجه خوانندگان به این ماهیت غیرخنثیِ فناوری و تشویق آن‌ها به تلاش برای بهره‌برداری از این عدم خنثی بودن به شیوه‌ای سودمند است. این فصل با بحث در مورد اینکه چرا فناوری هرگز یک ابزار خنثی نیست، آغاز می‌شود. با کمک مثال‌های مختلف مربوط به نرم‌افزار، تأثیر فناوری بر عناصر مختلف زندگی انسان مورد بحث قرار می‌گیرد. با توجه به ماهیت ذاتاً غیرخنثیِ فناوری و تأثیرات مشکل‌ساز بالقوه آن، مهم است که بفهمیم چگونه می‌توانیم از ثمرات فناوری بهره‌مند شویم و در عین حال آسیب‌های احتمالی آن را کاهش دهیم.

بنابراین، نیمه دوم فصل استدلال می‌کند که به‌طور ایده‌آل، ما باید از همان ابتدای طراحی فناوری، فعالانه با این عدم خنثی بودن برخورد کنیم. برای کمک به طراحان در این امر، این فصل ایده‌های اصلی زیربنای «طراحی حساس به ارزش» \lr{(VSD)} را معرفی می‌کند. از آنجا که ارائه یک دفترچه راهنمای کامل برای طراحی نرم‌افزار حساس به ارزش در اینجا ممکن نیست، هدف این فصل ارائه «خوراک فکری» کافی به خوانندگان است تا بتوانند خودشان در این سفرِ پیگیری گام بردارند.


% =================================================================
% شروع بخش ۲۱.۲ تا پایان ۲۱.۲.۲ (کد نهایی و دقیق)
% =================================================================

\raggedbottom % حذف فاصله‌های اضافی عمودی

% -----------------------------------------------------------------
% تنظیمات اجباری برای شماره‌گذاری دقیق (۲۱.۲ و ۲۱.۲.۱)
% -----------------------------------------------------------------
\setcounter{section}{21} 
\setcounter{subsection}{1} % تنظیم روی ۱، تا بعدی بشود ۲ (یعنی ۲۱.۲)
\renewcommand{\thesubsection}{21.\arabic{subsection}} 
\renewcommand{\thesubsubsection}{21.\arabic{subsection}.\arabic{subsubsection}}

% -----------------------------------------------------------------
% 21.2 The Good, the Bad, and the Never Neutral
% -----------------------------------------------------------------
\subsection{خوب، بد، و «هرگز خنثی»}
\label{sec:21-good-bad}

در این بخش، ما عمیقاً به موضوع عدم خنثی بودن فناوری خواهیم پرداخت. ابتدا، پیشینه دیدگاه عدم خنثی بودن مورد بحث قرار خواهد گرفت. پس از آن، این عدم خنثی بودن با جزئیات بیشتر و با رویکرد به فناوری از دو دیدگاه سطح خرد \lr{(micro-level)} و سطح کلان \lr{(macro-level)} توضیح داده خواهد شد.

% -----------------------------------------------------------------
% 21.2.1 Non-neutrality
% -----------------------------------------------------------------
\subsubsection{عدم خنثی بودن}
\label{sec:21-non-neutrality}

اهمیت فناوری برای زندگی انسان را به سختی می‌توان نادیده گرفت: جامعه و زندگی آن‌گونه که امروز می‌شناسیم، بدون توسعه و استفاده از فناوری وجود نداشت. فناوری به ما اجازه می‌دهد تا به اهداف خاصی دست یابیم، کارهایی را انجام دهیم و چیزهایی را درک کنیم که بدون استفاده از فناوری قادر به انجام آن‌ها نبودیم، و جهان را به روش‌های جدیدی برای ما آشکار می‌کند.

برای مثال، ما می‌توانیم تک‌سلول‌های بدن را از طریق میکروسکوپ ببینیم، با افرادی در آن سوی کره زمین از طریق تلفن مشورت کنیم، یا با دستگاه اکو به درون بدن نگاه کنیم. با فراهم کردن چنین تجربیات، اقدامات و ادراکات جدیدی از جهان، فناوری ما را قادر می‌سازد تا به روش‌های جدیدی با جهان ارتباط برقرار کنیم و بر تفسیر ما از جهان اطرافمان، و همچنین بر اعمال و قراردادهای اجتماعی ما تأثیر می‌گذارد \lr{(Kiran \& Verbeek, 2010; Verbeek, 2011)}.

به دلیل تأثیر شکل‌دهنده فناوری بر ادراک، تجربه، اعمال، اهداف و درک ما، فناوری از نقشِ صرفاً ابزار بودن فراتر می‌رود. در قرن گذشته، فیلسوفان تکنولوژی استدلال کردند که فناوری **ذاتاً خنثی نیست**: فناوری‌ها می‌توانند جهان را به روش‌های جدیدی برای ما آشکار کنند؛ انتخاب‌ها و امکانات جدیدی برای عمل ایجاد کنند؛ هویت‌های اجتماعی، روابط قدرت، و موقعیت‌های شمول و طرد را ایجاد نمایند؛ و بر ما، انتخاب‌های ما، فرهنگ ما و جهان‌بینی ما تأثیر بگذارند \lr{(see, e.g., Heidegger, 1954; Ihde, 1983; Latour, 1993; Feenberg, 2002; Verbeek, 2005)}. به دلیل این عدم خنثی بودن، فناوری دارای یک تأثیر هنجاری بر رابطه بین انسان‌ها و جهان آن‌هاست \lr{(Hildebrandt, 2015)}.

تحلیل تأثیر و معنای فناوری برای وجود انسان، منجر به پیدایش مکاتب فکری مختلفی در فلسفه تکنولوژی شد. به جای اینکه خواننده را درگیر بحث بین ایده‌های گوناگون کنیم، برای هدف این فصل ارزشمندتر خواهد بود که دو جهت‌گیری اصلیِ این دیدگاه‌ها را به صورت ساده‌سازی شده در نظر بگیریم و آن‌ها را مکمل یکدیگر بدانیم.

به زبان ساده، فناوری بر نحوه تعامل انسان‌ها با جهان تأثیر می‌گذارد که از یک **سطح خرد** (فردی و تجربی) \lr{(see, e.g., Ihde, 1983; Verbeek, 2005)}، تا یک **سطح کلان** (اجتماعی و انتزاعی) \lr{(see, e.g., Stiegler, 1998; Feenberg, 2002)} امتداد دارد.

در حالی که در نظر گرفتن تأثیر فناوری در سطح کلان برای درک دامنه و عمق تأثیر آن ضروری است، تحلیل متمرکز بر سطح خرد می‌تواند برای ردیابی مشکلات به ویژگی‌های ملموس فناوری بسیار مفید باشد. با این حال، من استدلال می‌کنم که سطح خرد و کلانِ تأثیر را نمی‌توان کاملاً از هم جدا کرد، زیرا مکانیسم‌های خرد به تأثیر کلان شکل می‌دهند و بالعکس. با وجود این، برای شفافیت ساختاری، با تمرکز بر سطح خرد شروع می‌کنم و سپس به سطح کلان می‌روم—اما خوانندگان باید توجه داشته باشند که این دو به هم پیوسته هستند.

% -----------------------------------------------------------------
% 21.2.2 Impact on a Micro-level
% -----------------------------------------------------------------
\subsubsection{تأثیر در سطح خرد}
\label{sec:21-micro-level}

در سطح خرد، فناوری با اجازه دادن به ما برای تجربه جهان با واسطه‌گری فناوری، بر ادراک، اعمال، رویه‌ها و اهداف انسانی تأثیر می‌گذارد \lr{(see, e.g., Ihde, 1983; Verbeek, 2005)}.

برای مثال، یک دماسنج می‌تواند دمای بدن ما را نشان دهد. اگر دماسنج عدد $38.5^\circ C$ را نشان دهد، ما احتمالاً نتیجه می‌گیریم که تب داریم، حتی اگر احساس بیماری نکنیم. با گفتن اینکه ما در واقع بیمار هستیم (در حالی که ممکن است احساس خوبی داشته باشیم)، فناوری بر نحوه درک ما از سلامتی‌مان تأثیر می‌گذارد. با انجام این کار، فناوری رابطه ما با جهان (در این مورد، بدن انسان) را «هم‌شکل» \lr{(co-shapes)} می‌دهد.



فناوری بر ادراک ما، اعمال ما و حتی نحوه تفکر و حافظه ما تأثیر می‌گذارد. مورد آخر به وضوح توسط تحقیقات در مورد تأثیرات موتورهای جستجو بر حافظه نشان داده شد: وقتی مردم می‌دانند که می‌توانند برای اطلاعات به موتور جستجو تکیه کنند، تمایل دارند به جای محتوا، «مکان» و «چگونگی» یافتن آن را به خاطر بسپارند \lr{(Sparrow et al., 2011, p. 778)}.

هنگامی که فناوری رابطه ما با جهان را میانجی‌گری می‌کند، عموماً تمرکز خاصی دارد: اغلب جنبه‌های خاصی از واقعیت را آشکار و برجسته می‌کند، در حالی که عناصر دیگر پنهان یا نادیده گرفته می‌شوند \lr{(Verbeek, 2005, p. 131)}. به تماس تلفنی فکر کنید: صدای تماس‌گیرنده برجسته می‌شود، در حالی که بقیه وجودِ فرد پنهان می‌ماند. فناوری بدین ترتیب رابطه خاصی بین انسان و جهان برقرار می‌کند؛ رابطه‌ای که جهت‌دار است. بنابراین می‌توانیم بگوییم که فناوری دارای نوعی «جهت‌مندی» \lr{(directionality)} است \lr{(Verbeek, 2005, p. 115)}.

این جهت‌مندی در طراحی مادی فناوری تعبیه شده است و یک «موضع» \lr{(stance)} خاص می‌گیرد: می‌تواند «پیشنهاد دهد، فعال کند، درخواست کند، ترغیب کند، تشویق کند و برخی اعمال را ممنوع یا ترویج کند» \lr{(Lazzarato \& Jordan, 2014, p. 30)}. یک طراح عموماً با القای ویژگی‌های خاص، قصد دارد جهت‌مندی خاصی به فناوری بدهد. مثلاً فروشگاه‌های آنلاین معمولاً ثبت سفارش را بدون پذیرش «شرایط و ضوابط» غیرممکن می‌سازند. با این حال، نفوذ طراح محدود است و فناوری وجود مستقلی دارد \lr{(Chabot, 2013, p. 15)} و ممکن است اثرات ناخواسته‌ای داشته باشد. اما چون طراحان ویژگی‌های مادی را تعیین می‌کنند، نقش محوری در شکل‌دهی به جهت‌مندی غیرخنثی دارند.

جهت‌مندی نرم‌افزار به ویژگی‌های فناوری و انتخاب‌های طراح بستگی دارد. دانش و محدودیت‌های توسعه‌دهنده پس‌زمینه طراحی را تشکیل می‌دهند \lr{(Kitchin, 2017, p. 18)}. این در یک فرآیند دو مرحله‌ای شکل می‌گیرد: توسعه‌دهنده باید (۱) وظیفه را تفسیر کند و (۲) آن را به کد ترجمه نماید. در این فرآیند، فرضیات و سوگیری‌های طراح در نرم‌افزار گنجانده می‌شود \lr{(Friedman \& Nissenbaum, 1996; Goldman, 2008)}. بنابراین، رویه‌های کدنویسی شده ناگزیر «آکنده از ارزش» هستند \lr{(see, e.g., Brey \& Søraker, 2009; Mittelstadt et al., 2016)}.

\vspace{0.4cm}

% -----------------------------------------------------------------
% Example 1 (مثال ۱ - داخل کادر)
% -----------------------------------------------------------------
\noindent
\fcolorbox{black}{gray!10}{% کادر خاکستری با حاشیه مشکی
	\begin{minipage}{\dimexpr\linewidth-2\fboxsep-2\fboxrule}
		\vspace{0.2cm}
		\textbf{مثال ۱} \\
		\textit{تصور کنید شرکتی شما را استخدام می‌کند تا یک آگهی استخدام برای راننده کامیون پخش کنید و تعدادی نامزد بالقوه را انتخاب نمایید. شما تصمیم می‌گیرید یک فرم آنلاین برای روند درخواست شغل ایجاد کنید. برای درخواست، متقاضیان باید نام و تاریخ تولد خود را پر کنند، یک رزومه آپلود کنند، یک گزینه را برای زن یا مرد بودن تیک بزنند، و گزینه‌ای را مبنی بر رضایت به پردازش داده‌های شخصی‌شان تیک بزنند. برای جلوگیری از فراموش کردن موارد توسط افراد، تمام فیلدها را اجباری می‌کنید.}
		
		\par \vspace{0.2cm}
		
		در حالی که چنین فرمی ساده به نظر می‌رسد، در همین کاربرد کوچک نرم‌افزار «جهت‌مندی» خاصی دارد که ممکن است مشکل‌ساز باشد.
		اولاً، آنلاین بودن فرم بلافاصله فرآیند را دیجیتالی می‌کند و ممکن است افراد با مهارت دیجیتال کم را حذف کند.
		
		دوماً، اجباری بودن فرم افراد را مجبور به افشای اطلاعات (یا دروغ گفتن) و شناسایی خود بر اساس گزینه‌های فرم می‌کند. فرم دیدگاهی خاص از جهان دارد: عناصر خاصی را مهم می‌داند و دیدگاه جنسیتی دوگانه (باینری) را بیان می‌کند. متقاضیان ممکن است این «تطبیق هویت با جعبه‌های نرم‌افزار» را مشکل‌ساز بدانند (مثلاً نگرانی حریم خصوصی برای تاریخ تولد، یا عدم تمایل به تعیین جنسیت). اما انتخاب‌های آن‌ها محدود به گزینه‌های فرم آنلاین است.
		\vspace{0.2cm}
	\end{minipage}
}
\vspace{0.4cm}

% -----------------------------------------------------------------
% ادامه متن بعد از مثال
% -----------------------------------------------------------------
شکل دادن به نرم‌افزار می‌تواند به ویژه هنگام طراحی نرم‌افزاری که نیاز به تولید تصمیمات بر اساس قوانین خاص دارد (مثل جریمه‌های خودکار سرعت یا یارانه مراقبت از کودک) دشوار باشد. در چنین مواردی، برنامه‌نویس باید قوانین حقوقی یا سیاست‌گذاری را به کد ترجمه کند و ممکن است با سوالاتی مواجه شود که دقیقاً چه زمانی یک مورد خاص تحت تعاریف قانون قرار می‌گیرد. با شکل دادن به چنین مرزهایی در کد، برنامه‌نویس مفاهیم قانونی را پر می‌کند و عملاً یک قاعده سیاستی را تثبیت می‌کند.

نقش انتخاب‌های طراحی در نرم‌افزار را به سختی می‌توان نادیده گرفت: کد کنترل می‌کند کاربران چه کاری می‌توانند و چه کاری نمی‌توانند انجام دهند. این همان چیزی است که لسیگ با بیانیه معروف خود «کد قانون است» منظور داشت \lr{(Lessig, 2006)}. با این حال، همان‌طور که پارایزر توضیح می‌دهد، کد نرم‌افزار با قدرت بیشتری نسبت به قانون رفتار کاربر را کنترل می‌کند، حداقل در ابتدا:

% -----------------------------------------------------------------
% Pariser Quote (نقل قول پارایزر - با تورفتگی/Shift)
% -----------------------------------------------------------------
\begin{quote}
	\small 
	«اگر کد قانون است، مهندسان نرم‌افزار و خوره‌های کامپیوتر کسانی هستند که آن را می‌نویسند. و این نوع عجیبی از قانون است که بدون هیچ سیستم قضایی ایجاد شده و فوراً اجرا می‌شود. حتی با وجود قوانین ضد خرابکاری، در دنیای فیزیکی شما هنوز می‌توانید سنگی را به پنجره فروشگاهی که دوست ندارید پرتاب کنید. اما اگر خرابکاری بخشی از طراحی یک دنیای آنلاین نباشد، به سادگی غیرممکن است. سعی کنید سنگی را به یک ویترین مجازی پرتاب کنید؛ شما فقط یک پیام خطا دریافت می‌کنید» \lr{(Pariser, 2011, pp. 96–97)}.
\end{quote}

% -----------------------------------------------------------------
% پایان بخش ۲۱.۲.۲
% -----------------------------------------------------------------
بنابراین توسعه‌دهنده از طریق معماری نرم‌افزار قدرت زیادی بر کاربران دارد. در این میان، **رابط کاربری** \lr{(interface)} نقش کلیدی ایفا می‌کند؛ از یک سو با پیشنهاد دادن به کاربر که نرم‌افزار چه کاری انجام می‌دهد و از سوی دیگر به عنوان قلمرو تعامل. رابط کاربری ادراک کاربران را شکل می‌دهد، دانش کار با نرم‌افزار را فراهم می‌کند و اقدامات آن‌ها را محدود می‌کند. معمولاً رابط گرافیکی \lr{(GUI)} کد منبع را پنهان می‌کند و عملکرد واقعی نرم‌افزار را غیرشفاف می‌سازد. علاوه بر این، رابط کاربری می‌تواند برای دستکاری یا «تلنگر» \lr{(nudging)} کاربران طراحی شود \lr{(see, e.g., Fogg, 1999; Thaler \& Sunstein, 2009)}. مثلاً استفاده از دکمه سبز بزرگ برای «پذیرش همه کوکی‌ها» در مقابل دکمه قرمز کوچک برای تنظیمات دیگر (به فصل گلرت در همین کتاب مراجعه کنید).

بسته به رابط کاربری، میزان بینش و انتخاب‌های کاربران تغییر می‌کند. این موضوع بر **خودمختاری** \lr{(autonomy)} کاربران تأثیر می‌گذارد: توانایی آن‌ها برای خود-حکمرانی که مستلزم آزادی تصمیم‌گیری آگاهانه است.\footnote{تعریف دقیق اینکه خودمختاری شامل چه چیزی است، بسته به دیدگاه اجتماعی و سیاسی متفاوت است. برای اهداف این فصل، من این مفهوم را نسبتاً باز نگه داشتم تا در رابطه با طراحی نرم‌افزار قابل استفاده باشد.}

برای مثال، برخی فروشگاه‌ها کاربران را ملزم به انتخاب جنسیت «زن» یا «مرد» می‌کنند، برخی گزینه «ترجیح می‌دهم نگویم» دارند و برخی اصلاً آن را اجباری نمی‌کنند. هرچه کاربران آزادانه‌تر انتخاب کنند، خودمختاری بیشتری دارند. کاهش خودمختاری و اجبار کاربران به مسیرهای خاص، می‌تواند آن‌ها را از وظیفه‌شان بیگانه کند. بنابراین دی مول و ون دن برگ استدلال می‌کنند: «آگاهی از، و بینش نسبت به "ماهیت متنی/اسکریپتی" مصنوع، و داشتن توانایی تأثیرگذاری بر آن، برای کاربران در پرتوِ واگذاری خودمختاری‌شان حیاتی است» \lr{(De Mul \& van den Berg, 2011, pp. 59–60)}.




