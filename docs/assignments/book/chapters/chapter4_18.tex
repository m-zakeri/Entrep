% =================================================================
% فصل ۱۸: دیدگاه‌هایی از قانون مالکیت فکری (نسخه نهایی)
% =================================================================

\raggedbottom 

% تنظیم شماره‌ها برای فصل ۱۸
\setcounter{section}{18}
\setcounter{subsection}{0}
\renewcommand{\thesection}{18}
\renewcommand{\thesubsection}{18.\arabic{subsection}}

% -----------------------------------------------------------------
% بلوک عنوان و نویسنده
% -----------------------------------------------------------------
\noindent
\fcolorbox{black}{gray!15}{%
	\begin{minipage}{\dimexpr\linewidth-2\fboxsep-2\fboxrule}
		\vspace{0.5cm}
		\begin{center}
			\textbf{\huge دیدگاه‌هایی از قانون مالکیت فکری}
			\vspace{0.4cm}
			
			\large
			\textit{لیزا ون دونگن} \\
			\lr{\textit{Lisa van Dongen}}
		\end{center}
		\vspace{0.5cm}
	\end{minipage}
}

\vspace{0.8cm}

% -----------------------------------------------------------------
% فهرست مطالب فصل ۱۸
% -----------------------------------------------------------------
\noindent
\textbf{\Large فهرست مطالب}
\vspace{0.3cm}

{ \small
	\noindent
	\textbf{۱۸.۱} \hspace{0.3cm} \textbf{مقدمه} \par
	\vspace{0.2cm}
	
	\noindent
	\textbf{۱۸.۲} \hspace{0.3cm} \textbf{برآورده کردن معیارها} \par
	\vspace{0.1cm}
	\hspace{0.8cm} ۱۸.۲.۱ \hspace{0.1cm} الزامات شکلی حق تکثیر \lr{(Copyright)} \par
	\hspace{0.8cm} ۱۸.۲.۲ \hspace{0.1cm} حق خاص پایگاه داده \lr{(Sui Generis)} \par
	\hspace{0.8cm} ۱۸.۲.۳ \hspace{0.1cm} حق اسرار تجاری \par
	\hspace{0.8cm} ۱۸.۲.۴ \hspace{0.1cm} خلاصه \par
	\vspace{0.2cm}
	
	\noindent
	\textbf{۱۸.۳} \hspace{0.3cm} \textbf{دامنه حفاظت} \par
	\vspace{0.1cm}
	\hspace{0.8cm} ۱۸.۳.۱ \hspace{0.1cm} حق تکثیر: موضوع مورد حمایت \par
	\hspace{0.8cm} ۱۸.۳.۲ \hspace{0.1cm} حفاظت خاص پایگاه داده \par
	\hspace{0.8cm} ۱۸.۳.۳ \hspace{0.1cm} حق اسرار تجاری \par
	\hspace{0.8cm} ۱۸.۳.۴ \hspace{0.1cm} خلاصه \par
	\vspace{0.2cm}
	
	\noindent
	\textbf{۱۸.۴} \hspace{0.3cm} \textbf{استثنائات و محدودیت‌ها} \par
	\vspace{0.1cm}
	\hspace{0.8cm} ۱۸.۴.۱ \hspace{0.1cm} محدودیت‌های حقوق \par
	\hspace{0.8cm} ۱۸.۴.۲ \hspace{0.1cm} استثنائات: وجه مشترک \par
	\hspace{0.8cm} ۱۸.۴.۳ \hspace{0.1cm} استثنائات خاص برای هر حق \par
	\vspace{0.2cm}
	
	\noindent
	\textbf{۱۸.۵} \hspace{0.3cm} \textbf{منابع جایگزین} \par
	\vspace{0.3cm}
	\noindent
	\textbf{منابع برای مطالعه بیشتر}
}

\vspace{0.8cm}

% -----------------------------------------------------------------
% اهداف یادگیری
% -----------------------------------------------------------------
\noindent
\fcolorbox{black}{gray!15}{%
	\begin{minipage}{\dimexpr\linewidth-2\fboxsep-2\fboxrule}
		\vspace{0.3cm}
		\textbf{\large اهداف یادگیری}
		
		\begin{itemize}
			\setlength\itemsep{0.5em}
			\item[\textbf{--}] حق تکثیر \lr{(copyright)}، حقوق خاص پایگاه داده \lr{(sui generis)} و اسرار تجاری شامل چه مواردی می‌شوند و چگونه می‌توان ذینفعان آن‌ها را تعیین کرد.
			
			\item[\textbf{--}] چه زمانی و چگونه استفاده از مجموعه داده‌های شخص ثالث توسط این حقوق محدود می‌شود و چه زمانی محدود نمی‌شود.
			
			\item[\textbf{--}] پتانسیل و محدودیت‌های منابع جایگزین برای تکمیل یا جایگزینی مجموعه داده‌های شخص ثالث، مانند حقوق قابلیت انتقال داده‌ها و اطلاعات بخش عمومی.
		\end{itemize}
		\vspace{0.2cm}
	\end{minipage}
}

\newpage

% =================================================================
% ۱۸.۱ مقدمه
% =================================================================
\subsection{مقدمه}
\label{sec:18-introduction}

اطلاعات بیشتر و بیشتری از طریق استفاده از دستگاه‌های هوشمند (مانند ترموستات هوشمند، تلفن هوشمند)، خدمات اینترنتی (مانند گوگل و فیس‌بوک)، حسگرها (مثلاً در اتومبیل‌ها، خانه‌های هوشمند و شهرها) و دوربین‌ها جمع‌آوری می‌شود. مجموعه داده‌های حاصل حاوی اطلاعات زیادی درباره افراد، و همچنین درباره جامعه به طور کلی هستند.

این مجموعه داده‌ها به ناظران خود اجازه می‌دهند تا مشکلات را شناسایی کرده و راه‌های رفع آن‌ها را بررسی کنند، اما همچنین فرصت‌ها را شناسایی کرده و راه‌های بهره‌برداری از آن‌ها را کاوش کنند. برای مثال، با مطالعه اطلاعات حاصل از حسگرهای اتومبیل‌ها، حسگرها و دوربین‌های متمرکز بر جاده‌ها و سیستم‌های چراغ راهنمایی، می‌توان علل تصادفات رانندگی را شناسایی کرد و راه‌حل‌هایی برای کاهش تعداد تصادفات در یک بلوک خاص پیشنهاد داد.

با این حال، دسترسی به چنین مجموعه داده‌هایی که توسط دیگران تولید شده‌اند، اغلب محدود است. گروه بزرگی از بازیگران وجود دارند که نمی‌خواهند دیگران از داده‌های «آن‌ها» استفاده کنند. یک عامل بسیار مهم که به چنین بازیگرانی کمک می‌کند تا دسترسی به مجموعه داده‌های خود را محدود کنند، «قانون مالکیت فکری» \lr{(intellectual property law)} است. دارنده حقوق مالکیت فکری بر روی یک مجموعه داده، توانایی محدود کردن دسترسی دیگران به (بخش‌هایی از) مجموعه داده خود و همچنین اعمال محدودیت‌هایی بر استفاده از آن را دارد.

برای درک چگونگی پیمایش در این حوزه حقوقی، ابتدا مهم است که درک کنیم حقوق مالکیت فکری چه هدفی را دنبال می‌کنند. همان‌طور که در «دستورالعمل اجرایی» \lr{(Enforcement Directive)} بیان شده است، دلیل اصلی زیربنایی در سیستم‌های فعلی قانون مالکیت فکری، ایجاد انگیزه (سرمایه‌گذاری) در نوآوری است. حقوق مالکیت فکری به عنوان حقوق مالکیت مصنوعی برای اصلاح برخی «شکست‌های بازار» ایجاد شده‌اند.

بازار را به عنوان زمینی پر از میوه تصور کنید. اگر همه آزاد باشند که بدون هیچ محدودیتی از زمین و میوه‌های آن استفاده کنند، احتمالاً بسیاری این کار را خواهند کرد. با این حال، بعید است که همه کسانی که از زمین استفاده می‌کنند، به صورت جداگانه در آن سرمایه‌گذاری کنند. این به دلیل عدم اطمینان از این موضوع است که آیا این سرمایه‌گذاری نتیجه‌ای برای آن‌ها خواهد داشت یا حتی به آن‌ها اجازه می‌دهد سرمایه خود را بازگردانند؛ زیرا همه آزادند بدون محدودیت از زمین استفاده کنند.

زمانی که عواید پیش‌بینی شده کمتر باشد، افراد کمتری مایل به سرمایه‌گذاری خواهند بود. علاوه بر این، هر سرمایه‌گذاری که انجام شود احتمالاً کوچکتر خواهد بود. اینجاست که حقوق مالکیت فکری وارد می‌شود. آن‌ها ابزارهایی برای اصلاح این «شکست بازار» هستند که با اعطای مجموعه‌ای از حقوق انحصاری برای مدت محدود به کسانی که در نوآوری سرمایه‌گذاری می‌کنند، پاداش می‌دهند. این حقوق ابزارهایی برای دارنده حق هستند تا دسترسی و استفاده از مالکیت فکری خود را به صورت قانونی محدود کند. این امر به دارنده حق اجازه می‌دهد تا قیمت‌های بالاتری را برای بازگشت سرمایه و کسب سود تعیین کند.

این فصل قصد دارد مقدمه‌ای بر مبانی حقوق مالکیت فکری در اتحادیه اروپا ارائه دهد. این فصل از ساده‌سازی‌هایی استفاده می‌کند و همیشه تصویر کاملی ارائه نمی‌دهد تا درک مطالب را به حداکثر برساند. چنین ساده‌سازی‌هایی عموماً ذکر می‌شوند و منابعی در مورد این موضوع برای کسانی که مایل به درک عمیق‌تر این مفهوم کمتر توسعه‌یافته هستند، در مراجع گنجانده شده است. بنابراین، این فصل نباید به عنوان جایگزینی برای مشاوره حقوقی یا به عنوان مبنایی برای بحث‌های آکادمیک استفاده شود.

علاوه بر این، در حالی که انواع مختلفی از حقوق مالکیت فکری وجود دارد، در اینجا تنها «حق تکثیر» \lr{(copyright)}، «حقوق خاص پایگاه داده» \lr{(sui generis database rights)} و «اسرار تجاری» \lr{(trade secrets)} مورد بحث قرار خواهند گرفت. تحت چارچوب حقوقی اتحادیه اروپا، موضوع و شرایط این حقوق مالکیت فکری ارتباط نزدیکی با داده‌ها و نرم‌افزار دارند، همان‌طور که در ادامه فصل مشخص خواهد شد.

سایر حقوق مانند حق اختراع \lr{(patents)} در حال حاضر نقش پیچیده‌تری در اتحادیه اروپا در زمینه داده‌ها و نرم‌افزار ایفا می‌کنند، از جمله به دلیل محدودیت‌ها در قابلیت ثبت اختراعِ موضوعاتی مانند روش‌های ریاضی و برنامه‌های کامپیوتری به عنوان چنین. چنین محدودیت‌هایی همچنین شروع به ایفای نقش بیشتری در مثلاً ایالات متحده کرده‌اند، همان‌طور که می‌توان از رویه قضایی دیوان عالی آن‌ها (یعنی در مورد مفهوم «ایده انتزاعی») بین سال‌های ۲۰۱۰ و ۲۰۱۴ استنباط کرد. بنابراین، این موضوع نیاز به توجه بیشتری دارد که در این نوشتار محدود نمی‌گنجد.

بنابراین، این فصل تنها بر این حقوق خاص مالکیت فکری از دیدگاه اتحادیه اروپا تمرکز دارد. سوالات بررسی شده در بخش‌های بعدی بر ایجاد این موضوع برای هر یک از این حقوق مالکیت فکری تمرکز خواهند کرد که چه زمانی اعمال می‌شوند (بخش ۱۸.۲)، و به دنبال آن، این موضوع برای استفاده شخص ثالث از داده‌ها چه معنایی دارد (بخش ۱۸.۳)، و همچنین محدودیت‌ها و استثنائات (بخش ۱۸.۴). این فصل با بحث در مورد راه‌های دسترسی قانونی به مجموعه داده‌های تحت پوشش یک یا چند مورد از این حقوق مالکیت فکری و منابع جایگزین پایان می‌بریم.

% =================================================================
% ۱۸.۲ برآورده کردن معیارها
% =================================================================
\subsection{برآورده کردن معیارها}
\label{sec:18-meeting-criteria}

% -----------------------------------------------------------------
% ۱۸.۲.۱ الزامات شکلی حق تکثیر
% -----------------------------------------------------------------
\subsubsection{الزامات شکلی حق تکثیر}

یک چیز ممکن است تحت حمایت قانون کپی‌رایت قرار گیرد اگر سه معیار تجمعی \lr{(cumulative criteria)} برای حفاظت کپی‌رایت را برآورده کند. طبق «کنوانسیون برن» \lr{(Berne Convention)}، این معیارها مستلزم آن هستند که اثر: (۱) یک «بیان/ابراز» \lr{(expression)} باشد، (۲) که «اصیل» \lr{(original)} باشد، (۳) در حوزه «ادبیات و هنر» باشد. در رژیم کپی‌رایت اتحادیه اروپا، عواملی مانند کار یا سرمایه‌گذاری بی‌ربط هستند.

سه عنصر از یک مجموعه داده وجود دارد که قادر به برآورده کردن این الزامات هستند:
\begin{itemize}
	\setlength\itemsep{0.5em}
	\item محتویات مجموعه داده
	\item انتخاب داده‌ها
	\item چیدمان/آرایش داده‌ها
\end{itemize}

اگر یک یا چند مورد از این عناصر الزامات شکلی را برآورده کنند، ممکن است حفاظت کپی‌رایت بر روی آن عناصر مجموعه داده وجود داشته باشد. در این صورت، محدودیت‌های قانونی برای استفاده از آن وجود خواهد داشت. بنابراین مهم است که این معیارها را درک کنیم تا بتوانیم احتمال حفاظت کپی‌رایت بر روی یک مجموعه داده را برای اطمینان از استفاده قانونی تعیین کنیم.

فرض اساسی معیار «بیان» \lr{(expression)} این است که حقایق و ایده‌ها خلق نمی‌شوند، بلکه کشف می‌شوند. این موضوع همچنین در پرونده \lr{Feist Publications, Inc., v. Rural Telephone Service Co.} تأیید شده است، که نشان می‌دهد ایالات متحده و اتحادیه اروپا به شیوه‌ای مشابه با این معیار برخورد می‌کنند. آنچه این موضوع برای کپی‌رایت معنا می‌دهد این است که کپی‌رایت از «آنچه گفته می‌شود» محافظت نمی‌کند، بلکه از «چگونگی بیان آن» محافظت می‌کند.

یک قاعده سرانگشتی خوب، نگاه کردن به آن به عنوان یک طیف است که در آن حقایق و ایده‌ها در یک طرف و بیان‌ها در طرف دیگر بر اساس خاص بودن \lr{(specificity)} قرار دارند. در اصل، هر چه یک حقیقت یا ایده خاص‌تر شود، عقربه عموماً به سمت بیان نزدیک‌تر می‌شود. استدلال پشت این موضوع این است که یک نویسنده می‌تواند یک حقیقت یا ایده را با انتخاب کلمات خود منتقل کند و بدین ترتیب چیزی فراتر و جدا از آن حقیقت یا ایده خلق کند. برای مثال، به تفاوت جزئیات در جملات جدول ۱۸.۱ نگاه کنید.

% جدول ۱۸.۱
\begin{table}[h!]
	\centering
	\caption{بیان \lr{(Expression)}}
	\vspace{0.2cm}
	\begin{tabular}{|p{0.45\linewidth}|p{0.45\linewidth}|}
		\hline
		\textbf{جمله نمونه} & \textbf{سطح جزئیات: واقعیت/ایده/بیان} \\
		\hline
		این خانه سبز است. & جزئیات بسیار کم و بسیار کلی \newline (واقعیت/ایده) \\
		\hline
		این خانه سه طبقه، سه طیف مختلف از رنگ سبز است. & جزئیات بیشتر، اما هنوز کاملاً کلی \newline (واقعیت/ایده، اما بیشتر به سمت بیان) \\
		\hline
		این محل سکونت سه طبقه، ترکیبی از طیف‌های سبز است، از جمله زیتونی، خزه‌ای و حتی برخی رگه‌های سبز متالیک در گوشه‌های پنجره‌ها و درهایش. & جزئیات زیاد و بسیار خاص \newline (بیان) \\
		\hline
	\end{tabular}
	\small{\newline \textbf{نکته جدول:} جدول متعلق به نویسنده است.}
	\label{tab:18-1-expression}
\end{table}

این الزام یک مانع احتمالی برای حفاظت کپی‌رایت بر روی یک مجموعه داده است. برای مثال، داده‌ها در چنین مجموعه داده‌هایی با هم ممکن است تصویر بسیار خاصی ایجاد کنند، اما اگر داده‌ها صرفاً به عنوان متغیرهایی در یک جدول نمایش داده شوند، داده‌ها فاقد «بیان» هستند.

در پرونده \lr{Football DataCo Ltd.}، عنصر دوم — اصالت \lr{(originality)} — به عنوان «حاشیه اختیار» برای انجام انتخاب‌های آزادانه و خلاقانه که مورد استفاده قرار گرفته، درک شد. به زبان ساده‌تر، این مستلزم آن است که خالق «مُهر شخصی» خود را بر روی آن زده باشد. البته این نباید به معنای تحت‌اللفظی گرفته شود. برای مثال، قرار دادن لوگوی خود روی چیزی آن را اصیل نمی‌کند.

سطح برآورده کردن این معیار در عمل خیلی بالا نیست. چنین انتخاب‌های خلاقانه‌ای می‌تواند به سادگیِ انتخاب نورپردازی، پس‌زمینه و زاویه برای گرفتن یک عکس، یا انتخاب کلمات در یک متن یا کد باشد. با این حال، مهم است تأکید شود که باید فضایی برای انجام چنین انتخاب‌هایی توسط خالق وجود داشته باشد.

برای مثال، یک عکس پاسپورت باید تعدادی از الزامات سخت‌گیرانه را برآورده کند. چنین تنظیمات از پیش تعیین شده‌ای بر فضایی که عکاس برای تصمیم‌گیری‌های خلاقانه خود دارد، تأثیر می‌گذارد. بنابراین برای یک عکس پاسپورت، بسیار بعید است که عکاس بتواند الزام اصالت را برآورده کند.

مثال دیگر الزامات عملکردی است. کد نرم‌افزار از زمان پیدایش «دستورالعمل نرم‌افزار» \lr{(Software Directive)} قادر به جذب حفاظت کپی‌رایت است، اما همان‌طور که در پرونده \lr{Bezpečnostní softwarová asociace} تأیید شد، بیان در کد اگر «توسط عملکرد فنی آن‌ها دیکته شده باشد»، نمی‌تواند به حد اصالت برسد.

بنابراین، شرایطی که نویسنده فضایی برای انجام انتخاب‌های خلاقانه داشته باشد، برای برآورده کردن الزام اصالت حیاتی است. علاوه بر این، در غیاب چنین الزاماتی، هنوز این مسئله وجود دارد که آیا انتخاب‌های خلاقانه واقعاً انجام شده‌اند یا خیر. انتخاب و/یا چیدمان داده‌ها در یک مجموعه داده می‌تواند، برای مثال، حداقل آستانه خلاقیت را برآورده کند، اما این انتخاب‌ها در عمل عموماً بر اساس مطلوبیت \lr{(utility)} انجام می‌شوند؛ انتخاب‌های انجام شده در انتخاب داده‌ها اغلب توسط کسب‌وکار اصلی یک شرکت تعیین می‌شوند و داده‌ها به دلایل عملی مانند ترتیب حروف الفبا یا تاریخ چیده می‌شوند.

آخرین معیار مستلزم آن است که اثر در حوزه «ادبیات و هنر» باشد. آنچه هنر یا ادبیات را تشکیل می‌دهد در رژیم کپی‌رایت بسیار گسترده درک می‌شود. برای مثال، ادبیات برای اهداف حفاظت کپی‌رایت می‌تواند اساساً شامل هر چیزی که شامل کلمه مکتوب است باشد. همان‌طور که در بالا ذکر شد، حتی می‌تواند کد نرم‌افزار را نیز پوشش دهد. این بدان معناست که داده‌ها — چه عددی و چه متنی — نیز در این دسته گسترده قرار می‌گیرند.

برخی دیگر از نمونه‌های آثار که ممکن است محافظت شوند عبارتند از کتاب‌ها، نقاشی‌ها، طرح‌ها، نقشه‌ها، معماری، مواد طراحی مقدماتی برای کد نرم‌افزار، فیلم‌ها، ترکیبات موسیقی، اشعار، توپوگرافی، آثار رقص‌پردازی و غیره: ماده ۲(۱) کنوانسیون برن شامل بیش از ۲۰ نمونه از انواع آثاری است که در حوزه ادبیات و هنر قرار می‌گیرند.

% =================================================================
% ۱۸.۲.۲ حق خاص پایگاه داده
% =================================================================
\subsubsection{حق خاص پایگاه داده}

اگر موادی مانند مجموعه داده‌ها و مواد طراحی مقدماتی برای کد نرم‌افزار بخشی از یک پایگاه داده باشند، استفاده از آن‌ها ممکن است توسط «حفاظت خاص پایگاه داده» محدود شود. به دلیل حفاظت محدودی که توسط رژیم کپی‌رایت در پایگاه‌های داده ارائه می‌شد، «دستورالعمل پایگاه داده» \lr{(Database Directive)} در سال ۱۹۹۶ برای تقویت بیشتر اقتصاد اطلاعات در اتحادیه اروپا تصویب شد. تا به امروز، این رژیم همچنان یک ابداع اروپایی است (برای مثال، هیچ معادل دقیقی در ایالات متحده ندارد).

حق خاص پایگاه داده بنابراین از پایگاه‌های داده‌ای محافظت می‌کند که فاقد اصالت هستند. با این حال، این بدان معنا نیست که اگر حفاظت کپی‌رایت بر روی محتوا، انتخاب و/یا چیدمان پایگاه داده وجود داشته باشد، نمی‌تواند حفاظت خاص پایگاه داده نیز وجود داشته باشد. این دو حق می‌توانند روی یک پایگاه داده واحد همزیستی داشته باشند.

یک مجموعه داده به احتمال زیاد تحت پوشش این حق قرار می‌گیرد اگر: (۱) یک «پایگاه داده» باشد که برای آن (۲) «سرمایه‌گذاری مرتبط» انجام شده است (۳) که آن سرمایه‌گذاری «قابل توجه» \lr{(substantial)} باشد.

برای اینکه یک مجموعه داده شرط اول — یعنی پایگاه داده بودن — را برآورده کند، ابتدا لازم است که مجموعه داده یک مجموعه یا تالیف از مواد باشد. چنین موادی شامل آثار کپی‌رایت شده، اعداد، حقایق و داده‌ها هستند، اما محدود به این دسته‌ها نمی‌شوند. سپس، چنین موادی باید سازماندهی، ذخیره و از طریق ابزارهای الکترونیکی یا غیرالکترونیکی قابل دسترسی باشند. این بدان معناست که یک سند مکتوب که تمام الزامات دیگر را برآورده می‌کند نیز می‌تواند به عنوان یک پایگاه داده واجد شرایط باشد. با این حال، برای یک پایگاه داده فیزیکی، لازم نیست که مواد به صورت فیزیکی به روشی سازمان‌یافته ذخیره شوند.

معیار دوم مستلزم آن است که یک سرمایه‌گذاری مرتبط انجام شده باشد. این بدان معناست که سرمایه‌گذاری باید در جمع‌آوری، تأیید و/یا ارائه داده‌ها برای پایگاه داده انجام شود. همان‌طور که در پرونده \lr{BHB v William Hill} روشن شد، سرمایه‌گذاری در دسته‌های دیگر مانند «خلق داده‌ها» برای برآورده کردن این معیار بی‌ربط است.

چنین سرمایه‌گذاری می‌تواند از طریق منابع مالی، منابع انسانی و منابع مادی انجام شود. سرمایه‌گذاری از طریق منابع انسانی می‌تواند، برای مثال، در تلاش یا زمان انجام شود. برای منابع مادی، سرمایه‌گذاری در تجهیزات برای ساخت پایگاه داده مانند سخت‌افزار و نرم‌افزار انجام می‌شود. البته، چنین نوع سرمایه‌گذاری‌هایی هزینه مالی نیز دارند. علاوه بر این، ورودی انسانی عموماً در کار با تجهیزات برای ساخت یک پایگاه داده مورد نیاز است. در واقعیت، ارتباط بین این سه نوع سرمایه‌گذاری اغلب ترکیبی از هر سه را با تأکید بر منابع مالی ایجاد می‌کند.

علاوه بر این، چنین سرمایه‌گذاری‌هایی نباید برای اهداف دیگری انجام شده باشند. برای مثال، رایانه‌هایی که برای ایجاد پایگاه داده استفاده می‌شوند اغلب صرفاً برای آن هدف خریداری نمی‌شوند. در آن صورت، سرمایه‌گذاری عموماً در ایجاد حفاظت خاص پایگاه داده به حساب نمی‌آید.

معیار آخر — اینکه سرمایه‌گذاری باید «قابل توجه» باشد — کمی مبهم‌تر است. دستورالعمل پایگاه داده راهنمایی قاطعی در مورد معنای این معیار یا چگونگی اعمال آن ارائه نمی‌دهد. رویه قضایی تاکنون بیشتر با مبالغ بالای سرمایه‌گذاری مالی سروکار داشته است، بنابراین این موارد نیز راهنمایی زیادی در مورد آستانه قابل توجه بودن ارائه نمی‌دهند. متأسفانه، سقف و کف دقیق این معیار نیز همچنان موضوع بحث‌های آکادمیک سنگین است، اما گنجاندن آن‌ها فراتر از اهداف این فصل خواهد بود.

این آستانه، برخلاف آنچه کلمه «قابل توجه» ممکن است پیشنهاد کند، نباید به عنوان «زیاد» تفسیر شود. در عوض، این معیار بهتر است به عنوان مستلزم سرمایه‌گذاری که «خیلی ناچیز نباشد» درک شود. این حدود در متن اصلی — نه زیاد، فقط نه خیلی ناچیز — عموماً در کشورهای عضو اتحادیه اروپا مانند آلمان پذیرفته شده‌اند. یک مثال روشن از چنین سرمایه‌گذاری ناچیزی، یک کارمند واحد از یک شرکت بزرگ است که تنها چند ساعت را به ساخت پایگاه داده اختصاص می‌دهد. مثالی از چیزی که واجد شرایط خواهد بود، سرمایه‌گذاری در تأیید مقدار زیادی داده با یک مجموعه داده دیگر است.

% =================================================================
% ۱۸.۲.۳ حق اسرار تجاری
% =================================================================
\subsubsection{حق اسرار تجاری}

طبق «دستورالعمل اسرار تجاری» \lr{(Trade Secret Directive)}، اگر مجموعه داده شامل (۱) اطلاعاتی باشد که در محافل مربوطه شناخته شده نیست، (۲) دارای ارزش تجاری باشد، و (۳) توسط شرکت مورد نظر مخفی نگه داشته شود، مجموعه داده ممکن است به عنوان یک راز تجاری محافظت شود.

معیار اول مستلزم آن است که اطلاعات مورد نظر به راحتی قابل دسترسی یا شناخته شده در محافل مربوطه نباشد. محافل مربوطه به افرادی اشاره دارد که عموماً با این نوع اطلاعات سروکار دارند، که به این معنی است که اگر موضوع محافظت شده شامل انواع مختلفی از اطلاعات باشد، محفل مربوطه ممکن است برای هر نوع اطلاعات متفاوت باشد. بنابراین، نمی‌تواند اطلاعات بی‌اهمیت یا نوعی را که از طریق تجربه شغلی عادی به دست می‌آید پوشش دهد. اطلاعاتی که می‌تواند تحت پوشش حق اسرار تجاری قرار گیرد حداقل شامل دانش فنی \lr{(know-how)}، اطلاعات تجاری یا اطلاعات فناوری است، اما ممکن است در قوانین داخلی گسترده‌تر تعریف شود.

دوم، اطلاعات باید دارای ارزش تجاری باشد. مهم نیست که این ارزش بالفعل باشد یا بالقوه. آنچه مهم است این است که اگر راز تجاری به خطر بیفتد، منافع دارنده حقِ راز تجاری — خواه ماهیت علمی، فنی، تجاری یا مالی داشته باشد — آسیب ببیند. بنابراین باید به دلیل مخفی بودن ارزش تجاری داشته باشد. اگر در صورت سوءاستفاده ارزش آن به طور منفی تحت تأثیر قرار نگیرد، برآورده شدن معیار دوم مورد تردید است.

در نهایت، دارنده حق راز تجاری باید تلاش‌های معقولی برای مخفی نگه داشتن اطلاعات انجام دهد. البته، این موضوع تابع شرایط پرونده است. در برخی موارد، ممکن است مخفی نگه داشتن اطلاعات دشوارتر باشد یا شرایط ممکن است اقدامات متفاوتی نسبت به موارد دیگر بطلبد.

این واقعیت که افراد زیادی می‌دانند، لزوماً به این معنی نیست که شرکت در تلاش خود برای برآورده کردن این معیار شکست خورده است. برای مثال، بسیاری از کارمندان ممکن است برای ساخت یک محصول به دانش (بخش‌هایی از) راز تجاری نیاز داشته باشند. تا زمانی که آن‌ها تحت تعهدات قراردادی رازداری باشند، مهم نیست که چه تعداد می‌دانند. همین امر برای توزیع‌کنندگانی که اطلاعات خاصی را تحت یک توافق‌نامه عدم افشا \lr{(NDA)} دریافت کرده‌اند تا بتوانند کار خود را انجام دهند، صادق است.

% =================================================================
% ۱۸.۲.۴ خلاصه
% =================================================================
\subsubsection{خلاصه}

الزامات شکلی هر یک از حقوق مالکیت فکری را می‌توان به سه مؤلفه اساسی تجزیه کرد. قرار دادن آن‌ها در کنار یکدیگر در یک جدول، تصویر زیر را ایجاد می‌کند (جدول ۱۸.۲).

% جدول ۱۸.۲
\begin{table}[h!]
	\centering
	\caption{الزامات شکلی \lr{(Source: Author's own table)}}
	\vspace{0.2cm}
	% تعریف ۳ ستون مساوی
	\begin{tabular}{|p{0.3\linewidth}|p{0.3\linewidth}|p{0.3\linewidth}|}
		\hline
		\textbf{حق تکثیر (Copyright)} & \textbf{حق خاص پایگاه داده (Sui generis)} & \textbf{حق اسرار تجاری (Trade secret)} \\
		\hline
		۱. بیان \lr{(Expression)} & ۱. پایگاه داده & ۱. عدم دسترسی آسان \\
		\hline
		۲. اصالت \lr{(Originality)} & ۲. سرمایه‌گذاری مرتبط & ۲. ارزش تجاری \\
		\hline
		۳. ادبیات و هنر & ۳. قابل توجه \lr{(Substantial)} & ۳. مخفی نگه داشته شود \\
		\hline
	\end{tabular}
	\label{tab:18-2-formal-requirements}
\end{table}

% =================================================================
% ۱۸.۳ دامنه حفاظت
% =================================================================
\subsection{دامنه حفاظت}
\label{sec:18-scope-protection}

% -----------------------------------------------------------------
% ۱۸.۳.۱ حق تکثیر: موضوع مورد حمایت
% -----------------------------------------------------------------
\subsubsection{حق تکثیر: موضوع مورد حمایت}

اگر یک مجموعه داده از طریق یک یا چند مورد از این مسیرها محافظت شود، هنوز محدودیتی در مورد اینکه این حقوق دقیقاً از چه چیزی و در برابر چه چیزی محافظت می‌کنند، وجود دارد.

زمانی که یک مجموعه داده یا کد نرم‌افزار الزامات حفاظت کپی‌رایت را برآورده می‌کند، این حفاظت تنها به «بیان اصیل» محدود می‌شود. این بدان معناست که حفاظت هرگز نمی‌تواند به مواردی مانند محتوای واقعی یا ایده‌ها گسترش یابد. علاوه بر این، اگر تنها انتخاب و/یا چیدمان یک مجموعه داده توسط کپی‌رایت محافظت شود و نه خود داده‌ها، بیان تنها در انتخاب و/یا چیدمان وجود دارد. برای کد نرم‌افزار، این بدان معناست که کپی‌رایت تنها می‌تواند بر روی کدی قرار گیرد که توسط عملکردهای فنی دیکته نشده باشد. بنابراین یک شخص ثالث می‌تواند از محتویات مجموعه داده یا چنین بخش‌های محافظت‌نشده‌ای از کد نرم‌افزار استفاده کند.

علاوه بر این، کپی‌رایت تنها از بیان اصیل در برابر انواع خاصی از استفاده توسط دیگران محافظت می‌کند. به عبارت دیگر، دارنده کپی‌رایت حقوق خاصی برای طرد کردن \lr{(exclude)} دارد. متفاوت از آنچه اصطلاح «کپی‌رایت» (حق تکثیر) نشان می‌دهد، این یک حق واحد نیست بلکه دسته‌ای از حقوق است. دسته حقوق شامل حقوق بهره‌برداری است که به عنوان حقوق اقتصادی شناخته می‌شوند.

چندین حق اقتصادی در «دستورالعمل جامعه اطلاعاتی» \lr{(InfoSoc Directive)} گنجانده شده است، اما تنها حق تکثیر \lr{(reproduction)} و حق در دسترس عموم قرار دادن \lr{(make public)} برای استفاده از داده‌ها و نرم‌افزار اهمیت ویژه‌ای دارند.

حق تکثیر مستلزم آن است که، در اصل، تنها دارنده کپی‌رایت حق دارد کپی‌هایی از اثر خود تهیه کند. علاوه بر این، مهم است توجه شود که تکثیر نباید دقیق باشد. گرفتن عکس از یک نقاشی نیز تکثیر اثر محسوب می‌شود. ابزارهای مورد استفاده برای کپی کردن برای این حق اهمیتی ندارند.

علاوه بر این، لازم نیست که کپی از کل اثر باشد. آنچه مهم است این است که باید به اندازه‌ای کپی شود که کار فکری و خلاقانه هنرمند را نمایش دهد. در پرونده \lr{Infopaq v Danske Dagblades Forening} مشخص شد که نمونه‌ای به کوچکی ۱۱ کلمه از مقالات روزنامه قادر به انجام این کار است. در نتیجه، قابل بحث است که بخش کوچکی از مجموعه داده یا کد نیز می‌تواند انتخاب‌های خلاقانه نویسنده را منتقل کند. اگر چنین باشد، در غیاب یک استثنای قابل اجرا، حتی استفاده از چنین گزیده‌های کوچکی نیاز به مجوز دارد.

دوم، حق در دسترس عموم قرار دادن است. برای مثال، فکر کنید به قرار دادن محتوای محافظت شده در یک وب‌سایت یا استفاده از هایپرلینک به محتوای محافظت شده. دقت کنید که این تا حدودی ساده‌سازی شده است. آنچه باید به عنوان در دسترس عموم قرار دادن درک شود و چه کسی باید به عنوان انجام‌دهنده این عمل درک شود، به دلیل تحولات قانونی و قضایی اخیر در سطح اتحادیه اروپا هنوز در حال تکامل است. در اکثر موارد، نوعی تکثیر لازم است تا بتوان آن را عمومی کرد. استثنائات قابل توجه در اینجا استفاده از هایپرلینک‌ها یا نمایش نسخه اصلی (مثلاً یک نقاشی در موزه) است.

آنچه این موضوع برای اشخاص ثالث معنا می‌دهد این است که آن‌ها نمی‌توانند بدون مجوز به طور قانونی در این استفاده‌ها از بیان اصیل مشارکت کنند. دارنده کپی‌رایت می‌تواند، برای مثال، از طریق لایسنس به دیگران اجازه دهد اثرش را تکثیر کنند. از آنجا که ۱۱ کلمه می‌تواند انتخاب‌های خلاقانه نویسنده را منتقل کند، الزام کسب مجوز خیلی سریع وارد عمل می‌شود.

در اصل، چنین مجوزی تنها می‌تواند از دارنده کپی‌رایت اخذ شود. دارنده حق عموماً یک شخص حقیقی است — نویسنده یا خالق. زمانی که اثری به سفارش خلق شده باشد، تخصیص کپی‌رایت بستگی به این دارد که چه کسی انتخاب‌های خلاقانه را انجام داده است.

در برخی موارد، انتخاب‌های خلاقانه ممکن است توسط چندین بازیگر انجام شده باشد که عموماً منجر به حقوق مشترک بر روی اثر می‌شود. با این حال، این موضوع در مورد خلق در حین استخدام متفاوت است. برای مثال، حقوق بهره‌برداری بر روی اثر اگر توسط کارمند در جریان استخدام و بنا به دستورات کارفرما خلق شده باشد، متعلق به کارفرما است. علاوه بر این، در مورد نرم‌افزار، اداره انتشارات در خلاصه دستورالعمل اجرایی خود روشن کرد که کشورهای عضو اتحادیه اروپا ممکن است مقرر کنند که اشخاص حقوقی یا نهادها نیز می‌توانند دارنده حق باشند. در برخی حوزه‌های قضایی، همه حقوق ممکن است همیشه از نویسنده به دیگری قابل انتقال نباشد.

% -----------------------------------------------------------------
% ۱۸.۳.۲ حفاظت خاص پایگاه داده
% -----------------------------------------------------------------
\subsubsection{حفاظت خاص پایگاه داده}

حق خاص پایگاه داده با در نظر گرفتن سرمایه‌گذار ایجاد شده است، بنابراین صرفاً سازنده واقعی بودن برای دارنده حق بودن کافی نیست. طبق دستورالعمل پایگاه داده، دارنده حق شخصی است که ابتکار عمل و ریسک سرمایه‌گذاری را بر عهده می‌گیرد. پیمانکاران فرعی و کار برای استخدام به صراحت از این تعریف مستثنی شده‌اند.

اگر پایگاه داده توسط یک کارمند ساخته شده باشد، تخصیص حقوق بستگی به معیارهای قانون ملی دارد. اگر چندین نفر یا نهاد در یک پایگاه داده مشارکت داشته باشند، ممکن است حقوق مشترک وجود داشته باشد. برخلاف کپی‌رایت، حق خاص پایگاه داده کاملاً قابل انتقال است.

مانند کپی‌رایت، حق خاص پایگاه داده یک حق واحد نیست. زمانی که یک پایگاه داده تحت پوشش حفاظت خاص پایگاه داده قرار می‌گیرد، دارنده حق حقوق انحصاری برای (۱) استخراج \lr{(extraction)} و (۲) استفاده مجدد \lr{(reutilization)} را دارد. این حقوق باید به شرح زیر درک شوند.

**استخراج** به انتقال پایگاه داده یا بخش قابل توجهی از آن اشاره دارد. این انتقال ممکن است دائمی یا موقت باشد. علاوه بر این، وسیله‌ای که از طریق آن منتقل می‌شود اهمیتی ندارد. همچنین بی‌ربط است که پایگاه داده به کجا منتقل می‌شود (نوع رسانه). آنچه مهم است این است که پایگاه داده یا بخش قابل توجهی از آن منتقل شود.

این بدان معناست که هر شخصی غیر از دارنده حق، در اصل نیاز به مجوز از دارنده حق دارد تا این عمل را به طور قانونی انجام دهد. با این حال، مجوز برای استخراج سیستماتیک بخش‌های ناچیز نیز مورد نیاز است. این موضوع در تعریف استخراج گنجانده شده است تا با «دوشیدن» \lr{(milking)} مبارزه شود. این فرآیندِ انتقال مکرر بخش‌های کوچک پایگاه داده است تا زمانی که کل پایگاه داده یا بخش قابل توجهی از آن منتقل شود.

نوع دیگر استفاده — **استفاده مجدد** — به در دسترس قرار دادن پایگاه داده یا بخش قابل توجهی از آن برای عموم اشاره دارد. این شامل توزیع یا اجاره کپی‌ها، انتقال آنلاین پایگاه داده و سایر انواع انتقالات است. هر روشی که در آن پایگاه داده عمومی شود، تحت این تعریف قرار می‌گیرد. در اصل، این حق به دارنده حق، حق انحصاری انجام استفاده مجدد ضمنی از (بخش قابل توجهی از) پایگاه داده را می‌دهد.

با این حال، درست مانند حق استخراج، حق استفاده مجدد نیز در برابر استفاده مجدد سیستماتیک از بخش‌های ناچیز محافظت می‌کند. باز هم، اگر این تعریف محدود به بخش‌های قابل توجه یا کل پایگاه داده بود، این فرصت را برای اشخاص ثالث فراهم می‌کرد که همچنان (بخش قابل توجهی از) پایگاه داده را ارتباط دهند، فقط هر بار بخش کوچکتری را. در نهایت، یک مورد آخر وجود دارد که در آن استفاده مجدد وجود دارد. این شامل استفاده از یک موتور جستجوی متا \lr{(meta search engine)} با عملکردهای خاص است.

یک موتور جستجوی متا، موتور جستجویی است که امکان جستجو در تعدادی از پایگاه‌های داده دیگر را فراهم می‌کند. عموماً این موتور کوئری جستجو را که توسط بازدیدکننده موتور جستجوی متا وارد شده، به سایر موتورهای جستجو منتقل می‌کند. چیزی از پایگاه‌های داده‌ای که در آن‌ها جستجو می‌کند کپی نمی‌کند، بلکه نتایج جستجو را نشان می‌دهد، از جمله نتایج سایر پایگاه‌های داده.

در پرونده \lr{Innoweb BV v Wegener} مشخص شد که چنین موتور جستجوی متایی احتمالاً (بخش قابل توجهی از) پایگاه داده را مورد استفاده مجدد قرار می‌دهد اگر سه عملکرد زیر وجود داشته باشد.

اول، فرم‌های جستجوی ارائه شده به کاربر نهایی توسط موتور جستجوی متا و پایگاه داده دیگر اساساً یکسان عمل می‌کنند. دوم، کوئری‌ها برای کاربر نهایی به صورت بلادرنگ به سایر موتورهای جستجو ترجمه می‌شوند. این بدان معناست که تمام اطلاعات پایگاه داده دیگر پس از شروع جستجو توسط کاربر نهایی موتور جستجوی متا، به صورت بلادرنگ جستجو می‌شود. سوم و در نهایت، نتایج همه با هم به ترتیبی ارائه می‌شوند که معیارهای مشابهی با معیارهای استفاده شده توسط پایگاه داده دیگر را منعکس می‌کند. برای این منظور، از قالب وب‌سایت خود موتور جستجوی متا در نمایش نتایج استفاده می‌شود و موارد تکراری را با هم به عنوان یک آیتم بلوک نشان می‌دهد.

برای تکرار، اگر یک موتور جستجوی متا که سایر پایگاه‌های داده را جستجو می‌کند به روش فوق عمل کند، اپراتور این موتور جستجوی متا احتمالاً درگیر استفاده مجدد از (بخش‌های قابل توجهی از) پایگاه داده دیگر است. البته، این بدان معنا نیست که اگر یک موتور جستجوی متا این ویژگی‌ها را نداشته باشد، با این وجود نمی‌تواند استفاده مجدد وجود داشته باشد.

برای هر دوی این حقوق، کلمه **«قابل توجه»** دوباره نقش بازی می‌کند. برای اهداف استخراج و استفاده مجدد، اصطلاح «قابل توجه» به حجم داده‌های یک پایگاه داده اشاره دارد، به طور خاص‌تر، حجم داده‌هایی که نسبت به کل پایگاه داده استخراج یا مورد استفاده مجدد قرار می‌گیرند \lr{(see BHB v William Hill)}. در اینجا پیوندی بین سرمایه‌گذاری و دو حق وجود دارد.

راه آسان برای برخورد با این موضوع، رویکرد کمی \lr{(quantitatively)} است. مثال زیر را در نظر بگیرید. سرمایه‌گذاری قابل توجهی در جمع‌آوری، تأیید و/یا ارائه داده‌ها انجام شده بود، اما تفاوت‌های معنی‌داری در سرمایه‌گذاری در سراسر داده‌ها وجود نداشت. یک شخص ثالث اکنون نیمی از داده‌های پایگاه داده را استخراج می‌کند. این بدان معناست که نیمی از سرمایه‌گذاری توسط بخش استخراج شده نمایندگی می‌شود. بخشی که استخراج شده است بنابراین احتمالاً قابل توجه است.

با این حال، اینکه استخراج یا استفاده مجدد قابل توجه است، می‌تواند به صورت کیفی \lr{(qualitatively)} نیز آزمایش شود. این کمی مبهم‌تر است. شرایط مثال ما تا حدودی تغییر می‌کند. اکنون، داده‌های خاصی در پایگاه داده وجود دارد که نیاز به سرمایه‌گذاری بسیار بیشتری در جمع‌آوری، تأیید و/یا ارائه آن‌ها نسبت به بقیه داده‌ها داشته است. داده‌های «گران‌تر» تنها بخش کوچکی از کل پایگاه داده هستند.

یک شخص ثالث اکنون تنها بخشی از پایگاه داده را که حاوی داده‌های «گران‌تر» است، مورد استفاده مجدد قرار می‌دهد. اگرچه داده‌های کمتری است، اما بخش بزرگتری از سرمایه‌گذاری را نشان می‌دهد. این بدان معناست که احتمالاً چنین استفاده مجددی توسط شخص ثالث از نظر کیفی قابل توجه خواهد بود. در هر دو مثال، شخص ثالث احتمالاً نمی‌تواند این اعمال را بدون مجوز از دارنده حق یا توسط قانون انجام دهد.

% -----------------------------------------------------------------
% ۱۸.۳.۳ حق اسرار تجاری
% -----------------------------------------------------------------
\subsubsection{حق اسرار تجاری}

دستورالعمل اسرار تجاری تصریح می‌کند که دارنده اسرار تجاری هر شخص حقیقی یا حقوقی است که به طور قانونی کنترل راز تجاری را در اختیار دارد. مانند حق خاص پایگاه داده، این حق می‌تواند کاملاً منتقل شود. حق اسرار تجاری در برابر اکتساب، استفاده و/یا افشای غیرقانونیِ موضوعِ محافظت شده محافظت می‌کند. این اعمال باید بسیار گسترده تفسیر شوند. هر عملی مغایر با رویه‌های تجاری صادقانه، دسترسی غیرمجاز و/یا تصاحب هر ماده‌ای که حاوی موضوع محافظت شده باشد، تحت اکتساب غیرقانونی قرار می‌گیرد.

همین امر برای موادی که اطلاعات راز تجاری می‌تواند از آن‌ها استخراج شود، صادق است. البته، اگر شخصی سپس اقدام به استفاده و/یا افشای راز تجاری کند، این نیز غیرقانونی خواهد بود. استفاده یا افشای موضوع محافظت شده در نقض یک وظیفه قراردادی — از جمله توافق‌نامه محرمانگی — یا هر وظیفه دیگری که محدودیت‌هایی بر آن اعمال تحمیل می‌کند نیز غیرقانونی است. علاوه بر این، استفاده غیرقانونی شامل تولید کالاهای ناقض، یا ارائه یا قرار دادن آن‌ها در بازار است. ذخیره‌سازی، واردات و صادرات کالاهای ناقض برای آن هدف نیز در این تعریف قرار می‌گیرند. یک کالا ناقض است اگر موضوع محافظت شده‌ای که به طور غیرقانونی اکتساب، استفاده یا افشا شده است، به روشی معنادار به (فرآیند تولید یا بازاریابی) یک محصول کمک کند.

حق اسرار تجاری مسلماً شکننده‌ترین حق مالکیت فکری است. زمانی که کپی‌رایت یا حق خاص پایگاه داده نقض شود، این حقوق مالکیت فکری به حیات خود ادامه می‌دهند. اما زمانی که داده‌های تحت پوشش یک حق اسرار تجاری به گونه‌ای مورد سوءاستفاده قرار گیرند که دیگر شرایط مربوط به محرمانه بودن خود را برآورده نکنند، حق منقضی می‌شود.

با این حال، مهم است تکرار شود که راز تجاری در برابر اعمال غیرمجاز محافظت می‌کند. مثال زیر را در نظر بگیرید. داده‌های تحت پوشش یک راز تجاری تحت یک توافق‌نامه عدم افشا در قبال پرداخت فاش می‌شوند. اگر وظایف ارائه‌دهنده و گیرنده — قراردادی و غیره — مانع این انتقال داده تحت شرایط نشوند، احتمالاً قانونی است. چنین افشایی احتمالاً راز تجاری را دست‌نخورده باقی می‌گذارد. معامله تحت یک توافق‌نامه عدم افشا لزوماً منجر به از دست دادن حق اسرار تجاری نمی‌شود. بنابراین قراردادهایی مانند قراردادهای استخدامی با بندهای محرمانگی و توافق‌نامه‌های عدم افشا ابزارهای حیاتی برای دارنده حق اسرار تجاری هستند.

% -----------------------------------------------------------------
% ۱۸.۳.۴ خلاصه
% -----------------------------------------------------------------
\subsubsection{خلاصه}

اگر یک مجموعه داده واجد شرایط کپی‌رایت، حفاظت خاص پایگاه داده و/یا حق اسرار تجاری باشد، حفاظت همچنان محدود به موضوع خاصی است. علاوه بر این، تنها در برابر اعمال غیرقانونی خاصی که توسط شخصی غیر از دارنده حق انجام شود، محافظت می‌شود (نگاه کنید به جدول ۱۸.۳). چنین اعمالی بدون مجوز ارائه شده توسط دارنده حق یا قانون (مثلاً استثنا) غیرقانونی هستند.

% جدول ۱۸.۳
\begin{table}[h!]
	\centering
	\caption{دامنه حفاظت \lr{(Source: Author's own table)}}
	\vspace{0.2cm}
	% تعریف ۳ ستون مساوی
	\begin{tabular}{|p{0.3\linewidth}|p{0.3\linewidth}|p{0.3\linewidth}|}
		\hline
		& \textbf{حق تکثیر (Copyright)} & \textbf{حق خاص پایگاه داده (Sui generis)} \\ 
		\hline
		\textbf{موضوع} & بیان اصیل & پایگاه داده \\
		\hline
		\textbf{حفاظت} & تکثیر و عمومی‌سازی & استخراج و استفاده مجدد \\
		\hline
	\end{tabular}
	\label{tab:18-3-scope}
\end{table}

% =================================================================
% ۱۸.۴ استثنائات و محدودیت‌ها
% =================================================================
\subsection{استثنائات و محدودیت‌ها}
\label{sec:18-exceptions-limitations}

% -----------------------------------------------------------------
% ۱۸.۴.۱ محدودیت‌های حقوق
% -----------------------------------------------------------------
\subsubsection{محدودیت‌های حقوق}

در برخی موارد، استفاده توسط شخص ثالث خارج از دامنه حق قرار می‌گیرد. محدودیت‌ها، همان‌طور که از نامشان پیداست، حفاظت را محدود می‌کنند. برای مثال، حقوق مالکیت فکری به طور نامحدود ادامه نمی‌یابد.

در اتحادیه اروپا، کپی‌رایت تا ۷۰ سال پس از مرگ نویسنده طبق «دستورالعمل مدت» \lr{(Term Directive)} ادامه می‌یابد. طبق دستورالعمل پایگاه داده، حفاظت خاص پایگاه داده به مدت ۱۵ سال از روز تکمیل پایگاه داده ادامه دارد، اما ساعت با هر تغییر و/یا سرمایه‌گذاری قابل توجه جدید دوباره شروع می‌شود.

حقوق اسرار تجاری در اینجا استثنا هستند: هیچ حداکثر مدتی برای حفاظت در دستورالعمل اسرار تجاری درج نشده است. حق اسرار تجاری تا زمانی که موضوع محافظت شده آن دیگر معیارها را برآورده نکند، ادامه خواهد داشت.

همان‌طور که قبلاً ذکر شد، کپی‌رایت به حقایق و ایده‌ها گسترش نمی‌یابد. علاوه بر این، حتی موضوعی که نه واقعیت است و نه ایده، زمانی که بخشی از بیان اصیل نباشد، می‌تواند خارج از دامنه حفاظت قرار گیرد. علاوه بر این، اصالت به این معنی است که انتخاب‌های خلاقانه توسط نویسنده انجام شده است، نه اینکه باید جدید باشد. این بدان معناست که در برابر «خلق مستقل» \lr{(independent creation)} محافظت نمی‌کند.

برای حق خاص پایگاه داده، حفاظت حول محور سرمایه‌گذاری می‌چرخد. اگر یک شخص ثالث به طور تصادفی بخش‌های ناچیزی را استخراج و/یا مورد استفاده مجدد قرار دهد، در اصل، این کار قانونی خواهد بود. با این حال، مرزهایی نیز در آنجا وجود دارد. در انجام این کار، دستورالعمل پایگاه داده شخص ثالث را ملزم می‌کند که مراقب باشد اعمالش با بهره‌برداری عادی از پایگاه داده توسط دارنده حق در تضاد نباشد یا به منافع او آسیب نامعقول نرساند. به طور خلاصه، اعمال شخص ثالث نباید به سرمایه‌گذاری «آسیب» برساند.

دامنه حفاظت ارائه شده توسط حق اسرار تجاری نیز محدودیت‌های خود را تحت دستورالعمل اسرار تجاری دارد. حق اسرار تجاری تنها در برابر اعمال غیرقانونی محافظت می‌کند. این بدان معناست که خلق یا کشف مستقل با حقوق اسرار تجاری تداخل ندارد.

علاوه بر این، «مهندسی معکوس» \lr{(reverse engineering)} پس از به دست آوردن قانونی یک محصول نیز حق اسرار تجاری را نقض نمی‌کند. این بدان معناست که برای یک شخص ثالث قانونی خواهد بود که محصولی را که توسط دارنده حق در بازار اتحادیه اروپا عرضه شده است بخرد و عملکرد آن را برای بهبود فرآیند تولید، محصول و/یا خدمات خود مطالعه کند. برای مثال، یک تولیدکننده خودرو می‌تواند سنسور خودرویی را که توسط رقیب در بازار عرضه شده است بخرد تا آن را مهندسی معکوس کند و از دانش به دست آمده برای بهبود سنسورهای خودروی خود استفاده کند.

در نهایت، چندین ارجاع به قابلیت انتقال حقوق اختصاص یافته به دارنده حق توسط این حقوق مالکیت فکری شده است. آنچه این بدان معناست این است که عموماً امکان «رزرو» یا انتقال قراردادی چنین حقوقی یا اجازه دادن به اعمال تحت شرایط خاص وجود دارد. دارندگان حق خودشان نیز می‌توانند به طور قراردادی حقوق خود را محدود کنند.

برای رزرو حقوق، برای مثال به وضعیت حقوق مشترک فکر کنید. ممکن است برای طرفین مفید باشد که به طور قراردادی تعیین کنند که مجوز همه دارندگان حق باید اخذ شود، نه فقط یکی. متناوباً، یک دارنده حق می‌تواند حق انحصاری را به یک توزیع‌کننده انحصاری منتقل کند تا حق مالکیت فکری را علیه ناقضان (ادعایی) اجرا کند و بدین ترتیب دستان خود را آزاد کند.

مثالی از اجازه دادن به اعمال تحت شرایط خاص را می‌توان در بسیاری از شرایط خدمات در صنعت بازی یافت. چنین شرایطی اغلب حاوی بندی است که به کاربران اجازه می‌دهد در اعمالی مانند پخش زنده بازی کردن خود شرکت کنند. مثال مناسب دیگر استفاده از یک آستانه است که به کاربران اجازه می‌دهد از مواد محافظت شده استفاده کنند تا زمانی که سودی بیش از تعداد مشخصی کسب نکنند یا به تعداد مشخصی از مشتریان نرسند (جدول ۱۸.۴).

% جدول ۱۸.۴
\begin{table}[h!]
	\centering
	\caption{محدودیت‌ها \lr{(Source: Author's own table)}}
	\vspace{0.2cm}
	\small
	\begin{tabular}{|p{0.22\linewidth}|p{0.22\linewidth}|p{0.22\linewidth}|p{0.22\linewidth}|}
		\hline
		& \textbf{حق تکثیر (Copyright)} & \textbf{حق خاص پایگاه داده (Sui generis)} & \textbf{حق اسرار تجاری (Trade secret)} \\
		\hline
		\textbf{حداکثر مدت} & ۷۰ سال پس از مرگ نویسنده & ۱۵ سال، اما قابل تمدید & — \\
		\hline
		\textbf{خارج از دامنه} & $\bullet$ حقایق \newline $\bullet$ ایده‌ها \newline $\bullet$ خلق مستقل & $\bullet$ استخراج بخش‌های ناچیز \newline $\bullet$ استفاده مجدد از بخش‌های ناچیز & $\bullet$ خلق مستقل \newline $\bullet$ کشف مستقل \newline $\bullet$ مهندسی معکوس \\
		\hline
		\textbf{محدودیت‌های قراردادی ممکن} & بله، روی حقوق بهره‌برداری\textsuperscript{الف} & بله & بله \\
		\hline
	\end{tabular}
	\vspace{0.1cm}
	\footnotesize{\newline \textsuperscript{الف} همان‌طور که قبلاً ذکر شد، دسته حقوق همیشه به طور کامل قابل انتقال نیست. با این حال، مهم است توجه شود که این عموماً در مورد حقوق بهره‌برداری صدق نمی‌کند.}
	\label{tab:18-4-limitations}
\end{table}

% -----------------------------------------------------------------
% ۱۸.۴.۲ استثنائات: وجه مشترک
% -----------------------------------------------------------------
\subsubsection{استثنائات: وجه مشترک}

اگر عملی تحت پوشش یک استثنا قرار گیرد، توسط قانون مجاز است. این بدان معناست که دارنده حق نمی‌تواند آن عمل را مجاز یا منع کند. طبق اسناد قانونی مانند کنوانسیون برن، موافقت‌نامه تریپس \lr{(TRIPS Agreement)} و دستورالعمل جامعه اطلاعاتی \lr{(InfoSoc Directive)}، استثنائات باید محدود به موارد خاص باشند و با بهره‌برداری عادی از اثر تداخل نداشته باشند و به منافع مشروع نویسنده آسیب نامعقول نرسانند. به طور کلی، این استثنائات در سراسر اتحادیه اروپا به نفع حفاظت بالای حقوق مالکیت فکری به صورت محدود اعمال می‌شوند.

استثنائات تا حدودی برای هر حق مالکیت فکری متفاوت است، اما وجه مشترکی وجود دارد. برای مثال، استثنا برای تدریس و تحقیق و اهداف امنیت عمومی یا یک رویه اداری یا قضایی هم در دستورالعمل جامعه اطلاعاتی (در مورد کپی‌رایت) و هم در دستورالعمل پایگاه داده وجود دارد.

در اولی، این‌ها استثنائاتی بر حق تکثیر دارنده حق هستند. در دومی، این استثنائات هم حق استخراج و هم حق استفاده مجدد را هدف قرار می‌دهند. با این حال، در رژیم خاص پایگاه داده، این استثنائات تنها توسط یک «کاربر قانونی» \lr{(lawful user)} قابل استناد هستند. برای مثال، دور زدن الزام اشتراک برای دسترسی به یک پایگاه داده غیرعمومی بدون مجوز را در نظر بگیرید. استخراج و/یا استفاده مجدد توسط چنین کاربری نمی‌تواند در دامنه این استثنائات قرار گیرد.

مثالی از تدریس و تحقیق می‌تواند نمایش کلیپ‌ها، (مواد طراحی مقدماتی برای) کد نرم‌افزار، متون کوچک، یا بخش‌هایی از یک پایگاه داده برای تصویرسازی به دانشجویان یا محققان باشد. برای واجد شرایط بودن، هر دو رژیم مستلزم آن هستند که اشخاص ثالث نباید چنین استفاده‌هایی را برای اهداف تجاری انجام دهند. در صورت امکان، منبع باید ذکر شود و استفاده نباید فراتر از آنچه برای هدف غیرتجاری دنبال شده لازم است، باشد.

برای استثنای اهداف امنیت عمومی یا یک رویه اداری یا قضایی، یک مثال می‌تواند کپی کردن یک اثر یا داده‌های خاص از یک پایگاه داده برای تأیید کالاهای وارداتی باشد. مثال دیگر می‌تواند گنجاندن چنین موادی در تصمیم کتبی یک پرونده دادگاه حول محور مسائل نقض کپی‌رایت و/یا حفاظت خاص پایگاه داده باشد. باز هم، چنین اعمالی نباید برای اهداف تجاری انجام شده باشند.

% -----------------------------------------------------------------
% ۱۸.۴.۳ استثنائات خاص برای هر حق
% -----------------------------------------------------------------
\subsubsection{استثنائات خاص برای هر حق}

رایج‌ترین و مرتبط‌ترین استثنائات خاص برای رژیم کپی‌رایت اتحادیه اروپا عبارتند از روزنامه‌نگاری، نقل قول برای نقد و بررسی، و کاریکاتور، پارودی و تقلید ادبی \lr{(pastiche)}. قانون اتحادیه اروپا، به طور خاص‌تر دستورالعمل جامعه اطلاعاتی، هیچ شرطی برای هیچ یک از این استثنائات ارائه نمی‌دهد. این بدان معناست که برای مثال، کشورهای عضو آزاد بودند که استثنائات را تنها به شرایط یا استفاده‌های خاص محدود کنند.

در نهایت — و شاید مهم‌تر از همه — استثنای «داده‌کاوی و متن‌کاوی» \lr{(text and data mining)} است که اخیراً در «دستورالعمل بازار واحد دیجیتال» \lr{(Digital Single Market Directive)} معرفی شده است. این مفهوم به بهترین وجه به عنوان هر تکنیک تحلیلی که خودکار است درک می‌شود. این تکنیک برای استخراج اطلاعات با تحلیل متن و داده‌ها به شکل دیجیتال استفاده می‌شود. این تمرین می‌تواند، برای مثال، برای کشف الگوها، روندها و همبستگی‌ها در یک مجموعه داده انجام شود.

دو نوع از این حق معرفی شده است، یکی متمرکز بر متن‌کاوی و داده‌کاوی برای اهداف علمی و دیگری عمومی. هر دو مستلزم آن هستند که دسترسی قانونی به آثاری که قرار است تحت متن‌کاوی و داده‌کاوی قرار گیرند، وجود داشته باشد. نوع اول اجازه ذخیره و نگهداری تکثیرهای آثار برای تحقیقات علمی را می‌دهد. با این حال، باید سطح مناسبی از امنیت در ذخیره‌سازی کپی‌های آثار وجود داشته باشد.

برای استثنای عمومی، پیش‌شرط این است که دارنده حق به صراحت استفاده از اثر خود را رزرو (ممنوع) نکرده باشد. در غیاب چنین رزروی، آثار می‌توانند تا زمانی که برای هدف دنبال شده با متن‌کاوی و داده‌کاوی لازم است، «کاوش»، نگهداری و ذخیره شوند.

تحت رژیم حقوق اسرار تجاری، مرتبط‌ترین استثنائات سه مورد زیر هستند. اول، اگر بتوان با موفقیت به آزادی بیان و حق دسترسی به اطلاعات استناد کرد، عمل ممکن است حق اسرار تجاری را نقض نکند.

علاوه بر این، اکتساب، استفاده یا افشای موضوع محافظت شده توسط یک حق اسرار تجاری در تعقیبِ آشکارسازی سوءرفتار، تخلف یا فعالیت غیرقانونی، مانند «افشاگری» \lr{(whistleblowing)}، نیز ممکن است قانونی باشد. علاوه بر این، مرتبط با یکی از محدودیت‌های ذکر شده قبلی، اگر وظایف مشروع آن‌ها به عنوان کارگران یا نمایندگان کارگران افشا را ایجاب کرده باشد، دارنده حق اسرار تجاری ممکن است نتواند علیه آن‌ها درخواست جبران خسارت کند (جدول ۱۸.۵).

% جدول ۱۸.۵
\begin{table}[h!]
	\centering
	\caption{خلاصه استثنائات \lr{(Source: Author's own table)}}
	\vspace{0.2cm}
	\small
	\begin{tabular}{|p{0.3\linewidth}|p{0.3\linewidth}|p{0.3\linewidth}|}
		\hline
		\textbf{حق تکثیر (Copyright)} & \textbf{حق خاص پایگاه داده (Sui generis)} & \textbf{حق اسرار تجاری (Trade secret)} \\
		\hline
		$\bullet$ تدریس و تحقیق \newline $\bullet$ اهداف امنیت عمومی و رویه اداری یا قضایی \newline $\bullet$ روزنامه‌نگاری \newline $\bullet$ نقل قول برای نقد و بررسی \newline $\bullet$ کاریکاتور، پارودی و تقلید ادبی \newline $\bullet$ متن‌کاوی و داده‌کاوی & $\bullet$ تدریس و تحقیق \newline $\bullet$ اهداف امنیت عمومی و/یا رویه اداری یا قضایی & $\bullet$ آزادی بیان و حق دسترسی به اطلاعات \newline $\bullet$ آشکارسازی سوءرفتار، تخلف یا فعالیت غیرقانونی \newline $\bullet$ وظایف مشروع کارگر(ان) (نمایندگان) \\
		\hline
	\end{tabular}
	\label{tab:18-5-exceptions}
\end{table}

% =================================================================
% ۱۸.۵ منابع جایگزین
% =================================================================
\subsection{منابع جایگزین}
\label{sec:18-alternative-sources}

این حقوق مالکیت فکری ممکن است از نظر دامنه و اهداف متفاوت باشند، اما کاملاً ممکن است که چندین مورد از آن‌ها برای (بخش‌هایی از) همان مجموعه داده یا کد قابل اجرا باشند. استثنائات این حقوق مختلف محدود به حقوق و اهداف خاص هستند. بنابراین، ممکن است عملی که تحت یک استثنا برای یک حق مالکیت فکری قرار می‌گیرد، به دلیل وجود حق دیگر مجاز نباشد. اگر شخص ثالثی نیاز به دسترسی به مجموعه داده‌هایی (تا حدی) تحت پوشش این حقوق داشته باشد، چندین گزینه برای به دست آوردن دسترسی قانونی وجود دارد.

ساده‌ترین گزینه، اخذ «مجوز» \lr{(license)} از دارنده حق برای استفاده از مجموعه داده‌های اوست. یک مجوز به صاحب امتیاز اجازه می‌دهد تا از موضوع محافظت شده مطابق با شرایط توافق شده، معمولاً در قبال پرداخت، استفاده کند. موضوع محافظت شده می‌تواند برای برخی یا تمام استفاده‌های تحت پوشش کپی‌رایت و/یا حفاظت خاص پایگاه داده مجوز داده شود، اما نویسنده یا سازنده مالک باقی می‌ماند. گزینه مشابه دیگر در اینجا ورود به یک توافق‌نامه موردی \lr{(ad hoc agreement)} یا شراکت با پرداخت مبلغ یا ارائه چیزی در ازای آن است.

متناوباً، گاهی اوقات امکان دسترسی به مجموعه داده‌های قابل مقایسه از طریق منابع دیگر، مانند «اطلاعات بخش عمومی» یا \lr{PSI} وجود دارد. این فرصتی بسیار جالب و مفید برای بررسی است زیرا دولت داده‌های زیادی در اختیار دارد — برای مثال به نقشه‌ها، تصمیمات دادگاه، داده‌های شرکت‌ها، آمار شهروندان و غیره فکر کنید — و ممکن است تعهدی برای انتشار آن داده‌ها و اجازه استفاده مجدد از آن‌ها داشته باشد (مانند قوانین آزادی اطلاعات)، اگرچه لزوماً رایگان نیست.

داده‌ها به احتمال زیاد زمانی تابع رژیم \lr{PSI} هستند که: (۱) مرتبط با اجرای فعالیت‌های دولتی باشند، (۲) هیچ حق مالکیت فکری متعلق به اشخاص ثالث بر روی آن‌ها وجود نداشته باشد، و (۳) داده‌ها به دلایل سیاست عمومی (از جمله حفاظت از داده‌ها) محرمانه نگه داشته نشوند.

بسته به مدل کسب‌وکار، گزینه دیگری که باید در نظر گرفت استفاده از نرم‌افزار یا داده‌های تابع طرح‌های مجوز باز («متن‌باز» یا \lr{Open Source}) است. استفاده از چنین داده‌ها یا نرم‌افزاری رایگان است، اما بسته به نوع مجوز، ممکن است انواع دیگری از محدودیت‌ها وجود داشته باشد. رایج‌ترین تقسیم‌بندی بین مجوزهای «سهل‌گیرانه» \lr{(permissive)} و مجوزهای «کپی‌لفت» (ضعیف یا قوی) \lr{(copyleft)} است.

این انواع مجوزها بهتر است به عنوان طیفی از کمترین محدودیت تا بیشترین محدودیت تصور شوند. هر دو نوع مجوز استفاده از موضوع را از نظر استفاده، تغییر و توزیع مجدد محدود نمی‌کنند، اما مجوزهای سهل‌گیرانه اجازه می‌دهند آثار مشتق شده اختصاصی (انحصاری) شوند در حالی که مجوزهای کپی‌لفت این اجازه را نمی‌دهند.

این بدان معناست که، برای مثال، یک شخص ثالث می‌تواند تغییراتی در موضوع تحت مجوز سهل‌گیرانه ایجاد کند و آن را تحت نوع متفاوتی از مجوز، مجوز دهد و توزیع کند. از طرف دیگر، یک مجوز کپی‌لفت ضعیف این اجازه را نمی‌دهد. چنین مجوزهایی حاوی بندی هستند که اختصاصی کردن مواد مشتق شده از موضوع آن یا مجوزدهی مجدد این مواد مشتق شده را ممنوع می‌کنند.

مجوزهای کپی‌لفت قوی علاوه بر این مستلزم آن هستند که موضوع آن نیز نتواند تحت مجوزی متفاوت از مجوز اصلی مجوز داده شود. این بدان معناست که اثری که تابع یک مجوز اختصاصی «عادی» است نمی‌تواند با اثر دیگری که تابع یک مجوز کپی‌لفت است ترکیب شود.

ارائه خدمات یا محصولات مکمل در بازار برای ایجاد یا به دست آوردن دسترسی به یک مجموعه داده مشابه نیز یک امکان است. برای مثال، یک شخص ثالث داده‌های مشابهی را می‌خواهد که توسط سنسورهای عرضه شده در بازار توسط یک رقیب تولید می‌شود. شخص ثالث می‌تواند تصمیم بگیرد نرم‌افزاری ارائه دهد که بتواند سنسورهای رقیب را راه اندازی کند یا سنسورهای رقیب ارائه دهد. گزینه دیگر در اینجا تبدیل مشتریان خودتان به جمع‌آوری‌کنندگان داده با واداشتن آن‌ها به اصلاح یا گزارش داده‌های خاص است. برای مثال، به گزارش افزودنی‌ها به یک نقشه یا تغییرات در یک خیابان فکر کنید.

در نهایت، اگر این مجموعه داده‌ها حاوی داده‌های شخصی هستند، می‌توانید از آن افراد بخواهید که از حق «قابلیت انتقال داده» \lr{(data portability)} خود از طریق تبلیغاتی برای مشتریان جدید یا موجودِ خدمات یا محصولات خودشان استفاده کنند. طبق مقررات عمومی حفاظت از داده‌ها \lr{(GDPR)}، این حق به اشخاص حقیقی فرصت می‌دهد تا داده‌های شخصی خود را از یک سرویس آنلاین به سرویس دیگر منتقل کنند.

الزاماتی که در اینجا باید برآورده شوند عبارتند از اینکه داده‌ها (۱) داده‌های شخصی باشند و (۲) توسط شخصی که داده‌های شخصی مربوط به اوست به کنترل‌کننده ارائه شده باشند. برای مثال، یک شرکت بیمه یا شهرداری می‌تواند در ازای انتقال داده‌های شخصی آن‌ها به شما، مزایایی مانند تخفیف در حق بیمه یا خدمات ارائه شده توسط شهرداری ارائه دهد.

به طور خلاصه، اگر حقوق مالکیت فکری بر روی یک مجموعه داده وجود داشته باشد و هیچ یک از استثنائات قابل اجرا نباشد، هنوز چندین راه برای به دست آوردن دسترسی قانونی وجود دارد. علاوه بر این، منابع جایگزین می‌توانند به عنوان منبع مکمل یا جایگزین برای مجموعه داده محافظت شده بررسی شوند.

% =================================================================
% نتیجه‌گیری (Conclusion)
% =================================================================
\vspace{0.8cm}
\noindent
\fcolorbox{black}{gray!15}{%
	\begin{minipage}{\dimexpr\linewidth-2\fboxsep-2\fboxrule}
		\vspace{0.3cm}
		\begin{center}
			\textbf{\Large نتیجه‌گیری}
		\end{center}
		\vspace{0.2cm}
		
		برای خلاصه کردن، هنگام برخورد با موضوعاتی مانند مجموعه داده‌ها و نرم‌افزار، مهم است که ابتدا تعیین کنید آیا حقوق مالکیت فکری ممکن است بر روی آن‌ها وجود داشته باشد یا خیر. اگر چنین باشد، استفاده از چنین موضوعاتی توسط اشخاص ثالث ممکن است محدود شود. اینکه کدام استفاده‌ها محدود شده‌اند و تحت چه شرایطی، بستگی به این دارد که کدام حق اعمال می‌شود و تا درجات مختلف، کدام رژیم (یعنی اتحادیه اروپا یا ایالات متحده) اعمال می‌شود.
		
		دوم، باید تعیین شود که دارنده حق کیست. اگر شما دارنده حق هستید، این بدان معناست که ممکن است بتوانید دسترسی و استفاده دیگران از موضوع محافظت شده را محدود کنید. اگر شخص دیگری است، چندین مسیر ممکن برای استفاده قانونی از موضوع محافظت شده آن طرف یا جایگزین‌هایی برای این موضوع وجود دارد، از اخذ رضایت از دارنده حق گرفته تا عمل در محدوده محدودیت‌ها یا استثنائات تا یافتن یا ایجاد منابع جایگزین.
		\vspace{0.3cm}
	\end{minipage}
}

% =================================================================
% پیام‌های کلیدی
% =================================================================
\vspace{0.8cm}
\noindent
\fcolorbox{black}{gray!15}{%
	\begin{minipage}{\dimexpr\linewidth-2\fboxsep-2\fboxrule}
		\vspace{0.3cm}
		\textbf{\large پیام‌های کلیدی}
		
		\begin{itemize}
			\setlength\itemsep{0.5em}
			
			\item[\textbf{--}] کپی‌رایت بر روی محتوا، انتخاب یا چیدمان یک مجموعه داده به دارنده حق، حق انحصاری تکثیر و در دسترس عموم قرار دادن مواد محافظت شده را می‌دهد.
			
			\item[\textbf{--}] حق خاص پایگاه داده \lr{(sui generis)} به دارنده حق، حق انحصاری استخراج و استفاده مجدد از بخش‌های قابل توجه پایگاه داده را می‌دهد.
			
			\item[\textbf{--}] حقوق اسرار تجاری بر روی داده‌ها از دارنده حق در برابر اکتساب، استفاده و افشای غیرقانونی مواد محافظت شده محافظت می‌کند.
			
			\item[\textbf{--}] یک شخص ثالث تنها در صورتی می‌تواند در استفاده قانونی از موضوع محافظت شده توسط این حقوق مشارکت کند که توسط دارنده حق یا توسط قانون (اگر استثنائات قابل اجرا باشند) مجاز شده باشد.
			
			\item[\textbf{--}] در غیاب مجوز، چندین راه وجود دارد که می‌توان به طور قانونی به (بخش‌هایی از) مجموعه داده یا نرم‌افزار یا به منابع قابل مقایسه دسترسی پیدا کرد.
		\end{itemize}
		\vspace{0.3cm}
	\end{minipage}
}

% =================================================================
% سوالات بحث و گفتگو
% =================================================================
\vspace{0.8cm}
\section*{سوالات بحث و گفتگو}
\addcontentsline{toc}{section}{سوالات بحث و گفتگو}

\begin{enumerate}
	\item چرا ما حقوق مالکیت فکری داریم؟
	\item تمایز بین ایده‌ها و بیان‌ها در قانون کپی‌رایت را چگونه توضیح می‌دهید؟
	\item لطفاً تمام انواع سرمایه‌گذاری‌های مرتبط برای حفاظت خاص پایگاه داده، از جمله ابزارهایی که از طریق آن‌ها چنین سرمایه‌گذاری‌هایی می‌تواند انجام شود را تعریف کنید.
	\item لطفاً به طور خلاصه وضعیت مهندسی معکوس تحت حفاظت اسرار تجاری را توضیح دهید.
\end{enumerate}

% =================================================================
% منابع برای مطالعه بیشتر
% =================================================================
\vspace{1cm}
\section*{منابع برای مطالعه بیشتر}
\addcontentsline{toc}{section}{منابع برای مطالعه بیشتر}

\begin{latin}
	\begin{itemize}
		\setlength\itemsep{0.5em}
		
		\item[] Agreement on Trade-Related Aspects of Intellectual Property Rights, Annex 1c of the Marrakesh Agreement Establishing the World Trade Organization, 1994.
		
		\item[] Berne Convention for the Protection of Literary and Artistic Works of 1886 as amended on September 28, 1979 (‘Berne Convention’).
		
		\item[] Case C-202/12 Innoweb BV v Wegener [2013] ECLI:EU:C:2013:850.
		
		\item[] Case C-203/02 British Horseracing Board v William Hill [2004] ECLI:EU:C:2004:695.
		
		\item[] Case C-393/09 Bezpečnostní softwarová asociace [2010] ECLI:EU:C:2010:816.
		
		\item[] Case C-5/08, Infopaq v Danske Dagblades Forening Case C-5/08, Infopaq v Danske Dagblades Forening [2009] ECLI:EU:C:2009:465.
		
		\item[] Convention on the Grant of European Patents of 1973 as amended last on 29 November 2000 (‘European Patent Convention’).
		
		\item[] Council Directive 93/98/EEC of 29 October 1993 harmonizing the term of protection of copyright and certain related rights [1993] OJ L290/9 (‘Term Directive’).
		
		\item[] Estelle Derclaye, ‘Database sui generis right: what is a substantial investment? A tentative definition’ (2005) IIC 36(1).
		
		\item[] Directive 2001/29/EC of the European Parliament and of the Council on the harmonisation of certain aspects of copyright and related rights in the information society [2001] OJ L167/10 (‘InfoSoc Directive’).
		
		\item[] Directive 2004/48/EC of the European Parliament and of the Council on the enforcement of intellectual property rights [2004] OJ L157/45 (‘Enforcement Directive’).
		
		\item[] Directive 2009/24/EC of the European Parliament and of the Council of 23 April 2009 on the legal protection of computer programs [2009] OJ L111/16 (‘Software Directive’).
		
		\item[] Directive (EU) 2016/943 of the European Parliament and of the Council on the protection of undisclosed know-how and business information (trade secrets) against their unlawful acquisition, use and disclosure [2016] OJ L 157/1 (‘Trade Secret Directive’).
		
		\item[] Directive 2019/790 of the European Parliament and of the Council on copyright and related rights in the Digital Single Market and amending Directives 96/9/EC and 2001/29/EC [2019] OJ L130/11 (‘Digital Single Market Directive’).
		
		\item[] Directive 96/6/EC of the European Parliament and of the Council of 11 March 1996 on the legal protection of databases [1996] OJ L77/20 (‘Database Directive’).
		
		\item[] Feist Publications, Inc., v. Rural Telephone Service Co., 499 U.S. 340 (1991).
		
		\item[] Husovec, M. (2019). How Europe Wants to Redefine Global Online Copyright Enforcement. TILEC Discussion Paper, 2019–2016.
		
		\item[] Publications Office in their Summary of Directive 2009/24/EC—the legal protection of computer programs, 23 January 2017.
		
		\item[] Regulation 2016/679 of the European Parliament and of the Council of 27 April 2016 on the protection of natural persons with regard to the processing of personal data and on the free movement of such data, and repealing Directive 95/46/EC [2016] OJ L 119/1 (‘General Data Protection Regulation’).
		
		\item[] Rosati, E. (2017). GS Media and its implications for the construction of the right of communication to the public within EU copyright architecture. \textit{Common Market Law Review}, 54(4), 1221–1242.
	\end{itemize}
\end{latin}