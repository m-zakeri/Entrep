% =================================================================
% فصل ۱۸: دیدگاه‌هایی از حقوق مالکیت فکری
% =================================================================

\clearpage
\raggedbottom 

% تنظیم شماره‌گذاری فصل روی ۱۸
\setcounter{section}{18}
\setcounter{subsection}{0}
\renewcommand{\thesection}{18}
\renewcommand{\thesubsection}{18.\arabic{subsection}}

% تنظیم عمق شماره‌گذاری برای زیربخش‌های احتمالی آینده
\setcounter{secnumdepth}{4}
\renewcommand{\theparagraph}{\thesubsubsection.\arabic{paragraph}}

% -----------------------------------------------------------------
% عنوان و نویسنده
% -----------------------------------------------------------------
\noindent
\fcolorbox{black}{gray!15}{%
	\begin{minipage}{\dimexpr\linewidth-2\fboxsep-2\fboxrule}
		\vspace{0.5cm}
		\begin{center}
			\textbf{\huge دیدگاه‌هایی از حقوق مالکیت فکری}
			\vspace{0.4cm}
			
			\large
			\textit{لیسا ون دونگن} \\
			\lr{\textit{Lisa van Dongen}}
		\end{center}
		\vspace{0.5cm}
	\end{minipage}
}

\vspace{0.8cm}

% -----------------------------------------------------------------
% فهرست مطالب داخلی
% -----------------------------------------------------------------
\noindent
\textbf{\Large فهرست مطالب}
\vspace{0.3cm}

{ \small
	\noindent \textbf{۱۸.۱} \hspace{0.2cm} \textbf{مقدمه} \par \vspace{0.1cm}
	
	\noindent \textbf{۱۸.۲} \hspace{0.2cm} \textbf{احراز معیارها} \par 
	\hspace{0.8cm} ۱۸.۲.۱ الزامات شکلی حق تکثیر (کپی‌رایت) \par
	\hspace{0.8cm} ۱۸.۲.۲ حق پایگاه داده خاص \lr{(Sui Generis)} \par
	\hspace{0.8cm} ۱۸.۲.۳ حق اسرار تجاری \par
	\hspace{0.8cm} ۱۸.۲.۴ خلاصه \par \vspace{0.1cm}
	
	\noindent \textbf{۱۸.۳} \hspace{0.2cm} \textbf{دامنه حفاظت} \par
	\hspace{0.8cm} ۱۸.۳.۱ حق تکثیر: موضوعات محافظت‌شده \par
	\hspace{0.8cm} ۱۸.۳.۲ حفاظت خاص از پایگاه داده \lr{(Sui Generis)} \par
	\hspace{0.8cm} ۱۸.۳.۳ حق اسرار تجاری \par
	\hspace{0.8cm} ۱۸.۳.۴ خلاصه \par \vspace{0.1cm}
	
	\noindent \textbf{۱۸.۴} \hspace{0.2cm} \textbf{استثنائات و محدودیت‌ها} \par
	\hspace{0.8cm} ۱۸.۴.۱ محدودیت‌های حقوق \par
	\hspace{0.8cm} ۱۸.۴.۲ استثنائات: وجوه مشترک \par
	\hspace{0.8cm} ۱۸.۴.۳ استثنائات خاصِ حق \par \vspace{0.1cm}
	
	\noindent \textbf{۱۸.۵} \hspace{0.2cm} \textbf{منابع جایگزین} \par \vspace{0.1cm}
	
	\noindent \textbf{مطالعه بیشتر}
}

\vspace{0.8cm}

% -----------------------------------------------------------------
% اهداف یادگیری
% -----------------------------------------------------------------
\noindent
\fcolorbox{black}{gray!15}{%
	\begin{minipage}{\dimexpr\linewidth-2\fboxsep-2\fboxrule}
		\vspace{0.3cm}
		\textbf{\large اهداف یادگیری}
		\vspace{0.2cm}
		\begin{itemize}
			\setlength\itemsep{0.5em}
			\item[\textbf{--}] اینکه حق تکثیر (کپی‌رایت)، حقوق خاص پایگاه داده \lr{(sui generis)} و اسرار تجاری شامل چه مواردی هستند و چگونه ذینفعان آن‌ها تعیین می‌شوند.
			\item[\textbf{--}] چه زمانی و چگونه استفاده از مجموعه داده‌های شخص ثالث توسط این حقوق محدود می‌شود و چه زمانی محدود نمی‌شود.
			\item[\textbf{--}] پتانسیل و محدودیت‌های منابع جایگزین برای تکمیل یا جایگزینی مجموعه داده‌های شخص ثالث، مانند حقوق قابلیت انتقال داده‌ها و اطلاعات بخش عمومی.
		\end{itemize}
		\vspace{0.2cm}
	\end{minipage}
}

\newpage

% =================================================================
% ۱۸.۱ مقدمه
% =================================================================
\subsection{مقدمه}
\label{sec:18-1-intro}

اطلاعات بیشتر و بیشتری از طریق استفاده از دستگاه‌های هوشمند (مانند ترموستات هوشمند، تلفن هوشمند)، خدمات اینترنتی (مانند گوگل و فیس‌بوک)، حسگرها (مثلاً در خودروها، خانه‌های هوشمند و شهرها) و دوربین‌ها جمع‌آوری می‌شود. مجموعه داده‌های حاصل حاوی اطلاعات زیادی در مورد افراد، و همچنین در مورد جامعه به طور کلی هستند.

این مجموعه داده‌ها به ناظران خود اجازه می‌دهند تا مشکلات را شناسایی کرده و راه‌هایی برای حل آن‌ها جستجو کنند، اما همچنین فرصت‌ها را شناسایی کرده و نحوه بهره‌برداری از آن‌ها را بررسی نمایند. برای مثال، با مطالعه اطلاعات حاصل از حسگرهای خودروها، حسگرها و دوربین‌های متمرکز بر جاده‌ها، و سیستم‌های چراغ راهنمایی، می‌توان علل تصادفات رانندگی را شناسایی کرد و راه‌حل‌هایی برای کاهش تعداد تصادفات در یک بلوک خاص پیشنهاد داد.

با این حال، دسترسی به چنین مجموعه داده‌هایی که توسط دیگران تولید شده‌اند، اغلب محدود است. گروه بزرگی از بازیگران وجود دارند که نمی‌خواهند دیگران از داده‌های «آن‌ها» استفاده کنند. عامل بسیار مهمی که به چنین بازیگرانی کمک می‌کند تا دسترسی به مجموعه داده‌های خود را محدود کنند، قانون مالکیت فکری است. دارنده حقوق مالکیت فکری بر روی یک مجموعه داده، توانایی محدود کردن دسترسی دیگران به (بخش‌هایی از) مجموعه داده خود و همچنین اعمال محدودیت‌هایی بر استفاده از آن را دارد. برای درک چگونگی پیمایش در این حوزه حقوقی، مهم است که ابتدا درک کنیم حقوق مالکیت فکری چه هدفی را دنبال می‌کنند.

همان‌طور که در «دستورالعمل اجرایی» \lr{(Enforcement Directive)} بیان شده است، دلیل اصلی زیربنایی در سیستم‌های فعلی حقوق مالکیت فکری، تشویق (سرمایه‌گذاری در) نوآوری است. حقوق مالکیت فکری به عنوان حقوق مالکیت مصنوعی برای اصلاح برخی شکست‌های بازار ایجاد شده‌اند. بازار را مانند مزرعه‌ای پر از میوه تصور کنید. اگر همه آزاد باشند بدون هیچ محدودیتی از مزرعه و میوه‌های آن استفاده کنند، احتمالاً بسیاری این کار را خواهند کرد.

با این حال، آنچه بعید است این است که همه کسانی که از مزرعه استفاده می‌کنند، به صورت جداگانه در آن سرمایه‌گذاری کنند. این به دلیل عدم اطمینان از این موضوع است که آیا این کار نتیجه‌ای برای آن‌ها خواهد داشت یا حتی به آن‌ها اجازه می‌دهد سرمایه خود را بازگردانند؛ زیرا در نهایت، همه آزادند بدون محدودیت از مزرعه استفاده کنند. وقتی عواید پیش‌بینی‌شده کمتر باشد، افراد کمتری مایل به سرمایه‌گذاری خواهند بود. علاوه بر این، هر سرمایه‌گذاری انجام شده احتمالاً کوچکتر خواهد بود.

اینجاست که حقوق مالکیت فکری وارد می‌شوند. آن‌ها ابزارهایی برای اصلاح این «شکست بازار» از طریق پاداش دادن به کسانی هستند که در نوآوری سرمایه‌گذاری می‌کنند، با مجموعه‌ای از حقوق انحصاری برای مدت محدود. این حقوق ابزارهایی برای دارنده حق هستند تا به‌طور قانونی دسترسی و استفاده از مالکیت فکری خود را محدود کند. این امر به دارنده حق اجازه می‌دهد تا قیمت‌های بالاتری را برای بازگشت سرمایه و کسب سود تعیین کند.

این فصل با هدف ارائه مقدمه‌ای بر مبانی حقوق مالکیت فکری در اتحادیه اروپا تدوین شده است. این فصل از ساده‌سازی‌هایی استفاده می‌کند و همیشه تصویر کامل را ارائه نمی‌دهد تا درک مطالب را به حداکثر برساند. چنین ساده‌سازی‌هایی عموماً گوشزد می‌شوند و منابعی در مورد موضوع برای کسانی که مایل به درک عمیق‌تر از چنین مفهوم توسعه‌نیافته‌ای هستند، در مراجع گنجانده شده است. بنابراین، نباید به عنوان جایگزینی برای مشاوره حقوقی یا مبنایی برای بحث‌های آکادمیک استفاده شود.

علاوه بر این، در حالی که انواع مختلفی از حقوق مالکیت فکری وجود دارد، تنها «حق تکثیر» (کپی‌رایت)، «حقوق خاص پایگاه داده» \lr{(sui generis)} و «اسرار تجاری» در اینجا بحث خواهد شد. تحت چارچوب حقوقی اتحادیه اروپا، موضوع و شرایط این حقوق مالکیت فکری ارتباط نزدیکی با داده‌ها و نرم‌افزار دارند، همان‌طور که در ادامه فصل مشخص خواهد شد.

سایر حقوق مانند «پتنت‌ها» (ثبت اختراع) در حال حاضر نقش پیچیده‌تری در اتحادیه اروپا در زمینه داده‌ها و نرم‌افزار ایفا می‌کنند، از جمله به دلیل محدودیت‌ها در قابلیت ثبت اختراعِ موضوعاتی مانند روش‌های ریاضی و برنامه‌های کامپیوتری به خودی خود. چنین محدودیت‌هایی همچنین شروع به ایفای نقش بیشتری در مثلاً ایالات متحده کرده‌اند، همان‌طور که می‌توان از رویه قضایی دیوان عالی آن‌ها (به‌ویژه در مورد مفهوم «ایده انتزاعی») بین سال‌های ۲۰۱۰ تا ۲۰۱۴ استنباط کرد. بنابراین این موضوع نیاز به توجه بیشتری دارد که نمی‌تواند در این نوشتار محدود بگنجد.

بنابراین این فصل تنها بر این حقوق مالکیت فکری خاص از دیدگاه اتحادیه اروپا تمرکز دارد. سوالاتی که در بخش‌های بعدی بررسی می‌شوند، برای هر یک از این حقوق مالکیت فکری بر این تمرکز خواهند داشت که چه زمانی قابل اعمال هستند (بخش ۱۸.۲)، و به دنبال آن، این موضوع چه معنایی برای استفاده شخص ثالث از داده‌ها دارد (بخش ۱۸.۳)، و همچنین محدودیت‌ها و استثنائات (بخش ۱۸.۴). این فصل با بحث در مورد راه‌های به دست آوردن دسترسی قانونی به مجموعه داده‌های تحت پوشش یک یا چند مورد از این حقوق مالکیت فکری و منابع جایگزین به پایان می‌رسد.

% =================================================================
% ۱۸.۲ احراز معیارها
% =================================================================
\subsection{احراز معیارها}
\label{sec:18-2-meeting-criteria}

\subsubsection{الزامات شکلی حق تکثیر (کپی‌رایت)}
\label{sec:18-2-1-copyright}

چیزی ممکن است توسط حق تکثیر محافظت شود اگر سه معیار تجمعی برای حفاظت از حق تکثیر را برآورده کند. طبق کنوانسیون برن، این معیارها ایجاب می‌کنند که آن چیز (۱) یک «بیان» \lr{(expression)} باشد که (۲) «اصیل» \lr{(original)} است و (۳) در حوزه «ادبیات و هنر» قرار دارد. در رژیم حق تکثیر اتحادیه اروپا، عواملی مانند کار یا سرمایه‌گذاری مرتبط نیستند.

سه عنصر از یک مجموعه داده وجود دارد که قادر به برآورده کردن این الزامات هستند:
\begin{itemize}
	\item محتویات مجموعه داده
	\item انتخاب داده‌ها
	\item آرایش داده‌ها
\end{itemize}

اگر یک یا چند مورد از این عناصر الزامات شکلی را برآورده کنند، ممکن است حفاظت حق تکثیر بر روی آن عناصرِ مجموعه داده وجود داشته باشد. در آن صورت، محدودیت‌های قانونی بر استفاده از آن وجود خواهد داشت. بنابراین مهم است که این معیارها را درک کنیم تا بتوانیم احتمال حفاظت حق تکثیر بر روی یک مجموعه داده را تعیین کنیم تا از استفاده قانونی اطمینان حاصل شود.

فرض زیربنایی معیار «بیان» این است که حقایق و ایده‌ها خلق نمی‌شوند بلکه کشف می‌شوند. این موضوع همچنین در پرونده \lr{Feist Publications, Inc., v. Rural Telephone Service Co.} تأیید شده است، که نشان می‌دهد ایالات متحده و اتحادیه اروپا با این معیار به روش مشابهی برخورد می‌کنند.

این موضوع برای حق تکثیر به این معنی است که از «آنچه» گفته می‌شود محافظت نمی‌کند، بلکه از «چگونگی» گفته شدن آن محافظت می‌کند. یک قانون سرانگشتی خوب این است که به آن به عنوان طیفی نگاه کنیم که در آن حقایق و ایده‌ها در یک طرف و بیان‌ها در طرف دیگر بر اساس خاص بودن \lr{(specificity)} قرار دارند.

در اصل، هر چه یک حقیقت یا ایده خاص‌تر شود، سوزن عموماً به سمت بیان حرکت می‌کند. استدلال پشت این موضوع این است که یک نویسنده می‌تواند یک حقیقت یا ایده را با انتخاب کلمات خود منتقل کند، و بدین ترتیب چیزی فراتر و جدا از حقیقت یا ایده خلق کند. برای مثال، به تفاوت در جزئیات جملات زیر در جدول ۱۸.۱ نگاه کنید.

این الزام مانعی احتمالی برای حفاظت حق تکثیر بر روی یک مجموعه داده است. برای مثال، داده‌ها در چنین مجموعه داده‌هایی ممکن است با هم تصویر بسیار خاصی ایجاد کنند، اما اگر داده‌ها صرفاً به عنوان متغیرهایی در یک جدول نمایش داده شوند، داده‌ها فاقد بیان هستند.

در پرونده \lr{Football DataCo Ltd.}، عنصر دوم — اصالت — به عنوان حاشیه صلاحدید برای انجام انتخاب‌های آزاد و خلاقانه درک شد که مورد استفاده قرار می‌گیرد. به زبان ساده‌تر، ایجاب می‌کند که خالق «مهر شخصی» خود را بر روی آن زده باشد. با این حال، این البته نباید به معنای تحت‌اللفظی گرفته شود. برای مثال، قرار دادن لوگوی خود بر روی چیزی آن را اصیل نمی‌کند.

استاندارد برای برآورده کردن این معیار در عمل خیلی بالا نیست. چنین انتخاب‌های خلاقانه‌ای می‌توانند به سادگیِ انتخاب نورپردازی، پس‌زمینه و زاویه برای گرفتن عکس، یا انتخاب کلمات در یک متن یا کد باشند. با این حال، مهم است تأکید شود که باید فضایی برای انجام چنین انتخاب‌هایی توسط خالق وجود داشته باشد.

برای مثال، یک عکس پاسپورت باید تعدادی از الزامات سخت‌گیرانه را برآورده کند. چنین تنظیمات از پیش تعیین شده‌ای بر فضایی که عکاس برای گرفتن تصمیمات خلاقانه خود دارد تأثیر می‌گذارد. بنابراین برای یک عکس پاسپورت، بسیار بعید است که عکاس بتواند الزام اصالت را برآورده کند.

مثال دیگر الزامات عملکردی است. کد نرم‌افزار از زمانی که دستورالعمل نرم‌افزار \lr{(Software Directive)} به وجود آمد، قادر به جذب حفاظت حق تکثیر است، اما همان‌طور که در پرونده \lr{Bezpečnostní softwarová asociace} تأیید شد، بیان در کد اگر «توسط عملکرد فنی آن‌ها دیکته شده باشد» نمی‌تواند به میزان اصالت برسد.

شرایطی که نویسنده فضایی برای انجام انتخاب‌های خلاقانه دارد بنابراین برای برآورده کردن الزام اصالت حیاتی است. علاوه بر این، در غیاب چنین الزاماتی، هنوز این مسئله وجود دارد که آیا انتخاب‌های خلاقانه واقعاً انجام شده‌اند یا خیر. انتخاب و/یا آرایش داده‌ها در یک مجموعه داده می‌تواند، برای مثال، حداقل آستانه خلاقیت را برآورده کند، اما این انتخاب‌ها عموماً بر اساس سودمندی در عمل انجام می‌شوند؛ انتخاب‌های انجام شده در انتخاب داده‌ها اغلب توسط کسب‌وکار اصلی یک شرکت تعیین می‌شوند، و داده‌ها به دلایل عملی مانند ترتیب الفبایی یا تاریخ مرتب می‌شوند.

آخرین معیار ایجاب می‌کند که آن یک اثر در حوزه ادبیات و هنر باشد. آنچه هنر یا ادبیات را تشکیل می‌دهد در رژیم حق تکثیر بسیار گسترده درک می‌شود. برای مثال، ادبیات به منظور حفاظت حق تکثیر می‌تواند اساساً هر چیزی را که شامل کلمه مکتوب باشد شامل شود. همان‌طور که در بالا ذکر شد، حتی می‌تواند کد در نرم‌افزار را پوشش دهد.

این بدان معناست که داده‌ها — چه عددی و چه متنی — نیز در این دسته گسترده قرار می‌گیرند. برخی مثال‌های دیگر از آثاری که ممکن است محافظت شوند کتاب‌ها، نقاشی‌ها، طرح‌ها، نقشه‌ها، معماری، مواد طراحی مقدماتی برای کد نرم‌افزار، فیلم‌ها، ترکیبات موسیقی، اشعار، توپوگرافی، آثار رقص‌پردازی و غیره هستند: ماده ۲(۱) کنوانسیون برن شامل بیش از ۲۰ مثال از انواع آثاری است که در حیطه ادبیات و هنر قرار می‌گیرند.

% جدول ۱۸.۱
\begin{table}[h!]
	\centering
	\caption{بیان \lr{(Expression)}}
	\vspace{0.2cm}
	\begin{tabular}{|p{0.5\linewidth}|p{0.2\linewidth}|p{0.2\linewidth}|}
		\hline
		\textbf{جمله مثال} & \textbf{سطح جزئیات} & \textbf{حقیقت/ایده یا بیان} \\
		\hline
		این خانه سبز است. & جزئیات بسیار کم و بسیار کلی & حقیقت/ایده \\
		\hline
		این خانه سه طبقه دارای سه سایه مختلف سبز است. & جزئیات بیشتر، اما هنوز کاملاً کلی & حقیقت/ایده، اما در حال حرکت به سمت بیان \\
		\hline
		این محل اقامت سه طبقه ترکیبی از سایه‌های سبز است، که در میان آن‌ها زیتونی، خزه‌ای و حتی نشانه‌هایی از سبز متالیک در اطراف گوشه‌های پنجره‌ها و درب‌هایش وجود دارد. & جزئیات زیاد و بسیار خاص & بیان \\
		\hline
	\end{tabular}
	\label{tab:18-1-expression}
	\vspace{0.1cm}
	\footnotesize{\textit{یادداشت جدول: جدول متعلق به نویسنده است.}}
\end{table}

\subsubsection{حق پایگاه داده خاص \lr{(Sui Generis Database Right)}}
\label{sec:18-2-2-sui-generis}

اگر موادی مانند مجموعه داده‌ها و مواد طراحی مقدماتی برای کد نرم‌افزار بخشی از یک پایگاه داده باشند، استفاده از آن‌ها ممکن است توسط حفاظت پایگاه داده خاص \lr{(sui generis)} محدود شود. به دلیل حفاظت محدود ارائه شده توسط رژیم حق تکثیر در پایگاه‌های داده، «دستورالعمل پایگاه داده» \lr{(Database Directive)} در سال ۱۹۹۶ برای تقویت بیشتر اقتصاد اطلاعات در اتحادیه اروپا تصویب شد. تا به امروز، این رژیم هنوز هم بسیار یک مخلوق اروپایی است (برای مثال، هیچ معادلی در ایالات متحده وجود ندارد).

حق پایگاه داده خاص بنابراین از پایگاه‌های داده بدون اصالت محافظت می‌کند. با این حال، این بدان معنا نیست که اگر حفاظت حق تکثیر بر روی محتویات، انتخاب و/یا آرایش پایگاه داده وجود داشته باشد، نمی‌تواند حفاظت پایگاه داده خاص نیز وجود داشته باشد. این دو حق می‌توانند بر روی یک پایگاه داده واحد همزیستی داشته باشند.

یک مجموعه داده احتمالاً تحت پوشش این حق قرار می‌گیرد اگر (۱) یک پایگاه داده باشد که برای آن (۲) سرمایه‌گذاری مرتبط انجام شده است (۳) که قابل توجه \lr{(substantial)} است.

برای اینکه یک مجموعه داده شرط اول را برآورده کند — که یک پایگاه داده است — ابتدا لازم است که مجموعه داده یک مجموعه یا گردآوری از مواد باشد. چنین موادی شامل آثار دارای حق تکثیر، اعداد، حقایق و داده‌ها هستند، اما محدود به آن دسته‌ها نیستند. سپس، چنین موادی باید سازماندهی، ذخیره و از طریق ابزارهای الکترونیکی یا غیرالکترونیکی قابل دسترسی باشند. این بدان معناست که یک سند مکتوب که تمام الزامات دیگر را برآورده می‌کند نیز می‌تواند به عنوان یک پایگاه داده واجد شرایط باشد. با این حال، برای یک پایگاه داده فیزیکی، لازم نیست که مواد به صورت فیزیکی به روشی سازمان‌یافته ذخیره شوند.

معیار دوم ایجاب می‌کند که یک سرمایه‌گذاری مرتبط انجام شده باشد. این بدان معناست که سرمایه‌گذاری باید در جمع‌آوری، تأیید و/یا ارائه داده‌ها برای پایگاه داده انجام شده باشد. همان‌طور که در پرونده \lr{BHB v William Hill} روشن شد، سرمایه‌گذاری در سایر دسته‌ها مانند در خلق داده‌ها برای برآورده کردن این معیار مرتبط نیست.

چنین سرمایه‌گذاری می‌تواند از طریق منابع مالی، منابع انسانی و منابع مادی انجام شود. سرمایه‌گذاری از طریق منابع انسانی می‌تواند، برای مثال، در تلاش یا زمان انجام شود. برای منابع مادی، سرمایه‌گذاری در تجهیزات برای ساخت پایگاه داده مانند سخت‌افزار و نرم‌افزار انجام می‌شود. البته، چنین نوع سرمایه‌گذاری‌هایی نیز هزینه مالی دارند. علاوه بر این، ورودی انسانی عموماً در کار با تجهیزات برای ساخت یک پایگاه داده مورد نیاز است. در واقعیت، ارتباط بین این سه نوع سرمایه‌گذاری اغلب ترکیبی از سه مورد با تأکید بر منابع مالی ایجاد می‌کند.

علاوه بر این، چنین سرمایه‌گذاری‌هایی نباید برای اهداف دیگر انجام شده باشند. برای مثال، رایانه‌هایی که برای ایجاد پایگاه داده استفاده می‌شوند اغلب صرفاً برای آن هدف خریداری نمی‌شوند. در آن صورت، سرمایه‌گذاری عموماً به حسابِ به وجود آمدن حفاظت پایگاه داده خاص گذاشته نمی‌شود.

آخرین معیار — اینکه سرمایه‌گذاری باید قابل توجه باشد — کمی مبهم‌تر است. دستورالعمل پایگاه داده راهنمایی قطعی در مورد اینکه این معیار به چه معناست یا چگونه باید اعمال شود ارائه نمی‌دهد. رویه قضایی تاکنون بیشتر با مبالغ بالای سرمایه‌گذاری مالی سروکار داشته است، بنابراین این پرونده‌ها نیز راهنمایی زیادی در مورد آستانه قابل توجه بودن ارائه نمی‌دهند. متأسفانه، سقف و کف دقیق این معیار نیز هنوز موضوع بحث‌های آکادمیک سنگین است، اما گنجاندن آن‌ها فراتر از اهداف این فصل خواهد بود.

این آستانه، برخلاف آنچه کلمه قابل توجه \lr{(substantial)} ممکن است القا کند، نباید به عنوان «زیاد» تفسیر شود. در عوض، این معیار بهتر است به عنوان الزامِ یک سرمایه‌گذاری درک شود که «خیلی ناچیز نیست». این حدود در متن اصلی — نه زیاد، فقط نه خیلی ناچیز — عموماً در کشورهای عضو اتحادیه اروپا مانند آلمان پذیرفته شده‌اند. یک مثال روشن از چنین سرمایه‌گذاری ناچیزی، یک کارمندِ یک شرکت بزرگ است که تنها چند ساعت را به ساخت پایگاه داده اختصاص می‌دهد. مثالی از چیزی که واجد شرایط خواهد بود، سرمایه‌گذاری در تأیید مقدار زیادی داده با یک مجموعه داده دیگر است.

\subsubsection{حق اسرار تجاری \lr{(Trade Secret Right)}}
\label{sec:18-2-3-trade-secret}

طبق دستورالعمل اسرار تجاری \lr{(Trade Secret Directive)}، اگر مجموعه داده شامل (۱) اطلاعاتی باشد که در محافل مربوطه شناخته شده نیست، (۲) دارای ارزش تجاری باشد، و (۳) توسط شرکت مورد نظر مخفی نگه داشته شود، مجموعه داده ممکن است به عنوان یک راز تجاری محافظت شود.

معیار اول ایجاب می‌کند که اطلاعات مورد نظر به راحتی قابل دسترسی یا در محافل مربوطه شناخته شده نباشد. محافل مربوطه به افرادی اشاره دارد که عموماً با این نوع اطلاعات سروکار دارند، که بدان معناست که محفل مربوطه ممکن است بسته به نوع اطلاعات متفاوت باشد اگر موضوع محافظت‌شده شامل انواع مختلفی از اطلاعات باشد. بنابراین، نمی‌تواند اطلاعات بی‌اهمیت یا نوعی که از طریق تجربه شغلی عادی به دست می‌آید را پوشش دهد. اطلاعاتی که می‌تواند توسط حق اسرار تجاری پوشش داده شود حداقل شامل دانش فنی \lr{(know-how)}، اطلاعات تجاری یا اطلاعات فناوری است، اما ممکن است در قانون داخلی گسترده‌تر تعریف شود.

دوم، اطلاعات باید دارای ارزش تجاری باشد. مهم نیست که این کار را بالفعل یا بالقوه انجام می‌دهد. آنچه مهم است این است که منافع دارنده حق راز تجاری — خواه ماهیت علمی، فنی، تجاری یا مالی داشته باشد — در صورت افشای راز تجاری آسیب ببیند. بنابراین باید ارزش تجاری داشته باشد زیرا مخفی است. اگر در صورت سوءاستفاده ارزش آن تحت تأثیر منفی قرار نگیرد، رضایت از معیار دوم قابل تردید است.

در نهایت، دارنده حق راز تجاری باید تلاش‌های معقولی در مخفی نگه داشتن اطلاعات انجام دهد. البته، این موضوع تابع شرایط پرونده است. در برخی موارد، ممکن است مخفی نگه داشتن اطلاعات دشوارتر باشد یا شرایط ممکن است اقدامات متفاوتی را نسبت به سایر موارد ایجاب کند. این واقعیت که افراد زیادی می‌دانند لزوماً به این معنی نیست که شرکت در تلاش خود برای برآورده کردن این معیار شکست خورده است. برای مثال، بسیاری از کارمندان ممکن است نیاز به دانش (بخش‌هایی از) راز تجاری داشته باشند تا بتوانند یک محصول بسازند. تا زمانی که آن‌ها تحت تعهدات قراردادی به رازداری هستند، مهم نیست چند نفر می‌دانند. همین امر برای توزیع‌کنندگانی صادق است که اطلاعات خاصی را تحت یک توافق‌نامه عدم افشا \lr{(NDA)} دریافت کرده‌اند تا بتوانند کار خود را انجام دهند.

\subsubsection{خلاصه}
\label{sec:18-2-4-summary}

الزامات شکلی هر یک از حقوق مالکیت فکری را می‌توان به سه جزء اساسی تقسیم کرد. اگر در کنار یکدیگر در یک جدول قرار گیرند، تصویر زیر را ایجاد می‌کنند (جدول ۱۸.۲).

% جدول ۱۸.۲
\begin{table}[h!]
	\centering
	\caption{الزامات شکلی \lr{(Formal requirements)}}
	\vspace{0.2cm}
	\small
	\begin{tabular}{|p{0.3\linewidth}|p{0.3\linewidth}|p{0.3\linewidth}|}
		\hline
		\textbf{حق تکثیر (کپی‌رایت)} & \textbf{حق پایگاه داده خاص} & \textbf{حق اسرار تجاری} \\
		\hline
		۱. بیان \lr{(Expression)} & ۱. پایگاه داده & ۱. به راحتی قابل دسترسی یا در محافل مربوطه شناخته شده نیست \\
		۲. اصالت \lr{(Originality)} & ۲. سرمایه‌گذاری مرتبط & ۲. ارزش تجاری \\
		۳. ادبیات و هنر & ۳. قابل توجه \lr{(Substantial)} & ۳. مخفی نگه داشته شده \\
		\hline
	\end{tabular}
	\label{tab:18-2-formal-requirements}
	\vspace{0.1cm}
	\footnotesize{\textit{یادداشت جدول: جدول متعلق به نویسنده است.}}
\end{table}



% =================================================================
% ۱۸.۳ دامنه حفاظت
% =================================================================
\subsection{دامنه حفاظت}
\label{sec:18-3-scope-of-protection}

\subsubsection{حق تکثیر: موضوعات محافظت‌شده}
\label{sec:18-3-1-copyright-scope}

اگر یک مجموعه داده از طریق یک یا چند مورد از این مسیرها محافظت شود، هنوز محدودیتی در مورد اینکه این حقوق دقیقاً از چه چیزی و در برابر چه چیزی محافظت می‌کنند وجود دارد. زمانی که یک مجموعه داده یا کد نرم‌افزار الزامات حفاظت حق تکثیر را برآورده می‌کند، این حفاظت تنها محدود به «بیان اصیل» است. این بدان معناست که حفاظت هرگز نمی‌تواند، از جمله موارد دیگر، به محتوای واقعی یا ایده‌ها گسترش یابد.

علاوه بر این، اگر تنها انتخاب و/یا آرایش یک مجموعه داده توسط حق تکثیر محافظت شود و نه خود داده‌ها، بیان تنها در انتخاب و/یا آرایش وجود دارد. برای کد نرم‌افزار، این بدان معناست که حق تکثیر تنها می‌تواند بر کدی استوار باشد که توسط عملکردهای فنی دیکته نشده است. بنابراین، یک شخص ثالث قادر خواهد بود از محتویات مجموعه داده یا چنین بخش‌های محافظت‌نشده‌ای از کد نرم‌افزار استفاده کند.

علاوه بر این، حق تکثیر تنها از بیان اصیل در برابر انواع خاصی از استفاده توسط دیگران محافظت می‌کند. به عبارت دیگر، دارنده حق تکثیر حقوق خاصی برای مستثنی کردن دارد. برخلاف آنچه اصطلاح «حق تکثیر» (کپی‌رایت) القا می‌کند، این حق یک حق واحد نیست بلکه مجموعه‌ای از حقوق است.

این مجموعه حقوق شامل حقوق بهره‌برداری، که در غیر این صورت به عنوان حقوق اقتصادی شناخته می‌شوند، است. چندین حق اقتصادی در «دستورالعمل جامعه اطلاعاتی» \lr{(InfoSoc Directive)} گنجانده شده است، اما تنها حق تکثیر و حق در دسترس عموم قرار دادن برای استفاده از داده‌ها و نرم‌افزار اهمیت ویژه‌ای دارند.

حق تکثیر مستلزم آن است که، در اصل، تنها دارنده حق تکثیر حق تهیه کپی از اثر خود را دارد. علاوه بر این، مهم است توجه شود که یک تکثیر لازم نیست دقیق باشد. گرفتن عکس از یک نقاشی نیز تکثیر اثر محسوب می‌شود. ابزار مورد استفاده برای تهیه کپی برای این حق اهمیتی ندارد.

علاوه بر این، لازم نیست کپی از کل اثر باشد. آنچه مهم است این است که مقدار کافی کپی شود تا کار فکری و خلاقانه هنرمند را نمایش دهد. نمونه‌ای به کوچکی ۱۱ کلمه از مقالات روزنامه در پرونده \lr{Infopaq v Danske Dagblades Forening} قادر به انجام این کار تشخیص داده شده است. در نتیجه، قابل استدلال است که بخش کوچکی از مجموعه داده یا کد نیز می‌تواند انتخاب‌های خلاقانه نویسنده را منتقل کند. اگر چنین باشد، در غیاب یک استثنای قابل اجرا، حتی استفاده از چنین گزیده‌های کوچکی نیاز به مجوز دارد.

دوم، حق در دسترس عموم قرار دادن وجود دارد. برای مثال، قرار دادن محتوای محافظت‌شده در یک وب‌سایت یا استفاده از ابرپیوندها \lr{(hyperlinks)} به محتوای محافظت‌شده را در نظر بگیرید. توجه داشته باشید که این کمی ساده‌سازی شده است. آنچه باید به عنوان در دسترس عموم قرار دادن درک شود و چه کسی باید به عنوان انجام‌دهنده این عمل درک شود، به دلیل تحولات قانونی و قضایی اخیر در سطح اتحادیه اروپا هنوز در حال تکامل است.

در اکثر موارد، نوعی تکثیر برای اینکه بتوان آن را در دسترس عموم قرار داد ضروری است. استثنائات قابل توجه در اینجا استفاده از ابرپیوندها یا نمایش نسخه اصلی (مثلاً یک نقاشی در موزه) هستند.

آنچه این موضوع برای اشخاص ثالث معنا دارد این است که آن‌ها نمی‌توانند بدون مجوز به طور قانونی در این استفاده‌ها از بیان اصیل شرکت کنند. دارنده حق تکثیر می‌تواند، برای مثال، به دیگران اجازه دهد تا اثر او را از طریق یک مجوز تکثیر کنند. از آنجا که ۱۱ کلمه می‌تواند انتخاب‌های خلاقانه نویسنده را منتقل کند، الزام دریافت مجوز به سرعت وارد عمل می‌شود.

در اصل، چنین مجوزی تنها می‌تواند از دارنده حق تکثیر دریافت شود. دارنده حق عموماً یک شخص حقیقی است — نویسنده یا خالق. زمانی که اثری به سفارش خلق شده باشد، تخصیص حق تکثیر بستگی به این دارد که چه کسی انتخاب‌های خلاقانه را انجام داده است. در برخی موارد، انتخاب‌های خلاقانه ممکن است توسط چندین بازیگر انجام شده باشد، که عموماً منجر به حقوق مشترک بر اثر می‌شود.

با این حال، این موضوع در مورد خلق تحت استخدام متفاوت است. برای مثال، حقوق بهره‌برداری بر روی اثری که توسط کارمند در جریان استخدامش و بنا به دستورات کارفرما خلق شده است، نزد کارفرما قرار دارد.

علاوه بر این، در مورد نرم‌افزار، «دفتر انتشارات» \lr{(Publications Office)} در خلاصه خود از دستورالعمل اجرایی روشن کرد که کشورهای عضو اتحادیه اروپا ممکن است مقرر کنند که اشخاص حقوقی یا نهادها نیز می‌توانند دارنده حق باشند. در برخی حوزه‌های قضایی، همه حقوق ممکن است همیشه از نویسنده به دیگری قابل انتقال نباشد.

\subsubsection{حفاظت خاص از پایگاه داده \lr{(Sui Generis Database Protection)}}
\label{sec:18-3-2-sui-generis-scope}

حق پایگاه داده خاص با در نظر گرفتن سرمایه‌گذار ایجاد شده است، بنابراین صرفاً سازنده واقعی بودن برای دارنده حق بودن کافی نیست. طبق دستورالعمل پایگاه داده، دارنده حق کسی است که ابتکار عمل و ریسک سرمایه‌گذاری را بر عهده می‌گیرد. پیمانکاران فرعی و کار در ازای استخدام \lr{(work for hire)} صراحتاً از این تعریف مستثنی شده‌اند.

اگر یک پایگاه داده توسط یک کارمند ساخته شود، تخصیص حقوق بستگی به معیارهای قانون ملی دارد. اگر افراد یا نهادهای متعددی در یک پایگاه داده مشارکت داشته باشند، ممکن است حقوق مشترک وجود داشته باشد. برخلاف حق تکثیر، حق پایگاه داده خاص کاملاً قابل انتقال است.

مانند حق تکثیر، حق پایگاه داده خاص یک حق واحد نیست. زمانی که یک پایگاه داده تحت پوشش حفاظت پایگاه داده خاص قرار می‌گیرد، دارنده حق دارای حقوق انحصاری برای (۱) استخراج \lr{(extraction)} و (۲) استفاده مجدد \lr{(reutilization)} است. این حقوق باید به صورت زیر درک شوند.

\textbf{استخراج} به انتقال پایگاه داده یا بخش قابل توجهی از آن اشاره دارد. این انتقال ممکن است دائمی یا موقت باشد. علاوه بر این، ابزاری که از طریق آن منتقل می‌شود مهم نیست. همچنین بی‌اهمیت است که پایگاه داده به کجا منتقل می‌شود (نوع رسانه). آنچه مهم است این است که پایگاه داده یا بخش قابل توجهی از آن منتقل شود.

این بدان معناست که هر شخصی غیر از دارنده حق در اصل نیاز به مجوز از دارنده حق دارد تا این عمل را به طور قانونی انجام دهد. با این حال، مجوز برای استخراج سیستماتیک بخش‌های ناچیز نیز مورد نیاز است. این در تعریف استخراج گنجانده شده است تا با «دوشیدن» \lr{(milking)} مقابله شود. این فرآیندِ انتقال مکرر بخش‌های کوچک یک پایگاه داده است تا زمانی که کل پایگاه داده یا بخش قابل توجهی از آن منتقل شود.

نوع دیگر استفاده — \textbf{استفاده مجدد} — به در دسترس عموم قرار دادن پایگاه داده یا بخش قابل توجهی از آن اشاره دارد. این شامل توزیع یا اجاره کپی‌ها، انتقال آنلاین پایگاه داده و سایر انواع انتقال‌ها می‌شود. هر روشی که در آن پایگاه داده عمومی شود، تحت این تعریف قرار می‌گیرد.

در اصل، این حق بنابراین به دارنده حق، حق انحصاری انجام استفاده مجدد اتفاقی از (بخش قابل توجهی از) پایگاه داده را می‌دهد. با این حال، درست مانند حق استخراج، حق استفاده مجدد نیز در برابر استفاده مجدد سیستماتیک از بخش‌های ناچیز محافظت می‌کند. باز هم، اگر این تعریف محدود به بخش‌های قابل توجه یا کل پایگاه داده می‌شد، این امر به اشخاص ثالث فرصت می‌داد تا همچنان (بخش قابل توجهی از) پایگاه داده را، فقط بخش کوچکتری در هر زمان، منتقل کنند.

در نهایت، یک مورد آخر وجود دارد که در آن استفاده مجدد وجود دارد. این شامل استفاده از یک «موتور فرا-جستجو» \lr{(meta search engine)} با عملکردهای خاص است.

یک موتور فرا-جستجو، موتور جستجویی است که جستجو در تعدادی از پایگاه‌های داده دیگر را امکان‌پذیر می‌سازد. عموماً عبارت جستجویی را که توسط بازدیدکننده موتور فرا-جستجو وارد می‌شود به سایر موتورهای جستجو منتقل می‌کند. چیزی را از پایگاه‌های داده‌ای که در آن‌ها جستجو می‌کند کپی نمی‌کند، بلکه نتایج جستجو را، از جمله نتایج سایر پایگاه‌های داده، نمایش می‌دهد.

در پرونده \lr{Innoweb BV v Wegener} تعیین شد که چنین موتور فرا-جستجویی احتمالاً (بخش قابل توجهی از) پایگاه داده را مورد استفاده مجدد قرار می‌دهد اگر سه عملکرد زیر وجود داشته باشد. اول، فرم‌های جستجوی ارائه شده به کاربر نهایی توسط موتور فرا-جستجو و پایگاه داده دیگر اساساً یکسان عمل کنند. دوم، پرس‌وجوها برای کاربر نهایی به صورت بلادرنگ به سایر موتورهای جستجو ترجمه شوند. این بدان معناست که تمام اطلاعات پایگاه داده دیگر به صورت بلادرنگ پس از شروع جستجو توسط کاربر نهایی موتور فرا-جستجو، جستجو می‌شود.

سوم و در نهایت، نتایج همه با هم به ترتیبی ارائه شوند که معیارهای مشابه با معیارهای استفاده شده توسط پایگاه داده دیگر را منعکس کند. بدین منظور، از فرمت وب‌سایت خود موتور فرا-جستجو در نمایش نتایج استفاده می‌شود و موارد تکراری را با هم به عنوان یک بلوک نمایش می‌دهد. برای تکرار، اگر یک موتور فرا-جستجو که سایر پایگاه‌های داده را جستجو می‌کند به روش مذکور عمل کند، اپراتور این موتور فرا-جستجو احتمالاً در استفاده مجدد از (بخش‌های قابل توجهی از) پایگاه داده دیگر شرکت دارد. البته، این بدان معنا نیست که اگر یک موتور فرا-جستجو این ویژگی‌ها را نداشته باشد، با این حال نمی‌تواند استفاده مجدد وجود داشته باشد.

برای هر دوی این حقوق، کلمه «قابل توجه» \lr{(substantial)} دوباره نقش ایفا می‌کند. برای اهداف استخراج و استفاده مجدد، اصطلاح «قابل توجه» به حجم داده‌ها از یک پایگاه داده، به طور خاص‌تر، حجم داده‌هایی که نسبت به کل پایگاه داده استخراج یا مورد استفاده مجدد قرار می‌گیرد اشاره دارد \lr{(see BHB v William Hill)}. در اینجا پیوندی بین سرمایه‌گذاری و این دو حق وجود دارد.

راه آسان برای برخورد با این موضوع به صورت کمی است. مثال زیر را در نظر بگیرید. سرمایه‌گذاری قابل توجهی در جمع‌آوری، تأیید و/یا ارائه داده‌ها انجام شده است، اما تفاوت قابل توجهی در سرمایه‌گذاری در سراسر داده‌ها وجود ندارد. یک شخص ثالث اکنون نیمی از داده‌های پایگاه داده را استخراج می‌کند. این بدان معناست که نیمی از سرمایه‌گذاری توسط بخش استخراج شده نمایندگی می‌شود. بخشی که استخراج می‌شود بنابراین احتمالاً قابل توجه است.

با این حال، اینکه آیا استخراج یا استفاده مجدد قابل توجه است، می‌تواند به صورت کیفی نیز آزمایش شود. این کمی مبهم‌تر است. شرایط مثال ما کمی تغییر می‌کند. اکنون، داده‌های خاصی در پایگاه داده وجود دارد که نیاز به سرمایه‌گذاری بسیار بیشتری در جمع‌آوری، تأیید و/یا ارائه آن‌ها نسبت به بقیه داده‌ها داشته است. داده‌های «گران‌تر» تنها بخش کوچکی از کل پایگاه داده هستند. یک شخص ثالث اکنون تنها بخشی از پایگاه داده را که حاوی داده‌های «گران» است مورد استفاده مجدد قرار می‌دهد. حتی اگر داده کمتری باشد، بخش بزرگتری از سرمایه‌گذاری را نمایندگی می‌کند. این بدان معناست که احتمالاً چنین استفاده مجددی توسط شخص ثالث به صورت کیفی قابل توجه خواهد بود. در هر دو مثال، شخص ثالث احتمالاً نمی‌تواند این اعمال را بدون مجوز از دارنده حق یا طبق قانون انجام دهد.

\subsubsection{حق اسرار تجاری}
\label{sec:18-3-3-trade-secret}

دستورالعمل اسرار تجاری تصریح می‌کند که دارنده راز تجاری هر شخص حقیقی یا حقوقی است که به طور قانونی راز تجاری را کنترل می‌کند. مانند حق پایگاه داده خاص، این حق می‌تواند به طور کامل منتقل شود.

حق اسرار تجاری در برابر کسب، استفاده و/یا افشای غیرقانونیِ موضوعِ محافظت‌شده حفاظت می‌کند. این اعمال باید بسیار گسترده تفسیر شوند. هر عملی مغایر با رویه‌های تجاری صادقانه، دسترسی غیرمجاز، و/یا تصاحب هر ماده‌ای که حاوی موضوع محافظت‌شده باشد، تحت کسب غیرقانونی قرار می‌گیرد. همین امر برای موادی که اطلاعات راز تجاری می‌تواند از آن‌ها استخراج شود صادق است. البته، اگر شخصی سپس اقدام به استفاده و/یا افشای راز تجاری کند، این نیز غیرقانونی خواهد بود.

استفاده یا افشای موضوع محافظت‌شده در نقض یک وظیفه قراردادی — از جمله توافق‌نامه محرمانگی — یا هر وظیفه دیگری که محدودیت‌هایی بر آن اعمال تحمیل می‌کند نیز غیرقانونی است. علاوه بر این، استفاده غیرقانونی شامل تولید کالاهای ناقض، یا عرضه یا قرار دادن آن‌ها در بازار است. ذخیره‌سازی، واردات و صادرات کالاهای ناقض به آن منظور نیز در آن تعریف قرار می‌گیرد. یک کالا ناقض است اگر موضوع محافظت‌شده‌ای که به طور غیرقانونی کسب، استفاده یا افشا شده است، به روشی معنادار به (فرآیند تولید یا بازاریابی) یک محصول کمک کند.

حق اسرار تجاری مسلماً شکننده‌ترین حق مالکیت فکری است. زمانی که حق تکثیر یا حق پایگاه داده خاص نقض می‌شود، این حقوق مالکیت فکری به وجود خود ادامه می‌دهند. زمانی که داده‌های تحت پوشش یک حق راز تجاری به گونه‌ای سوءاستفاده شوند که دیگر شرایط مربوط به مخفی بودن آن را برآورده نکنند، حق منقضی می‌شود.

با این حال، مهم است تکرار شود که راز تجاری در برابر اعمال غیرمجاز محافظت می‌کند. مثال زیر را در نظر بگیرید. داده‌های تحت پوشش یک راز تجاری تحت یک توافق‌نامه عدم افشا در قبال پرداخت افشا می‌شوند. اگر وظایف ارائه‌دهنده و دریافت‌کننده — قراردادی و غیره — مانع از این انتقال داده‌ها تحت آن شرایط نشود، احتمالاً قانونی است. چنین افشایی احتمالاً راز تجاری را دست‌نخورده باقی می‌گذارد. معامله تحت یک توافق‌نامه عدم افشا لزوماً منجر به از دست دادن حق راز تجاری نمی‌شود. قراردادهایی مانند قراردادهای کارمندی با بندهای محرمانگی و توافق‌نامه‌های عدم افشا بنابراین ابزارهای حیاتی برای دارنده حق راز تجاری هستند.

\subsubsection{خلاصه}
\label{sec:18-3-4-summary}

اگر یک مجموعه داده واجد شرایط حق تکثیر، حفاظت پایگاه داده خاص و/یا حق راز تجاری باشد، حفاظت همچنان محدود به موضوعات خاصی است. علاوه بر این، تنها در برابر اعمال غیرقانونی خاصی که توسط شخصی غیر از دارنده حق انجام می‌شود محافظت می‌شود (جدول ۱۸.۳ را ببینید). چنین اعمالی بدون مجوز ارائه شده توسط دارنده حق یا قانون (یعنی استثنا) غیرقانونی هستند.

% جدول ۱۸.۳
\begin{table}[h!]
	\centering
	\caption{دامنه حفاظت \lr{(Scope of protection)}}
	\vspace{0.2cm}
	\small
	\begin{tabular}{|p{0.3\linewidth}|p{0.2\linewidth}|p{0.2\linewidth}|p{0.2\linewidth}|}
		\hline
		& \textbf{حق تکثیر} & \textbf{حق پایگاه داده خاص} & \textbf{حق اسرار تجاری} \\
		\hline
		موضوع محافظت‌شده & بیان اصیل & پایگاه داده & راز تجاری \\
		\hline
		محافظت‌شده در برابر اعمال غیرقانونی & 
		\textbullet\ تکثیر \newline \textbullet\ در دسترس عموم قرار دادن & 
		\textbullet\ استخراج \newline \textbullet\ استفاده مجدد & 
		\textbullet\ کسب \newline \textbullet\ استفاده \newline \textbullet\ افشا \\
		\hline
	\end{tabular}
	\label{tab:18-3-scope-protection}
	\vspace{0.1cm}
	\footnotesize{\textit{یادداشت جدول: جدول متعلق به نویسنده است.}}
\end{table}



% =================================================================
% ۱۸.۴ استثنائات و محدودیت‌ها
% =================================================================
\subsection{استثنائات و محدودیت‌ها}
\label{sec:18-4-exceptions-limitations}

\subsubsection{محدودیت‌های حقوق}
\label{sec:18-4-1-limitations}

در برخی موارد، استفاده توسط شخص ثالث خارج از دامنه حق قرار می‌گیرد. محدودیت‌ها، همان‌طور که از کلمه پیداست، حفاظت را محدود می‌کنند. برای مثال، حقوق مالکیت فکری به طور نامحدود ادامه نمی‌یابند. در اتحادیه اروپا، حق تکثیر طبق «دستورالعمل مدت» \lr{(Term Directive)} تا ۷۰ سال پس از مرگ نویسنده ادامه می‌یابد.

طبق دستورالعمل پایگاه داده، حفاظت پایگاه داده خاص به مدت ۱۵ سال از روز تکمیل پایگاه داده ادامه می‌یابد، اما این زمان با هر تغییر و/یا سرمایه‌گذاری قابل توجه جدید، مجدداً شروع می‌شود. حقوق اسرار تجاری در اینجا استثنا هستند: هیچ حداکثر مدتی برای حفاظت در دستورالعمل اسرار تجاری درج نشده است. حق راز تجاری تا زمانی که موضوع محافظت‌شده آن دیگر معیارهای این حق را برآورده نکند، ادامه خواهد داشت.

همان‌طور که قبلاً ذکر شد، حق تکثیر به حقایق و ایده‌ها گسترش نمی‌یابد. علاوه بر این، حتی موضوعی که نه حقیقت است و نه ایده، زمانی که بخشی از بیان اصیل نباشد، می‌تواند خارج از دامنه حفاظت قرار گیرد. علاوه بر این، اصالت به این معنی است که انتخاب‌های خلاقانه توسط نویسنده انجام شده است، نه اینکه باید جدید باشد. این بدان معناست که در برابر خلق مستقل محافظت نمی‌کند.

برای حق پایگاه داده خاص، حفاظت حول محور سرمایه‌گذاری می‌چرخد. اگر شخص ثالثی به‌طور اتفاقی بخش‌های ناچیزی را استخراج و/یا مورد استفاده مجدد قرار دهد، در اصل، این کار قانونی خواهد بود. با این حال، در آنجا نیز مرزهایی وجود دارد. در انجام این کار، دستورالعمل پایگاه داده شخص ثالث را ملزم می‌کند که مراقب باشد اعمالش با بهره‌برداری عادی از پایگاه داده توسط دارنده حق در تضاد نباشد یا به منافع او به‌طور نامعقول آسیب نرساند. به‌طور خلاصه، اعمال شخص ثالث نباید به سرمایه‌گذاری «آسیب» برساند.

دامنه حفاظت ارائه شده توسط حق راز تجاری نیز محدودیت‌های خود را تحت دستورالعمل اسرار تجاری دارد. حق راز تجاری تنها در برابر اعمال غیرقانونی محافظت می‌کند. این بدان معناست که خلق یا کشف مستقل با حقوق اسرار تجاری تداخل ندارد. علاوه بر این، مهندسی معکوس پس از به دست آوردن قانونی یک محصول نیز حق راز تجاری را نقض نخواهد کرد.

این بدان معناست که برای شخص ثالث قانونی خواهد بود که محصولی را که توسط دارنده حق در بازار اتحادیه اروپا عرضه شده است خریداری کند و عملکرد آن را برای بهبود فرآیند تولید، محصول و/یا خدمات خود مطالعه کند. برای مثال، یک تولیدکننده خودرو می‌تواند حسگر خودرویی را که توسط یک رقیب در بازار عرضه شده است خریداری کند تا آن را مهندسی معکوس کند و از دانش به دست آمده برای بهبود حسگرهای خودروی خود استفاده نماید.

در نهایت، چندین ارجاع به قابلیت انتقال حقوقِ تخصیص یافته به دارنده حق توسط این حقوق مالکیت فکری شده است. این بدان معناست که عموماً امکان‌پذیر است که به صورت قراردادی چنین حقوقی را «حفظ» یا منتقل کرد یا تحت شرایط خاصی اجازه انجام اعمالی را داد. بنابراین دارندگان حق خودشان نیز می‌توانند به صورت قراردادی حقوق خود را محدود کنند.

برای حفظ حقوق، برای مثال، وضعیت حقوق مشترک را در نظر بگیرید. ممکن است برای طرفین مفید باشد که به صورت قراردادی تعیین کنند که مجوز همه دارندگان حق باید اخذ شود، نه فقط یکی. از سوی دیگر، یک دارنده حق می‌تواند حق انحصاری را به یک توزیع‌کننده انحصاری منتقل کند تا حق مالکیت فکری را علیه ناقضین (ادعایی) اجرا کند و بدین ترتیب دستان خود را آزاد کند.

مثالی از اجازه دادن به اعمال تحت شرایط خاص را می‌توان در بسیاری از شرایط خدمات در صنعت بازی یافت. چنین شرایطی اغلب حاوی بندی هستند که به کاربرانشان اجازه می‌دهد در اعمالی مانند پخش زنده بازی کردنِ خودشان شرکت کنند. مثال مناسب دیگر استفاده از یک آستانه است که به کاربران اجازه می‌دهد از مواد محافظت‌شده استفاده کنند تا زمانی که سودی بیش از یک عدد تعیین شده کسب نکنند یا به تعداد مشخصی از مشتریان نرسند (جدول ۱۸.۴).

% جدول ۱۸.۴
\begin{table}[h!]
	\centering
	\caption{محدودیت‌ها \lr{(Limitations)}}
	\vspace{0.2cm}
	\small
	\begin{tabular}{|p{0.2\linewidth}|p{0.25\linewidth}|p{0.25\linewidth}|p{0.2\linewidth}|}
		\hline
		& \textbf{حق تکثیر} & \textbf{حق پایگاه داده خاص} & \textbf{حق اسرار تجاری} \\
		\hline
		حداکثر مدت & ۷۰ سال پس از مرگ نویسنده & ۱۵ سال، اما قابل تمدید & -- \\
		\hline
		خارج از دامنه & 
		\textbullet\ حقایق \newline \textbullet\ ایده‌ها \newline \textbullet\ خلق مستقل & 
		\textbullet\ استخراج بخش‌های ناچیز \newline \textbullet\ استفاده مجدد از بخش‌های ناچیز & 
		\textbullet\ خلق مستقل \newline \textbullet\ کشف مستقل \newline \textbullet\ مهندسی معکوس \\
		\hline
		محدودیت‌های قراردادی ممکن است & بله، بر روی حقوق بهره‌برداری\footnotemark & بله & بله \\
		\hline
	\end{tabular}
	\label{tab:18-4-limitations}
	\vspace{0.1cm}
	\footnotesize{\textit{یادداشت جدول: جدول متعلق به نویسنده است.}}
\end{table}
\footnotetext{همان‌طور که قبلاً ذکر شد، مجموعه حقوق همیشه به طور کامل قابل انتقال نیست. با این حال، مهم است توجه شود که این عموماً در مورد حقوق بهره‌برداری صدق نمی‌کند.}

\subsubsection{استثنائات: وجوه مشترک}
\label{sec:18-4-2-exceptions-common}

اگر عملی تحت پوشش یک استثنا قرار گیرد، توسط قانون مجاز است. این بدان معناست که دارنده حق نمی‌تواند آن عمل را مجاز یا منع کند. طبق اسناد حقوقی مانند کنوانسیون برن، توافق‌نامه \lr{TRIPS} و دستورالعمل جامعه اطلاعاتی، استثنائات باید محدود به موارد خاص باشند و با بهره‌برداری عادی از اثر تداخل نداشته باشند و به منافع مشروع نویسنده آسیب نامعقول نرسانند. به طور کلی، این استثنائات در سراسر اتحادیه اروپا به نفع حفاظت بالای حقوق مالکیت فکری به صورت محدود اعمال می‌شوند.

استثنائات تا حدودی برای هر حق مالکیت فکری متفاوت است، اما وجوه مشترکی وجود دارد. برای مثال، استثنا برای تدریس و پژوهش و مقاصد امنیت عمومی یا یک رویه اداری یا قضایی هم در دستورالعمل جامعه اطلاعاتی (در مورد حق تکثیر) و هم در دستورالعمل پایگاه داده وجود دارد. در اولی، این‌ها استثنائاتی بر حق تکثیرِ دارنده حق هستند. در دومی، این استثنائات هم حق استخراج و هم حق استفاده مجدد را هدف قرار می‌دهند.

در رژیم پایگاه داده خاص، با این حال، این استثنائات تنها توسط یک «کاربر قانونی» قابل استناد هستند. برای مثال، دور زدن الزامِ اشتراک برای دسترسی به یک پایگاه داده غیرعمومی بدون مجوز را در نظر بگیرید. استخراج و/یا استفاده مجدد توسط چنین کاربری نمی‌تواند در دامنه این استثنائات قرار گیرد.

مثالی از تدریس و پژوهش می‌تواند نمایش کلیپ‌ها، (مواد طراحی مقدماتی برای) کد نرم‌افزار، متون کوچک، یا بخش‌هایی از یک پایگاه داده برای تصویرسازی به دانشجویان یا محققان باشد. برای واجد شرایط بودن، هر دو رژیم ایجاب می‌کنند که اشخاص ثالث نباید چنین استفاده‌هایی را برای مقاصد تجاری انجام دهند. در صورت امکان، منبع باید ارجاع داده شود و استفاده نباید فراتر از آنچه برای هدف غیرتجاری دنبال شده لازم است، برود.

برای استثنای مقاصد امنیت عمومی یا یک رویه اداری یا قضایی، یک مثال می‌تواند کپی کردن یک اثر یا داده‌های خاص از یک پایگاه داده برای تأیید کالاهای وارداتی باشد. مورد دیگر می‌تواند گنجاندن چنین موادی در تصمیم کتبی یک پرونده دادگاه باشد که حول محور مسائل نقض حق تکثیر و/یا حفاظت پایگاه داده خاص می‌چرخد. باز هم، چنین اعمالی نباید برای مقاصد تجاری انجام شده باشند.

\subsubsection{استثنائات خاصِ حق}
\label{sec:18-4-3-exceptions-specific}

رایج‌ترین و مرتبط‌ترین استثنائات خاصِ رژیم حق تکثیر اتحادیه اروپا عبارتند از روزنامه‌نگاری، نقل‌قول برای نقد و بررسی، و کاریکاتور، پارودی و تقلید ادبی \lr{(pastiche)}. قانون اتحادیه اروپا، به طور خاص‌تر دستورالعمل جامعه اطلاعاتی، هیچ شرطی برای هیچ یک از این استثنائات ارائه نمی‌دهد. این بدان معناست که، برای مثال، کشورهای عضو آزاد بودند تا استثنائات را تنها به شرایط یا استفاده‌های خاصی محدود کنند.

در نهایت — و شاید مهم‌تر از همه — استثنای متنی و داده‌کاوی \lr{(text and data mining)} است که اخیراً در دستورالعمل بازار واحد دیجیتال معرفی شده است. این مفهوم به بهترین وجه به عنوان هر تکنیک تحلیلی که خودکار است درک می‌شود. از آن برای استخراج اطلاعات با تجزیه و تحلیل متن و داده‌ها به شکل دیجیتال استفاده می‌شود. این تمرین می‌تواند، برای مثال، برای کشف الگوها، روندها و همبستگی‌ها در یک مجموعه داده انجام شود.

دو نوع از این حق معرفی شده است، یکی با تمرکز بر متن‌کاوی و داده‌کاوی برای مقاصد علمی و دیگری عمومی. هر دو ایجاب می‌کنند که دسترسی قانونی به آثاری که قرار است تحت متن‌کاوی و داده‌کاوی قرار گیرند وجود داشته باشد. نوع اول اجازه ذخیره‌سازی و نگهداری تکثیرهای آثار برای تحقیقات علمی را می‌دهد. با این حال، باید سطح مناسبی از امنیت بر روی ذخیره‌سازی کپی‌های آثار وجود داشته باشد.

برای استثنای عمومی، پیش‌شرط این است که دارنده حق به صراحت استفاده از اثر خود را حفظ نکرده باشد (منع نکرده باشد). در غیاب چنین حفظ حقی، آثار می‌توانند «کاوش» شوند، نگهداری شوند و تا زمانی که برای هدف دنبال شده با متن‌کاوی و داده‌کاوی لازم است، ذخیره شوند.

تحت رژیم حقوق اسرار تجاری، مرتبط‌ترین استثنائات سه مورد زیر هستند. اول، اگر بتوان با موفقیت به آزادی بیان و حق دسترسی به اطلاعات استناد کرد، عمل ممکن است حق راز تجاری را نقض نکند. علاوه بر این، کسب، استفاده یا افشای موضوع محافظت‌شده توسط حق راز تجاری در تعقیبِ افشای سوءرفتار، تخلف یا فعالیت غیرقانونی، مانند افشاگری \lr{(whistleblowing)}، نیز ممکن است قانونی باشد.

علاوه بر این، مرتبط با یکی از محدودیت‌های ذکر شده قبلی، اگر وظایف مشروع آن‌ها به عنوان کارگران یا نمایندگان کارگران افشا را ایجاب کرده باشد، دارنده حق راز تجاری نیز ممکن است نتواند علیه آن‌ها درخواست جبران خسارت کند (جدول ۱۸.۵).

% جدول ۱۸.۵
\begin{table}[h!]
	\centering
	\caption{خلاصه استثنائات \lr{(Summary of exceptions)}}
	\vspace{0.2cm}
	\small
	\begin{tabular}{|p{0.15\linewidth}|p{0.25\linewidth}|p{0.25\linewidth}|p{0.25\linewidth}|}
		\hline
		& \textbf{حق تکثیر} & \textbf{حق پایگاه داده خاص} & \textbf{حق اسرار تجاری} \\
		\hline
		استثنائات & 
		\textbullet\ تدریس و پژوهش \newline \textbullet\ مقاصد امنیت عمومی و رویه اداری یا قضایی \newline \textbullet\ روزنامه‌نگاری \newline \textbullet\ نقل‌قول برای نقد و بررسی \newline \textbullet\ کاریکاتور، پارودی و تقلید ادبی \newline \textbullet\ متن‌کاوی و داده‌کاوی & 
		\textbullet\ تدریس و پژوهش \newline \textbullet\ مقاصد امنیت عمومی و/یا رویه اداری یا قضایی & 
		\textbullet\ آزادی بیان و حق دسترسی به اطلاعات \newline \textbullet\ افشای سوءرفتار، تخلف یا فعالیت غیرقانونی \newline \textbullet\ وظایف مشروع کارگر(ان) (نمایندگان) \\
		\hline
	\end{tabular}
	\label{tab:18-5-exceptions-summary}
	\vspace{0.1cm}
	\footnotesize{\textit{یادداشت جدول: جدول متعلق به نویسنده است.}}
\end{table}


% =================================================================
% ۱۸.۵ منابع جایگزین
% =================================================================
\subsection{منابع جایگزین}
\label{sec:18-5-alternative-sources}

این حقوق مالکیت فکری ممکن است از نظر دامنه و اهداف متفاوت باشند، اما کاملاً محتمل است که چندین حق بر (بخش‌هایی از) همان مجموعه داده یا کد قابل اعمال باشند. استثنائات این حقوق مختلف محدود به حقوق و اهداف خاصی هستند. بنابراین، ممکن است عملی که تحت یک استثنا برای یک حق مالکیت فکری قرار می‌گیرد، به دلیل وجود حق دیگری مجاز نباشد.

اگر شخص ثالثی نیاز به دسترسی به مجموعه داده‌هایی داشته باشد که (تا حدی) تحت پوشش این حقوق هستند، چندین گزینه برای دسترسی قانونی وجود دارد. ساده‌ترین گزینه، دریافت مجوز از دارنده حق برای استفاده از مجموعه داده‌های اوست. یک مجوز به دارنده مجوز اجازه می‌دهد تا از موضوع محافظت‌شده مطابق با شرایط توافق شده، معمولاً در ازای پرداخت، استفاده کند. موضوع محافظت‌شده می‌تواند برای برخی یا تمام استفاده‌های تحت پوشش حق تکثیر و/یا حفاظت پایگاه داده خاص مجوز داده شود، اما نویسنده یا سازنده همچنان مالک باقی می‌ماند. گزینه مشابه دیگر در اینجا ورود به یک توافق‌نامه موردی \lr{(ad hoc)} یا مشارکت، چه با پرداخت مبلغی یا با ارائه چیزی در عوض است.

به‌عنوان جایگزین، گاهی اوقات دسترسی به مجموعه داده‌های قابل مقایسه از طریق منابع دیگر، مانند «اطلاعات بخش عمومی» یا \lr{“PSI”} نیز امکان‌پذیر است. این یک فرصت بسیار جالب و مفید برای در نظر گرفتن است زیرا دولت داده‌های زیادی در اختیار دارد — برای مثال نقشه‌ها، تصمیمات دادگاه، داده‌های شرکت‌ها، آمار شهروندان و غیره را در نظر بگیرید — و ممکن است تعهدی برای انتشار آن داده‌ها و اجازه استفاده مجدد از آن‌ها داشته باشد (مانند قوانین آزادی اطلاعات)، اگرچه لزوماً رایگان نیست.

داده‌ها زمانی احتمالاً تابع رژیم \lr{PSI} هستند که (۱) مرتبط با اجرای فعالیت‌های دولتی باشند، (۲) هیچ حق مالکیت فکری متعلق به اشخاص ثالث بر آن‌ها وجود نداشته باشد، و (۳) داده‌ها به دلایل سیاست عمومی (از جمله حفاظت از داده‌ها) مخفی نگه داشته نشوند.

بسته به مدل کسب‌وکار، گزینه دیگری که باید در نظر گرفته شود ممکن است استفاده از نرم‌افزار یا داده‌های مشمول طرح‌های مجوزدهی باز («متن‌باز») باشد. استفاده از چنین داده‌ها یا نرم‌افزارهایی رایگان است، اما بسته به نوع مجوز، ممکن است انواع دیگری از محدودیت‌ها وجود داشته باشد. رایج‌ترین تقسیم‌بندی بین «مجوزهای سهل‌گیرانه» \lr{(permissive)} و مجوزهای «کپی‌لفت» \lr{(copyleft)} (ضعیف یا قوی) است.

این انواع مجوزها بهتر است به عنوان طیفی از کمترین محدودیت تا بیشترین محدودیت تصور شوند. هر دو نوع مجوز استفاده از موضوع را از نظر استفاده، تغییر و توزیع مجدد محدود نمی‌کنند، اما مجوزهای سهل‌گیرانه اجازه می‌دهند آثار مشتق‌شده انحصاری \lr{(proprietary)} شوند در حالی که مجوزهای کپی‌لفت این اجازه را نمی‌دهند. این بدان معناست که، برای مثال، یک شخص ثالث می‌تواند تغییراتی در موضوع تحت مجوز سهل‌گیرانه ایجاد کند و آن را تحت نوع متفاوتی از مجوز، مجوزدهی و توزیع کند.

از سوی دیگر، یک مجوز کپی‌لفت ضعیف این اجازه را نمی‌دهد. چنین مجوزهایی حاوی بندی هستند که انحصاری کردنِ مواد مشتق‌شده از موضوعِ خود یا مجوزدهی مجدد این مواد مشتق‌شده را منع می‌کنند. مجوزهای کپی‌لفت قوی علاوه بر این ایجاب می‌کنند که موضوع آن‌ها نیز نمی‌تواند تحت مجوزی متفاوت از اصلی مجوزدهی شود. این بدان معناست که اثری که مشمول یک مجوز انحصاری «عادی» است نمی‌تواند با اثر دیگری که مشمول مجوز کپی‌لفت است ترکیب شود.

ارائه خدمات یا محصولات مکمل در بازار برای ایجاد یا به دست آوردن دسترسی به مجموعه داده‌های مشابه نیز یک امکان است. برای مثال، یک شخص ثالث داده‌هایی مشابه آنچه توسط حسگرهای عرضه شده در بازار توسط یک رقیب تولید می‌شود، می‌خواهد. شخص ثالث می‌تواند تصمیم بگیرد نرم‌افزاری ارائه دهد که بتواند حسگرهای رقیب را اداره کند یا حسگرهای رقیب را ارائه دهد. گزینه دیگر در اینجا تبدیل مشتریان خودتان به جمع‌آوری‌کنندگان داده با واداشتن آن‌ها به اصلاح یا گزارش داده‌های خاص است. برای مثال، گزارش اضافات به یک نقشه یا تغییرات در یک خیابان را در نظر بگیرید.

در نهایت، اگر این مجموعه داده‌ها حاوی داده‌های شخصی باشند، می‌توانید از آن افراد بخواهید که از «حق قابلیت انتقال داده‌های» خود از طریق تبلیغاتی برای مشتریان جدید یا فعلی خدمات یا محصولات خود استفاده کنند. طبق مقررات عمومی حفاظت از داده‌ها \lr{(GDPR)}، این حق به اشخاص حقیقی فرصت می‌دهد تا داده‌های شخصی خود را از یک سرویس آنلاین به سرویس دیگر منتقل کنند.

الزامات که باید در اینجا رعایت شوند این است که داده‌ها (۱) داده‌های شخصی باشند و (۲) توسط شخصی که داده‌های شخصی مربوط به اوست به کنترل‌کننده ارائه شده باشند. برای مثال، یک شرکت بیمه یا شهرداری می‌تواند در ازای انتقال داده‌های شخصی‌شان به شما، مزایایی مانند تخفیف در حق بیمه یا خدمات ارائه شده توسط شهرداری ارائه دهد.

به‌طور خلاصه، اگر حقوق مالکیت فکری بر روی یک مجموعه داده وجود داشته باشد و هیچ‌یک از استثنائات قابل اعمال نباشند، هنوز چندین راه برای دسترسی قانونی وجود دارد. علاوه بر این، منابع جایگزین می‌توانند به عنوان منبع مکمل یا جایگزین برای مجموعه داده محافظت‌شده مورد بررسی قرار گیرند.

% =================================================================
% نتیجه‌گیری (Conclusion)
% =================================================================
\vspace{0.8cm}
\noindent
\fcolorbox{black}{gray!15}{%
	\begin{minipage}{\dimexpr\linewidth-2\fboxsep-2\fboxrule}
		\vspace{0.3cm}
		\begin{center}
			\textbf{\Large نتیجه‌گیری}
		\end{center}
		\vspace{0.2cm}
		
		برای خلاصه کردن، هنگام کار با موضوعاتی مانند مجموعه داده‌ها و نرم‌افزار، مهم است که ابتدا تعیین کنید که آیا حقوق مالکیت فکری ممکن است بر روی آن‌ها وجود داشته باشد یا خیر. اگر چنین باشد، استفاده از چنین موضوعاتی توسط اشخاص ثالث ممکن است محدود شود. اینکه کدام استفاده‌ها محدود شده‌اند و تحت چه شرایطی، بستگی به این دارد که کدام حق اعمال می‌شود و (تا درجات مختلف) کدام رژیم اعمال می‌شود (یعنی اتحادیه اروپا یا ایالات متحده).
		
		دوم، باید مشخص شود که دارنده حق کیست. اگر شما دارنده حق هستید، این بدان معناست که ممکن است بتوانید دسترسی و استفاده دیگران از موضوع محافظت‌شده را محدود کنید. اگر شخص دیگری است، چندین مسیر ممکن برای استفاده قانونی از موضوع محافظت‌شده آن طرف یا جایگزین‌هایی برای این موضوع وجود دارد؛ از دریافت رضایت از دارنده حق گرفته تا عمل در محدوده محدودیت‌ها یا استثنائات، تا یافتن یا ایجاد منابع جایگزین.
		\vspace{0.3cm}
	\end{minipage}
}

% =================================================================
% پیام‌های کلیدی (Take-Home Messages)
% =================================================================
\vspace{0.8cm}
\noindent
\fcolorbox{black}{gray!15}{%
	\begin{minipage}{\dimexpr\linewidth-2\fboxsep-2\fboxrule}
		\vspace{0.3cm}
		\centering \textbf{\large پیام‌های کلیدی}
		\vspace{0.2cm}
		\begin{itemize}
			\setlength\itemsep{0.5em}
			\item[\textbf{--}] حق تکثیر بر روی محتویات، انتخاب یا آرایش یک مجموعه داده به دارنده حق، حق انحصاری تکثیر و در دسترس عموم قرار دادن مواد محافظت‌شده را می‌دهد.
			\item[\textbf{--}] حق پایگاه داده خاص \lr{(sui generis)} به دارنده حق، حق انحصاری استخراج و استفاده مجدد از بخش‌های قابل توجه پایگاه داده را می‌دهد.
			\item[\textbf{--}] حقوق اسرار تجاری بر روی داده‌ها، دارنده حق را در برابر کسب، استفاده و افشای غیرقانونی مواد محافظت‌شده محافظت می‌کند.
			\item[\textbf{--}] شخص ثالث تنها در صورتی می‌تواند در استفاده قانونی از موضوع محافظت‌شده توسط این حقوق شرکت کند که توسط دارنده حق یا توسط قانون مجاز باشد (اگر استثنائات قابل اجرا باشند).
			\item[\textbf{--}] در غیاب مجوز، چندین راه وجود دارد که از طریق آن‌ها می‌توان به طور جایگزین به (بخش‌هایی از) مجموعه داده یا نرم‌افزار یا منابع قابل مقایسه دسترسی قانونی پیدا کرد.
		\end{itemize}
		\vspace{0.2cm}
	\end{minipage}
}

% =================================================================
% سوالات بحث
% =================================================================
\vspace{1cm}
\section*{سوالات بحث}
\addcontentsline{toc}{section}{سوالات بحث}

\begin{enumerate}
	\setlength\itemsep{0.6em}
	\item چرا ما حقوق مالکیت فکری داریم؟
	\item تمایز بین ایده‌ها و بیان‌ها را در قانون حق تکثیر چگونه توضیح می‌دهید؟
	\item لطفاً تمام انواع سرمایه‌گذاری‌های مرتبط برای حفاظت پایگاه داده خاص، از جمله ابزارهایی که از طریق آن‌ها چنین سرمایه‌گذاری‌هایی می‌تواند انجام شود را تعریف کنید.
	\item لطفاً به اختصار وضعیت مهندسی معکوس را تحت حفاظت اسرار تجاری توضیح دهید.
\end{enumerate}

% =================================================================
% مطالعه بیشتر (Further Reading)
% =================================================================
\vspace{1cm}
\section*{مطالعه بیشتر}
\addcontentsline{toc}{section}{مطالعه بیشتر}

\begin{latin}
	\begin{itemize}
		\setlength\itemsep{0.6em}
		
		\item[] Agreement on Trade-Related Aspects of Intellectual Property Rights, Annex 1c of the Marrakesh Agreement Establishing the World Trade Organization, 1994.
		
		\item[] Berne Convention for the Protection of Literary and Artistic Works of 1886 as amended on September 28, 1979 (‘Berne Convention’).
		
		\item[] Case C-202/12 Innoweb BV v Wegener [2013] ECLI:EU:C:2013:850.
		
		\item[] Case C-203/02 British Horseracing Board v William Hill [2004] ECLI:EU:C:2004:695.
		
		\item[] Case C-393/09 Bezpečnostní softwarová asociace [2010] ECLI:EU:C:2010:816.
		
		\item[] Case C-5/08, Infopaq v Danske Dagblades Forening [2009] ECLI:EU:C:2009:465.
		
		\item[] Convention on the Grant of European Patents of 1973 as amended last on 29 November 2000 (‘European Patent Convention’).
		
		\item[] Council Directive 93/98/EEC of 29 October 1993 harmonizing the term of protection of copyright and certain related rights [1993] OJ L290/9 (‘Term Directive’).
		
		\item[] Estelle Derclaye, ‘Database sui generis right: what is a substantial investment? A tentative definition’ (2005) IIC 36(1).
		
		\item[] Directive 2001/29/EC of the European Parliament and of the Council on the harmonisation of certain aspects of copyright and related rights in the information society [2001] OJ L167/10 (‘InfoSoc Directive’).
		
		\item[] Directive 2004/48/EC of the European Parliament and of the Council on the enforcement of intellectual property rights [2004] OJ L157/45 (‘Enforcement Directive’).
		
		\item[] Directive 2009/24/EC of the European Parliament and of the Council of 23 April 2009 on the legal protection of computer programs [2009] OJ L111/16 (‘Software Directive’).
		
		\item[] Directive (EU) 2016/943 of the European Parliament and of the Council on the protection of undisclosed know-how and business information (trade secrets) against their unlawful acquisition, use and disclosure [2016] OJ L 157/1 (‘Trade Secret Directive’).
		
		\item[] Directive 2019/790 of the European Parliament and of the Council on copyright and related rights in the Digital Single Market and amending Directives 96/9/EC and 2001/29/EC [2019] OJ L130/11 (‘Digital Single Market Directive’).
		
		\item[] Directive 96/6/EC of the European Parliament and of the Council of 11 March 1996 on the legal protection of databases [1996] OJ L77/20 (‘Database Directive’).
		
		\item[] Feist Publications, Inc., v. Rural Telephone Service Co., 499 U.S. 340 (1991).
		
		\item[] Husovec, M. (2019). How Europe Wants to Redefine Global Online Copyright Enforcement. TILEC Discussion Paper, 2019–2016.
		
		\item[] Publications Office in their Summary of Directive 2009/24/EC—the legal protection of computer programs, 23 January 2017.
		
		\item[] Regulation 2016/679 of the European Parliament and of the Council of 27 April 2016 on the protection of natural persons with regard to the processing of personal data and on the free movement of such data, and repealing Directive 95/46/EC [2016] OJ L 119/1 (‘General Data Protection Regulation’).
		
		\item[] Rosati, E. (2017). GS Media and its implications for the construction of the right of communication to the public within EU copyright architecture. \textit{Common Market Law Review}, 54(4), 1221–1242.
		
	\end{itemize}
\end{latin}

