% =================================================================
% فصل ۱۹: مسائل مسئولیت و قرارداد در خصوص داده‌ها
% =================================================================

\clearpage
\raggedbottom 

% تنظیم شماره‌گذاری فصل روی ۱۹
\setcounter{section}{19}
\setcounter{subsection}{0}
\renewcommand{\thesection}{19}
\renewcommand{\thesubsection}{19.\arabic{subsection}}

% تنظیم عمق شماره‌گذاری
\setcounter{secnumdepth}{4}
\renewcommand{\theparagraph}{\thesubsubsection.\arabic{paragraph}}

% -----------------------------------------------------------------
% عنوان و نویسنده
% -----------------------------------------------------------------
\noindent
\fcolorbox{black}{gray!15}{%
	\begin{minipage}{\dimexpr\linewidth-2\fboxsep-2\fboxrule}
		\vspace{0.5cm}
		\begin{center}
			\textbf{\huge مسائل مسئولیت و قرارداد در خصوص داده‌ها}
			\vspace{0.4cm}
			
			\large
			\textit{اریک تیونگ جین تای} \\
			\lr{\textit{Eric Tjong Tjin Tai}}
		\end{center}
		\vspace{0.5cm}
	\end{minipage}
}

\vspace{0.8cm}

% -----------------------------------------------------------------
% فهرست مطالب داخلی
% -----------------------------------------------------------------
\noindent
\textbf{\Large فهرست مطالب}
\vspace{0.3cm}

{ \small
	\noindent \textbf{۱۹.۱} \hspace{0.2cm} \textbf{مقدمه} \par \vspace{0.1cm}
	
	\noindent \textbf{۱۹.۲} \hspace{0.2cm} \textbf{ویژگی‌های عمومی حقوق خصوصی} \par \vspace{0.1cm}
	
	\noindent \textbf{۱۹.۳} \hspace{0.2cm} \textbf{داده چیست؟} \par \vspace{0.1cm}
	
	\noindent \textbf{۱۹.۴} \hspace{0.2cm} \textbf{قراردادها و داده‌ها} \par
	\hspace{0.8cm} ۱۹.۴.۱ تشکیل قراردادها \par
	\hspace{0.8cm} ۱۹.۴.۲ محتوای قراردادها \par
	\hspace{0.8cm} ۱۹.۴.۳ جبران خسارت‌های قراردادی \par \vspace{0.1cm}
	
	\noindent \textbf{۱۹.۵} \hspace{0.2cm} \textbf{حقوق مسئولیت مدنی ( شبه‌جرم) و داده‌ها} \par
	\hspace{0.8cm} ۱۹.۵.۱ مسئولیت مبتنی بر تقصیر \par
	\hspace{0.8cm} ۱۹.۵.۲ مسئولیت محض \par
	\hspace{0.8cm} ۱۹.۵.۳ رابطه‌ی سببی و دفاعیات \par
	\hspace{0.8cm} ۱۹.۵.۴ خسارات و سایر جبران‌ها در مسئولیت مدنی \par \vspace{0.1cm}
	
	\noindent \textbf{منابع}
}

\vspace{0.8cm}

% -----------------------------------------------------------------
% اهداف یادگیری
% -----------------------------------------------------------------
\noindent
\fcolorbox{black}{gray!15}{%
	\begin{minipage}{\dimexpr\linewidth-2\fboxsep-2\fboxrule}
		\vspace{0.3cm}
		\textbf{\large اهداف یادگیری}
		\vspace{0.2cm}
		\begin{itemize}
			\setlength\itemsep{0.5em}
			\item[\textbf{--}] درک چگونگی عملکرد قواعد حقوق خصوصی.
			\item[\textbf{--}] درک معانی مختلف «داده» در قانون.
			\item[\textbf{--}] ارزیابی یک قرارداد و شناسایی بندهای قراردادی مهم.
			\item[\textbf{--}] درک مفهوم زیان اقتصادی محض و ارتباط آن با داده‌ها.
			\item[\textbf{--}] آشنایی با مهم‌ترین مبانی مسئولیت، به ویژه موارد مرتبط با مسائل داده.
		\end{itemize}
		\vspace{0.2cm}
	\end{minipage}
}

\newpage

% =================================================================
% ۱۹.۱ مقدمه
% =================================================================
\subsection{مقدمه}
\label{sec:19-1-introduction}

این فصل مقدمه‌ای بر حوزه‌های خاصی از قانون را تا جایی که برای دانشمندان داده مرتبط است، فراهم می‌کند. اگر به عنوان یک دانشمند داده کار می‌کنید، ممکن است با سوالات حقوقی مواجه شوید. ممکن است نیاز به مذاکره در مورد یک قرارداد داشته باشید یا ممکن است نگران مسئولیت احتمالی باشید. یک مقدمه معمولی بر حقوق مانند آنچه توسط \lr{Ventura (2005)} یا \lr{Wacks (2015)} ارائه شده است، تنها کمک محدودی خواهد کرد، زیرا داده‌ها مشکلات حقوقی خاصی را ایجاد می‌کنند که ادبیات عمومی به آن‌ها پاسخ نمی‌دهد \lr{(Mak et al. 2018)}.

هدف این فصل تجهیز شما به دانش پایه حقوق قراردادها و حقوق مسئولیت مدنی است که باید شما را با اصول اساسی درگیر آشنا کند. همچنین، این فصل حاوی نکاتی برای اجتناب از دام‌های احتمالی است. از آنجا که این تنها یک مقدمه مختصر است، امکان ورود به قواعد دقیقی که ممکن است در پرونده‌های واقعی اعمال شوند وجود ندارد. در صورت تردید، با یک وکیل مشورت کنید.

ابتدا، با یک مثال مختصر و چند نکته کلی شروع خواهیم کرد. این با یک تحلیل حقوقی از اینکه داده چیست دنبال می‌شود. متعاقباً بحثی در مورد حقوق قراردادها با جزئیات بیشتر خواهد آمد. در نهایت، مسئولیت در حقوق شبه‌جرم (مسئولیت مدنی) مورد بحث قرار می‌گیرد.

\vspace{0.4cm}
% کادر مثال آلیس و باب
\begin{center}
	\colorbox{gray!15}{%
		\begin{minipage}{0.9\linewidth}
			\vspace{0.2cm}
			\noindent
			$\blacktriangleright$ \textbf{مثال}
			\vspace{0.2cm}
			
			آلیس کسب‌وکاری را راه‌اندازی کرده است که تحلیل پروفایل‌های مشتریان را برای شرکت‌های بزرگ ارائه می‌دهد. باب، آلیس را استخدام می‌کند تا داده‌های تجاری او را تحلیل کند. تحلیل توسط ایو \lr{(Eve)}، کارمند آلیس، انجام می‌شود. پس از تکمیل تحلیل، ایو به طور تصادفی پایگاه داده باب را حذف می‌کند و باب نسخه پشتیبان \lr{(backup)} نداشته است. آیا آلیس باید برای از دست رفتن پایگاه داده به باب خسارت بپردازد، و اگر چنین است، چقدر؟ $\blacktriangleleft$
			\vspace{0.2cm}
		\end{minipage}
	}
\end{center}

% =================================================================
% ۱۹.۲ ویژگی‌های عمومی حقوق خصوصی
% =================================================================
\subsection{ویژگی‌های عمومی حقوق خصوصی}
\label{sec:19-2-private-law}

همان‌طور که از مثال می‌بینید، سوالات مختلفی مطرح می‌شود. قانون، به ویژه حوزه‌ای که «حقوق خصوصی» \lr{(private law)} نامیده می‌شود، با این سوالات سروکار دارد. حقوق خصوصی بخشی از قانون است که ادعاها و روابط بین افراد خصوصی را پوشش می‌دهد: دو بخش عمده حقوق خصوصی، «حقوق قراردادها» و «حقوق مسئولیت مدنی» (که با مسئولیت سروکار دارد) هستند.

اگر ادعا یا اختلاف دیگری در مورد حقوق خصوصی دارید، می‌توانید در نهایت به دادگاه بروید تا تصمیمی در مورد اختلاف دریافت کنید. دادگاه ممکن است ادعای شما را بپذیرد که منجر به یک جبران خسارت می‌شود (نگاه کنید به بخش ۱۹.۴.۳).

در حقوق خصوصی، ما بین مواردی که قراردادی بین طرفین وجود دارد، یا جایی که ادعای مسئولیت وجود دارد در حالی که هیچ قراردادی بین قربانی و شخصی که ادعا می‌شود به اشتباه عمل کرده (مرتکب شبه‌جرم) وجود ندارد، تمایز قائل می‌شویم. نوع اول پرونده توسط حقوق قراردادها (بخش ۱۹.۴) و نوع دوم پرونده توسط حقوق مسئولیت مدنی یا مسئولیت ناشی از جرم (بخش ۱۹.۵) اداره می‌شود.

حقوق خصوصی، تا جایی که در اینجا مرتبط است، متشکل از قواعد (و استثنائات) است که تعیین می‌کند آیا یک پیامد حقوقی خاص وجود دارد یا خیر. یک تحلیل حقوقی معمولاً شامل یافتن قواعد حقوقی مربوطه و سپس ارزیابی این است که آیا این قواعد بر واقعیات پرونده اعمال می‌شوند و نتیجه چیست.

برای مثال، اگر ایجاب و قبول وجود داشته باشد، قرارداد تشکیل می‌شود. اما اگر عیبی در اراده مانند اشتباه وجود داشته باشد، قرارداد، اگرچه از نظر شکلی معتبر است، می‌تواند باطل شود (بخش ۱۹.۴.۱). علاوه بر این، اگر قرارداد معتبر باشد، اما تعهدی از قرارداد نقض شود (بخش ۱۹.۴.۳)، طلبکار (شخصی که حق دارد تعهد برایش اجرا شود) ممکن است ادعای خسارت کند (بخش ۱۹.۴.۳).

در شبه‌کد زبان برنامه‌نویسی، ساختار و رابطه بین چنین قواعدی می‌تواند به صورت زیر بیان شود (برای ارائه مثال):

\begin{latin}
	\begin{verbatim}
		if (offer & acceptance) {
			contract.valid()
		}
		if contract.mistake == TRUE {
			contract.invalid()
		}
		if contract.valid() {
			if contract.breached(case) {
				/* further conditions */
				creditor.money += contract.breached.damages(case)
			}
		}
	\end{verbatim}
\end{latin}

این نشان می‌دهد که چگونه بسته به چندین شرایط، نتیجه ممکن است این باشد که یک قرارداد معتبر یا نامعتبر است و اینکه طلبکار ممکن است حق دریافت خسارت داشته باشد. قواعد واقعی قانون بسیار پیچیده‌تر از آن چیزی هستند که این مثال نشان می‌دهد، اما حداقل ممکن است ایده‌ای از نحوه عملکرد قواعد حقوقی به شما بدهد.

برخی اطلاعات بیشتر برای جلوگیری از سوءتفاهم لازم است. اول از همه، حقوق خصوصی تعهداتی را بر افراد انسانی (اشخاص حقیقی) و همچنین بر شرکت‌ها و سایر سازمان‌ها تحمیل می‌کند. با چنین «اشخاص حقوقی» عموماً به همان روش افراد انسانی رفتار می‌شود: آن‌ها می‌توانند قرارداد منعقد کنند و ممکن است در مسئولیت مدنی مسئول شناخته شوند.

قواعد خاصی بر اشخاص حقوقی حاکم است که موضوع حقوق تجارت است و در اینجا به آن پرداخته نمی‌شود. در عمل، اشخاص حقوقی توسط نمایندگان و پرسنل مجاز (مانند مدیر عامل یا مدیر اجرایی) نمایندگی می‌شوند.

ثانیاً، باید آگاه باشید که حقوق خصوصی محلی و وابسته به زمان است: در درجه اول ملی است و ممکن است در طول زمان تغییر کند. می‌توانیم قیاسی با نحوه ایجاد برنامه‌های کامپیوتری در یک نسخه خاص از یک زبان برنامه‌نویسی خاص و سیستم عامل انجام دهیم. شما باید بدانید محیطی که برنامه در آن اجرا می‌شود چیست: حتی اگر یک برنامه ممکن است در چندین نسخه مختلف اجرا شود، تضمینی نیست که در نسخه‌ای که برای آن نوشته یا آزمایش نشده است اجرا شود. به طور مشابه، وکلا تنها می‌توانند پاسخ‌های دقیق در مورد قانون را در رابطه با سیستم حقوقی که اعمال می‌شود ارائه دهند.\footnote{این موضوع توسط آنچه حقوق بین‌الملل خصوصی نامیده می‌شود، که تعارض قوانین نیز نامیده می‌شود، اداره می‌شود.} با این حال، توصیف خطوط کلی قانون آنطور که در اکثر سیستم‌ها قابل اجراست، مشابه نحوه توصیف یک الگوریتم در شبه‌کد با انتزاع از جزئیات زبان‌های برنامه‌نویسی واقعی، امکان‌پذیر است. در این فصل، ما از چنین رویکرد انتزاعی به قانون استفاده می‌کنیم.

یک تمایز مهم که نمی‌توانیم از آن انتزاع کنیم، تمایز بین کشورهایی\footnote{یا به طور خاص‌تر، بخش‌هایی از یک کشور که سیستم یکسانی دارند: این‌ها همچنین حوزه‌های قضایی \lr{(jurisdictions)} نامیده می‌شوند.} است که دارای سیستم «کامنی‌لا» \lr{(common law)} هستند و کشورهایی که دارای سیستم «حقوق نوشته» \lr{(civil law)} هستند.

حوزه‌های قضایی کامن‌لا عبارتند از انگلستان و ولز (نه بریتانیا، زیرا اسکاتلند سیستم حقوقی متفاوتی دارد)، ایالات متحده و مستعمرات سابق انگلیس (اکثراً بخشی از کشورهای مشترک‌المنافع). اکثر کشورهای دیگر دارای سیستم‌های حقوق نوشته هستند:\footnote{همچنین برخی استثنائات وجود دارد که در هیچ یک از این دسته‌ها قرار نمی‌گیرند، به ویژه سیستم‌های ترکیبی مانند آفریقای جنوبی که دارای عناصری از کامن‌لا و حقوق نوشته و/یا سایر سیستم‌ها هستند.} آن‌ها یک کد، یک قانون مکتوب دارند که اکثر قواعد حقوق قراردادها و حقوق مسئولیت مدنی را جمع‌آوری می‌کند.

کامن‌لا ویژگی‌هایی دارد که به طور قابل توجهی از قواعد کشورهای حقوق نوشته منحرف می‌شود: ما چند مثال را در زیر بحث خواهیم کرد. به طور کلی، کامن‌لا بر تشریفات و معنای تحت‌اللفظی قراردادها تأکید دارد و طرفین را مسئول تدوین قرارداد برای بیان دقیق آنچه می‌خواهند می‌داند. سیستم‌های حقوق نوشته تمایل دارند بر قصد واقعی طرفین تأکید کنند و به دادگاه‌ها آزادی عمل بیشتری برای تفسیر قرارداد می‌دهند.

% =================================================================
% ۱۹.۳ داده چیست؟
% =================================================================
\subsection{داده چیست؟}
\label{sec:19-3-what-is-data}

قبل از اینکه بتوانیم جنبه‌های قراردادی داده‌ها و مسئولیت داده‌ها را بحث کنیم، باید اطمینان حاصل کنیم که می‌فهمیم داده واقعاً چیست، هم در واقعیت و هم در قانون.

به عنوان اولین رویکرد، ممکن است در نظر بگیرید که مردم واقعاً چگونه با داده‌ها کار می‌کنند. داده‌ها ممکن است به شکل اسناد واژه‌پرداز به عنوان پیوست ایمیل، فایل‌های موسیقی و عکس‌های دیجیتال آپلود شده در پایگاه‌های داده ابری استفاده شوند. از نظر فنی، همه این‌ها فایل‌های داده هستند. علاوه بر این، داده همچنین با اطلاعات موجود در چنین فایل‌هایی شناسایی می‌شود، مانند زمانی که از «داده‌های شخصی» صحبت می‌کنیم. در نهایت، عبارت «کلان‌داده» \lr{(big data)} مد شده است.

کلان‌داده نه آنقدر به یک پایگاه داده یا فایل مشخص، بلکه به یک سیستم مداوم اشاره دارد که به موجب آن داده‌ها به طور مداوم دریافت و پردازش می‌شوند. برای حفظ چنین سیستمی، یک سازمان نیاز به امکانات فنی (سیستم‌های مدیریت پایگاه داده، سرورها)، خدمات (تغذیه مداوم داده‌ها) و منابع انسانی (دانشمندان داده، کارکنان پشتیبانی \lr{IT}) دارد که همه آن‌ها نیاز به پشتیبانی حقوقی (قراردادهای مجوز، قراردادهای استخدام) دارند. این را می‌توان به صورت گرافیکی به شکل زیر در شکل ۱۹.۱ نمایش داد (که جریان‌های داده بین منابع مختلف و امکانات ذخیره‌سازی را نشان می‌دهد).

% استفاده از کادر جایگزین به جای TikZ برای جلوگیری از خطا
\begin{figure}[h!]
	\centering
	\fbox{
		\begin{minipage}{0.8\linewidth}
			\centering
			\vspace{0.2cm}
			\textbf{[نمودار جریان داده‌ها]}
			\vspace{0.2cm}
			
			\small
			این نمودار نشان می‌دهد که سرور مرکزی با عناصر زیر در ارتباط است:
			\begin{itemize}
				\item رابط کاربری (PC)
				\item تحلیل‌گران داده و کاربران
				\item پایگاه‌های داده (داخلی و خارجی)
				\item بسته‌های نرم‌افزاری استاندارد و سفارشی
				\item حسگرها (داخلی و خارجی)
			\end{itemize}
			\vspace{0.2cm}
		\end{minipage}
	}
	\caption{نمایش گرافیکی جریان داده‌ها. منبع: شکل متعلق به نویسنده.}
	\label{fig:19-1-graphical-representation}
\end{figure}

این سه شکل از داده، یعنی (۱) اطلاعات، (۲) فایل‌های داده، و (۳) کلان‌داده، منجر به مسائل حقوقی مختلفی می‌شوند. در ادامه، من بر دو شکل اول داده تمرکز خواهم کرد، زیرا کلان‌داده مسئله‌ای جداگانه است که نیاز به فضای بیشتری نسبت به آنچه در اینجا موجود است دارد. به طور خلاصه، کلان‌داده را می‌توان با نگاه کردن به هر عنصر به نوبه خود بهتر درمان کرد.

سوال بعدی این است که داده از نظر حقوقی چیست. داده چگونه می‌تواند توصیف شود؟ وقتی آن را بدانیم، همچنین می‌دانیم چگونه با داده در قراردادها برخورد کنیم، یا اینکه آیا داده می‌تواند منجر به مسئولیت شود.

در مورد کلان‌داده، آن چیزی خاص در قانون نیست. عناصر مختلف کلان‌داده می‌توانند به نوبه خود توصیف شوند. اطلاعات به خودی خود یک موضوع حقوقی نیست. اطلاعات ممکن است منجر به مسئولیت شود و امکان قرارداد بستن در مورد اطلاعات وجود دارد. ما برخی مثال‌ها را در زیر خواهیم دید. اما اطلاعات به خودی خود یک شیء به رسمیت شناخته شده قانونی نیست. این یک شیء ملموس نیست و به خودی خود توسط قانون محافظت نمی‌شود.

ممکن است اطلاعات تا حدی توسط یک حق مالکیت فکری محافظت شود زیرا دارای حق تکثیر یا موارد مشابه است، و قانون اسرار تجاری اطلاعات را به شرط داشتن ارزش تجاری و مخفی بودن فراهم می‌کند (نگاه کنید به فصل ۱۸ که به طور گسترده در مورد حقوق مالکیت فکری بحث می‌کند). اما اطلاعات به خودی خود محافظت نمی‌شود.

فایل‌های داده نیاز به توجه کمی بیشتر دارند. بسیاری از وکلا تمایل دارند فایل‌های داده را با اطلاعات برابر بدانند و به طور مشابه از اعطای حفاظت خاص به فایل‌های داده خودداری می‌کنند. در واقع، حقوق مالکیت فکری و حقوق پایگاه داده ممکن است خود پایگاه‌های داده را پوشش دهند، اما از فایل‌های داده به خودی خود محافظت نمی‌کنند: آن‌ها فقط کپی و توزیع شیء محافظت شده را ممنوع می‌کنند. اسرار تجاری اغلب داده‌های تجاری مرتبط را پوشش می‌دهند (نگاه کنید به فصل ۲۵).

با این حال، یک چیز توسط حقوق مالکیت فکری محافظت نمی‌شود. این کنترل واقعی بر روی یک فایل داده است. غیر وکلا کاملاً خوشحال هستند که در مورد داده صحبت کنند که دارایی کسی است یا کسی مالک داده است. وکلا در انجام این کار مردد هستند. این به دلیل آن است که داده فاقد ویژگی‌هایی است که برای اشیاء عادی مالکیت، به ویژه کالاهای ملموس مانند ماشین‌ها، کتاب‌ها و گلدان‌ها رایج است.

فایل‌های داده ناملموس هستند و فایل‌های داده انحصاری نیستند: شما می‌توانید یک کپی تهیه کنید و از این کپی بدون ایجاد مانع برای «مالک» فایل داده اصلی استفاده کنید.\footnote{وکلا از اصطلاح فنی «رقابت‌پذیر» \lr{(rivalrous)} برای نشان دادن ماهیت انحصاری مالکیت کالاهای ملموس استفاده می‌کنند.} در قانون، معمولاً تنها اشیاء ملموس توسط حقوق مالکیت محافظت می‌شوند که به شما حق بازگرداندن تصرف شیء را می‌دهند. ناملموس‌ها معمولاً تنها از دیدگاه مالکیت فکری درک می‌شوند. همان‌طور که در فصل ۱۸ دیدیم، حقوق مالکیت فکری مانند حقوق مالکیت عادی عمل نمی‌کنند زیرا آن‌ها تنها در برابر نقض محافظت می‌کنند؛ آن‌ها حقی برای به دست آوردن مجدد کنترل شیء خود (زیرا آن خلق غیرمادی است) نمی‌دهند. برای مثال، شما نمی‌توانید آهنگ \lr{Yesterday} را بدزدید. حتی اگر آن را سرقت ادبی کنید، موسیقی و متن ترانه برای دیگران در دسترس باقی می‌ماند. شما نمی‌توانید اطلاعات را به خودی خود کنترل کنید.

با این وجود، با فایل‌های داده، نوعی کنترل واقعی وجود دارد، صرفاً به این دلیل که شما از دسترسی دیگران به فایل داده جلوگیری می‌کنید. این شکل از کنترل ناملموس‌ها جدید است. در گذشته، شما فقط می‌توانستید اطلاعات را در مغز خود کنترل کنید، اما با فایل‌های داده، امکان کنترل اطلاعات خارجی وجود دارد. فایل داده شکل خاصی از اطلاعات است، درست همان‌طور که یک پرینت از داده‌ها شکل خاصی است (که توسط قانون مالکیت محافظت می‌شود، زیرا یک دسته کاغذ یک کالای ملموس است).

برابر دانستن فایل داده با اطلاعات به خودی خود نادرست است: یک فایل داده مزایایی را نسبت به اطلاعات صرفاً انتزاعی بدون توجه به فرم ارائه می‌دهد. این موضوع به راحتی آشکار می‌شود اگر عمل اسکن کردن یک کتاب کاغذی و انجام تشخیص متن روی اسکن را در نظر بگیریم: اگرچه فایل متنی حاصل نباید حاوی اطلاعات بیشتر یا متفاوتی نسبت به کتاب باشد، فایل متنی ممکن است برای اهدافی مانند تحلیل داده‌ها به روشی که کتاب کاغذی نیست، مفید باشد.

در حال حاضر، قانون در اکثر کشورها از کنترل بر فایل‌های داده تا حد معینی محافظت می‌کند، و در برخی موارد حتی به شما اجازه می‌دهد بازگرداندن فایل داده‌ای را که دزدیده شده است ادعا کنید. این حفاظتی مشابه حقوق مالکیت برای داده‌ها فراهم می‌کند. با این حال، این در مورد همه کشورها صدق نمی‌کند \lr{(Tjong Tjin Tai, 2018b)}. سایر حقوق مربوط به مالکیت (برای مثال حقوق وثیقه، مانند رهن و گرو) برای فایل‌های داده به سختی قابل تنظیم هستند.

با توجه به موقعیت نامطمئن فایل‌های داده در قانون، حفاظت از فایل‌های داده عمدتاً غیرمستقیم است: زیرا مداخله در داده‌ها یک شبه‌جرم است، یا زیرا شما تعهدات قراردادی برای رفتار با داده‌ها به شیوه صحیح را تنظیم کرده‌اید.



% =================================================================
% ۱۹.۴ قراردادها و داده‌ها
% =================================================================
\subsection{قراردادها و داده‌ها}
\label{sec:19-4-contracts-and-data}

قرارداد چیست؟ به زبان ساده، قرارداد توافقی بین دو نفر است که هر کدام تعهدات خاصی را در قبال طرف دیگر بر عهده می‌گیرند.\footnote{ممکن است بیش از دو طرف در یک قرارداد وجود داشته باشد؛ ما در مورد آن‌ها بحث نخواهیم کرد.} مثال کلاسیک، قرارداد بیع (فروش) است: خریدار باید قیمت خرید را بپردازد، و فروشنده موظف است شیء فروخته شده را به خریدار ارائه دهد.

هنگام در نظر گرفتن یک قرارداد، مهم است که بین کاغذ (یا فایل دیجیتال) که اثبات قرارداد را فراهم می‌کند و رابطه حقوقی قراردادی بین طرفین (که ناشی از عمل امضا است، که پذیرش را ثابت می‌کند) تمایز قائل شوید. کاغذ امضا شده نیز «قرارداد» نامیده می‌شود. با این حال، اگر کاغذ گم شود، خودِ قرارداد (رابطه حقوقی) همچنان وجود خواهد داشت و معتبر خواهد بود. در اینجا، ما بر رابطه حقوقی حاصل تمرکز می‌کنیم. مقدمات عمومی عبارتند از \lr{(Bix 2012, Cartwright 2013, Smits 2014)}.

موقعیت‌هایی وجود دارد که بلافاصله مشخص نیست آیا قراردادی منعقد شده است یا خیر. اینکه آیا واقعاً قراردادی وجود دارد ممکن است به قانون قابل اجرا و سایر شرایط نیز بستگی داشته باشد. مثالی از این مورد، شرایط و ضوابط \lr{(T\&C)} است که بر دسترسی و استفاده از یک وب‌سایت حاکم است: مسلماً، این یک قرارداد است زیرا به نظر می‌رسد شما را ملزم به رعایت این شرایط در حین دسترسی به وب‌سایت می‌کند، که در ازای آن دسترسی دریافت می‌کنید. اما \lr{T\&C} همچنین می‌تواند به عنوان صرفاً شرایطی که تحت آن به شما اجازه دسترسی داده می‌شود تفسیر شود، که تعهدات اضافی را فراهم نمی‌کند.

تمایز بین دو دیدگاه زمانی روشن می‌شود که، برای مثال، \lr{T\&C} جریمه‌ای را در صورت ارسال نقد منفی از وب‌سایت در جای دیگر تعیین کند. در تفسیر اول، این می‌تواند یک شرط الزام‌آور باشد؛ در شکل دوم اینطور نیست، زیرا شرط دسترسی نیست بلکه یک تعهد اضافی است که اگر قراردادی وجود نداشته باشد نمی‌تواند تحمیل شود. ما در مورد این مسائل بیشتر بحث نخواهیم کرد زیرا این‌ها منجر به بحث‌های حقوقی پیچیده می‌شوند و به راحتی قابل توضیح یا حل نیستند.

برای داده‌ها، «مجوز» \lr{(license)} از اهمیت ویژه‌ای برخوردار است. این در درجه اول به اجازه استفاده از مالکیت فکریِ کسی اشاره دارد اما با تعمیم برای اجازه استفاده از داده‌ها نیز استفاده می‌شود. یک مجوز می‌تواند بخشی از یک توافق‌نامه گسترده‌تر باشد، و می‌تواند موضوع یک «فروش» باشد، برای مثال زمانی که شما برنامه‌ای را برای تلفن هوشمند خود «می‌خرید». چنین قراردادی شامل حق دریافت فایل برنامه و یک مجوز (حق استفاده) برای برنامه است.

شما به هر دوی این عناصر نیاز دارید. ممکن است مجوزی داشته باشید در حالی که فایل را گم کرده‌اید (چون تلفن هوشمند جدیدی دارید)، که در این صورت از حق خود برای دریافت مجدد فایل برنامه استفاده خواهید کرد. برعکس، اگر کپی غیرقانونی از فایل برنامه را به دست آورده باشید، طبیعتاً مجاز به استفاده از آن نیستید: شما به اجازه، یعنی مجوز، نیاز دارید.

\subsubsection{تشکیل قراردادها}
\label{sec:19-4-1-formation}

قراردادها با «ایجاب» \lr{(offer)} و «قبول» \lr{(acceptance)} منعقد می‌شوند. یک طرف ایجابی برای یک قرارداد می‌دهد، و طرف دیگر ایجاب را می‌پذیرد.\footnote{لازم نیست که طرفین یک ایجاب صریح و روشن ارائه دهند که متعاقباً به عنوان یک عمل جداگانه پذیرفته شود؛ کافی است که طرفین به توافق متقابل در مورد قرارداد برسند، برای مثال با پیش‌نویس کردن قرارداد با هم و متعاقباً امضای قرارداد.}

در اکثر موارد، هیچ الزام شکلی وجود ندارد: قرارداد می‌تواند به صورت کتبی، دیجیتالی یا شفاهی منعقد شود؛ رضایت به قرارداد می‌تواند از اقدامات (مانند بالا بردن دست) یا حتی (در موقعیت‌های خاص) از سکوت (رضایت ضمنی) استنتاج شود. با این حال، برای برخی از انواع قراردادها، تشریفات اضافی وجود دارد. مثالی از این مورد توافق‌نامه‌های پیش از ازدواج است که عموماً نیاز به نوعی کمک حقوقی (توسط سردفتر یا سایر ارائه‌دهندگان خدمات حقوقی) دارد تا اطمینان حاصل شود که رضایت آزادانه و آگاهانه داده شده است.

در کامن‌لا، یک الزام اضافی برای یک قرارداد معتبر وجود دارد: «عوض» \lr{(consideration)}. این بدان معناست که باید نوعی اجرای متقابل در ازای آن وجود داشته باشد. اگر یک طرف به طور یک‌جانبه تعهداتی را بر عهده بگیرد، بدون دریافت چیزی در عوض، این به عنوان یک قرارداد معتبر در نظر گرفته نمی‌شود.

الزام عوض مسلماً می‌تواند منجر به مشکلاتی در مورد نرم‌افزار متن‌باز شود. چنین نرم‌افزاری به صورت رایگان ارائه می‌شود، اما ممکن است طرفی را که از نرم‌افزار استفاده می‌کند ملزم به پذیرش محدودیت‌های خاصی در استفاده از نرم‌افزار کند، به ویژه ممکن است تعهدی برای عمومی کردن هرگونه تغییری که در نرم‌افزار متن‌باز ایجاد می‌شود تحت همان مجوز متن‌باز تحمیل کند.\footnote{این نسخه به اصطلاح کپی‌لفت \lr{(copyleft)} از مجوزهای متن‌باز است. نسخه‌های دیگری نیز وجود دارد.} با این حال، دادگاه‌ها پذیرفته‌اند که استفاده از نرم‌افزار متن‌باز نیز شامل نوعی عوض به معنای انطباق با الزامات مجوز است \lr{(Jacobsen v. Katzer, 2008)}. بنابراین، پذیرش مجوز متن‌باز ممکن است یک قرارداد معتبر تحت کامن‌لا ایجاد کند.

اگرچه قراردادها عموماً تشریفات دیگری ندارند، ممکن است تعهدات اطلاعاتی وجود داشته باشد که یک طرف را ملزم به اطلاع‌رسانی کافی به طرف دیگر هنگام ارائه ایجاب برای قرارداد کند. نمونه‌هایی از آن هویت و محل کسب‌وکار هنگام قرارداد بستن در اینترنت است (برای موارد اروپایی).

یک مسئله خاص در حقوق قراردادها امکان این است که قراردادی که معتبر به نظر می‌رسد بعداً معلوم شود که واقعاً نامعتبر است. دو دلیل اصلی این است که «عیب در اراده» وجود دارد (رضایت به روش مناسبی شکل نگرفته است، مثلاً با تهدید به خشونت) یا اینکه قرارداد «نظم عمومی» را نقض می‌کند (مانند قرارداد با یک آدمکش). قراردادهای خلاف نظم عمومی باطل و بی‌اثر هستند (هیچ اثر الزام‌آوری ندارند؛ گویی هرگز منعقد نشده‌اند).

در عمل، مهم‌ترین عیب در اراده در صورت سوءتفاهم بین طرفین ایجاد می‌شود. در سیستم‌های حقوق نوشته، این ممکن است ناشی از «اشتباه» \lr{(mistake)} باشد (یک سوءتفاهم بین طرفین در مورد موضوع قرارداد، برای مثال اینکه آیا یک تلفن هوشمند کاملاً نو است یا یک مدل بازسازی شده). این بدان معناست که طرفین باید اطلاعات مرتبط را به ابتکار خود فاش کنند، و اگر طرفی در انجام این کار کوتاهی کند، طرف دیگر ممکن است به اشتباه استناد کند.

در کشورهای کامن‌لا، ساختار اصلی «اظهار خلاف واقع» \lr{(misrepresentation)} است: یکی از طرفین در قرارداد اظهار کرده (یعنی بیان کرده که چیزی چنین است) که رایانه نو است و بازسازی شده نیست. اگر معلوم شود که این درست نیست، طرف دیگر بر اساس اظهار خلاف واقع ادعایی دارد. این بدان معناست که طرفین باید ابتکار عمل را برای پرسش در مورد ویژگی‌هایی که مهم می‌دانند به دست بگیرند. پیامد اشتباه و اظهار خلاف واقع این است که طرف زیان‌دیده می‌تواند قرارداد را فسخ کند، که این اثر را دارد که طرفین به وضعیت قبل از انعقاد قرارداد بازگردانده می‌شوند و هر آنچه قبلاً اجرا شده است باید بازگردانده (اعاده) شود.

\subsubsection{محتویات قراردادها}
\label{sec:19-4-2-content}

در یک قرارداد، می‌توانید مشخص کنید که طرفین باید چه کاری انجام دهند و ممکن است چه انتظاری از یکدیگر داشته باشند. یک قرارداد معمولاً شامل چندین ماده یا بند است که این تعهدات و پیش‌فرض‌ها را توصیف می‌کند و سایر مسائل را نیز تنظیم می‌کند. تفسیر این بندها تمایل دارد بر متن بندها تمرکز کند (زیرا معمولاً فرض می‌شود که طرفین عمداً فرمول‌بندی خاصی را انتخاب کرده‌اند)، اما معنای تحت‌اللفظی ممکن است بر اساس نیات طرفین یا سایر شرایط نیز اصلاح شود.\footnote{این امر به ویژه در کشورهای حقوق نوشته صدق می‌کند.}

یک تمایز مهم بین «تعهدات به وسیله» \lr{(obligations of means)} و «تعهدات به نتیجه» \lr{(obligations of result)} است. اگر موافقت کنید که تحلیل داده انجام دهید، ممکن است قرارداد ببندید که نتیجه منجر به صرفه‌جویی ۱۰٪ از هزینه‌های جاری شود. با این حال، شما احتمالاً مایل به انجام این کار نخواهید بود، زیرا هیچ راهی وجود ندارد که بتوانید اطمینان حاصل کنید که این اتفاق می‌افتد. در آن صورت، ترجیح می‌دهید آن را به عنوان تعهد به وسیله بیان کنید: شما از بهترین تلاش‌های خود برای انجام یک تحلیل پیشرفته استفاده خواهید کرد. این شبیه به پزشکی است که تنها باید از مراقبت و مهارت، دقت حرفه‌ای، برای تلاش جهت بهبود وضعیت بیمار استفاده کند، بدون تضمین اینکه بیمار سلامت خود را به طور کامل باز خواهد یافت. برای دادن اطمینان به طرف دیگر، می‌توانید در مورد برخی تعهدات فرعی عینی‌تر توافق کنید. برای مثال، توافق می‌کنید که از روش تحلیل خاصی استفاده کنید و حداقل سه تحلیل یا گزارش مختلف ارائه دهید.

یک قرارداد همچنین می‌تواند به عنوان موضوع، «فروش» یا مجوزدهی داده‌ها را داشته باشد.\footnote{دقیق‌تر بگوییم، داده‌ها نمی‌توانند فروخته شوند زیرا فروش تنها شامل اشیاء ملموس می‌شود. با این حال، به نظر می‌رسد قانون در حال توسعه است تا معنای «فروش» را به قراردادهای مربوط به محتوای دیجیتال نیز گسترش دهد.} قرارداد می‌تواند شامل بندهایی در مورد کیفیت داده‌ها باشد (منبع چیست، تا چه سطح جزئیات/چند بیت، چه دوره‌ای، آیا مداوم است، یا آیا آپ‌تایم ۹۹.۹۹٪ وجود دارد). ممکن است مربوط به یک مجموعه داده ثابت (مانند دمای مجموعه‌ای از حسگرها در طول یک سال) یا قراردادی در مورد تغذیه مداوم داده باشد.

مهم است درک کنید که مجوز داده ممکن است این اثر را داشته باشد که دارنده مجوز استفاده دائمی از داده‌ها یا نوعی تغییر داده‌ها را داشته باشد. اگر داده‌ها، برای مثال، برای آموزش یک الگوریتم استفاده شوند، به نوعی در الگوریتم گنجانده شده‌اند و نمی‌توانند از آن حذف شوند. هنگام تنظیم پیش‌نویس مجوز موقت برای چنین استفاده‌ای، منطقی است که تصریح شود که طرفین در مورد این تثبیت قطعی داده‌ها توافق دارند، در حالی که قرارداد می‌بندند که پس از مدت مجوز، دارنده مجوز داده‌های خام را حذف کند.

علاوه بر این، اگر یک مجوز ابدی و نامحدود باشد (با حق اعطای مجوزهای فرعی)، این در واقع به معنای تبدیل دارنده مجوز به «مالک» نسخه کپی داده‌های خودش است. حتی اگر شما «مالک» داده‌های خود باقی بمانید (مگر اینکه مجوز انحصاری ارائه دهید، که در واقع دارنده مجوز را مالک جدید واقعی می‌کند)، نمی‌توانید دارنده مجوز را از انجام هر کاری که شما می‌توانید با داده‌ها انجام دهید، بازدارید. این یکی از راه‌هایی است که شرکت‌ها ممکن است اسماً بگویند که شما «مالک» باقی می‌مانید در حالی که عملاً از طریق یک مجوز گسترده، اختیاراتی شبیه به مالک به دست می‌آورند.

علاوه بر این، قراردادها اغلب حاوی بندهای زیادی هستند که نوع جبران خسارت‌های موجود برای طرفین را تنظیم می‌کنند. این بندها به ویژه در کامن‌لا مهم هستند، زیرا قوانین انگلستان و ایالات متحده تنها برای انواع خاصی از بندهای قراردادی، به اصطلاح ضمانت‌ها \lr{(warranties)} و شروط \lr{(terms)}، جبران خسارت ارائه می‌دهند،\footnote{اصطلاحات دقیق در حقوق انگلستان با حقوق ایالات متحده متفاوت است.} در حالی که قرارداد باید تصریح کند که کدام جبران خسارت‌ها برای این بندها اعمال می‌شود. در حقوق نوشته، نیازی نیست که اینقدر دقیق باشید: دادگاه حقوق نوشته هر تعهدی را که در قرارداد گنجانده شده است، بدون نیاز به فرم دقیق اجرا خواهد کرد.

یک دسته مهم از بندها، «انتخاب قانون» و «انتخاب دادگاه» است. یک قرارداد می‌تواند (و اغلب می‌کند) مشخص کند که کدام قانون اعمال می‌شود و به کدام دادگاه می‌توانید مراجعه کنید. همچنین می‌تواند حاوی بندی باشد که می‌گوید شما اصلاً نمی‌توانید به دادگاه بروید بلکه (برای مثال) باید پرونده را داوری کنید. انتخاب قانون می‌تواند این اثر را داشته باشد که شما نوعی از حفاظت را که تحت سیستم خودتان داشتید از دست بدهید. برخی انواع دیگر بندها در زیر (بخش ۱۹.۴.۲) مورد بحث قرار می‌گیرند زیرا مربوط به جبران خسارت‌ها هستند.

قانون برخی بررسی‌ها را بر روی محتوای قرارداد از طریق دکترین‌های «شروط ناعادلانه» و «رویه‌های تجاری ناعادلانه» فراهم می‌کند. به زبان ساده، این‌ها شروط به وضوح نامعقول و از نظر اجتماعی غیرقابل قبول را مجاز نمی‌دانند و ممکن است این پیامد را داشته باشند که شروطی که به نظر منطقی می‌رسد ضمنی باشند، به عنوان بخشی از قرارداد در نظر گرفته شوند.\footnote{در کامن‌لا، این تا حدی با دکترین شروط ضمنی همپوشانی دارد.} برای مثال، یک قرارداد فروش فرض می‌شود که دلالت بر این دارد که شیء فروخته شده برای هدفی که معمولاً دارد مناسب است.\footnote{در ایالات متحده، کد تجاری یکنواخت بخش ۲-۳۱۵ ضمانت ضمنیِ تناسب برای یک هدف خاص را کدگذاری می‌کند.}

در حالی که این شکل مهمی از کنترل برای اطمینان از معقول بودن قراردادها است، قواعد حقوقی کاملاً کلی هستند. آن‌ها برای روش‌های خاصی که قراردادهای شامل داده ممکن است نامعقول یا ناعادلانه باشند به‌روز نشده‌اند، و همیشه مشخص نیست که آیا می‌توان آن‌ها را برای تناسب با جامعه داده تفسیر مجدد کرد.

دکترین شروط ناعادلانه مربوط به دکترین کلی‌تری در کشورهای حقوق نوشته است، یعنی اینکه قراردادها فرض می‌شود شامل وظیفه‌ای برای تفسیر و اجرای قرارداد بر اساس «حسن نیت» \lr{(good faith)} هستند: عمل کردن مانند یک فرد معقول، و نه صرفاً تمرکز بر متن تحت‌اللفظی قرارداد. این ابزار همچنین می‌تواند برای پر کردن شکاف‌های قرارداد استفاده شود. یک نقطه ضعف استفاده از حسن نیت این است که می‌تواند منجر به انحراف دادگاه‌ها از نیات اصلی طرفین شود. در کامن‌لا، حسن نیت به طور کلی به عنوان یک ابزار پذیرفته نشده است، و بحث‌هایی در مورد اینکه آیا می‌توان از آن برای اصلاح قراردادها استفاده کرد وجود دارد. با این وجود، دادگاه‌های ایالات متحده نیز پذیرفته‌اند که در هر قراردادی، میثاق ضمنی حسن نیت و معامله منصفانه وجود دارد.

\subsubsection{جبران خسارت‌های قراردادی}
\label{sec:19-4-3-remedies}

اگر قرارداد به درستی اجرا نشود (که یعنی اگر طرف دیگر به طور کامل آنطور که در قرارداد وعده داده بود عمل نکند، یعنی او یک یا چند مورد از تعهدات خود را نقض کند)، به اصطلاح «نقض قرارداد» \lr{(breach of contract)} رخ داده است. پس از نقض، طرفی که تعهد به او مدیون بوده است (ما معمولاً این طرف را طلبکار می‌نامیم) ممکن است به یک «جبران خسارت» \lr{(remedy)} استناد کند. جبران خسارت پاسخ قانونی است که دادگاه به طلبکار ارائه می‌دهد، که هدف آن تشویق بدهکار (طرفی که تعهد بر او بوده است) به اجرای واقعی تعهدش است.

طلبکار معمولاً باید ثابت کند که بدهکار تعهد را به درستی انجام نداده است. در مورد تعهد به نتیجه این آسان است، او فقط باید نشان دهد که نتیجه محقق نشده است. در مورد تعهد به وسیله این دشوارتر است: او باید نشان دهد که بدهکار مراقبت و مهارتی را که لازم بود ارائه نداده است، که معمولاً تنها به طور غیرمستقیم قابل تعیین است.

\paragraph{۱۹.۴.۳.۱ پیش‌نیازهای استناد به جبران خسارت}
\label{sec:19-4-3-1-prerequisites}
\mbox{}\\
قبل از اینکه بتوانید ادعای جبران خسارت کنید، اکثر سیستم‌های حقوقی ابتدا نیاز دارند که بدهکار در «قصور» \lr{(default)} باشد. این بدان معناست که او قطعاً تعهد خود را انجام نداده است، و تقصیر اوست که این کار را نکرده است. این دلالت بر دو الزام دارد: اخطار قصور و فقدان عذر موجه.

\begin{enumerate}
	\item[(الف)] \textbf{بدهکار معمولاً باید شانس دومی برای انجام تعهد خود داشته باشد.}
	
	ممکن است او از عدم اجرا آگاه نبوده و در صورت شکایت شما با کمال میل وضعیت را اصلاح کند. برای مثال، اگر یک تلفن هوشمند از آمازون سفارش دهید، فروشنده نمی‌داند اگر شما بسته را دریافت نکرده‌اید. اگر شکایت کنید، او باید فرصتی برای جبران وضعیت با ارسال تلفن دیگر داشته باشد. بنابراین، بسیاری از سیستم‌های حقوقی از شما می‌خواهند که ابتدا یک «اخطار قصور» \lr{(notice of default)} ارسال کنید: بیانیه‌ای روشن مبنی بر اینکه نقض تعهد قراردادی وجود داشته است، و مهلتی که در آن می‌خواهید نقض برطرف شود. تنها پس از انقضای مهلت، بدهکار در قصور است و می‌توانید از دادگاه درخواست جبران خسارت کنید.
	
	در برخی موارد، زمانی که روشن است که بدهکار به فرصتی برای جبران نقض پاسخ نخواهد داد، یا زمانی که قرارداد خود یک مهلت قطعی را مشخص کرده است (لحظه‌ای که تعهد قطعاً باید در آن انجام شود، مانند تاریخ تحویل کیک عروسی)، ارسال اخطار قصور لازم نیست. در عوض، بدهکار به محض گذشتن مهلت و عدم انجام تعهد، فوراً در قصور تعهد خود قرار می‌گیرد.
	
	\item[(ب)] \textbf{حتی اگر تعهدی آنطور که وعده داده شده انجام نشود، ممکن است بدهکار عذر موجهی داشته باشد.}
	
	علت واقعی نقض ممکن است یک عامل خارجی باشد. برای مثال، بسته‌ای تحویل داده نشد زیرا کل ساختمان به دلیل حمله تروریستی غیرقابل دسترسی بود. در چنین موردی، رسماً نقض قرارداد وجود دارد، اما چون حمله عذر موجهی را فراهم می‌کند، بدهکار مقصر نیست و طلبکار ممکن است جبران خسارتی نداشته باشد. سوال این است: چگونه تعیین می‌کنید که آیا عذر موجهی وجود دارد؟
	
	این معمولاً با نگاه کردن به علت واقعی عدم اجرا یا نقض، و در نظر گرفتن اینکه آیا این علت «قابل انتساب» به اوست انجام می‌شود: آیا او برای آن علت مقصر است، یا مسئولیت اوست. اگر او مقصر نباشد، چنین علتی «فورس ماژور» \lr{(force majeure)} در نظر گرفته می‌شود. این موضوع روشنی نیست. برای مثال، اگر تولیدکننده‌ای نتواند تراشه‌های کامپیوتری را به دلیل اعتصاب در کارخانه‌اش به شما عرضه کند، می‌توانید این را فورس ماژور در نظر بگیرید. با این حال، اگر اعتصاب به این دلیل شروع شد که مدیران با پرسنل خود بدرفتاری کردند، می‌توانید استدلال کنید که خودشان مقصر اعتصاب هستند.
	
	برای اجتناب از این نوع بحث‌ها، قراردادها اغلب حاوی یک «بند فورس ماژور» هستند که رویدادهایی را که فورس ماژور تلقی می‌شوند فهرست می‌کند. چنین لیستی ممکن است حصری باشد یا ممکن است تنها تمثیلی باشد (نشان دادن نوع مواردی که فورس ماژور را تشکیل می‌دهند، و اجازه دادن به اینکه سایر موارد مشابه نیز فورس ماژور را تشکیل دهند).
\end{enumerate}

در سیستم‌های حقوق نوشته، عموماً تمام تعهدات قراردادی (همانطور که از تفسیر بندهای قرارداد ناشی می‌شود) در صورت عدم اجرای صحیح تعهد، منجر به جبران خسارت می‌شوند. در سیستم‌های کامن‌لا، عمدتاً انگلستان و ایالات متحده، تنها بندهای قراردادی خاص ممکن است منجر به نقض قرارداد و جبران خسارت شوند.

در انگلستان، تمایز بین اظهارات \lr{(representations)} (که منجر به دعوای اظهار خلاف واقع می‌شود) و شروط \lr{(terms)} (که منجر به نقض قرارداد می‌شود) است. چنین شروطی بیشتر به شروط اصلی \lr{(conditions)} (که منجر به فسخ می‌شود)، ضمانت‌ها \lr{(warranties)} (که علت خسارت هستند)، و شروط میانی \lr{(intermediate terms)} (که ممکن است منجر به فسخ و/یا خسارت شوند) تقسیم می‌شوند.

در ایالات متحده، اظهارات نیز منجر به دعوای اظهار خلاف واقع می‌شوند. با این حال، در حالی که در حقوق انگلستان بندی که اظهار نیست تمایل دارد به طور خودکار شرطی باشد که منجر به جبران خسارت می‌شود، در حقوق ایالات متحده، بندی که اظهار نیست به طور خودکار منجر به جبران خسارت نمی‌شود. تنها به اصطلاح «ضمانت‌ها» به طلبکار اجازه می‌دهند تا به جبران خسارت استناد کند. قرارداد اغلب نوع جبران خسارت‌هایی را که به نقض ضمانت‌های خاص متصل هستند مشخص می‌کند. در سیستم‌های حقوق نوشته، نیازی نیست که وارد جزئیات شوید که کدام جبران خسارت‌ها اعمال می‌شوند و به چه روشی. با این حال، هنوز هم مفید است که این موضوع را تحت حقوق نوشته نیز تصریح کنید، زیرا این امر ممکن است از بحث‌های پیچیده در صورت بروز نقض جلوگیری کند.

بنابراین، چهار الزام احتمالی قبل از اینکه بتوانید به جبران خسارت استناد کنید وجود دارد:
\begin{itemize}
	\item نقض یک تعهد قراردادی
	\item قصور \lr{(Default)}
	\item قابلیت انتساب (بدون عذر، بدون فورس ماژور)
	\item در کامن‌لا: تعهد در بندی قرار داده شده است که اجازه جبران خسارت می‌دهد (مانند ضمانت یا شرط)
\end{itemize}

\paragraph{۱۹.۴.۳.۲ جبران خسارت‌های موجود}
\label{sec:19-4-3-2-available-remedies}
\mbox{}\\
به طور کلی، سه جبران خسارت قراردادی وجود دارد:
\begin{itemize}
	\item اجرای عین تعهد \lr{(Specific performance)}
	\item اعطای خسارت \lr{(Award of damages)}
	\item فسخ \lr{(Termination)}
\end{itemize}

این جبران خسارت‌ها را می‌توان از دادگاه درخواست کرد. فسخ ممکن است، بسته به حوزه قضایی، خارج از دادگاه نیز در دسترس باشد (فسخ با اخطار).

\textbf{اجرای عین تعهد} به این معناست که دادگاه به بدهکار دستور می‌دهد تا تعهد خود را انجام دهد. برای مثال، شما قرارداد بسته‌اید تا یک پایگاه داده بسازید، اما از انجام آن امتناع می‌کنید، و دادگاه متعاقباً به شما دستور می‌دهد تا آنچه را وعده داده‌اید انجام دهید. در سیستم‌های حقوق نوشته، این جبران خسارت اولیه است. در کامن‌لا، اجرای عین تعهد همیشه در دسترس نیست: بستگی به نوع تعهد دارد. دلیل این نگرش ممکن است این باشد که در کامن‌لا اجرای عین تعهد در اصل توسط دادگاه نظارت می‌شود (که برای دادگاه وقت‌گیر است)، زیرا نقض اجرای عین تعهد به معنای «تمرد از دستور دادگاه» است. در سیستم‌های حقوق نوشته، اجرای عین تعهد تعهدی به طلبکار است و اغلب از طریق یک «جریمه» خصوصی اجرا می‌شود که طلبکار می‌تواند بدون زحمت دادن به دادگاه به آن استناد کند.\footnote{در حقوق فرانسه، این «astreinte» است، نوعی جریمه که زمانی که تعهد پس از یک دوره معین هنوز انجام نشود، قابل پرداخت است.}

\textbf{اعطای خسارت} جبران خسارت اولیه در اکثر موارد است، چه در حقوق نوشته و چه در کامن‌لا. این بدان معناست که دادگاه به طلبکار خسارت اعطا می‌کند، یعنی بدهکار باید خسارت را به طلبکار بپردازد. خسارت \lr{(damages)} به معنای مبلغی پول است، با هدف جبران پیامدهای منفی نقض برای طلبکار. توجه داشته باشید: خسارات \lr{(damages)} مبلغی پول است؛ خسارت \lr{(damage)} (بدون s) به معنای زیان یا آسیب واقعی است. خسارات برای جبران خسارت در نظر گرفته شده‌اند. وکلا چندین نوع خسارت را متمایز می‌کنند.

یک تمایز مهم در اشکال خسارت بین موارد زیر است:
\begin{enumerate}
	\item صدمه شخصی \lr{(personal injury)}
	\item خسارت به اموال \lr{(property damage)}
	\item زیان اقتصادی محض \lr{(pure economic loss)}
\end{enumerate}

\textbf{صدمه شخصی} به معنای خسارتی است که از آسیب به بدن یا سلامت یک فرد ناشی می‌شود، مانند هزینه‌های پزشکی و از دست دادن درآمد (به دلیل صدمه).

\textbf{خسارت به اموال} آسیب به اموال فیزیکی و پیامدهای این آسیب است. نمونه‌ها کاهش ارزش خودروی آسیب‌دیده، هزینه‌های تعمیر خودرو، و همچنین هزینه حمل‌ونقل جایگزین در زمانی است که از خودرو نمی‌توان استفاده کرد.

\textbf{زیان اقتصادی محض} شامل تمام انواع دیگر خسارات می‌شود: این به معنای زیان‌هایی است که نه از صدمه شخصی ناشی می‌شوند و نه از خسارت به اموال. مثالی از آن یک بیانیه تهمت‌آمیز است: این با صدمه یا خسارت به اموال شروع نمی‌شود، بلکه با یک آسیب غیرمادی شروع می‌شود. نمونه‌ها از دست دادن سود، زمان تلف شده و زیان‌های بازار سهام هستند. همچنین خسارت به یک پایگاه داده یا فایل داده زیان اقتصادی محض است. بنابراین، یک زیان اقتصادی (مانند از دست دادن درآمد) تنها در صورتی زیان اقتصادی محض است که پیامد صدمه یا خسارت به اموال نباشد؛ در غیر این صورت، تحت یکی از دو شکل دیگر خسارت قرار می‌گیرد.

اهمیت این تمایز این است که دو نوع اول خسارت معمولاً قابل جبران هستند، از جمله زیان‌های تبعی ناشی از صدمه یا خسارت اولیه. در مقابل آن، برخی حوزه‌های قضایی (به ویژه کامن‌لا) اجازه جبران زیان اقتصادی محض برای نقض قرارداد را نمی‌دهند. بنابراین، حتی اگر شما ممکن است رسماً جبران خسارتی برای نقض قرارداد داشته باشید، در واقعیت، ممکن است زیان شما با اعطای خسارت جبران نشود، زیرا زیان، زیان اقتصادی محض محسوب می‌شود. اگر خواهان جبران برای این نوع زیان‌ها هستید، باید بندی برای «خسارات مقطوع» \lr{(liquidated damages)} برای نقض‌های خاص اضافه کنید (در ادامه ببینید).

مثالی که ممکن است تفاوت بین این نوع زیان‌ها را نشان دهد: ارائه‌دهنده خدمات \lr{IT} شما به طور تصادفی پایگاه داده شما را حذف می‌کند. از آنجا که حذف ناشی از خسارت به اموال یا صدمه شخصی نیست، این زیان اقتصادی محض محسوب می‌شود و ممکن است در انگلستان قابل جبران نباشد. اگر ارائه‌دهنده خدمات اتاق سرور شما را آتش می‌زد، که به موجب آن پایگاه داده و تمام نسخه‌های پشتیبان (روی هارد دیسک‌ها) از بین می‌رفت، از دست رفتن پایگاه داده زیان اقتصادی تبعی ناشی از خسارت به اموال می‌بود. این زیان قابل جبران می‌بود. بنابراین توصیف بستگی به خود زیان ندارد، بلکه به روشی که در آن زیان رخ داده است (به عنوان پیامد خسارت به اموال یا صدمه شخصی، یا نه) بستگی دارد.

بندهای محدودیت یا بندهای معافیت بندهایی هستند که میزان یا نوع خسارتی را که بدهکار باید در صورت نقض جبران کند، محدود می‌کنند. مثالی که اغلب در مجوزهای نرم‌افزار یافت می‌شود این است که تنها خسارت ناشی از صدمه شخصی یا خسارت فیزیکی به اموال جبران می‌شود.\footnote{به عبارت دیگر، جبران زیان اقتصادی محض صراحتاً مستثنی شده است.} از آنجا که نرم‌افزار معمولاً باعث صدمه شخصی یا خسارت به اموال نمی‌شود، نتیجه این است که تولیدکننده نرم‌افزار معمولاً مجبور به پرداخت هیچ خسارتی نخواهد بود (مگر اینکه بند ناعادلانه باشد). یک بند خسارت همچنین ممکن است میزان خسارتی را که ممکن است اعطا شود محدود کند، برای مثال به حداکثر ۱۰,۰۰۰ یورو.

بندهای خسارات مقطوع بندهایی هستند که میزان خسارات اعطایی برای انواع خاصی از نقض را تثبیت می‌کنند. مزیت این است که هر دو طرف از قبل می‌دانند چه مقدار خسارت برای نقض‌های خاص باید پرداخت شود، که به موجب آن طرفین نیازی به بحث‌های طولانی در مورد مثلاً ارزش یک پروژه اتوماسیون ناموفق ندارند. اگر مبلغ بسیار بیشتر از زیان واقعی متحمل شده باشد، چنین بندی به منزله «بند جریمه» \lr{(penalty clause)} است.\footnote{بندهای جریمه معمولاً مجاز هستند، اما برخی حوزه‌های قضایی آن‌ها را مجاز نمی‌دانند. در سایر حوزه‌های قضایی، اگر آن‌ها به طور نامتناسبی بالا باشند، ممکن است توسط دادگاه تعدیل شوند.}

بندهای محدودیت و بندهای خسارات مقطوع می‌توانند به منزله یک شرط ناعادلانه باشند، در این صورت مجاز نیستند (بخش ۱۹.۴.۲). برای مثال، معمولاً مجاز نیست که مسئولیت مرگ یا صدمه شخصی را به طور کامل سلب کنید.

علاوه بر این، خسارت تنها در صورتی جبران می‌شود که خیلی دور از نقض نباشد. در حقوق انگلستان، الزام «قابلیت پیش‌بینی» وجود دارد: خسارت باید قابل پیش‌بینی باشد. نقض قرارداد ممکن است برای مثال نیاز به جبران هزینه‌های جایگزینی داشته باشد، اما نه جبران هزینه‌های روان‌درمانی کارمندی که احساس می‌کند شخصاً توسط نقض بی‌احترامی شده است. این میزان مسئولیت را محدود می‌کند.

\textbf{فسخ} به این معناست که قرارداد به دلیل نقض به پایان می‌رسد. نتیجه این است که هیچ تعهد دیگری بین طرفین وجود ندارد، به جز تعهدات لازم برای خاتمه دادن به قرارداد. فسخ ممکن است این اثر را داشته باشد که اجرای قراردادی باید بازگردانده شود: هر آنچه انجام شده یا داده شده است خنثی می‌شود.

این ممکن است برای مثال به این معنی باشد که پایگاه داده‌ای که ایجاد شده و به یک شرکت تحویل داده شده است باید به توسعه‌دهنده بازگردانده شود، اما پرداختی که توسعه‌دهنده دریافت کرده است نیز ممکن است مجبور باشد بازگردانده شود.\footnote{برای مثال، بخش ۶:۲۷۱ کد مدنی هلند.} یک پیامد همچنین می‌تواند این باشد که شرکت ممکن است پایگاه داده را نگه دارد اما باید مبلغ مشخصی پول برای آن به توسعه‌دهنده بپردازد. همان‌طور که این مثال روشن می‌کند، احتمالات زیادی وجود دارد. مفید است که مهم‌ترین پیامدهای فسخ را با جزئیات تنظیم کنید تا اطمینان حاصل شود که پس از فسخ در موقعیتی که می‌خواهید خواهید بود.

انواع دیگر فسخ وجود دارد که نیاز به نقض ندارد (برای مثال، بندی که اجازه فسخ می‌دهد اگر طرف قرارداد شما توسط رقیب شما تصاحب شود).

اکنون می‌توانیم به مثال بخش ۱۹.۱ بازگردیم: حذف پایگاه داده به وضوح به عنوان بخشی از قرارداد در نظر گرفته نشده است، در حالی که قابل تصور است که آلیس باید اقدامات احتیاطی را در برابر چنین اتفاق ناگواری انجام می‌داد. بنابراین، ممکن است گفته شود که این نقض قرارداد را تشکیل می‌دهد. اگر کامن‌لا اعمال شود، همچنین ممکن است لازم باشد که ضمانت یا شرط دیگری وجود داشته باشد که زمینه‌ای برای نقض قرارداد در صورت حذف پایگاه داده فراهم کند. اگرچه اشتباه توسط ایو انجام شده است، آلیس به عنوان کارفرمای او مسئول است زیرا نمی‌تواند خود را با اشاره به تقصیر کارمند تبرئه کند (این فورس ماژور محسوب نمی‌شود).

با این حال، زیان، زیان اقتصادی محض است که تحت کامن‌لا ممکن است بر اساس نقض قرارداد قابل بازیابی نباشد. این ممکن است متفاوت باشد اگر قرارداد حاوی بند خسارات مقطوع باشد. در یک حوزه قضایی حقوق نوشته، زیان ممکن است قابل جبران باشد، و خسارات ممکن است به ارزش بازار پایگاه داده یا هزینه‌های (باز)تولید پایگاه داده ارزیابی شود (نگاه کنید به بخش ۶.۴).



% =================================================================
% ۱۹.۵ حقوق مسئولیت مدنی (شبه‌جرم) و داده‌ها
% =================================================================
\subsection{حقوق مسئولیت مدنی (شبه‌جرم) و داده‌ها}
\label{sec:19-5-tort-law-and-data}

اگر قراردادی بین دو نفر وجود نداشته باشد، مسئولیت باید بر اساس حقوق مسئولیت مدنی \lr{(tort law)} باشد.\footnote{در سیستم‌های حقوق نوشته، نام‌های جایگزینی مانند حقوق مسئولیت ناشی از جرم \lr{(delictual liability)} یافت می‌شود. ادبیات عمومی در مورد حقوق مسئولیت مدنی: \lr{Van Dam 2013, Tjong Tjin Tai 2022}.} یک شبه‌جرم، به زبان ساده، برای موارد خاص توصیف می‌کند که تحت چه شرایطی یک شخص مسئول است. مثالی از آن افترا \lr{(defamation)} است: این شبه‌جرم تنظیم می‌کند که چه زمانی کسی برای بیان اظهارات افتراآمیز مسئول است. انواع مختلفی از شبه‌جرم‌ها وجود دارد. در حقوق مسئولیت مدنی، ما بین دو شکل از مسئولیت تمایز قائل می‌شویم: مسئولیت مبتنی بر تقصیر و مسئولیت محض.

\subsubsection{مسئولیت مبتنی بر تقصیر \lr{(Fault Liability)}}
\label{sec:19-5-1-fault-liability}

ایده کلی مسئولیت مبتنی بر تقصیر این است که کسی (وکلا این شخص را «مرتکب شبه‌جرم» یا \lr{tortfeasor} می‌نامند) مسئول است اگر مرتکب تقصیری شده باشد: او کاری انجام داده است که از نظر قانونی نباید انجام می‌داد، و این تقصیر باعث آسیب به قربانی شده است. بنابراین، سه شرط وجود دارد:
\begin{itemize}
	\item رفتار نادرست (تقصیر)
	\item آسیب \lr{(Harm)}
	\item رابطه سببی بین تقصیر و آسیب
\end{itemize}

آسیب به معنای نوعی عدم مزیت، حادثه یا زیان است. این ممکن است یک حادثه فیزیکی، آسیب به کالا، آسیب به شهرت و از دست دادن حریم خصوصی باشد. رابطه سببی در بخش ۱۹.۵.۳ بیشتر مورد بحث قرار می‌گیرد.

اکثر شبه‌جرم‌ها با نوع تقصیری که پوشش می‌دهند تعریف می‌شوند، اما نوع آسیب نیز ممکن است مرتبط باشد. مثالی از آن دوباره افترا است: این مورد برای نوع خاصی از اظهارات اعمال می‌شود، اما همچنین نیاز به وجود آسیب به شهرت قربانی دارد. سیستم‌های حقوقی روش‌های مختلفی برای تعیین اینکه آیا رفتار خاصی (که ممکن است شامل فعل و همچنین ترک فعل باشد)\footnote{در سیستم‌های کامن‌لا، ترک فعل‌های محض (که ناشی از یک عمل مسامحه‌آمیز قبلی نیستند) ممکن است منجر به مسئولیت نشوند.} نادرست است یا خیر، دارند.

مهم‌ترین شبه‌جرم، «غفلت» یا «مسامحه» \lr{(negligence)} است.\footnote{این نام در کامن‌لا است؛ در سیستم‌های حقوق نوشته، نام‌های دیگری یافت می‌شود، اما غفلت ممکن است برای نشان دادن شبه‌جرم محلی در ترجمه انگلیسی استفاده شود.} دقیق‌تر بگوییم، این یک شبه‌جرم در کامن‌لا است، اما سیستم‌های حقوق نوشته نیز قواعدی دارند که مسئولیت را در مواردی که تحت پوشش شبه‌جرم غفلت قرار می‌گیرند، فراهم می‌کنند. غفلت به این معناست که شما مراقبت یا دقت کافی را نسبت به منافع قربانی رعایت نکرده‌اید: این مستلزم نقض «وظیفه مراقبت» \lr{(duty of care)} است.

غفلت می‌تواند برای مثال زمانی اعمال شود که شما الگوریتمی برای یک خودروی خودران توسعه داده‌اید بدون اینکه در طول توسعه مراقبت کافی برای جلوگیری از اشتباهات در شناسایی موانع انجام داده باشید، و خودرو عابری را کشت زیرا فکر می‌کرد یک جعبه مقوایی است. غفلت مفید است زیرا یک هنجار باز \lr{(open-ended norm)} ارائه می‌دهد که می‌تواند برای تحولات جدید اعمال شود. نقطه ضعف این است که کاربرد واقعی ممکن است نامشخص باشد: چه زمانی رفتار مسامحه‌آمیز است؟ آزمون معمول این است که آیا یک «فرد معقول» \lr{(reasonable person)} به همان روش عمل می‌کرد یا خیر.

شبه‌جرم‌های زیادی غیر از غفلت وجود دارند که ممکن است برای داده‌ها اعمال شوند. نمونه‌ها نقض حریم خصوصی و افترا هستند. این‌ها توسط قواعد و شرایط خاصی اداره می‌شوند؛ این مرور کلی جای پرداختن به جزئیات نیست. علاوه بر این، بسیاری از سوءاستفاده‌ها یا مداخلات نادرست در داده‌ها و رایانه‌ها جرم محسوب می‌شوند.\footnote{به ویژه به دلیل تأثیر کنوانسیون‌های مختلف در مورد جرایم سایبری.} نقض حقوق کیفری معمولاً به عنوان یک شبه‌جرم نیز تحت عنوان شبه‌جرم «نقض وظیفه قانونی» \lr{(breach of a statutory duty)} در نظر گرفته می‌شود.

\subsubsection{مسئولیت محض \lr{(Strict Liability)}}
\label{sec:19-5-2-strict-liability}

مسئولیت محض به این معناست که شما مسئول هستید حتی اگر شخصاً مرتکب تقصیری نشده باشید. نمونه‌ها عبارتند از:
\begin{itemize}
	\item مسئولیت کارفرما برای شبه‌جرم‌های ارتکابی توسط کارمند
	\item مسئولیت مالک خودرو برای حوادث شامل خودرو
	\item مسئولیت محصولِ تولیدکننده صنعتیِ محصولات ملموس
\end{itemize}

اکثر حوزه‌های قضایی این سه شکل از مسئولیت محض را به رسمیت می‌شناسند. برای اعمال این اشکال از مسئولیت محض، حداقل موارد زیر مورد نیاز است:
\begin{enumerate}
	\item رابطه خاصی بین شخصی که مسئول شناخته می‌شود و شیء یا شخصی که باعث خسارت شده است (رابطه استخدامی، مالکیت، کنترل).
	\item شکل خاصی از رفتار یا فعالیت شیء یا شخص (عمل شبه‌جرمی، تحقق یک خطر خاص).
\end{enumerate}

در بسیاری از کشورها، اشکال دیگری از مسئولیت محض نیز وجود دارد، مانند مسئولیت برای کودکان، حیوانات، اشیاء خطرناک و فعالیت‌های خطرناک. با این حال، این اشکال همه جا یا به یک اندازه پذیرفته نشده‌اند. مزیت مسئولیت محض این است که حفاظت بیشتری برای قربانیان فراهم می‌کند، که می‌توانند جبران خسارت را از طرفی که واقعاً از فعالیت پرخطر یا استخدام شخص سود می‌برد (یا تصمیم به ورود به آن گرفته است) تأمین کنند. در ادبیات، مسئولیت محض اغلب به عنوان مدلی برای تنظیم مسئولیت ربات‌ها و الگوریتم‌ها پیشنهاد می‌شود \lr{(Tjong Tjin Tai 2018a, and references therein)}.

برای اطمینان از اینکه مسئولیت محض بیش از حد گسترده نمی‌شود، چندین محدودیت وجود دارد. به ویژه، محدودیت‌های عمومی علیت و ارزیابی خسارات، و همچنین دفاعیات اعمال می‌شوند (بخش‌های ۱۹.۵.۳ و ۱۹.۵.۴).

\subsubsection{رابطه سببی و دفاعیات}
\label{sec:19-5-3-causality-defenses}

یک شرط برای دریافت جبران خسارت برای یک شبه‌جرم این است که آسیبی وجود داشته باشد که توسط تقصیر ایجاد شده باشد: «علیت» \lr{(causality)}. رابطه سببی بین تقصیر و آسیب ابتدا با نگاه کردن به رابطه سببی واقعی (علیت واقعی) ارزیابی می‌شود: آیا اگر تقصیر رخ نمی‌داد، خسارت نیز رخ می‌داد؟ این به اصطلاح «آزمونِ اگر-نبود» \lr{(but-for test)} (یا \lr{conditio sine qua non} در سیستم‌های حقوق نوشته) یک شرط لازم برای فرض یک شبه‌جرم است.

پس از آن، یک رابطه سببی دوم مورد نیاز است: «علیت حقوقی» \lr{(legal causality)}. آسیب و خسارت نباید خیلی دور باشد. این آزمونی مشابه با مسئولیت قراردادی است: قابلیت پیش‌بینی یا دوری عمومی \lr{(remoteness)} ممکن است به عنوان معیار استفاده شود. جزئیات بستگی به شبه‌جرم و سیستم حقوقی دارد. دوری کمک می‌کند تا قرار گرفتن احتمالی در معرض مسئولیت محدود شود. برای مثال، اگر شما اشتباهی در به‌روزرسانی یک روتین متن‌باز مرتکب شوید که در نهایت باعث اختلال در عملکرد سرورها در سراسر جهان شود، مسئولیت شما ممکن است محدود شود اگر این پیامدها خیلی دور تشخیص داده شوند.

اگر چندین مرتکب شبه‌جرم وجود داشته باشد، علیت معمولاً «تضامنی» \lr{(joint and several)} است: مرتکبین شبه‌جرم به صورت فردی و همچنین جمعی مسئول هستند. هر کدام ممکن است به تنهایی برای کل مبلغ خسارت مورد شکایت قرار گیرند؛ پس از آن، مرتکبی که خسارت را پرداخت کرده است ممکن است سهمی را از سایر مرتکبین دریافت کند.

اگر دفاعیه‌ای اعمال شود، مسئولیت کاهش می‌یابد یا محدود می‌شود. یک دفاع مهم «غفلت مشارکتی» \lr{(contributory negligence)} است: اگر قربانی کاری انجام داده باشد که همچنین در وقوع آسیب یا میزان خسارت نقش داشته باشد، دادگاه ممکن است اعطای خسارت را به نسبتی که قربانی در خسارت نقش داشته است کاهش دهد.

سایر دفاعیات ممکن است مانع مسئولیت شوند. مثالی از آن دفاع فورس ماژور در مسئولیت محض است.\footnote{این تا حدودی شبیه به فورس ماژور در حقوق قراردادها است (بخش ۱۹.۴.۳.۱).} اگر برای مثال حادثه‌ای نه توسط خودرو یا راننده آن، بلکه به دلیل برخورد کامیونی از عقب به خودرو و در نتیجه هل دادن آن به سمت خودروی جلویی ایجاد شده باشد، این ممکن است مالک خودرو را از مسئولیت تبرئه کند.

\subsubsection{خسارات و سایر جبران‌ها در مسئولیت مدنی}
\label{sec:19-5-4-damages-tort}

اگر شرایط شبه‌جرم محقق شود، قربانی حق دریافت جبران خسارت دارد. مفهوم کلی جبران خسارت در شبه‌جرم شبیه به جبران خسارت در قرارداد است (بحث شده در بخش ۱۹.۴.۳)، اما الزامات متفاوت است و همه جبران خسارت‌ها در هر دو موقعیت اعمال نمی‌شوند.

جبران خسارت اصلی برای یک شبه‌جرم، اعطای خسارت است. سایر جبران خسارت‌های مهم ممکن است در قالب حکم دادگاه \lr{(court order)} داده شوند. احکام دادگاه اعلامیه‌هایی توسط دادگاه هستند. انواع مختلفی از احکام وجود دارد؛ احکامی که به عنوان ارائه دهنده جبران خسارت مهم هستند آن‌هایی هستند که به طرفی دستور می‌دهند عمل یا اعمال خاصی را انجام دهد، یا از رفتار خاصی خودداری کند. مثالی از آن حکم منع کننده \lr{(restraining order)} است که رفتار خاصی را ممنوع می‌کند. نوع خاصی از حکم، «دستور موقت» \lr{(injunction)} است که انجام یک عمل خاص را ممنوع یا دستور می‌دهد.\footnote{تعاریف دقیق احکام و دستورات موقت ممکن است بین سیستم‌های حقوقی مختلف متفاوت باشد.}

اعطای خسارت به این معناست که زیانی که شبه‌جرم ایجاد کرده است باید توسط مرتکب جبران شود. دادگاه خسارت را ارزیابی می‌کند. با این حال، چندین پیچیدگی وجود دارد.

اول از همه، برای بسیاری از شبه‌جرم‌ها، همه انواع خسارت جبران نمی‌شوند. به ویژه، زیان اقتصادی محض اغلب جبران نمی‌شود.\footnote{برای مثال، شبه‌جرم غفلت در بسیاری از موارد اجازه جبران زیان اقتصادی محض را نمی‌دهد، اگرچه چنین جبرانی همیشه مستثنی نیست.} از آنجا که پرونده‌های مربوط به داده‌ها اغلب تنها باعث زیان اقتصادی محض می‌شوند، چنین شبه‌جرم‌هایی ممکن است عملاً جبران خسارت مناسبی نداشته باشند.

شکل‌های جایگزین خسارت وجود دارد که اشکال معمول خسارت را تکمیل می‌کند. یک مثال خاص «خسارات تنبیهی» \lr{(punitive damages)} است (اعطای مبلغی پول که از خسارت واقعی فراتر می‌رود: این به عنوان مجازات مرتکب شبه‌جرم تحمیل می‌شود تا به عنوان بازدارنده عمل کند).

آسیب‌هایی که ممکن است از داده‌ها ناشی شوند چیست؟ به طور خیلی کلی، می‌توانیم مسائل زیر را در نظر بگیریم:
\begin{itemize}
	\item فقدان کیفیت
	\item خطا در تحلیل
	\item از دست دادن داده‌ها
	\item نشت داده‌ها/از دست دادن حریم خصوصی
\end{itemize}

در مورد موضوع اول، اطلاعات بسیار کمی در این مورد وجود دارد \lr{(Zeno-Zencovich, 2018)}. در کلان‌داده، کیفیت به عنوان یک امر مسلم فرض نمی‌شود؛ بلکه ایده کلان‌داده اغلب کار با آنچه دارید، آلوده یا نه، با این فرض است که لکه‌های جزئی از نظر آماری خنثی می‌شوند. با این حال، داده‌هایی که به درستی ساختار نیافته‌اند یا به نحو دیگری نادرست هستند ممکن است باعث پیامدهای منفی شوند: کاربر ممکن است مجبور شود تلاش قابل توجهی برای نرمال‌سازی یا بازسازی داده‌ها صرف کند، و ممکن است نتیجه‌گیری‌های نادرستی بگیرد که منجر به تصمیمات بد شود.

خارج از قرارداد، دلیل کمی برای فرض اینکه شما دعوی برای فقدان کیفیت دارید وجود دارد، زیرا شما برای داده‌ها پولی پرداخت نکرده‌اید. اگر داده‌ها را در دسترس قرار می‌دهید، می‌تواند مفید باشد که تصریح کنید هیچ تضمینی در مورد کیفیت داده‌ها ارائه نمی‌دهید. اگر داده‌ها بر اساس قرارداد ارائه شوند، پیامدهای منفی در تئوری می‌تواند جبران شود، اما این‌ها معمولاً زیان اقتصادی محض هستند و ممکن است خارج از جبران خسارت قرار گیرند. علاوه بر این، اکثر قراردادها حاوی بندهایی در مورد چنین خسارتی هستند، که یا قواعد روشن خسارات مقطوع یا حتی بندهای استثنا/محدودیت را ارائه می‌دهند.

در مورد خطا در تحلیل: این احتمالاً یک مسئله قراردادی خواهد بود. قواعد قرارداد اعمال خواهد شد، به ویژه بندهای خسارات مقطوع یا بندهای محدودیت.

برای از دست دادن داده‌ها، ایده کلی این خواهد بود که از دست دادن ارزش داده‌ها می‌تواند جبران شود، یا احتمالاً هزینه بازسازی پایگاه داده.

برای نشت داده‌ها و از دست دادن حریم خصوصی، اغلب هیچ جبران خسارت مؤثری وجود ندارد \lr{(Peters 2014, Varuhas 2018)}. مثال یک وب‌سایت اجتماعی را در نظر بگیرید که هک شده است، که به موجب آن تمام داده‌های خصوصی شما در معرض هکرها قرار گرفته است. در حالی که این ممکن است تجاوز جدی به حریم خصوصی باشد، هیچ زیان مادی از این نقض وجود ندارد. حریم خصوصی به خودی خود ارزش روشنی ندارد؛ در بهترین حالت، ممکن است مقدار نمادین یا اسمی خسارت برای آسیب غیرمادی، یا در موارد نادر فجیع خسارات تنبیهی اعطا شود.

در حالی که داده‌های خصوصی شما می‌تواند برای هک کردن سایر حساب‌ها و متعاقباً ایجاد زیان مادی (مانند سرقت پول) استفاده شود، این معمولاً زیان اقتصادی محض خواهد بود. حتی اگر چنین زیانی در سیستم حقوقی شما قابل جبران باشد، اثبات رابطه سببی بین نشت داده و زیان دشوار خواهد بود. تنها در مورد نقض حریم خصوصی افراد مشهور ممکن است امکان اعطای خسارت قابل توجه وجود داشته باشد. بنابراین، در مورد شبه‌جرم‌های مربوط به داده‌ها، به دست آوردن خسارات قابل توجه ممکن است دشوار باشد.

ممکن است آموزنده باشد که دوباره به مثال بخش ۱۹.۱ نگاه کنیم، این بار از دیدگاه حقوق مسئولیت مدنی. فرض کنید که ایو این بار پایگاه داده شخص ثالثی (چارلی) را که اتفاقاً روی همان سرورِ پایگاه داده مشتری ذخیره شده بود، حذف کرده است. حذف پایگاه داده توسط ایو احتمالاً یک عمل مسامحه‌آمیز (غفلت) را تشکیل می‌دهد. آلیس، به عنوان کارفرما، مسئولیت نیابتی \lr{(vicariously liable)} برای غفلت ایو دارد. از آنجا که زیان، زیان اقتصادی محض است (ناشی از خسارت به اموال یا صدمه شخصی نیست)، ممکن است در کامن‌لا قابل بازیابی نباشد. در سیستم‌های حقوق نوشته، زیان ممکن است جبران شود. می‌توان استدلال کرد که غفلت مشارکتی از طرف چارلی وجود دارد: اگر پایگاه داده آنقدر مهم است، او نیز می‌توانست نسخه پشتیبان تهیه کند. در یک موقعیت قراردادی، اگر قرارداد بدهکار (آلیس) را ملزم به تهیه نسخه پشتیبان کند، این دفاع ممکن است کارساز نباشد.




% =================================================================
% نتیجه‌گیری (Conclusion)
% =================================================================
\vspace{0.8cm}
\noindent
\fcolorbox{black}{gray!15}{%
	\begin{minipage}{\dimexpr\linewidth-2\fboxsep-2\fboxrule}
		\vspace{0.3cm}
		\begin{center}
			\textbf{\Large نتیجه‌گیری}
		\end{center}
		\vspace{0.2cm}
		
		مسائل متعددی وجود دارند که هنگام قرارداد بستن در مورد داده‌ها و هنگام در نظر گرفتن مسئولیت برای داده‌ها نیاز به توجه دارند. علاوه بر نکات حقوقی عمومی ( که در این فصل معرفی شدند)، پیچیدگی‌های خاصی نیز وجود دارد که عمدتاً نتیجه ماهیت ناملموس داده‌هاست. 
		
		باید درک کرد که «داده» یک مفهوم ثابت نیست و ممکن است عناصری از زیرساخت‌های اطراف را نیز در بر بگیرد. توجه ویژه‌ای در مورد جبران خسارت محدودی که ممکن است به دست آید و انواع گسترده‌تر آسیب‌هایی که ممکن است توسط داده‌ها ایجاد شود، مورد نیاز است.
		\vspace{0.3cm}
	\end{minipage}
}

% =================================================================
% پیام‌های کلیدی (Take-Home Messages)
% =================================================================
\vspace{0.8cm}
\noindent
\fcolorbox{black}{gray!15}{%
	\begin{minipage}{\dimexpr\linewidth-2\fboxsep-2\fboxrule}
		\vspace{0.3cm}
		\centering \textbf{\large پیام‌های کلیدی فصل}
		\vspace{0.2cm}
		\begin{itemize}
			\setlength\itemsep{0.5em}
			\item[\textbf{--}] داده یک مفهوم به خوبی تعریف شده در حقوق نیست. ممکن است به فایل‌های داده، اطلاعات موجود در فایل‌ها یا کلان‌داده (مجموعه‌ای از عناصر مختلف) اشاره داشته باشد. فایل‌های داده به عنوان یک نهاد در قانون شناخته نمی‌شوند؛ اطلاعات تنها توسط حقوق مالکیت فکری (از جمله قانون اسرار تجاری) محافظت می‌شود.
			\item[\textbf{--}] در قراردادها، صراحت در مورد انتظارات و پیامدهای عدم تحقق آن‌ها بسیار مهم است. این امر در مورد اظهارات، ضمانت‌ها و تعهدات قراردادی به طور کلی صدق می‌کند.
			\item[\textbf{--}] آسیب ممکن است شامل صدمه شخصی، خسارت به اموال یا زیان اقتصادی محض باشد. زیان اقتصادی محض در بسیاری از موارد به دلیل محدودیت‌های حقوق قراردادها جبران نمی‌شود. در مورد داده‌ها، اغلب فقط زیان اقتصادی محض وجود دارد؛ لذا ممکن است هیچ خسارتی دریافت نکنید.
			\item[\textbf{--}] مسئولیت برای داده‌ها مستلزم وجود یک شبه‌جرم (مسئولیت مدنی) قابل اعمال است که با داده‌ها سروکار دارد و نیاز به عمل نادرست، آسیب و رابطه سببی دارد.
			\item[\textbf{--}] شما ممکن است نه تنها برای اعمال نادرست مستقیم، بلکه برای اعمال شخص دیگر (مسئولیت نیابتی) یا پیامدهای اشیائی که تحت کنترل دارید، مسئول باشید.
		\end{itemize}
		\vspace{0.2cm}
	\end{minipage}
}

% =================================================================
% پرسش‌ها و پاسخ‌ها
% =================================================================
\vspace{1cm}
\section*{پرسش‌ها و پاسخ‌ها}
\addcontentsline{toc}{section}{پرسش‌ها و پاسخ‌ها}

\noindent \textbf{\large ؟ پرسش‌ها}
\begin{enumerate}
	\setlength\itemsep{0.5em}
	\item شروط قراردادی ناعادلانه (سوءاستفاده‌گرانه) به چه روشی تنظیم می‌شوند؟
	\item الزامات استناد به جبران خسارت برای نقض قرارداد چیست؟
	\item زیان اقتصادی محض \lr{(Pure economic loss)} چیست؟
	\item چگونه می‌توان ریسک مسئولیت قراردادی را محدود کرد؟
	\item الزامات اساسی مسئولیت مدنی (شبه‌جرم) چیست؟
\end{enumerate}

\vspace{0.5cm}
\noindent \textbf{\large $\checkmark$ پاسخ‌ها}
\begin{enumerate}
	\setlength\itemsep{0.5em}
	\item از طریق کنترل شروط ناعادلانه یا رویه‌های تجاری ناعادلانه، یا از طریق اصل حسن نیت.
	\item نقض یک تعهد قراردادی، قصور \lr{(default)}، و قابلیت انتسابِ علتِ نقض. در کامن‌لا همچنین: نقض یک ضمانت \lr{(warranty)}.
	\item زیان‌هایی که پیامد صدمه فیزیکی یا خسارت به اموال نیستند.
	\item با محدود کردن دامنه تعهدات/ضمانت‌ها، استفاده از بندهای محدودیت/معافیت، داشتن بند فورس ماژور گسترده و بندهای خسارت مقطوع.
	\item تقصیر، علیت (واقعی و حقوقی) و آسیب.
\end{enumerate}

% =================================================================
% مراجع (References) - اصلاح شده برای رفع خطاها
% =================================================================
\vspace{1cm}
\section*{منابع}
\addcontentsline{toc}{section}{منابع}

\begin{latin}
	\begin{itemize}
		\setlength\itemsep{0.6em}
		
		\item[] Bix, B. H. (2012). \textit{Contract law. Rules, theory, and context}. Cambridge University Press.
		
		\item[] Cartwright, J. (2013). \textit{Contract law. An introduction to the English law of contract for the civil lawyer}. Hart, Oxford.
		
		\item[] Mak, V., Tjong Tjin Tai, T. F. E., \& Berlee, A. (Eds.). (2018). \textit{Research handbook in data science and law}. Elgar Publishing.
		
		\item[] Peters, R. M. (2014). So you’ve been notified, now what? the problem with current data-breach notification laws. \textit{Arizona Law Review}, 56(4), 1171–1202.
		
		\item[] Smits, J. M. (2014). \textit{Contract law: A comparative introduction}. Elgar.
		
		\item[] Tjong Tjin Tai, T. F. E. (2018a). Liability for (semi-)autonomous systems, in: Mak et al. (2018), pp. 55–82.
		
		\item[] Tjong Tjin Tai, T. F. E. (2018b). Data ownership and consumer protection. \textit{Journal of European Consumer and Market Law}, 7(4), 136–140.
		
		\item[] Tjong Tjin Tai, T. F. E. (2022). \textit{Tort Law: A Comparative Introduction}. Elgar.
		
		\item[] US Federal Court of Appeals, \textit{Jacobsen v. Katzer}, 535 F.3d 1373 (Fed. Cir. 2008).
		
		\item[] van Dam, C. C. (2013). \textit{European tort law} (2nd ed.). Oxford University Press.
		
		\item[] Varuhas, J. N. E. (2018). Varieties of damages for breach of privacy. In J. N. E. Varuhas \& N. A. Moreham (Eds.), \textit{Remedies for breach of privacy}. Hart Publishing.
		
		\item[] Ventura, J. (2005). \textit{Law For dummies} (2nd ed.). Wiley.
		
		\item[] Wacks, R. (2015). \textit{Law: A very short introduction}. Oxford University Press.
		
		\item[] Zeno-Zencovich, V. (2018). Liability for data loss, in: Mak et al. (2018), pp. 39–54.
		
	\end{itemize}
\end{latin}