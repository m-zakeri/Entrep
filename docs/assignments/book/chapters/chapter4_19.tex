% =================================================================
% فصل ۱۹: مسائل مربوط به مسئولیت و قرارداد در داده‌ها (نسخه نهایی)
% =================================================================

\clearpage

% -----------------------------------------------------------------
% تنظیمات حیاتی شماره‌گذاری (حل مشکل شماره‌گذاری ۱۹.۱)
% -----------------------------------------------------------------
\setcounter{section}{19}
\setcounter{subsection}{0}
\renewcommand{\thesubsection}{19.\arabic{subsection}}

% -----------------------------------------------------------------
% بلوک عنوان و نویسنده
% -----------------------------------------------------------------
\noindent
\fcolorbox{black}{gray!15}{%
	\begin{minipage}{\dimexpr\linewidth-2\fboxsep-2\fboxrule}
		\vspace{0.5cm}
		\begin{center}
			\textbf{\huge مسائل مربوط به مسئولیت و قرارداد در داده‌ها}
			\vspace{0.4cm}
			
			\large
			\textit{اریک تیجونگ تین تای} \\
			\lr{\textit{Eric Tjong Tjin Tai}}
		\end{center}
		\vspace{0.5cm}
	\end{minipage}
}

\vspace{0.8cm}

% -----------------------------------------------------------------
% فهرست مطالب فصل ۱۹
% -----------------------------------------------------------------
\noindent
\textbf{\Large فهرست مطالب}
\vspace{0.3cm}

{ \small
	\noindent
	\textbf{۱۹.۱} \hspace{0.3cm} \textbf{مقدمه} \par
	\vspace{0.2cm}
	
	\noindent
	\textbf{۱۹.۲} \hspace{0.3cm} \textbf{ویژگی‌های کلی حقوق خصوصی} \par
	\vspace{0.2cm}
	
	\noindent
	\textbf{۱۹.۳} \hspace{0.3cm} \textbf{داده چیست؟} \par
	\vspace{0.2cm}
	
	\noindent
	\textbf{۱۹.۴} \hspace{0.3cm} \textbf{قراردادها و داده‌ها} \par
	\vspace{0.1cm}
	\hspace{0.8cm} ۱۹.۴.۱ \hspace{0.1cm} تشکیل قراردادها \par
	\hspace{0.8cm} ۱۹.۴.۲ \hspace{0.1cm} محتوای قراردادها \par
	\hspace{0.8cm} ۱۹.۴.۳ \hspace{0.1cm} جبران خسارت‌های قراردادی \par
	\vspace{0.2cm}
	
	\noindent
	\textbf{۱۹.۵} \hspace{0.3cm} \textbf{حقوق مسئولیت مدنی و داده‌ها} \par
	\vspace{0.1cm}
	\hspace{0.8cm} ۱۹.۵.۱ \hspace{0.1cm} مسئولیت مبتنی بر تقصیر \par
	\hspace{0.8cm} ۱۹.۵.۲ \hspace{0.1cm} مسئولیت محض \par
	\hspace{0.8cm} ۱۹.۵.۳ \hspace{0.1cm} رابطه سبیت و دفاعیات \par
	\hspace{0.8cm} ۱۹.۵.۴ \hspace{0.1cm} خسارات و سایر جبران‌ها در مسئولیت مدنی \par
	\vspace{0.3cm}
	
	\noindent
	\textbf{منابع}
}

\vspace{0.8cm}

% -----------------------------------------------------------------
% اهداف یادگیری
% -----------------------------------------------------------------
\noindent
\fcolorbox{black}{gray!15}{%
	\begin{minipage}{\dimexpr\linewidth-2\fboxsep-2\fboxrule}
		\vspace{0.3cm}
		\textbf{\large اهداف یادگیری}
		
		\begin{itemize}
			\setlength\itemsep{0.5em}
			\item[\textbf{--}] درک چگونگی عملکرد قوانین حقوق خصوصی.
			\item[\textbf{--}] درک معانی مختلف «داده» در قانون.
			\item[\textbf{--}] ارزیابی یک قرارداد و شناسایی بندهای مهم قراردادی.
			\item[\textbf{--}] درک مفهوم زیان اقتصادی محض و ارتباط آن با داده‌ها.
			\item[\textbf{--}] آشنایی با مهم‌ترین مبانی مسئولیت، به ویژه موارد مرتبط با مسائل داده.
		\end{itemize}
		\vspace{0.2cm}
	\end{minipage}
}

\newpage

% =================================================================
% ۱۹.۱ مقدمه
% =================================================================
\subsection{مقدمه}
\label{sec:19-introduction}

این فصل مقدمه‌ای بر حوزه‌های خاصی از قانون را تا جایی که برای دانشمندان داده مرتبط است، فراهم می‌کند. اگر به عنوان یک دانشمند داده کار می‌کنید، ممکن است با سوالات حقوقی روبرو شوید. ممکن است نیاز به مذاکره در مورد یک قرارداد داشته باشید یا ممکن است نگران مسئولیت احتمالی باشید. یک مقدمه معمولی بر حقوق مانند آنچه توسط ونتورا (۲۰۰۵) یا وکس (۲۰۱۵) ارائه شده است، تنها کمک محدودی خواهد کرد، زیرا داده‌ها مشکلات حقوقی خاصی را ایجاد می‌کنند که ادبیات عمومی به آن‌ها پاسخ نخواهد داد \lr{(Mak et al. 2018)}.

هدف این فصل تجهیز شما به دانش پایه حقوق قراردادها و حقوق مسئولیت مدنی است که باید شما را با اصول اساسی درگیر آشنا کند. همچنین، این فصل حاوی نکاتی برای اجتناب از دام‌های احتمالی است. از آنجا که این تنها یک مقدمه کوتاه است، امکان ورود به قواعد دقیقی که ممکن است در پرونده‌های واقعی اعمال شوند وجود ندارد. در صورت تردید، با یک وکیل مشورت کنید.

ابتدا با یک مثال کوتاه و چند نکته کلی شروع خواهیم کرد. به دنبال آن تحلیلی حقوقی از اینکه داده چیست ارائه می‌شود. متعاقباً بحثی در مورد حقوق قراردادها با جزئیات بیشتر خواهد آمد. در نهایت، مسئولیت در حقوق شبه‌جرم (مسئولیت مدنی) مورد بحث قرار می‌گیرد.

\vspace{0.4cm}

% کادر مثال
\begin{center}
	\colorbox{gray!10}{%
		\begin{minipage}{0.9\linewidth}
			\vspace{0.2cm}
			\noindent
			$\blacktriangleright$ \textbf{مثال}
			
			\vspace{0.2cm}
			آلیس کسب‌وکاری را راه‌اندازی کرده است که تحلیل پروفایل‌های مشتریان را برای شرکت‌های بزرگ ارائه می‌دهد. باب، آلیس را استخدام می‌کند تا داده‌های کسب‌وکارش را تحلیل کند. تحلیل توسط «ایو»، کارمند آلیس، انجام می‌شود. پس از تکمیل تحلیل، ایو به طور تصادفی پایگاه داده باب را حذف می‌کند و باب نسخه پشتیبان \lr{(backup)} ندارد.
			
			آیا آلیس باید برای از دست رفتن پایگاه داده به باب خسارت بپردازد و اگر بله، چه مقدار؟ $\blacktriangleleft$
			\vspace{0.2cm}
		\end{minipage}
	}
\end{center}

% =================================================================
% ۱۹.۲ ویژگی‌های کلی حقوق خصوصی
% =================================================================
\subsection{ویژگی‌های کلی حقوق خصوصی}
\label{sec:19-private-law}

همان‌طور که از مثال پیداست، سوالات مختلفی مطرح می‌شود. قانون، به ویژه حوزه‌ای که «حقوق خصوصی» نامیده می‌شود، به این سوالات می‌پردازد. حقوق خصوصی بخشی از قانون است که دعاوی و روابط بین افراد خصوصی را پوشش می‌دهد: دو بخش عمده حقوق خصوصی عبارتند از «حقوق قراردادها» و «حقوق مسئولیت مدنی» (که با مسئولیت سروکار دارد).

اگر شما ادعا یا اختلاف دیگری در مورد حقوق خصوصی داشته باشید، در نهایت می‌توانید برای دریافت تصمیم در مورد اختلاف به دادگاه مراجعه کنید. دادگاه ممکن است حکم به نفع ادعای شما صادر کند که منجر به یک جبران خسارت \lr{(remedy)} می‌شود (نگاه کنید به بخش ۱۹.۴.۳).

در حقوق خصوصی، ما بین مواردی که بین طرفین «قرارداد» وجود دارد، و مواردی که ادعای مسئولیت وجود دارد در حالی که هیچ قراردادی بین قربانی و شخصی که ادعا می‌شود عمل اشتباهی انجام داده (مرتکب شبه‌جرم) وجود ندارد، تمایز قائل می‌شویم. نوع اول پرونده توسط حقوق قراردادها (بخش ۱۹.۴) و نوع دوم توسط حقوق مسئولیت مدنی یا مسئولیت ناشی از جرم (بخش ۱۹.۵) اداره می‌شود.

حقوق خصوصی، تا آنجا که در اینجا مرتبط است، متشکل از قواعد (و استثنائات) است که تعیین می‌کند آیا یک پیامد حقوقی خاص وجود دارد یا خیر. یک تحلیل حقوقی معمولاً شامل ابتدا یافتن قواعد حقوقی مرتبط و سپس ارزیابی این است که آیا این قواعد بر حقایق پرونده اعمال می‌شوند و نتیجه چیست.

برای مثال، یک قرارداد اگر «ایجاب» \lr{(offer)} و «قبول» \lr{(acceptance)} وجود داشته باشد، تشکیل می‌شود. اما اگر نقصی در اراده مانند «اشتباه» وجود داشته باشد، قرارداد، اگرچه از نظر شکلی معتبر است، می‌تواند باطل شود (بخش ۱۹.۴.۱). علاوه بر این، اگر قرارداد معتبر باشد، اما تعهدی از قرارداد نقض شود (بخش ۱۹.۴.۳)، طلبکار (شخصی که حق دارد تعهد برایش اجرا شود) ممکن است ادعای خسارت کند (بخش ۱۹.۴.۳).

در زبان برنامه‌نویسی (شبه‌کد)، ساختار و رابطه بین چنین قواعدی می‌تواند به صورت زیر بیان شود (به عنوان مثال):

\begin{latin}
	\begin{verbatim}
		if (offer & acceptance) { contract.valid()}
		if contract.mistake == TRUE { contract.invalid()
			if contract.valid() { 
				if contract.breached(case) {
					/* further conditions */
					creditor.money+= contract.breached.damages(case)
				}
			}
		\end{verbatim}
	\end{latin}
	
	این نشان می‌دهد که چگونه، بسته به چندین شرط، نتیجه ممکن است این باشد که یک قرارداد معتبر یا نامعتبر است، و اینکه طلبکار ممکن است حق دریافت خسارت را داشته باشد. قواعد واقعی قانون بسیار پیچیده‌تر از آن چیزی است که این مثال نشان می‌دهد، اما حداقل ممکن است ایده‌ای به شما بدهد که قواعد حقوقی چگونه عمل می‌کنند.
	
	اطلاعات بیشتری برای جلوگیری از سوءتفاهم مورد نیاز است. اول از همه، حقوق خصوصی تعهداتی را بر افراد انسانی (اشخاص حقیقی) و همچنین بر شرکت‌ها و سایر سازمان‌ها تحمیل می‌کند. با چنین اشخاص حقوقی عموماً به همان روش افراد انسانی رفتار می‌شود: آن‌ها می‌توانند قرارداد منعقد کنند و ممکن است در مسئولیت مدنی مسئول شناخته شوند. قواعد خاصی بر اشخاص حقوقی حاکم است که موضوع حقوق تجارت است و در اینجا به آن پرداخته نمی‌شود. در عمل، اشخاص حقوقی توسط نمایندگان و پرسنل مجاز (مانند مدیرعامل یا مدیر اجرایی) نمایندگی می‌شوند.
	
	ثانیاً، باید آگاه باشید که حقوق خصوصی «محلی» و «وابسته به زمان» است: این حقوق عمدتاً ملی است و ممکن است در طول زمان تغییر کند. می‌توانیم قیاسی انجام دهیم با اینکه چگونه برنامه‌های کامپیوتری در یک نسخه خاص از یک زبان برنامه‌نویسی خاص و سیستم‌عامل ایجاد می‌شوند. شما باید بدانید محیطی که برنامه در آن اجرا خواهد شد چیست: حتی اگر یک برنامه ممکن است در چندین نسخه متفاوت اجرا شود، تضمینی نیست که در نسخه‌ای که برای آن نوشته یا آزمایش نشده است اجرا شود.
	
	به طور مشابه، وکلا تنها می‌توانند پاسخ‌های دقیقی در مورد قانون در رابطه با سیستم حقوقی که اعمال می‌شود ارائه دهند.\footnote{این موضوع توسط آنچه «حقوق بین‌الملل خصوصی» یا «تعارض قوانین» نامیده می‌شود، اداره می‌گردد.} با این حال، توصیف خطوط کلی قانون که در اکثر سیستم‌ها قابل اجراست امکان‌پذیر است، شبیه به اینکه چگونه می‌توانید یک الگوریتم را در شبه‌کد توصیف کنید و از جزئیات زبان‌های برنامه‌نویسی واقعی انتزاع کنید. در این فصل، ما از چنین رویکرد انتزاعی به قانون استفاده می‌کنیم.
	
	یک تمایز مهم که نمی‌توانیم از آن انتزاع کنیم، تمایز بین کشورهایی\footnote{یا به طور خاص‌تر، بخش‌هایی از یک کشور که سیستم یکسانی دارند: به آن‌ها نیز حوزه‌های قضایی \lr{(jurisdictions)} گفته می‌شود.} است که دارای سیستم «کامن لا» \lr{(common law)} هستند و کشورهایی که سیستم «حقوق مدنی» \lr{(civil law)} دارند.
	
	حوزه‌های قضایی کامن لا عبارتند از انگلستان و ولز (نه بریتانیا، زیرا اسکاتلند سیستم حقوقی متفاوتی دارد)، ایالات متحده و مستعمرات سابق انگلیس (اکثراً بخشی از کشورهای مشترک‌المنافع). اکثر کشورهای دیگر دارای سیستم‌های حقوق مدنی هستند\footnote{همچنین برخی استثنائات وجود دارد که در هیچ یک از این دسته‌ها قرار نمی‌گیرند، به ویژه سیستم‌های ترکیبی مانند آفریقای جنوبی که عناصری از کامن لا و حقوق مدنی و/یا سایر سیستم‌ها را دارند.}: آن‌ها دارای یک «کد» (قانون مدون) هستند، یک قانون مکتوب که اکثر قواعد حقوق قراردادها و حقوق مسئولیت مدنی را جمع‌آوری می‌کند.
	
	کامن لا ویژگی‌هایی دارد که به طور قابل توجهی از قواعد کشورهای حقوق مدنی متفاوت است: ما چند مثال را در زیر بحث خواهیم کرد. به طور کلی، کامن لا بر تشریفات و معنای تحت‌اللفظی قراردادها تأکید دارد و طرفین را مسئول تدوین قرارداد برای بیان دقیق آنچه می‌خواهند می‌داند. سیستم‌های حقوق مدنی تمایل دارند بر قصد واقعی طرفین تأکید کنند و به دادگاه‌ها آزادی بیشتری برای تفسیر قرارداد می‌دهند.
	
	% =================================================================
	% ۱۹.۳ داده چیست؟
	% =================================================================
	\subsection{داده چیست؟}
	\label{sec:19-what-is-data}
	
	قبل از اینکه بتوانیم جنبه‌های قراردادی داده‌ها و مسئولیت برای داده‌ها را بحث کنیم، باید مطمئن شویم که می‌فهمیم داده واقعاً چیست، هم در واقعیت و هم در قانون.
	
	به عنوان اولین رویکرد، می‌توانید در نظر بگیرید که مردم واقعاً چگونه با داده‌ها کار می‌کنند. داده‌ها ممکن است در قالب اسناد واژه‌پرداز به عنوان پیوست ایمیل، فایل‌های موسیقی و عکس‌های دیجیتال آپلود شده در پایگاه‌های داده ابری استفاده شوند. از نظر فنی، همه این‌ها فایل‌های داده هستند. علاوه بر این، داده با اطلاعات موجود در چنین فایل‌ها و پایگاه‌های داده نیز شناخته می‌شود، مانند زمانی که از «داده‌های شخصی» صحبت می‌کنیم.
	
	در نهایت، عبارت «کلان‌داده» \lr{(big data)} مد شده است. کلان‌داده نه آنقدر به یک پایگاه داده یا فایل کاملاً تعریف شده، بلکه به یک سیستم در جریان اشاره دارد که در آن داده‌ها به طور مداوم دریافت و پردازش می‌شوند. برای فعال نگه داشتن چنین سیستمی، یک سازمان نیاز به امکانات فنی (سیستم‌های مدیریت پایگاه داده، سرورها)، خدمات (تغذیه مداوم داده‌ها) و منابع انسانی (دانشمندان داده، کارکنان پشتیبانی \lr{IT}) دارد که همه آن‌ها نیاز به پشتیبانی حقوقی (قراردادهای لایسنس، قراردادهای استخدام) دارند. این موضوع در شکل ۱۹.۱ نشان داده شده است.
	
	% -----------------------------------------------------------------
	% جایگزینی تصویر با کد لاتک (بدون نیاز به فایل عکس)
	% -----------------------------------------------------------------
	\vspace{0.5cm}
	\begin{figure}[h!]
		\centering
		% رسم نمودار ساده با fbox و minipage
		\setlength{\fboxsep}{10pt}
		\fbox{
			\begin{minipage}{0.7\linewidth}
				\centering
				\textbf{منابع داده} \\
				\small (شبکه‌های اجتماعی، سنسورها، وب) \\
				\vspace{0.2cm}
				$\Bigg\downarrow$ \\ % فلش به پایین
				\vspace{0.2cm}
				\textbf{زیرساخت و ذخیره‌سازی} \\
				\small (سرورها، پایگاه داده ابری) \\
				\vspace{0.2cm}
				$\Bigg\downarrow$ \\ % فلش به پایین
				\vspace{0.2cm}
				\textbf{پردازش و تحلیل} \\
				\small (دانشمندان داده، الگوریتم‌ها)
			\end{minipage}
		}
		\caption{نمایش گرافیکی جریان‌های داده (بین منابع، ذخیره‌سازی و پردازش). منبع: شکل متعلق به نویسنده.}
		\label{fig:19-1-data-flows}
	\end{figure}
	\vspace{0.5cm}
	
	این سه شکل از داده، یعنی (۱) اطلاعات، (۲) فایل‌های داده، و (۳) کلان‌داده، منجر به مسائل حقوقی مختلفی می‌شوند. در ادامه، من بر دو شکل اول تمرکز خواهم کرد.
	
	اطلاعات به خودی خود موضوع قانون نیست. اطلاعات ممکن است منجر به مسئولیت شود و امکان قرارداد بستن در مورد اطلاعات وجود دارد. در ادامه چند مثال خواهیم دید. اما اطلاعات به عنوانِ «خود» \lr{(as such)} یک شیء قانونیِ به رسمیت شناخته شده نیست. این یک شیء ملموس نیست و به خودی خود توسط قانون محافظت نمی‌شود.
	
	ممکن است اطلاعات تا حدی توسط یک حق مالکیت فکری محافظت شود (مثلاً چون دارای کپی‌رایت است)، و قانون اسرار تجاری نیز به شرطی که اطلاعات دارای ارزش تجاری و محرمانه باشد، حفاظت‌هایی را فراهم می‌کند (نگاه کنید به فصل ۱۸ که به طور گسترده در مورد حقوق مالکیت فکری بحث می‌کند). اما اطلاعات به خودی خود محافظت نمی‌شود.
	
	فایل‌های داده نیاز به توجه کمی بیشتر دارند. بسیاری از وکلا تمایل دارند فایل‌های داده را با اطلاعات برابر بدانند و به طور مشابه از اعطای حفاظت خاص به فایل‌های داده خودداری می‌کنند. در واقع، حقوق مالکیت فکری و حقوق پایگاه داده ممکن است پایگاه‌های داده را پوشش دهند، اما آن‌ها از «فایل‌های داده» به عنوانِ خود محافظت نمی‌کنند: آن‌ها فقط کپی‌برداری و توزیع شیء محافظت شده را ممنوع می‌کنند. اسرار تجاری اغلب داده‌های تجاری مرتبط را پوشش می‌دهند (نگاه کنید به فصل ۲۵).
	
	با این حال، یک چیز توسط حقوق مالکیت فکری محافظت نمی‌شود. این «کنترل واقعی» بر یک فایل داده است. غیر وکلا خیلی راحت در مورد اینکه داده‌ها «مالکیت» کسی هستند یا کسی «مالک» داده‌هاست صحبت می‌کنند. وکلا در انجام این کار مردد هستند.
	
	این به این دلیل است که داده‌ها فاقد ویژگی‌هایی هستند که برای اشیاء عادیِ مالکیت، به ویژه کالاهای ملموس مانند اتومبیل، کتاب و گلدان مشترک است. فایل‌های داده غیرملموس هستند و فایل‌های داده انحصاری نیستند: شما می‌توانید یک کپی تهیه کنید و از این کپی بدون ایجاد مزاحمت برای «مالک» فایل داده اصلی استفاده کنید.\footnote{وکلا از اصطلاح فنی «رقابت‌پذیر» \lr{(rivalrous)} برای نشان دادن ماهیت انحصاریِ مالکیت کالاهای ملموس استفاده می‌کنند.}
	
	در قانون، معمولاً تنها اشیاء ملموس توسط حقوق مالکیت محافظت می‌شوند که به شما حق می‌دهد در تصرف شیء بازگردانده شوید. غیرملموس‌ها معمولاً تنها از دیدگاه مالکیت فکری درک می‌شوند. همان‌طور که در فصل ۱۸ دیدیم، حقوق مالکیت فکری مانند حقوق مالکیت عادی عمل نمی‌کنند زیرا آن‌ها فقط در برابر نقض محافظت می‌کنند؛ آن‌ها حقی برای بازیابی کنترل شیء خود (چون آن یک خلق غیرمادی است) نمی‌دهند. برای مثال، شما نمی‌توانید آهنگ \lr{Yesterday} را بدزدید. حتی اگر آن را سرقت ادبی کنید، موسیقی و متن ترانه برای دیگران در دسترس باقی می‌ماند. شما نمی‌توانید اطلاعات را به عنوانِ خود کنترل کنید.
	
	با این وجود، در مورد فایل‌های داده، نوعی کنترل واقعی وجود دارد، صرفاً به این دلیل که شما مانع دسترسی دیگران به فایل داده می‌شوید. این شکل از کنترل بر غیرملموس‌ها جدید است. در گذشته، شما فقط می‌توانستید اطلاعات را در مغز خود کنترل کنید، اما با فایل‌های داده، کنترل اطلاعات خارجی امکان‌پذیر است.
	
	فایل داده شکل خاصی از اطلاعات است، درست همان‌طور که پرینتِ داده‌ها یک شکل خاص است (که توسط قانون مالکیت محافظت می‌شود، زیرا یک دسته کاغذ یک کالای ملموس است). برابر دانستن فایل داده با اطلاعات به عنوانِ خود نادرست است: یک فایل داده مزایایی را نسبت به اطلاعات صرفاً انتزاعی فارغ از شکل آن فراهم می‌کند.
	
	این موضوع اگر عمل اسکن کردن یک کتاب کاغذی و انجام تشخیص متن \lr{(OCR)} روی اسکن را در نظر بگیریم، به وضوح آشکار می‌شود: اگرچه فایل متنی حاصل نباید حاوی اطلاعات بیشتر یا متفاوتی نسبت به کتاب باشد، اما فایل متنی ممکن است برای اهدافی مانند تحلیل داده‌ها به شیوه‌ای مفید باشد که کتاب کاغذی نیست.
	
	در حال حاضر، قانون در اکثر کشورها از کنترل بر فایل‌های داده تا حد مشخصی محافظت می‌کند، و در برخی موارد حتی به شما اجازه می‌دهد تا بازگشت یک فایل داده که دزدیده شده است را مطالبه کنید. این حفاظتی شبیه به حقوق مالکیت برای داده‌ها فراهم می‌کند. با این حال، این موضوع در همه کشورها صدق نمی‌کند \lr{(Tjong Tjin Tai, 2018b)}. سایر حقوق مربوط به مالکیت (برای مثال حقوق وثیقه، مانند رهن و گروگذاری) برای فایل‌های داده به سختی قابل تنظیم هستند.
	
	با توجه به موقعیت نامشخص فایل‌های داده در قانون، حفاظت از فایل‌های داده عمدتاً غیرمستقیم است: زیرا مداخله در داده‌ها یک «شبه‌جرم» \lr{(tort)} است، یا زیرا شما تعهدات قراردادی را برای رفتار صحیح با داده‌ها تنظیم کرده‌اید.
	
	% =================================================================
	% ۱۹.۴ قراردادها و داده‌ها
	% =================================================================
	\subsection{قراردادها و داده‌ها}
	\label{sec:19-contracts-and-data}
	
	قرارداد چیست؟ به زبان ساده، قرارداد توافقی بین دو نفر است که هر کدام تعهدات خاصی را در قبال طرف دیگر بر عهده می‌گیرند.\footnote{همچنین ممکن است بیش از دو طرف در یک قرارداد وجود داشته باشند؛ ما در مورد آن‌ها بحث نخواهیم کرد.} مثال کلاسیک قرارداد بیع (فروش) است: خریدار باید قیمت خرید را بپردازد، و فروشنده موظف است شیء فروخته شده را به خریدار تحویل دهد.
	
	هنگام در نظر گرفتن یک قرارداد، مهم است که بین کاغذ (یا فایل دیجیتال) که اثبات قرارداد را فراهم می‌کند و رابطه حقوقی قراردادی بین طرفین (که ناشی از عمل امضا است، که پذیرش را ثابت می‌کند) تمایز قائل شویم. کاغذ امضا شده نیز «قرارداد» نامیده می‌شود. با این حال، اگر کاغذ گم شود، خودِ قرارداد (رابطه حقوقی) همچنان وجود خواهد داشت و معتبر خواهد بود. در اینجا، ما بر رابطه حقوقی حاصل تمرکز می‌کنیم. مقدمات عمومی عبارتند از: \lr{(Bix 2012, Cartwright 2013, Smits 2014)}.
	
	موقعیت‌هایی وجود دارد که فوراً مشخص نیست آیا قراردادی منعقد شده است یا خیر. اینکه آیا واقعاً قراردادی وجود دارد ممکن است به قانون حاکم و سایر شرایط نیز بستگی داشته باشد.
	
	یک مثال، «شرایط و ضوابط» \lr{(T\&C)} است که بر دسترسی و استفاده از یک وب‌سایت حاکم است: مسلماً این یک قرارداد است زیرا به نظر می‌رسد شما را ملزم می‌کند که هنگام دسترسی به وب‌سایت به این شرایط پایبند باشید، و در عوض شما دسترسی پیدا می‌کنید. اما \lr{T\&C} همچنین می‌تواند صرفاً به عنوان شرایطی تفسیر شود که تحت آن به شما اجازه دسترسی داده می‌شود، که تعهدات اضافی ایجاد نمی‌کنند.
	
	تمایز بین این دو دیدگاه زمانی روشن می‌شود که برای مثال، \lr{T\&C} جریمه‌ای را تعیین کند اگر شما یک نقد منفی از وب‌سایت در جای دیگری منتشر کنید. در تفسیر اول، این می‌تواند یک شرط الزام‌آور باشد؛ در شکل دوم اینطور نیست، زیرا شرطِ دسترسی نیست بلکه یک تعهد اضافی است که اگر قراردادی وجود نداشته باشد نمی‌تواند تحمیل شود. ما این مسائل را بیشتر بحث نخواهیم کرد زیرا منجر به بحث‌های حقوقی پیچیده می‌شود و به آسانی قابل توضیح یا حل نیست.
	
	برای داده‌ها، «مجوز» \lr{(License)} اهمیت ویژه‌ای دارد. این اصطلاح در درجه اول به اجازه استفاده از مالکیت فکری کسی اشاره دارد اما با تعمیم برای اجازه استفاده از داده‌ها نیز استفاده می‌شود. یک مجوز می‌تواند بخشی از یک توافق گسترده‌تر باشد، و می‌تواند موضوع یک «فروش» باشد، برای مثال زمانی که شما یک اپلیکیشن برای تلفن هوشمند خود «می‌خرید».
	
	چنین قراردادی شامل حقی برای دریافت فایل برنامه اپلیکیشن، و یک مجوز (حق استفاده) برای اپلیکیشن است. شما به هر دوی این عناصر نیاز دارید. ممکن است شما مجوز داشته باشید در حالی که فایل را گم کرده‌اید (چون تلفن هوشمند جدیدی دارید)، که در این صورت حق خود را برای دریافت مجدد فایل برنامه اعمال خواهید کرد. برعکس، اگر شما یک کپی غیرقانونی از فایل برنامه به دست آورده باشید، طبیعتاً مجاز به استفاده از آن نیستید: شما نیاز به اجازه، یعنی یک مجوز دارید.
	
	% =================================================================
	% ۱۹.۴.۱ تشکیل قراردادها
	% =================================================================
	\subsubsection{تشکیل قراردادها}
	
	قراردادها با «ایجاب» \lr{(offer)} و «قبول» \lr{(acceptance)} منعقد می‌شوند. یک طرف ایجابی برای یک قرارداد ارائه می‌دهد و طرف دیگر آن ایجاب را می‌پذیرد.\footnote{لازم نیست طرفین یک ایجاب صریح و روشن ارائه دهند که متعاقباً به عنوان یک عمل جداگانه پذیرفته شود؛ کافی است طرفین به توافق متقابل در مورد قرارداد برسند، مثلاً با تدوین قرارداد با هم و متعاقباً امضای آن.}
	
	در اکثر موارد، الزامات رسمی وجود ندارد: یک قرارداد می‌تواند به صورت کتبی، دیجیتالی یا شفاهی منعقد شود؛ رضایت به قرارداد می‌تواند از اعمال (مانند بالا بردن دست) یا حتی (در موقعیت‌های خاص) از سکوت (رضایت ضمنی) استنتاج شود. با این حال، برای برخی از انواع قراردادها، تشریفات اضافی وجود دارد. مثالی از آن توافقات پیش از ازدواج است که عموماً نیاز به نوعی کمک حقوقی (توسط سردفتر اسناد رسمی یا سایر ارائه‌دهندگان خدمات حقوقی) دارد تا اطمینان حاصل شود که رضایت آزادانه و آگاهانه داده شده است.
	
	در «کامن لا» (حقوق عرفی)، یک الزام اضافی برای یک قرارداد معتبر وجود دارد: «عوض» \lr{(consideration)}. این بدان معناست که باید نوعی عملکرد متقابل در ازای آن وجود داشته باشد. اگر یک طرف به طور یک‌جانبه تعهداتی را بر عهده بگیرد، بدون اینکه چیزی در ازای آن دریافت کند، این یک قرارداد معتبر تلقی نمی‌شود.
	
	الزام «عوض» مسلماً می‌تواند منجر به مشکلاتی در مورد نرم‌افزارهای متن‌باز \lr{(open-source)} شود. چنین نرم‌افزاری به صورت رایگان ارائه می‌شود، اما ممکن است طرفی را که از نرم‌افزار استفاده می‌کند ملزم به پذیرش محدودیت‌های خاصی در استفاده از نرم‌افزار کند، به ویژه ممکن است تعهدی را تحمیل کند که هر تغییری که در نرم‌افزار متن‌باز ایجاد می‌شود، تحت همان مجوز متن‌باز در دسترس عموم قرار گیرد.\footnote{این نسخه به‌اصطلاح «کپی‌لفت» \lr{(copyleft)} از مجوزهای متن‌باز است. نسخه‌های دیگری نیز وجود دارد.}
	
	با این حال، دادگاه‌ها پذیرفته‌اند که استفاده از نرم‌افزار متن‌باز نیز شامل نوعی «عوض» به معنای انطباق با الزامات مجوز است \lr{(Jacobsen v. Katzer, 2008)}. بنابراین، پذیرش مجوز متن‌باز ممکن است یک قرارداد معتبر تحت کامن لا ایجاد کند.
	
	اگرچه قراردادها عموماً تشریفات دیگری ندارند، ممکن است تعهدات اطلاعاتی وجود داشته باشد که یک طرف را ملزم کند هنگام ارائه ایجاب برای قرارداد، به طرف دیگر به اندازه کافی اطلاع‌رسانی کند. نمونه‌هایی از آن هویت و محل کسب‌وکار هنگام قرارداد بستن در اینترنت است (برای موارد اروپایی).
	
	یک مسئله خاص در حقوق قراردادها، این احتمال است که قراردادی که معتبر به نظر می‌رسد، بعداً معلوم شود که در واقع نامعتبر است. دو دلیل اصلی این است که «نقص در اراده» \lr{(defect of will)} وجود دارد (رضایت به روش مناسب شکل نگرفته است، مثلاً با تهدید به خشونت) یا اینکه قرارداد «نظم عمومی» \lr{(public policy)} را نقض می‌کند (مانند قرارداد بستن با یک آدمکش). قراردادهای خلاف نظم عمومی باطل و بی‌اثر هستند (هیچ اثر الزام‌آوری ندارند؛ گویی هرگز منعقد نشده‌اند).
	
	در عمل، مهم‌ترین نقص اراده در صورت سوءتفاهم بین طرفین ایجاد می‌شود. در سیستم‌های «حقوق مدنی» \lr{(civil law)}، این ممکن است ناشی از «اشتباه» \lr{(mistake)} باشد (سوءتفاهم بین طرفین در مورد موضوع قرارداد، مثلاً اینکه آیا یک تلفن هوشمند کاملاً نو است یا یک مدل بازسازی شده \lr{refurbished}). این دلالت بر این دارد که طرفین باید اطلاعات مرتبط را به ابتکار خود فاش کنند، و اگر طرفی در انجام این کار کوتاهی کند، طرف دیگر ممکن است به اشتباه استناد کند.
	
	در کشورهای «کامن لا»، سازه اولیه «تدلیس» \lr{(misrepresentation)} است: یکی از طرفین در قرارداد اظهار داشته (یعنی بیان کرده که چیزی چنین است) که رایانه نو است و بازسازی شده نیست. اگر معلوم شود که این درست نیست، طرف دیگر بر اساس تدلیس ادعایی دارد. این دلالت بر این دارد که طرفین باید ابتکار عمل را به دست بگیرند و در مورد ویژگی‌هایی که مهم می‌دانند سوال کنند.
	
	پیامد اشتباه و تدلیس این است که طرف زیان‌دیده می‌تواند قرارداد را فسخ \lr{(annul)} کند، که این اثر را دارد که طرفین به وضعیت قبل از انعقاد قرارداد بازگردانده می‌شوند و هر آنچه که قبلاً انجام شده است باید خنثی (بازگردانده) شود.
	
	% =================================================================
	% ۱۹.۴.۲ محتوای قراردادها
	% =================================================================
	\subsubsection{محتوای قراردادها}
	
	در یک قرارداد، می‌توانید مشخص کنید که طرفین چه کاری باید انجام دهند و چه انتظاری می‌توانند از یکدیگر داشته باشند. یک قرارداد معمولاً از چندین ماده یا بند تشکیل شده است که این تعهدات و پیش‌فرض‌ها را توصیف می‌کند و سایر مسائل را نیز تنظیم می‌کند. تفسیر این بندها تمایل دارد بر متن بندها تمرکز کند (زیرا معمولاً فرض می‌شود که طرفین عمداً فرمول‌بندی خاصی را انتخاب کرده‌اند)، اما معنای تحت‌اللفظی ممکن است بر اساس مقاصد طرفین یا سایر شرایط نیز اصلاح شود.\footnote{این به ویژه در کشورهای دارای حقوق مدنی صدق می‌کند.}
	
	یک تمایز مهم بین «تعهدات به وسیله» \lr{(obligations of means)} و «تعهدات به نتیجه» \lr{(obligations of result)} است. اگر موافقت کنید که تحلیل داده انجام دهید، ممکن است قرارداد ببندید که نتیجه منجر به صرفه‌جویی ۱۰٪ در هزینه‌های جاری شود. با این حال، احتمالاً دوست ندارید این کار را انجام دهید، زیرا راهی وجود ندارد که بتوانید مطمئن شوید این اتفاق می‌افتد.
	
	در آن صورت، ترجیح می‌دهید آن را به عنوان یک «تعهد به وسیله» بیان کنید: شما بهترین تلاش خود را برای انجام یک تحلیل پیشرفته به کار خواهید گرفت. این شبیه پزشکی است که تنها باید از مراقبت و مهارت، و تلاش حرفه‌ای استفاده کند تا سعی در بهبود وضعیت بیمار داشته باشد، بدون اینکه تضمین کند بیمار به طور کامل سلامتی خود را باز می‌یابد.
	
	برای دادن اطمینان به طرف دیگر، می‌توانید در مورد برخی تعهدات جانبی عینی‌تر توافق کنید. برای مثال، توافق می‌کنید که از یک روش تحلیل خاص استفاده خواهید کرد و حداقل سه تحلیل یا گزارش مختلف ارائه خواهید داد.
	
	یک قرارداد همچنین می‌تواند موضوعش «فروش» یا صدور مجوز داده‌ها باشد.\footnote{به بیان دقیق، داده‌ها نمی‌توانند فروخته شوند زیرا فروش فقط بر اشیاء ملموس اعمال می‌شود. با این حال، به نظر می‌رسد قانون در حال توسعه است تا معنای «فروش» را به قراردادهای مربوط به محتوای دیجیتال نیز گسترش دهد.} قرارداد می‌تواند حاوی بندهایی در مورد کیفیت داده‌ها باشد (منبع چیست، با چه سطح جزئیاتی/چند بیت، کدام دوره، آیا پیوسته است، یا آیا آپتایم ۹۹.۹۹٪ وجود دارد). ممکن است مربوط به یک مجموعه داده ثابت (مانند دماهای مجموعه‌ای از حسگرها در طول یک سال) یا قراردادی در مورد تغذیه مداوم داده باشد.
	
	مهم است درک کنیم که مجوز داده‌ها ممکن است این اثر را داشته باشد که صاحب امتیاز \lr{(licensee)} استفاده دائم از داده‌ها یا نوعی تغییر شکل داده‌ها را داشته باشد. برای مثال، اگر داده‌ها برای آموزش یک الگوریتم استفاده شوند، به نوعی در الگوریتم گنجانده شده‌اند و نمی‌توانند از آن حذف شوند.
	
	هنگام تنظیم یک مجوز موقت برای چنین استفاده‌ای، منطقی است تصریح شود که طرفین در مورد این تثبیت قطعی داده‌ها توافق دارند، در حالی که قرارداد می‌بندند که پس از مدت مجوز، صاحب امتیاز داده‌های خام را حذف خواهد کرد.
	
	علاوه بر این، اگر یک مجوز ابدی و نامحدود باشد (با حق اعطای مجوزهای فرعی)، این در واقع به منزله تبدیل صاحب امتیاز به «مالک» کپیِ داده‌های خودش است. اگرچه شما «مالک» داده‌های خود باقی می‌مانید (مگر اینکه مجوز انحصاری ارائه دهید، که در واقع صاحب امتیاز را «مالک» جدید واقعی می‌کند)، نمی‌توانید صاحب امتیاز را از انجام هر کاری که خودتان می‌توانید با داده‌ها انجام دهید، بازدارید. این یکی از راه‌هایی است که شرکت‌ها ممکن است اسماً بگویند شما «مالک» باقی می‌مانید در حالی که در واقع از طریق یک مجوز گسترده، اختیاراتی شبیه مالک را به دست می‌آورند.
	
	علاوه بر این، قراردادها اغلب حاوی بندهای زیادی هستند که نوع جبران خسارت‌های در دسترس طرفین را تنظیم می‌کنند. این بندها به ویژه در کامن لا مهم هستند، زیرا قوانین انگلیس و ایالات متحده تنها برای انواع خاصی از بندهای قراردادی، به‌اصطلاح «ضمانت‌ها» \lr{(warranties)} و «شروط» \lr{(terms)}، جبران خسارت ارائه می‌دهند،\footnote{اصطلاحات دقیق در حقوق انگلیس با حقوق ایالات متحده متفاوت است.} در حالی که قرارداد باید تصریح کند کدام جبران خسارت‌ها بر این بندها اعمال می‌شود. در حقوق مدنی، نیازی نیست اینقدر دقیق باشید: دادگاه حقوق مدنی هر تعهدی را که در قرارداد گنجانده شده است، بدون نیاز به فرم دقیق اجرا خواهد کرد.
	
	یک دسته مهم از بندها، «انتخاب قانون» \lr{(choice of law)} و «انتخاب دادگاه» \lr{(choice of forum)} است. یک قرارداد می‌تواند (و اغلب این کار را می‌کند) مشخص کند که کدام قانون اعمال می‌شود و به کدام دادگاه می‌توانید مراجعه کنید. همچنین می‌تواند حاوی بندی باشد که می‌گوید اصلاً نمی‌توانید به دادگاه بروید بلکه (مثلاً) باید پرونده را به داوری ارجاع دهید. انتخاب قانون می‌تواند این اثر را داشته باشد که شما نوعی از حفاظت را که تحت سیستم خودتان داشتید، از دست بدهید.
	
	برخی از انواع دیگر بندها در ادامه (بخش ۱۹.۴.۳) مورد بحث قرار می‌گیرند زیرا مربوط به جبران خسارت‌ها هستند.
	
	قانون برخی کنترل‌ها را بر محتوای قرارداد از طریق دکترین‌های «شروط ناعادلانه» \lr{(unfair terms)} و «شیوه‌های تجاری ناعادلانه» \lr{(unfair commercial practices)} فراهم می‌کند. به زبان ساده، این‌ها شروط کاملاً نامعقول و از نظر اجتماعی غیرقابل قبول را ممنوع می‌کنند و ممکن است این پیامد را داشته باشند که شروطی که به نظر منطقی می‌رسد ضمنی باشند، بخشی از قرارداد در نظر گرفته شوند.\footnote{در کامن لا، این تا حدی با دکترین «شروط ضمنی» \lr{(implied terms)} همپوشانی دارد.} برای مثال، فرض بر این است که یک قرارداد فروش دلالت بر این دارد که شیء فروخته شده برای هدفی که معمولاً دارد مناسب است.\footnote{در ایالات متحده، کد تجاری یکنواخت \lr{(UCC)} بخش ۲-۳۱۵ ضمانت ضمنی مناسب بودن برای یک هدف خاص را کدگذاری می‌کند.}
	
	در حالی که این شکل مهمی از کنترل برای اطمینان از معقول بودن قراردادها است، قواعد حقوقی کاملاً کلی هستند. آن‌ها برای روش‌های خاصی که قراردادهای شامل داده‌ها ممکن است نامعقول یا ناعادلانه باشند به‌روزرسانی نشده‌اند، و همیشه مشخص نیست که آیا می‌توانند برای انطباق با جامعه داده‌ها تفسیر مجدد شوند یا خیر.
	
	دکترین شروط ناعادلانه با دکترین کلی‌تری در کشورهای حقوق مدنی مرتبط است، یعنی اینکه قراردادها فرض می‌شود شامل وظیفه تفسیر و اجرای قرارداد بر اساس «حسن نیت» \lr{(good faith)} باشند: عمل کردن مانند یک فرد معقول، و نه صرفاً تمرکز بر متن تحت‌اللفظی قرارداد. این ابزار همچنین می‌تواند برای پر کردن شکاف‌ها در قرارداد استفاده شود.
	
	یک نقطه ضعف استفاده از حسن نیت این است که می‌تواند منجر به انحراف دادگاه‌ها از مقاصد اصلی طرفین شود. در کامن لا، حسن نیت به طور کلی به عنوان یک ابزار پذیرفته نشده است، و بحثی در مورد اینکه آیا می‌توان از آن برای اصلاح قراردادها استفاده کرد وجود دارد. با این وجود، دادگاه‌های ایالات متحده نیز پذیرفته‌اند که در هر قرارداد، یک پیمان ضمنی حسن نیت و معامله منصفانه وجود دارد.
	
	% =================================================================
	% ۱۹.۴.۳ جبران خسارت‌های قراردادی
	% =================================================================
	\subsubsection{جبران خسارت‌های قراردادی}
	
	اگر قرارداد به درستی اجرا نشود (به این معنی که اگر طرف دیگر کاملاً طبق وعده‌ای که در قرارداد داده بود عمل نکند، یعنی او یک یا چند تعهد خود را نقض کرده باشد)، با اصطلاحاً «نقض قرارداد» \lr{(breach of contract)} مواجه هستیم. پس از نقض، طرفی که تعهد به نفع او بوده است (ما عموماً این طرف را «طلبکار» یا متعهدله می‌نامیم) می‌تواند به یک «جبران خسارت» \lr{(remedy)} متوسل شود. جبران خسارت، پاسخ قانونی است که دادگاه به طلبکار ارائه می‌دهد، که هدف آن ایجاد انگیزه در بدهکار (طرفی که تعهد بر عهده او بوده) برای انجام واقعی تعهدش است.
	
	طلبکار معمولاً باید ثابت کند که بدهکار تعهد را به درستی انجام نداده است. در مورد تعهد به نتیجه، این کار آسان است؛ او فقط باید نشان دهد که نتیجه محقق نشده است. در مورد تعهد به وسیله، این دشوارتر است: او باید نشان دهد که بدهکار مراقبت و مهارتی را که مورد نیاز بوده است، ارائه نکرده است، که معمولاً فقط به طور غیرمستقیم قابل تعیین است.
	
	% -----------------------------------------------------------------
	% ۱۹.۴.۳.۱ پیش‌نیازهای استناد به جبران خسارت
	% -----------------------------------------------------------------
	\vspace{0.4cm}
	\noindent
	\textbf{۱۹.۴.۳.۱ پیش‌نیازهای استناد به جبران خسارت}
	
	قبل از اینکه بتوانید ادعای جبران خسارت کنید، اکثر سیستم‌های حقوقی ابتدا نیاز دارند که بدهکار در وضعیت «قصور» \lr{(default)} باشد. این بدان معناست که او قطعاً تعهد خود را انجام نداده است، و تقصیر اوست که چنین نکرده است. این مستلزم دو شرط است: اخطار قصور و فقدان عذر موجه.
	
	\begin{itemize}
		\setlength\itemsep{0.5em}
		\item[(الف)] \textbf{بدهکار معمولاً باید شانس دومی برای انجام تعهد خود داشته باشد.}
		ممکن است او از عدم اجرا آگاه نبوده و اگر شکایت کنید با میل و رغبت وضعیت را اصلاح کند. برای مثال، اگر شما یک تلفن هوشمند از آمازون سفارش دهید، فروشنده نمی‌داند که آیا شما بسته را دریافت نکرده‌اید. اگر شکایت کنید، او باید فرصتی برای جبران وضعیت با ارسال یک تلفن دیگر داشته باشد.
		
		بنابراین، بسیاری از سیستم‌های حقوقی از شما می‌خواهند ابتدا یک «اخطار قصور» \lr{(notice of default)} ارسال کنید: بیانیه‌ای روشن مبنی بر اینکه نقض تعهد قراردادی صورت گرفته است، و مهلتی که در آن می‌خواهید نقض اصلاح شود. تنها پس از انقضای مهلت است که بدهکار در وضعیت قصور قرار می‌گیرد و شما می‌توانید از دادگاه درخواست جبران خسارت کنید.
		
		در برخی موارد، زمانی که مشخص است بدهکار به فرصتِ جبران نقض پاسخ نخواهد داد، یا زمانی که خود قرارداد یک «مهلت قطعی» \lr{(fatal term)} را مشخص کرده است (لحظه‌ای که تعهد باید قطعاً انجام شود، مانند تاریخ تحویل کیک عروسی)، ارسال اخطار قصور لازم نیست. در عوض، بدهکار بلافاصله پس از گذشت مهلت و عدم انجام تعهد، در وضعیت قصور قرار می‌گیرد.
		
		\item[(ب)] \textbf{حتی اگر تعهدی طبق وعده انجام نشود، ممکن است بدهکار عذر موجهی داشته باشد.}
		علت واقعی نقض ممکن است یک عامل خارجی باشد. برای مثال، یک بسته تحویل داده نشده زیرا کل ساختمان به دلیل حمله تروریستی غیرقابل دسترسی بوده است. در چنین موردی، رسماً نقض قرارداد وجود دارد، اما چون حمله عذر موجهی فراهم می‌کند، بدهکار مقصر نیست و طلبکار ممکن است جبران خسارتی نداشته باشد.
		
		سوال این است: چگونه تعیین می‌کنید که آیا عذر موجهی وجود دارد؟ این معمولاً با نگاه کردن به علت واقعی عدم اجرا یا نقض، و در نظر گرفتن اینکه آیا این علت «قابل انتساب» \lr{(attributable)} به اوست، انجام می‌شود: آیا او مقصر آن علت است، یا مسئولیت اوست. اگر او مقصر نباشد، چنین علتی «قوه قهریه» یا «فورس ماژور» \lr{(force majeure)} تلقی می‌شود.
		
		این یک مسئله کاملاً روشن نیست. برای مثال، اگر یک تولیدکننده نتواند تراشه‌های کامپیوتری را به دلیل اعتصاب در کارخانه‌اش به شما عرضه کند، می‌توانید این را فورس ماژور در نظر بگیرید. با این حال، اگر اعتصاب به این دلیل شروع شده باشد که مدیران با پرسنل خود بدرفتاری کرده‌اند، می‌توانید استدلال کنید که خود آن‌ها مقصر اعتصاب هستند. برای جلوگیری از این نوع بحث‌ها، قراردادها اغلب حاوی یک «بند فورس ماژور» هستند که رویدادهایی را که فورس ماژور تلقی می‌شوند لیست می‌کند. چنین لیستی ممکن است جامع باشد یا ممکن است فقط تمثیلی باشد.
	\end{itemize}
	
	در سیستم‌های حقوق مدنی، عموماً تمام تعهدات قراردادی (که از تفسیر بندهای قرارداد ناشی می‌شوند) زمانی که تعهد به درستی انجام نشود، منجر به جبران خسارت می‌شوند. در سیستم‌های کامن لا، عمدتاً انگلیس و ایالات متحده، تنها بندهای قراردادی خاص ممکن است منجر به نقض قرارداد و جبران خسارت شوند.
	
	در انگلیس، تمایز بین «اظهارات» \lr{(representations)} (که منجر به دعوای تدلیس می‌شوند، نگاه کنید به بخش ۱۹.۴.۱) و «شروط» \lr{(terms)} (که منجر به نقض قرارداد می‌شوند) است. چنین شروطی بیشتر به «شرط‌های اساسی» \lr{(conditions)} (که منجر به فسخ می‌شوند)، «ضمانت‌ها» \lr{(warranties)} (که مبنای خسارت هستند)، و «شروط میانی» (که ممکن است منجر به فسخ و/یا خسارت شوند) تقسیم می‌شوند.
	
	در ایالات متحده، اظهارات نیز منجر به دعوای تدلیس می‌شوند. با این حال، در حالی که در حقوق انگلیس بندی که اظهار نیست تمایل دارد به طور خودکار شرطی باشد که منجر به جبران خسارت شود، در حقوق ایالات متحده، بندی که اظهار نیست به طور خودکار منجر به جبران خسارت نمی‌شود. تنها به‌اصطلاح «ضمانت‌ها» به طلبکار اجازه می‌دهند تا به جبران خسارت متوسل شود. قرارداد اغلب نوع جبران خسارت‌هایی را که به نقض ضمانت‌های خاص متصل هستند، مشخص می‌کند.
	
	بنابراین، چهار الزام احتمالی قبل از اینکه بتوانید به جبران خسارت متوسل شوید وجود دارد:
	\begin{itemize}
		\setlength\itemsep{0.3em}
		\item نقض یک تعهد قراردادی
		\item قصور \lr{(Default)}
		\item قابلیت انتساب (بدون عذر، بدون فورس ماژور)
		\item در کامن لا: تعهد در بندی قرار داده شده باشد که اجازه جبران خسارت می‌دهد (مانند ضمانت یا شرط)
	\end{itemize}
	
	% -----------------------------------------------------------------
	% ۱۹.۴.۳.۲ جبران خسارت‌های موجود
	% -----------------------------------------------------------------
	\vspace{0.4cm}
	\noindent
	\textbf{۱۹.۴.۳.۲ جبران خسارت‌های موجود}
	
	به طور کلی، سه نوع جبران خسارت قراردادی وجود دارد:
	\begin{itemize}
		\setlength\itemsep{0.3em}
		\item اجرای عین تعهد \lr{(Specific performance)}
		\item حکم خسارت \lr{(Award of damages)}
		\item فسخ \lr{(Termination)}
	\end{itemize}
	
	این جبران‌ها می‌توانند از دادگاه مطالبه شوند. فسخ ممکن است، بسته به حوزه قضایی، خارج از دادگاه نیز در دسترس باشد (فسخ با اخطار).
	
	**اجرای عین تعهد** به این معنی است که دادگاه به بدهکار دستور می‌دهد تعهد خود را انجام دهد. برای مثال، شما قرارداد بسته‌اید که یک پایگاه داده بسازید، اما از انجام آن امتناع می‌کنید، و دادگاه متعاقباً به شما دستور می‌دهد آنچه را وعده داده‌اید انجام دهید. در سیستم‌های حقوق مدنی، این جبران اولیه است. در کامن لا، اجرای عین تعهد همیشه در دسترس نیست: بستگی به نوع تعهد دارد. دلیل این نگرش ممکن است این باشد که در کامن لا اجرای عین تعهد اصولاً توسط دادگاه نظارت می‌شود (که وقت دادگاه را می‌گیرد)، زیرا نقض اجرای عین تعهد به منزله «توهین به دادگاه» است. در سیستم‌های حقوق مدنی، اجرای عین تعهد تعهدی به طلبکار است و اغلب از طریق یک «جریمه» خصوصی اجرا می‌شود که طلبکار می‌تواند بدون مزاحمت برای دادگاه به آن متوسل شود.\footnote{در حقوق فرانسه، این «astreinte» است، نوعی جریمه که زمانی که تعهد پس از یک دوره معین هنوز انجام نشده باشد، سررسید می‌شود.}
	
	**حکم خسارت** در اکثر موارد، هم در حقوق مدنی و هم در کامن لا، جبران اولیه است. این بدان معناست که دادگاه به طلبکار خسارت اعطا می‌کند، یعنی بدهکار باید به طلبکار خسارت بپردازد. خسارت \lr{(Damages)} به معنای مبلغی پول است، با هدف جبران پیامدهای منفی نقض برای طلبکار. توجه داشته باشید: خسارت (با s جمع در انگلیسی) مبلغ پول است؛ خسارت/آسیب (بدون s) به معنای زیان یا جراحت واقعی است. مبلغ خسارت برای جبران آسیب در نظر گرفته شده است.
	
	وکلا انواع مختلفی از آسیب را متمایز می‌کنند. یک تمایز مهم در اشکال آسیب بین موارد زیر است:
	\begin{enumerate}
		\item صدمه شخصی \lr{(personal injury)}
		\item خسارت به اموال \lr{(property damage)}
		\item زیان اقتصادی محض \lr{(pure economic loss)}
	\end{enumerate}
	
	صدمه شخصی به معنای آسیبی است که از صدمه به بدن یا سلامتی شخص ناشی می‌شود، مانند هزینه‌های پزشکی و از دست دادن درآمد (به دلیل جراحت).
	خسارت به اموال، صدمه به اموال فیزیکی و پیامدهای این صدمه است. مثال‌ها شامل کاهش ارزش خودروی آسیب‌دیده، هزینه‌های تعمیر خودرو، و همچنین هزینه حمل‌ونقل جایگزین در زمانی که خودرو قابل استفاده نیست می‌باشد.
	
	**زیان اقتصادی محض** شامل تمام انواع دیگر خسارات است: این به معنای زیان‌هایی است که نه از صدمه شخصی ناشی می‌شوند و نه از خسارت به اموال. مثالی از آن یک بیانیه افتراآمیز است: این با جراحت یا خسارت به اموال شروع نمی‌شود، بلکه با یک صدمه غیرمادی شروع می‌شود. مثال‌ها شامل از دست دادن سود، زمان تلف شده و زیان‌های بازار سهام است. همچنین آسیب به یک پایگاه داده یا فایل داده، زیان اقتصادی محض است. بنابراین، یک زیان اقتصادی (مانند از دست دادن درآمد) تنها زمانی زیان اقتصادی محض است که پیامد جراحت یا خسارت به اموال نباشد.
	
	اهمیت این تمایز این است که دو نوع اول آسیب معمولاً قابل جبران هستند، از جمله زیان‌های تبعی ناشی از آسیب یا خسارت اولیه. در مقابل، برخی حوزه‌های قضایی (به ویژه کامن لا) اجازه جبران زیان اقتصادی محض برای نقض قرارداد را نمی‌دهند. بنابراین، حتی اگر شما رسماً جبران خسارتی برای نقض قرارداد داشته باشید، در واقعیت، ممکن است زیان شما با حکم خسارت جبران نشود، زیرا زیان مصداق زیان اقتصادی محض است.
	
	اگر خواهان جبران برای این نوع زیان‌ها هستید، باید بندی برای «وجه التزام» \lr{(liquidated damages)} برای نقض‌های خاص اضافه کنید (در ادامه ببینید).
	
	مثالی که ممکن است تفاوت بین این انواع زیان‌ها را روشن کند: ارائه‌دهنده خدمات \lr{IT} شما به طور تصادفی پایگاه داده شما را حذف می‌کند. از آنجا که حذف ناشی از خسارت به اموال یا صدمه شخصی نیست، این مصداق زیان اقتصادی محض است و ممکن است در انگلیس قابل جبران نباشد. اگر ارائه‌دهنده خدمات اتاق سرور شما را آتش می‌زد، که در آن پایگاه داده و تمام نسخه‌های پشتیبان (روی دیسک‌های سخت) از بین می‌رفتند، از دست رفتن پایگاه داده زیان اقتصادی تبعی بود که ناشی از خسارت به اموال است. این زیان قابل جبران بود. بنابراین صلاحیت به خود زیان بستگی ندارد، بلکه به روشی که زیان رخ داده است بستگی دارد.
	
	**بندهای محدودیت یا معافیت** بندهایی هستند که میزان یا نوع خسارتی را که بدهکار باید پس از نقض جبران کند، محدود می‌کنند. مثالی که اغلب در مجوزهای نرم‌افزار یافت می‌شود این است که تنها خسارت ناشی از صدمه شخصی یا خسارت فیزیکی به اموال جبران می‌شود.\footnote{به عبارت دیگر، جبران خسارت برای زیان اقتصادی محض به صراحت مستثنی شده است.} از آنجا که نرم‌افزار معمولاً باعث صدمه شخصی یا خسارت به اموال نمی‌شود، نتیجه این است که تولیدکننده نرم‌افزار معمولاً مجبور نخواهد بود هیچ خسارتی بپردازد (مگر اینکه بند ناعادلانه باشد).
	
	**بندهای وجه التزام** \lr{(Liquidated damages)} بندهایی هستند که مبلغ خسارت اعطایی برای انواع خاصی از نقض را تثبیت می‌کنند. مزیت این است که هر دو طرف از قبل می‌دانند چه مقدار خسارت باید برای نقض‌های خاص پرداخت شود، که بدین وسیله طرفین نیازی ندارند وارد بحث‌های طولانی در مورد مثلاً ارزش یک پروژه اتوماسیون ناموفق شوند. اگر مبلغ بسیار بالاتر از زیان واقعی متحمل شده باشد، چنین بندی به منزله «بند جریمه» \lr{(penalty clause)} است.\footnote{بندهای جریمه معمولاً مجاز هستند، اما برخی حوزه‌های قضایی اجازه آن‌ها را نمی‌دهند. در سایر حوزه‌های قضایی، اگر به طور نامتناسبی بالا باشند، ممکن است توسط دادگاه تعدیل شوند.}
	
	**فسخ** \lr{(Termination)} به این معنی است که قرارداد به دلیل نقض به پایان می‌رسد. نتیجه این است که هیچ تعهد دیگری بین طرفین وجود ندارد، به جز تعهدات لازم برای تصفیه قرارداد. فسخ ممکن است این اثر را داشته باشد که عملکرد قراردادی باید «بازگردانده» شود: هر آنچه انجام شده یا داده شده است خنثی می‌شود. این ممکن است مثلاً به این معنی باشد که پایگاه داده‌ای که ایجاد و به یک شرکت تحویل داده شده است باید به توسعه‌دهنده بازگردانده شود، اما پرداختی که توسعه‌دهنده دریافت کرده است نیز ممکن است لازم باشد بازگردانده شود.\footnote{یک پیامد همچنین می‌تواند این باشد که شرکت ممکن است پایگاه داده را نگه دارد اما باید مبلغ معینی پول برای آن به توسعه‌دهنده بپردازد.}
	
	اکنون می‌توانیم به مثال بخش ۱۹.۱ بازگردیم: حذف پایگاه داده به وضوح به عنوان بخشی از قرارداد در نظر گرفته نشده است، در حالی که احتمالاً آلیس باید اقدامات احتیاطی در برابر چنین پیشامدی انجام می‌داد. بنابراین، این می‌تواند به عنوان نقض قرارداد تلقی شود. اگر کامن لا اعمال شود، ممکن است لازم باشد که ضمانت یا شرط دیگری وجود داشته باشد که مبنای نقض قرارداد در صورت حذف پایگاه داده را فراهم کند.
	
	اگرچه اشتباه توسط ایو انجام شده است، آلیس به عنوان کارفرمای او مسئول است زیرا نمی‌تواند با اشاره به تقصیر یک کارمند خود را معاف کند (این فورس ماژور تلقی نمی‌شود). با این حال، زیان از نوع زیان اقتصادی محض است که تحت کامن لا ممکن است بر اساس نقض قرارداد قابل جبران نباشد. این ممکن است در صورتی که قرارداد حاوی بند وجه التزام باشد متفاوت باشد. در یک حوزه قضایی حقوق مدنی، زیان ممکن است قابل جبران باشد، و خسارت ممکن است بر اساس ارزش بازار پایگاه داده یا هزینه‌های (باز)تولید پایگاه داده ارزیابی شود.
	
	
	% =================================================================
	% ۱۹.۵ حقوق مسئولیت مدنی و داده‌ها
	% =================================================================
	\subsection{حقوق مسئولیت مدنی و داده‌ها}
	\label{sec:19-tort-law}
	
	اگر هیچ قراردادی بین دو نفر وجود نداشته باشد، مسئولیت باید بر اساس «حقوق مسئولیت مدنی» (شبه‌جرم) باشد.\footnote{در سیستم‌های حقوق مدنی، نام‌های جایگزینی مانند «حقوق مسئولیت ناشی از جرم» \lr{(delictual liability)} یافت می‌شود. ادبیات عمومی در مورد حقوق مسئولیت مدنی: \lr{Van Dam 2013, Tjong Tjin Tai 2022}.}
	
	یک «شبه‌جرم» \lr{(Tort)}، به زبان ساده، برای موارد خاص توصیف می‌کند که تحت چه شرایطی یک شخص مسئول است. یک مثال «افترا» \lr{(defamation)} است: این شبه‌جرم تنظیم می‌کند که چه زمانی کسی برای بیان اظهارات افتراآمیز مسئول است. انواع مختلفی از شبه‌جرم‌ها وجود دارد. در حقوق مسئولیت مدنی، ما بین دو شکل از مسئولیت تمایز قائل می‌شویم: «مسئولیت مبتنی بر تقصیر» و «مسئولیت محض».
	
	% -----------------------------------------------------------------
	% ۱۹.۵.۱ مسئولیت مبتنی بر تقصیر
	% -----------------------------------------------------------------
	\subsubsection{مسئولیت مبتنی بر تقصیر}
	
	ایده کلی مسئولیت مبتنی بر تقصیر این است که کسی (وکلا این شخص را «مرتکب زیان» یا \lr{tortfeasor} می‌نامند) مسئول است اگر «تقصیری» \lr{(fault)} مرتکب شده باشد: او کاری را انجام داده که قانوناً نباید انجام می‌داد، و این تقصیر باعث آسیب به قربانی شده است. بنابراین، سه شرط وجود دارد:
	
	\begin{itemize}
		\setlength\itemsep{0.5em}
		\item رفتار زیان‌بار (تقصیر)
		\item آسیب (زیان)
		\item رابطه سببیت بین تقصیر و آسیب
	\end{itemize}
	
	«آسیب» به معنای نوعی عدم مزیت، حادثه یا زیان است. این ممکن است یک حادثه فیزیکی، خسارت به کالا، لطمه به شهرت و از دست دادن حریم خصوصی باشد. رابطه سببیت در بخش ۱۹.۵.۳ بیشتر مورد بحث قرار می‌گیرد.
	
	اکثر شبه‌جرم‌ها با نوع تقصیری که پوشش می‌دهند تعریف می‌شوند، اما نوع آسیب نیز ممکن است مرتبط باشد. یک مثال دوباره افترا است: این مورد بر نوع خاصی از اظهارات اعمال می‌شود، اما همچنین مستلزم وجود لطمه به شهرت قربانی است. سیستم‌های حقوقی راه‌های مختلفی برای تعیین اینکه آیا رفتاری خاص (که ممکن است شامل «فعل» و همچنین «ترک فعل» باشد)\footnote{در سیستم‌های کامن لا، ترک فعل محض (که ناشی از یک عمل غفلت‌آمیز قبلی نباشد) ممکن است منجر به مسئولیت نشود.} زیان‌بار است یا خیر، دارند.
	
	مهم‌ترین شبه‌جرم، «غفلت» \lr{(Negligence)} است.\footnote{این نام در کامن لا است؛ در سیستم‌های حقوق مدنی، نام‌های دیگری یافت می‌شود، اما \lr{negligence} ممکن است برای نشان دادن شبه‌جرم محلی در ترجمه انگلیسی استفاده شود.} به بیان دقیق، این یک شبه‌جرم در کامن لا است، اما سیستم‌های حقوق مدنی نیز قواعدی دارند که در مواردی که تحت پوشش شبه‌جرم غفلت قرار می‌گیرند، مسئولیت ایجاد می‌کنند.
	
	غفلت به این معنی است که شما مراقبت یا تلاش کافی را نسبت به منافع قربانی رعایت نکرده‌اید: این مستلزم نقض یک «وظیفه مراقبت» \lr{(duty of care)} است.
	
	برای مثال، غفلت می‌تواند زمانی اعمال شود که شما الگوریتمی برای یک خودروی خودران توسعه داده‌اید بدون اینکه در طول توسعه مراقبت کافی برای اجتناب از اشتباهات در شناسایی موانع به خرج داده باشید، و خودرو عابری را بکشد زیرا فکر کرده که او یک جعبه مقوایی است.
	
	غفلت مفید است زیرا یک هنجار باز ارائه می‌دهد که می‌تواند برای پیشرفت‌های جدید اعمال شود. نقطه ضعف آن این است که کاربرد واقعی ممکن است نامشخص باشد: چه زمانی رفتار غفلت‌آمیز است؟ آزمون معمول این است که آیا یک «فرد معقول» به همان روش عمل می‌کرد یا خیر.
	
	شبه‌جرم‌های زیادی علاوه بر غفلت وجود دارند که ممکن است بر داده‌ها اعمال شوند. مثال‌ها عبارتند از نقض حریم خصوصی و افترا. این‌ها توسط قواعد و شرایط خاصی اداره می‌شوند؛ این مرور کلی جای پرداختن به جزئیات نیست.
	
	علاوه بر این، بسیاری از سوءاستفاده‌ها یا مداخلات زیان‌بار در داده‌ها و رایانه‌ها جرم محسوب می‌شوند.\footnote{به ویژه به دلیل تأثیر کنوانسیون‌های مختلف در مورد جرایم سایبری.} نقض حقوق کیفری عموماً یک شبه‌جرم نیز محسوب می‌شود، تحت عنوان شبه‌جرمِ «نقض وظیفه قانونی» \lr{(breach of a statutory duty)}.
	
	% =================================================================
	% ۱۹.۵.۲ مسئولیت محض
	% =================================================================
	\subsubsection{مسئولیت محض}
	
	مسئولیت محض به این معنی است که شما مسئول هستید حتی اگر شخصاً مرتکب تقصیری نشده باشید. مثال‌ها عبارتند از:
	\begin{itemize}
		\setlength\itemsep{0.3em}
		\item مسئولیت کارفرما برای شبه‌جرم‌های ارتکابی توسط کارمند.
		\item مسئولیت مالک خودرو برای حوادث مربوط به خودرو.
		\item مسئولیت محصولِ تولیدکننده صنعتیِ محصولات ملموس.
	\end{itemize}
	
	اکثر حوزه‌های قضایی این سه شکل از مسئولیت محض را به رسمیت می‌شناسند. برای اعمال این اشکال از مسئولیت محض، حداقل موارد زیر لازم است:
	\begin{enumerate}
		\item یک رابطه خاص بین شخصی که مسئول شناخته می‌شود و شیء یا شخصی که باعث خسارت شده است (رابطه استخدامی، مالکیت، کنترل).
		\item شکل خاصی از رفتار یا فعالیت شیء یا شخص (عمل زیان‌بار، تحقق یک خطر خاص).
	\end{enumerate}
	
	در بسیاری از کشورها، اشکال دیگری از مسئولیت محض نیز وجود دارد، مانند مسئولیت برای کودکان، حیوانات، اشیاء خطرناک و فعالیت‌های خطرناک. با این حال، این اشکال در همه جا یا به همان میزان پذیرفته نشده‌اند.
	
	مزیت مسئولیت محض این است که حفاظت بیشتری برای قربانیان فراهم می‌کند، که می‌توانند خسارت را از طرفی که واقعاً از فعالیت پرخطر یا به کارگیری شخص سود می‌برد (یا تصمیم به ورود به آن گرفته است) دریافت کنند. در ادبیات، مسئولیت محض اغلب به عنوان مدلی برای تنظیم مسئولیت ربات‌ها و الگوریتم‌ها پیشنهاد می‌شود \lr{(Tjong Tjin Tai 2018a, and references therein)}.
	
	برای اطمینان از اینکه مسئولیت محض بیش از حد گسترده نشود، محدودیت‌های متعددی وجود دارد. به ویژه، محدودیت‌های عمومیِ رابطه سببیت و ارزیابی خسارات، و همچنین دفاعیات اعمال می‌شوند (بخش‌های ۱۹.۵.۳ و ۱۹.۵.۴).
	
	% =================================================================
	% ۱۹.۵.۳ رابطه سببیت و دفاعیات
	% =================================================================
	\subsubsection{رابطه سببیت و دفاعیات}
	
	یک الزام برای دریافت جبران خسارت برای شبه‌جرم این است که آسیبی وجود داشته باشد که توسط تقصیر ایجاد شده باشد: رابطه سببیت \lr{(causality)}. پیوند سببی بین تقصیر و آسیب ابتدا با نگاه کردن به «رابطه سببی واقعی» (سببیت واقعی) ارزیابی می‌شود: آیا اگر تقصیر رخ نداده بود، آسیب همچنان اتفاق می‌افتاد؟ این آزمونِ به‌اصطلاح «اگر نبود» \lr{(but-for test)} (یا شرط لازم \lr{conditio sine qua non} در سیستم‌های حقوق مدنی) شرط لازم برای فرض یک شبه‌جرم است.
	
	پس از آن، دومین پیوند سببی مورد نیاز است: «سببیت قانونی» \lr{(legal causality)}. آسیب و خسارت نباید خیلی دور باشند. این آزمونی مشابه با مسئولیت قراردادی است: قابلیت پیش‌بینی یا دوری عمومی ممکن است به عنوان معیار استفاده شود. جزئیات بستگی به شبه‌جرم و سیستم حقوقی دارد.
	
	«دوری» \lr{(remoteness)} کمک می‌کند تا قرار گرفتن در معرض مسئولیت محدود شود. برای مثال، اگر شما اتفاقاً اشتباهی در به‌روزرسانی یک روال متن‌باز مرتکب شوید که در نهایت باعث اختلال در سرورهای سراسر جهان شود، اگر این پیامدها خیلی دور تشخیص داده شوند، مسئولیت شما ممکن است محدود شود.
	
	اگر چندین مرتکب زیان وجود داشته باشند، مسئولیت تمایل دارد «تضامنی» \lr{(joint and several)} باشد: مرتکبان زیان هم به صورت فردی و هم جمعی مسئول هستند. هر کدام ممکن است به تنهایی برای کل مبلغ خسارت مورد شکایت قرار گیرند؛ پس از آن، مرتکبی که خسارت را پرداخت کرده است ممکن است سهمی از سایر مرتکبان دریافت کند.
	
	اگر یک «دفاع» \lr{(defense)} اعمال شود، مسئولیت کاهش می‌یابد یا محدود می‌شود. یک دفاع مهم «غفلت مشارکتی» \lr{(contributory negligence)} است: اگر قربانی کاری انجام داده باشد که به وقوع آسیب یا میزان خسارت کمک کرده باشد، دادگاه ممکن است حکم خسارت را به نسبتی که قربانی در خسارت سهیم بوده کاهش دهد.
	
	سایر دفاعیات ممکن است مانع مسئولیت شوند. یک مثال دفاع فورس ماژور در مسئولیت محض است.\footnote{این تا حدودی شبیه به فورس ماژور در حقوق قراردادها است (بخش ۱۹.۴.۳.۱).} برای مثال اگر تصادفی نه توسط خودرو یا راننده‌اش، بلکه به این دلیل رخ داده باشد که کامیونی از عقب به خودرو برخورد کرده و بدین ترتیب آن را به خودروی جلویی کوبیده است، این ممکن است مالک خودرو را از مسئولیت مبری کند.
	
	% =================================================================
	% ۱۹.۵.۴ خسارات و سایر جبران‌ها در مسئولیت مدنی
	% =================================================================
	\subsubsection{خسارات و سایر جبران‌ها در مسئولیت مدنی}
	
	اگر شرایط شبه‌جرم برآورده شود، قربانی حق جبران خسارت دارد. مفهوم کلی جبران خسارت در شبه‌جرم شبیه به جبران خسارت در قرارداد است (که در بخش ۱۹.۴.۳ بحث شد)، اما الزامات متفاوت است و همه جبران‌ها در هر دو موقعیت اعمال نمی‌شوند.
	
	جبران اصلی برای یک شبه‌جرم، «حکم خسارت» است. سایر جبران‌های مهم ممکن است به شکل «دستور دادگاه» \lr{(court order)} داده شوند. دستورات دادگاه اعلامیه‌هایی توسط دادگاه هستند. انواع مختلفی از دستورات وجود دارد؛ دستوراتی که به عنوان ارائه جبران مهم هستند، آن‌هایی هستند که به یک طرف دستور می‌دهند عمل یا اعمال خاصی را انجام دهد، یا از رفتار خاصی خودداری کند. یک مثال «دستور منع» \lr{(restraining order)} است که رفتار خاصی را ممنوع می‌کند. نوع خاصی از دستور، «حکم توقیف/بازدارنده» \lr{(injunction)} است که انجام یک عمل خاص را ممنوع یا دستور می‌دهد.\footnote{تعاریف دقیق دستورات و احکام بازدارنده ممکن است بین سیستم‌های حقوقی مختلف متفاوت باشد.}
	
	حکم خسارت به این معنی است که زیانی که شبه‌جرم ایجاد کرده است باید توسط مرتکب زیان جبران شود. دادگاه خسارت را ارزیابی می‌کند. با این حال، چندین پیچیدگی وجود دارد.
	
	اول از همه، برای بسیاری از شبه‌جرم‌ها، همه انواع خسارت جبران نمی‌شوند. به ویژه، «زیان اقتصادی محض» اغلب جبران نمی‌شود.\footnote{برای مثال، شبه‌جرم غفلت در بسیاری از موارد اجازه جبران زیان اقتصادی محض را نمی‌دهد، اگرچه چنین جبرانی همیشه مستثنی نیست.} از آنجا که موارد مربوط به داده‌ها اغلب تنها باعث زیان اقتصادی محض می‌شوند، چنین شبه‌جرم‌هایی ممکن است عملاً جبران مناسبی نداشته باشند.
	
	اشکال جایگزینی از خسارات وجود دارد که اشکال معمول خسارت را تکمیل می‌کنند. یک مثال خاص «خسارات تنبیهی» \lr{(punitive damages)} است (حکم مبلغی پول که از خسارت واقعی فراتر می‌رود: این به عنوان مجازات مرتکب زیان تحمیل می‌شود تا به عنوان بازدارنده عمل کند).
	
	چه آسیب‌هایی ممکن است از داده‌ها ناشی شود؟ به طور کلی، می‌توانیم موارد زیر را در نظر بگیریم:
	\begin{itemize}
		\setlength\itemsep{0.3em}
		\item فقدان کیفیت
		\item خطا در تحلیل
		\item از دست دادن داده‌ها
		\item نشت داده‌ها / از دست دادن حریم خصوصی
	\end{itemize}
	
	در مورد موضوع اول، اطلاعات بسیار کمی در این مورد وجود دارد \lr{(Zeno-Zencovich, 2018)}. در کلان‌داده، کیفیت به عنوان یک امر مسلم فرض نمی‌شود؛ بلکه ایده کلان‌داده اغلب این است که با آنچه دارید کار کنید، آلوده یا نه، با این فرض که لکه‌های جزئی از نظر آماری خنثی می‌شوند. با این حال، داده‌هایی که به درستی ساختار نیافته‌اند یا در غیر این صورت نادرست هستند، ممکن است باعث پیامدهای منفی شوند.
	
	در مورد خطا در تحلیل: این احتمالاً یک مسئله قراردادی خواهد بود. قواعد قرارداد اعمال خواهد شد.
	
	برای از دست دادن داده‌ها، ایده کلی این خواهد بود که کاهش ارزش داده‌ها می‌تواند جبران شود، یا احتمالاً هزینه بازسازی پایگاه داده.
	
	برای نشت داده‌ها و از دست دادن حریم خصوصی، اغلب جبران مؤثری وجود ندارد \lr{(Peters 2014, Varuhas 2018)}. مثال وب‌سایت اجتماعی را در نظر بگیرید که هک شده است. اگرچه این ممکن است تهاجم جدی به حریم خصوصی باشد، اما هیچ زیان مادی از نقض وجود ندارد. حریم خصوصی به خودی خود ارزش روشنی ندارد؛ در بهترین حالت، مقداری خسارت نمادین یا اسمی ممکن است برای صدمه غیرمادی اعطا شود.
	
	بنابراین، در مورد شبه‌جرم‌های مربوط به داده‌ها، ممکن است دریافت خسارت قابل توجه دشوار باشد.
	
	ممکن است آموزنده باشد که دوباره به مثال بخش ۱۹.۱ نگاه کنیم، این بار از دیدگاه حقوق مسئولیت مدنی. فرض کنید که ایو این بار پایگاه داده شخص ثالثی (چارلی) را که اتفاقاً روی همان سرورِ پایگاه داده مشتری ذخیره شده بود، حذف کرده است. حذف پایگاه داده توسط ایو احتمالاً یک عمل غفلت‌آمیز است. آلیس، به عنوان کارفرما، مسئولیت نیابتی \lr{(vicariously liable)} برای غفلت ایو دارد. از آنجا که زیان، زیان اقتصادی محض است (ناشی از خسارت به اموال یا صدمه شخصی نیست)، ممکن است در کامن لا قابل جبران نباشد. در سیستم‌های حقوق مدنی، زیان ممکن است جبران شود.
	
	% =================================================================
	% نتیجه‌گیری
	% =================================================================
	\vspace{0.8cm}
	\noindent
	\fcolorbox{black}{gray!15}{%
		\begin{minipage}{\dimexpr\linewidth-2\fboxsep-2\fboxrule}
			\vspace{0.3cm}
			\begin{center}
				\textbf{\Large نتیجه‌گیری}
			\end{center}
			\vspace{0.2cm}
			
			مسائل متعددی وجود دارد که هنگام قرارداد بستن در مورد داده‌ها و هنگام در نظر گرفتن مسئولیت برای داده‌ها نیاز به توجه دارد. علاوه بر نکات حقوقی عمومی (که در این فصل معرفی شد)، عوارض خاصی نیز وجود دارد که عمدتاً نتیجه ماهیت غیرملموس داده‌ها است.
			
			مهم است که درک کنیم «داده» یک مفهوم ثابت نیست و ممکن است عناصری از زیرساخت‌های اطراف را نیز در بر گیرد. توجه ویژه‌ای به جبران محدود که ممکن است به دست آید، و انواع گسترده‌تر آسیب‌هایی که ممکن است توسط داده‌ها ایجاد شود، مورد نیاز است.
			\vspace{0.3cm}
		\end{minipage}
	}
	
	% =================================================================
	% پیام‌های کلیدی
	% =================================================================
	\vspace{0.8cm}
	\noindent
	\fcolorbox{black}{gray!15}{%
		\begin{minipage}{\dimexpr\linewidth-2\fboxsep-2\fboxrule}
			\vspace{0.3cm}
			\textbf{\large پیام‌های کلیدی}
			
			\begin{itemize}
				\setlength\itemsep{0.5em}
				\item[\textbf{--}] داده مفهوم کاملاً تعریف شده‌ای در قانون نیست. ممکن است به فایل‌های داده، اطلاعات موجود در فایل‌ها، یا کلان‌داده اشاره داشته باشد. فایل‌های داده به عنوان یک موجودیت در قانون شناخته نمی‌شوند؛ اطلاعات فقط توسط حقوق مالکیت فکری (شامل قانون اسرار تجاری) محافظت می‌شود.
				
				\item[\textbf{--}] در قراردادها، مهم است تصریح کنید که چه انتظاری دارید و اگر انتظارات شما برآورده نشود چه پیامدهایی دارد. این شامل اظهارات، ضمانت‌ها و تعهدات قراردادی به طور کلی می‌شود.
				
				\item[\textbf{--}] خسارت ممکن است شامل صدمه شخصی، خسارت به اموال، یا زیان اقتصادی محض باشد. زیان اقتصادی محض در بسیاری از موارد جبران نمی‌شود. در مورد داده‌ها، اغلب فقط زیان اقتصادی محض وجود دارد؛ بنابراین، اگر فقط داده درگیر باشد، ممکن است هیچ خسارتی دریافت نکنید.
				
				\item[\textbf{--}] مسئولیت برای داده‌ها مستلزم وجود یک شبه‌جرم قابل اجرا است که با داده‌ها سروکار دارد. مسئولیت مستلزم یک عمل زیان‌بار یا تقصیر، آسیب، و یک رابطه سببیت بین عمل و آسیب است.
				
				\item[\textbf{--}] شما ممکن است نه تنها برای اعمال زیان‌بار مستقیم، بلکه برای اعمال شخص دیگری (مسئولیت نیابتی) یا پیامدهای اشیائی که کنترل می‌کنید نیز مسئول باشید.
			\end{itemize}
			\vspace{0.3cm}
		\end{minipage}
	}
	
	% =================================================================
	% سوالات و پاسخ‌ها
	% =================================================================
	\vspace{0.8cm}
	\section*{سوالات و پاسخ‌ها}
	\addcontentsline{toc}{section}{سوالات و پاسخ‌ها}
	
	\subsection*{سوالات}
	\begin{enumerate}
		\item بندهای قراردادی سوءاستفاده‌گرایانه چگونه تنظیم می‌شوند؟
		\item الزامات استناد به جبران خسارت برای نقض قرارداد چیست؟
		\item زیان اقتصادی محض چیست؟
		\item چگونه می‌توانید ریسک مسئولیت قراردادی را محدود کنید؟
		\item الزامات اساسی مسئولیت مدنی (شبه‌جرم) چیست؟
	\end{enumerate}
	
	\subsection*{پاسخ‌ها}
	\begin{enumerate}
		\item با کنترل شروط ناعادلانه یا شیوه‌های تجاری ناعادلانه، یا توسط حسن نیت.
		\item نقض یک تعهد قراردادی، قصور، قابلیت انتساب علت نقض. در کامن لا همچنین: نقض یک ضمانت.
		\item زیان‌هایی که پیامد صدمه فیزیکی یا خسارت به اموال نیستند.
		\item با محدود کردن دامنه تعهدات/ضمانت‌های خود، با بندهای محدودیت/معافیت، با داشتن یک بند فورس ماژور گسترده، با بندهای وجه التزام.
		\item تقصیر، رابطه سببیت (واقعی و قانونی)، و آسیب.
	\end{enumerate}
	
	% =================================================================
	% منابع
	% =================================================================
	\vspace{1cm}
	\section*{منابع}
	\addcontentsline{toc}{section}{منابع}
	
	\begin{latin}
		\begin{itemize}
			\setlength\itemsep{0.5em}
			
			\item[] Bix, B. H. (2012). \textit{Contract law. Rules, theory, and context}. Cambridge University Press.
			
			\item[] Cartwright, J. (2013). \textit{Contract law. An introduction to the English law of contract for the civil lawyer}. Hart, Oxford.
			
			\item[] Mak, V., Tjong Tjin Tai, T. F. E., \& Berlee, A. (Eds.). (2018). \textit{Research handbook in data science and law}. Elgar Publishing.
			
			\item[] Peters, R. M. (2014). So you’ve been notified, now what? The problem with current data-breach notification laws. \textit{Arizona Law Review}, 56(4), 1171–1202.
			
			\item[] Smits, J. M. (2014). \textit{Contract law: A comparative introduction}. Elgar.
			
			\item[] Tjong Tjin Tai, T. F. E. (2018a). Liability for (semi-)autonomous systems. In Mak et al. (2018), pp. 55–82.
			
			\item[] Tjong Tjin Tai, T. F. E. (2018b). Data ownership and consumer protection. \textit{Journal of European Consumer and Market Law}, 7(4), 136–140.
			
			\item[] Tjong Tjin Tai, T. F. E. (2022). \textit{Tort Law: A Comparative Introduction}. Elgar.
			
			\item[] US Federal Court of Appeals, Jacobsen v. Katzer, 535 F.3d 1373 (Fed. Cir. 2008).
			
			\item[] van Dam, C. C. (2013). \textit{European tort law} (2nd ed.). Oxford University Press.
			
			\item[] Varuhas, J. N. E. (2018). Varieties of damages for breach of privacy. In J. N. E. Varuhas \& N. A. Moreham (Eds.), \textit{Remedies for breach of privacy}. Hart Publishing.
			
			\item[] Ventura, J. (2005). \textit{Law For dummies} (2nd ed.). Wiley.
			
			\item[] Wacks, R. (2015). \textit{Law: A very short introduction}. Oxford University Press.
			
			\item[] Zeno-Zencovich, V. (2018). Liability for data loss. In Mak et al. (2018), pp. 39–54.
			
		\end{itemize}
	\end{latin}