% =================================================================
% شروع فصل ۲۲: علم داده برای کارآفرینی (فایل chapter4_22.tex)
% =================================================================

\raggedbottom % حذف فاصله‌های اضافی عمودی

% -----------------------------------------------------------------
% تنظیمات شماره‌گذاری برای فصل ۲۲
% -----------------------------------------------------------------
\setcounter{section}{22}   % تنظیم شماره فصل روی ۲۲
\setcounter{subsection}{0} % ریست کردن زیربخش‌ها
\renewcommand{\thesection}{22}
\renewcommand{\thesubsection}{22.\arabic{subsection}}

% -----------------------------------------------------------------
% بلوک عنوان فصل و نویسندگان (شبیه تصویر ارسالی)
% -----------------------------------------------------------------
\noindent
\fcolorbox{black}{gray!15}{% کادر خاکستری برای عنوان
	\begin{minipage}{\dimexpr\linewidth-2\fboxsep-2\fboxrule}
		\vspace{0.5cm}
		\begin{center}
			% عنوان اصلی
			\textbf{\huge علم داده برای کارآفرینی: راه پیش رو}
			
			\vspace{0.5cm}
			
			% نام نویسندگان
			\large
			\textit{ویلم-جان ون دن هوول، ورنر لیبرگتس و آریان ون دن بورن} \\
			\lr{\textit{Willem-Jan van den Heuvel, Werner Liebregts and Arjan van den Born}}
		\end{center}
		\vspace{0.5cm}
	\end{minipage}
}

\vspace{0.8cm}

% -----------------------------------------------------------------
% فهرست مطالب فصل (Contents)
% -----------------------------------------------------------------
\noindent
\textbf{\Large فهرست مطالب}

\vspace{0.3cm}

\noindent
\begin{minipage}{\linewidth}
	\setlength{\parskip}{0.3em} % فاصله بین خطوط فهرست
	
	\textbf{۲۲.۱} \hspace{0.3cm} \textbf{مقدمه} 
	
	\textbf{۲۲.۲} \hspace{0.3cm} \textbf{راه پیش رو} 
	
	\hspace{1cm} ۲۲.۲.۱ \hspace{0.2cm} نرم‌افزار هوش مصنوعی \lr{(AI Software)}
	
	\hspace{1cm} ۲۲.۲.۲ \hspace{0.2cm} عملیات یادگیری ماشین \lr{(MLOps)}
	
	\hspace{1cm} ۲۲.۲.۳ \hspace{0.2cm} رایانش لبه‌ای \lr{(Edge Computing)}
	
	\hspace{1cm} ۲۲.۲.۴ \hspace{0.2cm} دوقلوهای دیجیتال \lr{(Digital Twins)}
	
	\hspace{1cm} ۲۲.۲.۵ \hspace{0.2cm} آزمایش در مقیاس بزرگ \lr{(Large-Scale Experimentation)}
	
	\hspace{1cm} ۲۲.۲.۶ \hspace{0.2cm} فرصت‌های کلان‌داده و هوش مصنوعی \lr{(Big Data and AI Opportunities)}
	
	\hspace{1cm} ۲۲.۲.۷ \hspace{0.2cm} مقررات دولتی \lr{(Government Regulation)}
	
	\vspace{0.3cm}
	\textbf{منابع}
\end{minipage}

\vspace{1cm}
% اینجا فضای خالی برای شروع متن اصلی (بخش ۲۲.۱) در مراحل بعدی است


% =================================================================
% بخش اهداف یادگیری (Learning Objectives)
% =================================================================

\vspace{0.5cm}

\noindent
\fcolorbox{black}{gray!15}{% کادر خاکستری
	\begin{minipage}{\dimexpr\linewidth-2\fboxsep-2\fboxrule}
		\vspace{0.3cm}
		% تیتر
		\textbf{\large اهداف یادگیری}
		
		\vspace{0.2cm}
		پس از خواندن این فصل، شما قادر خواهید بود تا:
		
		\begin{itemize}
			\setlength\itemsep{0.5em} % تنظیم فاصله بین آیتم‌ها
			
			\item[\textbf{--}] شناسایی و تبیین چند تحول مهمِ در حال انجام که بر کارآفرینی داده و در نتیجه بر پژوهش‌های کارآفرینی داده تأثیر می‌گذارند (و ادامه خواهند داد).
			
			\item[\textbf{--}] ترسیم چگونگی تکامل احتمالی حوزه عملی کارآفرینی داده در سال‌های پیش رو.
			
			\item[\textbf{--}] مشخص کردن تعدادی از مسیرهای امیدوارکننده برای پژوهش‌های آتی در محل تلاقی رشته‌های علم داده و کارآفرینی.
		\end{itemize}
		\vspace{0.2cm}
	\end{minipage}
}

\vspace{0.5cm}


% =================================================================
% بخش ۲۲.۱: مقدمه
% =================================================================

\vspace{0.6cm}

% تیتر بخش ۲۲.۱ (دستی)
\subsection*{۲۲.۱ مقدمه}
\addcontentsline{toc}{subsection}{۲۲.۱ مقدمه}
\label{sec:22-introduction}

این کتاب به دو حوزه‌ای پرداخته است که—تا همین اواخر—در انزوای کامل نسبت به هم قرار داشتند، یعنی علم داده و کارآفرینی. همان‌طور که در این کتاب بررسی کردیم، هر دو رشته یکدیگر را در رشته نوظهورِ کارآفرینیِ داده‌محور، مبتنی بر داده، یا کارآفرینی علم داده، که اغلب به اختصار «کارآفرینی داده» \lr{(data entrepreneurship)} نامیده می‌شود، پیدا می‌کنند.

کارآفرینی داده نیازمند دانش (حداقل پایه‌ای) در حوزه‌های مهندسی داده و تحلیل داده، و به نوبه خود، دانشِ آنچه ما «داده و جامعه» نامیده‌ایم (یعنی بافت تجاری و اجتماعی، مانند قوانین حاکم و دیدگاه‌های عموماً پذیرفته‌شده در مورد رفتار اخلاقی در قبال داده‌ها) است. از این رو، چهار بخش این کتاب هر کدام موضوعات مرتبط متعددی را در حوزه‌های مربوطه خود پوشش دادند.

اکنون که درک عمیق‌تری از دانشِ روز در تمام این حوزه‌ها به دست آورده‌ایم، وقت آن است که نگاهی به آینده (نزدیک) بیندازیم. ما تحولات مهم مختلفی را در حال ظهور می‌بینیم که—دیر یا زود—بر کارآفرینی داده تأثیر خواهند گذاشت و روش‌های جدید و وسوسه‌انگیزی را برای ایجاد ارزش تجاری بیشتر با بهره‌برداری از فرصت‌هایی که علم داده به ارمغان می‌آورد، فراهم می‌کنند: به اختصار (بسیار) کوتاه، «علم داده برای کارآفرینی».

سوال محوری این فصل، زیرعنوان این کتاب را بیشتر توضیح می‌دهد: کارآفرینان چگونه می‌توانند از کلان‌داده \lr{(big data)} و هوش مصنوعی \lr{(AI)} برای خلق ارزش جدید بهره‌برداری کنند؟ این تحولات همچنین راه‌های کاملاً جدیدی را برای پژوهش‌های آتی در محل تلاقی علم داده و کارآفرینی می‌گشایند. بنابراین، ما همچنین به اختصار در مورد پیامدهای آن‌ها برای پژوهش توسط محققان کارآفرینی بحث می‌کنیم.

در این فصل، ما ابتدا مروری بر آنچه معتقدیم مهم‌ترین تحولات هستند ارائه می‌دهیم و به اختصار در مورد پیامدها و تبعات آن‌ها از دیدگاه کارآفرینانه و علمی بحث می‌کنیم. پس از آن، ما این فصل خاص و کل کتاب را به پایان می‌رسانیم.




% =================================================================
% بخش ۲۲.۲: راه پیش رو
% =================================================================

\vspace{0.6cm}

% تیتر بخش ۲۲.۲ (دستی)
\subsection*{۲۲.۲ راه پیش رو}
\addcontentsline{toc}{subsection}{۲۲.۲ راه پیش رو}
\label{sec:22-road-ahead}

این بخش مسیری از فرصت‌ها و چالش‌ها را می‌گشاید که معتقدیم به شدت بر گفتمان نوظهورِ علم داده و کارآفرینی تأثیر خواهند گذاشت. برخی از این تحولات مربوط به فناوری‌های جدیدی هستند که افراد و شرکت‌ها را قادر می‌سازند تا پیشنهادها و ترکیبات محصول-بازار جدیدی را توسعه دهند (نوآوری‌های محصول و خدمت)، و سایر تحولات فناوری افراد و شرکت‌ها را قادر می‌سازند تا تحویل مؤثرتر یا کارآمدتری را دنبال کنند (نوآوری فرآیند). چنین شرکت‌هایی می‌توانند هم جدید و هم از قبل موجود باشند.

پنج تا ده سال پیش، زمانی که حوزه میان‌رشته‌ای کارآفرینی داده شروع به ظهور کرد، فناوری‌های مذکور وجود نداشتند یا صرفاً نقاطی کم‌اهمیت در دستور کار بودند. در حال حاضر، می‌بینیم که این فناوری‌ها بالغ‌تر شده‌اند و نقش فزاینده‌ای برای شرکت‌هایی که به دنبال مزیت رقابتی هستند، ایفا می‌کنند.

برجسته‌ترین توسعه فناورانه در چارچوبِ به‌اصطلاح «انقلاب صنعتی چهارم» (یا صنعت ۴.۰) مربوط به **هوش مصنوعی** \lr{(AI)} است. هوش مصنوعی به‌طور انکارناپذیری ویژگی‌های یک فناوریِ تحول‌آفرین با هدف عمومی \lr{(general-purpose technology)} را داراست \lr{(Brynjolfsson \& McAfee, 2014; Cockburn et al., 2018)}. و بنابراین، همان‌طور که چالمرز و همکاران \lr{(Chalmers et al., 2020)} بیان می‌کنند: «هوش مصنوعی ... پیامدهای عمیقی برای چگونگی توسعه، طراحی و مقیاس‌دهی سازمان‌ها توسط کارآفرینان دارد» (ص ۱۵). علاوه بر این تحولات فناورانه، ما شاهد تغییرات مهم اجتماعی و اقتصادی در چشم‌انداز خود نیز هستیم. به روندهای کلان مانند جهانی‌سازی و بین‌المللی‌سازی فکر کنید.

همان‌طور که کسب‌وکارها از نظر پذیرش و استفاده از داده‌ها بالغ‌تر شده‌اند و الگوریتم‌های جدید و محصولات و خدمات داده‌محور توسعه داده‌اند، چشم‌انداز رقابتی نیز متراکم‌تر شده است. دوران کشف و اکتشاف اولیه، و در نتیجه رقابت محدود، به سرعت در حال پایان است. اکثر شرکت‌ها و صنایع خدمات خود را به خدمات دیجیتال تبدیل کرده‌اند. در این فرآیند، راهکارهای سهل‌الوصول (اصطلاحاً میوه‌های پایین‌دست) قبلاً توسعه یافته و اتخاذ شده‌اند، و تنها تعداد محدودی از بازارها استثناهای قابل توجه هستند. خدمات دیجیتال مبتنی بر الگوریتم‌ها نیز بیشتر به یک کالای عمومی \lr{(commodity)} تبدیل شده‌اند.

در نهایت، بحث‌ها در مورد تنظیم‌گریِ پلتفرم‌ها و/یا الگوریتم‌ها تقریباً وجود نداشت، در حالی که امروزه تنظیم‌گری پلتفرم‌ها \lr{(Newman, 2019)} و الگوریتم‌ها \lr{(Parikh et al., 2019)} هر دو زمینه‌های مهم پژوهشی هستند. علاوه بر این، دولت‌ها به‌طور فزاینده‌ای قوانین جدید و اغلب سخت‌گیرانه‌تری را برای تنظیم بازارهای دیجیتال و خدمات اتخاذ می‌کنند \lr{(e.g., European Commission, 2020; U.S. House of Representatives, 2020)}.

در باقیمانده این بخش، ما چند پیشرفت و روند اخیر در علم داده را (بدون ترتیب خاصی) برجسته خواهیم کرد، که—به اعتقاد ما—به شدت بر کارآفرینی داده در سال‌های آینده، و در نتیجه بر پژوهش‌های کارآفرینی داده تأثیر خواهند گذاشت. این یک لیست جامع نیست، اما چند تحول عمده با تأثیر مورد انتظارِ قابل توجه را برجسته می‌کند.

% =================================================================
% بخش ۲۲.۲.۱: نرم‌افزار هوش مصنوعی
% =================================================================

\vspace{0.4cm}

% تیتر زیربخش ۲۲.۲.۱ (دستی)
\subsubsection*{۲۲.۲.۱ نرم‌افزار هوش مصنوعی}
\addcontentsline{toc}{subsubsection}{۲۲.۲.۱ نرم‌افزار هوش مصنوعی}
\label{sec:22-ai-software}

پس از چند دهه رشد به عنوان یک رشته علمی و عملی، هوش مصنوعی اکنون با هزاران کاربرد در تجارت و جامعه به سرعت در حال بلوغ است. با بهره‌گیری از پتانسیل هوش مصنوعی، نسل جدیدی از برنامه‌های نرم‌افزاری ظهور کرده است که اغلب به عنوان **نرم‌افزار هوش مصنوعی** \lr{(AI software)} نامیده می‌شود.

در واقع، نرم‌افزار هوش مصنوعی دیگر خود را به برنامه‌های «اسباب‌بازی» \lr{(toy applications)} نسبتاً آزمایشی و غیرمقیایس‌پذیر که فاقد هرگونه ارزش تجاری هستند، محدود نمی‌کند. در اینجا، بهترین‌هایِ دو دنیایی که تا همین اواخر جداگانه عمل می‌کردند، به هم پل زده شده‌اند؛ یعنی دنیای هوش مصنوعی و دنیای مهندسی نرم‌افزار.

هوش مصنوعی تکنیک‌ها و ابزارهای قابل توجهی را برای کاوشِ راهکارهای بهینه در موارد بسیار بدون ساختار، پیچیده، مبهم، غیرقابل پیش‌بینی و/یا ناقص به ارمغان آورده است. از سوی دیگر، مهندسی نرم‌افزار ارزش خود را در تبدیل فضاهای راهکارِ به‌خوبی درک‌شده، نسبتاً پایدار و به‌وضوح مرزبندی‌شده به کد اثبات کرده است \lr{(Ford, 1987)}.

این آخرین روند هوش مصنوعی، جامعه مهندسی نرم‌افزار را بر آن داشته است تا به‌طور فزاینده‌ای فناوری‌ها و پلتفرم‌های هوش مصنوعی (مانند پلتفرم هوش مصنوعی گوگل، تنسورفلو \lr{TensorFlow}، واتسون استودیو آی‌بی‌ام و آژور مایکروسافت) را تزریق کنند و سری جدیدی از مدل‌ها و شیوه‌های مهندسی نرم‌افزار را توسعه دهند که تولید خودکار کد، تست و یکپارچه‌سازی مداوم، و طراحی نرم‌افزار را تقویت می‌کند. بنابراین، فرصت‌های تجاری و پژوهشی هیجان‌انگیز جدیدی در این حوزه نوظهورِ نرم‌افزار هوش مصنوعی یافت می‌شود.



% =================================================================
% بخش‌های ۲۲.۲.۲ تا ۲۲.۲.۵ (ادامه بخش راه پیش رو)
% =================================================================

\raggedbottom % حذف فاصله‌های اضافی عمودی

% -----------------------------------------------------------------
% 22.2.2 MLOps
% -----------------------------------------------------------------
\subsubsection*{۲۲.۲.۲ عملیات یادگیری ماشین \lr{(MLOps)}}
\addcontentsline{toc}{subsubsection}{۲۲.۲.۲ عملیات یادگیری ماشین \lr{(MLOps)}}
\label{sec:22-mlops}

توسعه مهم دیگری که مشاهده می‌کنیم، تبدیل هوش مصنوعی و یادگیری ماشین به یک رشته مهندسی و بهبود همکاری و هماهنگی بین متخصصان مهندسی داده (شامل برنامه‌نویسان و کارکنان نگهداری نرم‌افزار)، دانشمندان داده (شامل متخصصان یادگیری ماشین) و متخصصان دامنه \lr{(domain experts)} است. این توسعه به زیبایی در پذیرش نسل جدیدی از «عملیات یادگیری ماشین» \lr{(MLOps)} که منضبط، تکرارپذیر و شفاف هستند، منعکس شده است.

\lr{MLOps} همراه با تکنیک‌های خودکار برای پیاده‌سازی خطوط لوله \lr{(pipelines)} یادگیری ماشین با توسعه نرم‌افزار، و فرهنگی است که از تیم‌های مدل‌سازی که از نزدیک با هم کار می‌کنند، حمایت می‌کند.

\lr{MLOps} عمدتاً از فلسفه «دِواپس» \lr{(DevOps)} \lr{(Ebert et al., 2016)} و شیوه‌های مرتبط با آن که جریان کاری توسعه نرم‌افزار و فرآیندهای تحویل را ساده و به‌طور تنگاتنگی یکپارچه می‌کنند، الهام گرفته شده است. مانند \lr{DevOps}، \lr{MLOps} نیز چرخه «یکپارچه‌سازی مداوم» \lr{(continuous integration)} و «تست مداوم» \lr{(continuous testing)} را برای تولید و استقرار ریز-انتشارها \lr{(micro-releases)} و نسخه‌های جدیدِ آماده‌ی تولید از برنامه‌های کاربردی هوشمند سازمانی اتخاذ می‌کند.

این امر مستلزم تغییر فرهنگ بین مهندسان داده، تحلیلگران داده، مهندسان استقرار و سیستم، و متخصصان دامنه است، که با مدیریت وابستگیِ بهبودیافته—و در نتیجه شفافیت—بین توسعه مدل، آموزش، اعتبارسنجی و استقرار همراه است. به این ترتیب، \lr{MLOps} به وضوح نیازمند سیاست‌های پیچیده‌ای مبتنی بر معیارهای عملکرد و تله‌متری \lr{(telemetry)}، مانند امتیازات \lr{F1}، دقت \lr{(accuracy)} و کیفیت نرم‌افزار است \lr{(Nogueira et al., 2018)}.

با توجه به اینکه مرزهای دقیق بین \lr{MLOps} و \lr{DevOps} مبهم است، یک سناریوی کاربردیِ بنیادین از \lr{MLOps} را می‌توان در خدمات وب آمازون \lr{(AWS)} یافت که از یک جریان کاری یکپارچه یادگیری ماشین برای ساخت، تست و یکپارچه‌سازی پشتیبانی می‌کند و تحویل مداوم را با کنترل منبع \lr{(source control)} و خدمات نظارتی ارائه می‌دهد.

% -----------------------------------------------------------------
% 22.2.3 Edge Computing
% -----------------------------------------------------------------
\subsubsection*{۲۲.۲.۳ رایانش لبه‌ای \lr{(Edge Computing)}}
\addcontentsline{toc}{subsubsection}{۲۲.۲.۳ رایانش لبه‌ای \lr{(Edge Computing)}}
\label{sec:22-edge-computing}

رایانش لبه‌ای یک پارادایم رایانشی جدید است که امکان پردازش و تحلیل بسیار توزیع‌شده‌ی حجم عظیمی از داده‌ها را در لبه‌های شبکه، و در نزدیک‌ترین مکان به محل مورد نیاز، فراهم می‌کند.



به این ترتیب، پردازش از ابر \lr{(cloud)} به لبه‌های شبکه منتقل می‌شود و پردازش، ذخیره‌سازی و تحلیلِ بسیار غیرمتمرکز را فراهم می‌کند. این نشان می‌دهد که رایانش لبه‌ای مدلی از رایانش توزیع‌شده را به جای رایانش متمرکز (که در مدل‌های رایانش ابری مرسوم وجود دارد) در آغوش می‌گیرد \lr{(Khan et al., 2019)}.

مزایای بالقوه شامل تأخیر \lr{(latency)} کمتر که باعث آزادسازی پهنای باند می‌شود، وابستگی کمتر به شبکه، و نزدیکی به کاربر است، که البته به قیمت کاهش قابلیت اطمینان و ظرفیت پردازشِ کمترِ ارائه شده توسط دستگاه‌های لبه تمام می‌شود \lr{(Bagchi et al., 2019)}. این امر نیازمند مکانیسم‌های امنیتی مقیاس‌پذیر و مستحکمی است که باید روی دستگاه‌های لبه توزیع شوند.

سیستم‌های رایانش لبه‌ای معمولاً تحت مالکیت ارائه‌دهندگان خدمات مختلف هستند و ممکن است تحت مفاد مدل‌های تجاری گوناگون عمل کنند. هر کسب‌وکاری طبق استراتژی‌های تجاری و سیاست‌های مدیریتی متفاوتی اداره می‌شود، در حالی که از قوانین و مقررات متفاوتی بر اساس سازمانِ عملیاتی خود پیروی می‌کند \lr{(Khan et al., 2019)}. به همین ترتیب، دستگاه‌های لبه توسط فروشندگان مختلف توسعه می‌یابند و رابط‌های خاص خود را دارند که بر عملکرد تأثیر می‌گذارد و هزینه‌های بالایی را به همراه دارد. به منظور غلبه بر مسائل فوق، یک مدل تجاری مدیریت و استقرارِ مشترک برای اطمینان از عملکرد بالا و ارائه خدمات کم‌هزینه به کاربران نهایی حیاتی است.

% -----------------------------------------------------------------
% 22.2.4 Digital Twins
% -----------------------------------------------------------------
\subsubsection*{۲۲.۲.۴ دوقلوهای دیجیتال \lr{(Digital Twins)}}
\addcontentsline{toc}{subsubsection}{۲۲.۲.۴ دوقلوهای دیجیتال \lr{(Digital Twins)}}
\label{sec:22-digital-twins}

در حالی که ناسا \lr{(NASA)} برای اولین بار از دهه ۱۹۶۰ با مفاهیم به‌اصطلاح دوقلوی دیجیتال برای شبیه‌سازی و تحلیلِ (مثلاً) شرایط زندگی در سفینه‌های فضایی مانند آپولو ۱۳ تمرین کرد، خودِ این اصطلاح توسط مایکل گریوز \lr{(Michael Grieves)} در سال ۲۰۰۲ در زمینه معرفی یک موسسه جدید مدیریت چرخه عمر محصول معرفی شد \lr{(Grieves, 2005)}.

پذیرش سریع فناوری دوقلوی دیجیتال با بلوغ فناوری‌های توانمندساز مختلف، از جمله یادگیری ماشین، همجوشی داده‌ها \lr{(data fusion)}، ارتباطات داده، اینترنت اشیاء \lr{(IoT)}، واقعیت افزوده، واقعیت مجازی و تحلیل کلان‌داده‌ها تقویت شده است.

در اصل، دوقلوهای دیجیتال را می‌توان به عنوان کپی‌های دیجیتالی از اشیاء فیزیکی (زنده یا غیرزنده) تعریف کرد \lr{(Shafto et al., 2012)}. آن‌ها اساساً فراتر از بازنمایی‌های دیجیتال موجود، مانند مدل‌های \lr{CAD} می‌روند و از سیستم‌های سایبری-فیزیکی در تمام چرخه عمر، یعنی از طراحی تا تولید و تا اجرا و مدیریت واقعی، پشتیبانی می‌کنند.



به این ترتیب، دوقلوهای دیجیتال از هوش مصنوعی و یادگیری ماشین بهره‌برداری می‌کنند تا داده‌های عملیاتی و بی‌درنگ \lr{(real-time)} به دست آمده از اشیاء فیزیکی و مجازیِ مجهز به دستگاه‌های اینترنت اشیاء را تجسم کنند و در نتیجه، تصمیم‌گیری انسانی را تقویت نمایند. این امر اطلاعات مربوط به اشیاء (مانند ساختمان‌ها و خطوط تولید) یا مفاهیم (مانند برنامه‌ریزی تولید) را برای ارتباط روان‌تر و شهودی‌تر بین ذینفعان مجاز، به راحتی در دسترس قرار می‌دهد.

در اینجا، می‌توان به داده‌های تاریخی، گزارش‌های وضعیت و فراداده‌های بافتی (مانند گزارش‌های آب‌وهوا) فکر کرد. با افزودن قابلیت‌های مبتنی بر هوش مصنوعی، دوقلوهای دیجیتال حتی می‌توانند موقعیت‌های مختلف را شبیه‌سازی کرده و درباره آن‌ها استدلال کنند و برای مثال سناریوهای «چه می‌شود اگر» \lr{(what-if)} را با بهره‌گیری از قابلیت‌های تشخیصی، پیش‌بینی و بهینه‌سازی اجرا کنند.

% -----------------------------------------------------------------
% 22.2.5 Large-Scale Experimentation
% -----------------------------------------------------------------
\subsubsection*{۲۲.۲.۵ آزمایش در مقیاس بزرگ}
\addcontentsline{toc}{subsubsection}{۲۲.۲.۵ آزمایش در مقیاس بزرگ}
\label{sec:22-large-scale-experimentation}

یک نقد مکرر بر علم داده این است که به جای روابط علی و معلولی \lr{(causal relationships)} و «خلاف‌واقع‌ها» \lr{(counterfactuals)}، بر همبستگی‌ها \lr{(correlations)} و تداعی‌ها تمرکز دارد. در اینجا، استدلال به این صورت است که اکثر همبستگی‌ها حتی با وجود پایگاه‌های داده بزرگ، بنا به تعریف «کاذب» \lr{(spurious)} هستند \lr{(Calude \& Longo, 2017)}. در حالی که این نقد تا حدودی معتبر است، اما این نکته را نادیده می‌گیرد که تحلیل مجموعه داده‌های بزرگ لزوماً به معنای تمرکز بر همبستگی‌ها نیست. مجموعه داده‌های بزرگ همچنین می‌توانند در کشف علیت و بسطِ خلاف‌واقع‌ها مفید باشند.

امروزه، اکثر شرکت‌های دیجیتال، از جمله \lr{Airbnb}، آمازون، \lr{Booking.com}، \lr{eBay}، فیس‌بوک، گوگل، لینکدین، مایکروسافت، نتفلیکس، توییتر و اوبر، آزمایش‌های کنترل‌شده تصادفی آنلاین را در مقیاسی (بسیار) بزرگ اجرا می‌کنند \lr{(Kohavi et al., 2020)}. این امر آن‌ها را قادر می‌سازد تا از داده‌های (کلان) برای یافتن عوامل علیِ زیربنایی استفاده کنند. معمولاً، شرکت‌های بزرگتر روزانه صدها تا هزاران مورد از این آزمایش‌های کنترل‌شده را، گاهی بر روی میلیون‌ها کاربر، اجرا می‌کنند.

در حالی که به‌اصطلاح «کارآزمایی‌های تصادفی کنترل‌شده» \lr{(RCTs)} در پزشکی اغلب به دلیل گران و پیچیده بودن مورد انتقاد قرار می‌گیرند، در محیط‌های دیجیتال، هزینه نهایی چنین آزمایش‌هایی بسیار کم است و ارزش افزوده کشف روابط علی را نباید دست‌کم گرفت. اگر این کار به درستی انجام شود، پذیرش آزمایش در مقیاس بزرگ مستقیماً منجر به نوآوری (تدریجی) و افزایش درآمد می‌شود. در نتیجه، با رشد سالانه تعداد چنین آزمایش‌هایی، شرکت‌های دیجیتال دو، سه یا حتی چهار برابر شده‌اند.

خوشبختانه، در حالی که شرکت‌های بزرگ ممکن است از نظر اندازه و هزینه‌های آزمایش (به دلیل صرفه به مقیاس) مزیت داشته باشند، \lr{RCT}ها به عنوان چنین روشی می‌توانند برای شرکت‌های کوچک و متوسط \lr{(SMEs)} نیز ارزشمند باشند. زیرساخت اجرای چنین آزمایش‌های در مقیاس بزرگی به‌طور فزاینده‌ای برای انواع شرکت‌ها در دسترس قرار گرفته است \lr{(Fabijan et al., 2018; Tang et al., 2010)}.

برای مثال، \lr{Google Optimize} یک ابزار تستِ تقسیم‌شده \lr{(split-testing)} آنلاین است که به وب‌سایت‌ها متصل می‌شود و بدین ترتیب \lr{SME}ها را قادر می‌سازد تا با روش‌های مختلفِ ارائه محتوا آزمایش کنند. مثال دیگر پلتفرم آزمایش آی‌بی‌ام است که هدف آن عملیات هوش مصنوعی است \lr{(Rausch et al., 2020)}. این تحولات با توجه به ارزش بالقوه کشف روابط علی، افزایش مداوم داده‌ها (مثلاً به دلیل ظهور اینترنت اشیاء، همچنین نگاه کنید به \lr{Attaran (2017)})، و کاهش مداوم هزینه‌های انجام آزمایش، احتمالاً ادامه خواهند یافت. مورد آخر تا حدی به دلیل ظهور پلتفرم‌های آزمایش در صنایع و حوزه‌های مختلف است.



% =================================================================
% بخش‌های ۲۲.۲.۶ و ۲۲.۲.۷ (ادامه بخش راه پیش رو)
% =================================================================

\raggedbottom % حذف فاصله‌های اضافی عمودی

% -----------------------------------------------------------------
% 22.2.6 Big Data and AI Opportunities
% -----------------------------------------------------------------
\subsubsection*{۲۲.۲.۶ فرصت‌های کلان‌داده و هوش مصنوعی}
\addcontentsline{toc}{subsubsection}{۲۲.۲.۶ فرصت‌های کلان‌داده و هوش مصنوعی}
\label{sec:22-big-data-ai-opportunities}

پنج تا ده سال پیش، مفهوم کلان‌داده \lr{(big data)} تازه در حال بلوغ بود \lr{(Provost \& Fawcett, 2013)}، و راهکارهای هوش مصنوعی عمدتاً توسط شرکت‌های بزرگ فناوری با دسترسی به انبوهی از داده‌ها، مقادیر هنگفت بودجه و کارکنان بااستعدادشان استفاده می‌شد.

در این دوره، همه شرکت‌های بزرگ و همچنین بسیاری از شرکت‌های نسبتاً بزرگ در میان گروه \lr{SME}ها (شرکت‌های کوچک و متوسط) قبلاً در حال آزمایش با علم داده بودند. معمولاً، این شرکت‌ها با راه‌اندازی یک تیم پروژه، یا حتی یک آزمایشگاه داده با متخصصان شروع کردند تا ببینند علم داده چه چیزی می‌تواند به آن‌ها ارائه دهد. گاهی اوقات، این شرکت‌ها از شرکت‌های مشاوره خارجی برای کمک به توسعه قابلیت‌های داده خود استفاده می‌کردند. بدون استثناء، این شرکت‌ها کشف کردند که علم داده به هیچ وجه آسان نیست. موانع فرهنگی، فنی، مدیریتی و سازمانی زیادی برای غلبه بر آن‌ها وجود دارد.

با این حال، شرکت‌هایی که استقامت کردند اغلب در داده‌ها ارزش یافتند. بینش‌های به دست آمده از داده‌ها حداقل برای تصمیم‌گیری داخلی مفید بود. شرکت‌های دیگر حتی قادر به توسعه محصولات و خدمات (دیجیتال) جدید مبتنی بر داده بودند. در حالی که کاوش و توسعه محصولات و خدمات دیجیتال دشوار است، و کسب درآمد از داده‌ها اغلب حتی دشوارتر است \lr{(Bataineh et al., 2020; Wixom \& Ross, 2017)}، در صورتی که این شرکت‌ها قادر به غلبه بر همه موانع بودند، فرصت‌های فراوانی را با رقابت نسبتاً محدود پیدا کردند \lr{(Zuboff, 2019)}. در این تنظیمات، شرکت‌ها اغلب قادر به بهره‌برداری از «مزیت پیشرو بودن» \lr{(first-mover advantage)} خود بودند \lr{(Varadarajan et al., 2008)}.

بعدها، شرکت‌های بسیار بیشتری توانستند بر موانع اولیه غلبه کنند و قابلیت‌های علم داده خود را بر این اساس بسازند \lr{(Davenport \& Ronanki, 2018; Fountaine et al., 2019)}. بنابراین، امروزه شانس ایجاد مزیت رقابتی پیرامون یک قابلیت علم داده منحصر به فرد اندک است. این احتمالاً نیاز به دسترسی به مجموعه داده‌های منحصر به فرد و حفاظت شده، استفاده از فناوری‌های هوش مصنوعیِ روز، و/یا تعهد کارکنان فوق‌العاده بااستعداد دارد.



به نظر می‌رسد شرکت‌های دارای قابلیت‌های «تحلیل کلان‌داده» \lr{(BDA)} عملکرد کلی بهتری دارند، اما اندازه اثرات به شدت به گرایش کارآفرینانه شرکت، صنعتی که در آن فعالیت می‌کند و پویایی محیطی بستگی دارد \lr{(Dubey et al., 2020; Müller et al., 2018; Wamba et al., 2017)}. در هر صورت، کاربرد علم داده با ترکیب مجموعه داده‌های معمولی و همه‌جا حاضر و استفاده از آمار استاندارد دیگر کافی نیست. این امروزه «بلیط ورود به بازی» است، اما دیگر به شرکت‌ها مزیت رقابتی نخواهد داد.

% -----------------------------------------------------------------
% 22.2.7 Government Regulation
% -----------------------------------------------------------------
\subsubsection*{۲۲.۲.۷ مقررات دولتی}
\addcontentsline{toc}{subsubsection}{۲۲.۲.۷ مقررات دولتی}
\label{sec:22-government-regulation}

همان‌طور که در بالا ذکر شد، دشت‌های رقابتیِ باز و وسیعِ دهه‌های اول این هزاره به پایان رسیده است. این نه تنها در مورد سطح رقابت صدق می‌کند، بلکه در مورد حضور آژانس‌های دولتی نیز صادق است. تاریخ نشان می‌دهد که مداخله و مقررات دولتی همیشه هنگام معرفی فناوری جدید عقب‌تر است \lr{(Wiener, 2004)}.

علم داده و هوش مصنوعی به هیچ وجه از این قاعده مستثنی نیستند. حتی می‌توان گفت که پیچیدگی و تازگی هوش مصنوعی در ترکیب با عدم شفافیتِ تأثیر اجتماعی خدمات دیجیتال، باعث شده است که دولت‌ها در سراسر جهان رویکرد «صبر و مشاهده» را اتخاذ کنند. با این حال، به نظر می‌رسد که این روزها دیگر به پایان رسیده است. انتقاد از فناوری دیجیتال و شرکت‌های دیجیتال در همه جای جهان رو به افزایش است: از چین تا اروپا و از آفریقا تا ایالات متحده.

این انتقاد در امتداد خطوط متعددی شکل گرفته است. اول و مهم‌تر از همه، انتقاداتی بر ویژگی‌های «برنده همه‌چیز را می‌برد» \lr{(winner-takes-all)} در بازارهای دیجیتال و رفتار غول‌های بزرگ فناوری، مانند اپل، گوگل، فیس‌بوک و مایکروسافت وجود دارد، زیرا آن‌ها از قدرت بازار خود برای محدود کردن رقابت استفاده (سوءاستفاده؟) می‌کنند. همان‌طور که گفته شد، تنظیم‌کنندگان بازار به‌طور فزاینده‌ای در حال توسعه و تصویب قوانین جدید ضد انحصار هستند. در زمان نگارش این متن، فهرستی از این قوانین جدید هنوز در حال بحث بود، اما دولت‌ها به آرامی اما مطمئن قدرت بازارِ غول‌های فناوری \lr{(Big Tech)} را هدف قرار می‌دهند.

در دموکراسی‌های غربی، دولت‌ها نگران رقابت و رفاه مصرف‌کننده هستند، در حالی که در دولت‌های اقتدارگراتر، دولت‌ها نگران قدرت شرکت‌های بزرگ فناوری در برابر دولت هستند.

با این حال، قدرت بازار به هیچ وجه تنها دلیلی نیست که دولت‌ها مایل به تنظیم شرکت‌های داده‌محور هستند. حریم خصوصی سال‌هاست که استدلال مهمی بوده و منجر به معرفی چارچوب \lr{GDPR} در اتحادیه اروپا شده است. در ایالات متحده، انتظار می‌رود ریاست‌جمهوری جدید سیاست‌های جدیدی را در زمینه جنبه‌هایی مانند قانون فدرال حریم خصوصی، انتقال بین‌المللی داده‌ها و بی‌طرفی شبکه ارائه دهد. اخیراً، مسائل اخلاقی و اجتماعی بیشتر، مانند تبعیض در الگوریتم‌ها و حباب‌های فیلترِ نوظهور در جامعه، به نقاط اصلی مورد توجه تبدیل شده‌اند.

در مجموع، روزهای دسترسی نامحدود به داده‌ها و ارائه بدون محدودیت و نظارتِ خدمات دیجیتال به پایان رسیده است. بی‌اعتمادی جهانی به غول‌های فناوری سر به فلک کشیده است و دولت‌ها با تقاضاها، قوانین، چارچوب‌ها و نهادهای دولتیِ روزافزون برای نظارت بر بازار دیجیتال خواهند آمد. برای رقابت در این بازار، آگاهی از این قوانین نوظهور و ایجاد روابط قابل اعتماد و بلندمدت با سازمان‌های دولتی در همه سطوح حیاتی است.




% =================================================================
% بخش نتیجه‌گیری (قسمت اول) - صفحه جاری
% =================================================================

\raggedbottom 

\vspace{0.8cm}

\noindent
\fcolorbox{black}{gray!15}{% کادر خاکستری قسمت اول
	\begin{minipage}{\dimexpr\linewidth-2\fboxsep-2\fboxrule}
		\vspace{0.4cm}
		\begin{center}
			\textbf{\Large نتیجه‌گیری}
		\end{center}
		\vspace{0.3cm}
		
		کارآفرینی داده ماندگار است. با رشد سالانه درک شده داده‌ها به میزان بیش از ۴۰ درصد، اهمیت و فراگیری داده‌ها در سال‌های آینده تنها افزایش خواهد یافت. داده‌های اینترنت اشیاء، داده‌های حسگرها، داده‌های ژنومی، داده‌های سلامت شخصی، داده‌های صوتی، داده‌های ویدئویی و بسیاری از انواع (جدید) دیگر داده‌ها به رشد انفجاری خود ادامه خواهند داد. بنابراین، کارآفرینی داده، که به طور خلاصه به عنوان اکتشاف و بهره‌برداری از فرصت‌ها با استفاده از علم داده تعریف می‌شود، بدون شک از نظر اهمیت نیز رشد خواهد کرد.
		
		با این حال، به نظر می‌رسد روزهای بهره‌برداری از «میوه‌های پایین‌دست» (فرصت‌های سهل‌الوصول) به پایان رسیده است. شناسایی و کسب درآمد از فرصت‌های جدید با علم داده به تدریج نیازمند دسترسی به مجموعه داده‌های منحصر به فرد، استفاده از فناوری‌های نوظهور، و/یا مشارکت افراد فوق‌العاده بااستعداد است. امروزه، صرفاً کاوش در یک مجموعه داده و ارائه آمار توصیفی به عنوان بینش، برای به دست آوردن و حفظ (موقتی) مزیت رقابتی بسیار ناکافی است.
		
		خوشبختانه برای شرکت‌های دیجیتالی که در خط مقدم علم داده هستند و آن‌هایی که عقب مانده‌اند، فناوری‌ها هنوز به سرعت در حال تکامل هستند. علاوه بر این، چارچوب‌های فناوری جدید، مانند نرم‌افزار هوش مصنوعی و \lr{MLOps}، به شرکت‌ها اجازه می‌دهند تا قابلیت‌های هوش مصنوعی خود را مقیاس‌دهی کنند. در نهایت، قوانین و مقررات ناشی از دولت (یا تغییرات در آن‌ها) در رابطه با پذیرش و استفاده از علم داده، باری بر دوش برخی شرکت‌ها خواهد بود، اما در عین حال، ممکن است فرصت‌های جدیدی را برای آن دسته از شرکت‌هایی فراهم کند که می‌توانند به راحتی با چنین قوانین (جدیدی) سازگار شوند و شاید حتی آن‌ها را شکل دهند.
		
		بدیهی است که همه این تحولات با پیامدهای عملی شدید برای کارآفرینان داده و توسعه‌دهندگان کسب‌وکار داده‌محور (یا کارآفرینان سازمانیِ داده)، بر حوزه پژوهش کارآفرینی داده نیز تأثیر خواهد گذاشت. این حوزه پژوهشی هنوز در دوران نوزادی خود است. با این حال، چند موضوع، هرچند بسیار اخیراً، توجه بیشتری را به خود جلب کرده‌اند. این موضوعات شامل فرصت‌هایی است که فناوری‌های دیجیتال مانند هوش مصنوعی برای کارآفرینی به ارمغان می‌آورند \lr{(e.g., Ransbotham et al., 2017; Townsend \& Hunt, 2019; Von Briel et al., 2018; Von Krogh, 2018)}، تأثیر هوش مصنوعی بر مدیران و تصمیم‌گیری آن‌ها \lr{(e.g., Huang et al., 2019; Raisch \& Krakowski, 2021; Shrestha et al., 2019)}، و رابطه بین قابلیت‌های تحلیل کلان‌داده \lr{(BDA)} و عملکرد شرکت \lr{(e.g., Dubey et al., 2020; Müller et al., 2018; Wamba et al., 2017)}.
		\vspace{0.2cm}
	\end{minipage}
}

% =================================================================
% شکستن صفحه برای بخش دوم (قسمت بیرون زده)
% =================================================================
\newpage 

% =================================================================
% بخش نتیجه‌گیری (قسمت دوم) - صفحه جدید
% =================================================================
\noindent
\fcolorbox{black}{gray!15}{% کادر خاکستری قسمت دوم (ادامه متن)
	\begin{minipage}{\dimexpr\linewidth-2\fboxsep-2\fboxrule}
		\vspace{0.3cm}
		با این وجود، به‌طور کلی، ما هنوز فاقد درک عمیقی از چگونگی، زمان و چراییِ استفاده کارآفرینان و کارآفرینان سازمانی از علم داده برای خلق ارزش جدید (یا عدم خلق آن) هستیم. موضوعات پژوهشی خاصی که نیاز به توجه بیشتر دارند عبارتند از:
		(۱) آگاهی و شایستگی‌های کارآفرینانه و مدیریتی در رابطه با تحولات فناورانه جدید؛
		(۲) موانع پذیرش و استفاده از علم داده در میان شرکت‌های تمام رده‌های سنی و اندازه؛
		(۳) تعیین‌کننده‌ها و پیامدهای داشتن سطوح مختلفِ به‌اصطلاح «بلوغ داده» در شرکت‌ها؛
		(۴) مزایا و معایب اشتراک‌گذاری داده (باز) برای کارآفرینی و نوآوری؛
		و (۵) تأثیر قوانین و مقررات جدید بر اکتشاف و بهره‌برداری از داده‌ها توسط شرکت‌ها.
		نمونه‌های مرتبط دیگر توسط چالمرز و همکاران \lr{(Chalmers et al., 2020)} به تفصیل بحث شده‌اند.
		
		بدین‌وسیله ما خواستار نظریه‌پردازی دقیق و بررسی تجربی گسترده در مورد هر یک از موضوعات فوق هستیم. بنا به تعریف، محققان باید در پژوهش‌های چندرشته‌ای شرکت کنند و بین رشته‌های علم داده و کارآفرینی پل بزنند. عصر جدیدی آغاز شده است \lr{(Obschonka \& Audretsch, 2020)}، و بیایید به جامعه‌ای کارآفرینانه کمک کنیم که در آن کلان‌داده و هوش مصنوعی توسط کارآفرینان به سازنده‌ترین روش‌ها مورد بهره‌برداری قرار می‌گیرند.
		\vspace{0.3cm}
	\end{minipage}
}

\vspace{1cm} % فضای خالی بین دو کادر طبق تصویر

% =================================================================
% بخش نکات بحث (Discussion Points)
% =================================================================
\noindent
\fcolorbox{black}{gray!15}{% کادر خاکستری برای نکات بحث
	\begin{minipage}{\dimexpr\linewidth-2\fboxsep-2\fboxrule}
		\vspace{0.3cm}
		\textbf{\large نکات بحث}
		
		\begin{enumerate}
			\setlength\itemsep{0.8em} % فاصله بین آیتم‌ها
			
			\item در این فصل، ما پیشرفت‌ها و روندهای اخیر مختلفی را برجسته کردیم که—به اعتقاد ما—بر کارآفرینی داده تأثیر خواهند گذاشت (و ادامه خواهند داد). فکر می‌کنید کدام یک از آن‌ها به احتمال زیاد قوی‌ترین تأثیر را خواهد داشت و چرا؟
			
			\item پیشنهاد شده است که اجرای قوانین و مقررات جدید (برای مثال، برای کاهش قدرت بازار شرکت‌های بزرگ فناوری) همچنین می‌تواند فرصت‌های کارآفرینانه جدیدی را برای برخی فراهم کند. حداقل یک نمونه از چنین فرصتی را که ممکن است به عنوان پیامد تصویب قوانین جدید ضد انحصار ایجاد شود، نام ببرید و توضیح دهید.
			
			\item یکی از مسیرهای امیدوارکننده برای پژوهش کارآفرینی داده مربوط به موانع پذیرش و استفاده از علم داده است که توسط شرکت‌ها درک می‌شود. تمام موانعی را که می‌توانید به آن‌ها فکر کنید فهرست کنید، و بحث کنید که چگونه هر یک از این موانع می‌توانند کاهش یابند یا حتی کاملاً از بین بروند.
		\end{enumerate}
		\vspace{0.3cm}
	\end{minipage}
}


% =================================================================
% بخش پیام‌های کلیدی (Take-Home Messages)
% =================================================================

\raggedbottom % حذف فاصله‌های اضافی عمودی

\vspace{0.8cm}

\noindent
\fcolorbox{black}{gray!15}{% کادر خاکستری
	\begin{minipage}{\dimexpr\linewidth-2\fboxsep-2\fboxrule}
		\vspace{0.3cm}
		% تیتر
		\textbf{\large پیام‌های کلیدی}
		
		\begin{itemize}
			\setlength\itemsep{0.5em} % تنظیم فاصله بین آیتم‌ها
			
			\item[\textbf{--}] از زمان پیدایش آن در ۵ تا ۱۰ سال پیش، کارآفرینی داده بسیار سریع تکثیر شده و همچنان در تمام عناصر زندگی روزمره و جامعه ما اهمیت بیشتری پیدا می‌کند.
			
			\item[\textbf{--}] فرصت‌ها و چالش‌های جدید را باید در فناوری‌های نوظهور، مانند نرم‌افزار هوش مصنوعی، عملیات یادگیری ماشین \lr{(MLOps)}، رایانش لبه‌ای و دوقلوهای دیجیتال یافت.
			
			\item[\textbf{--}] با رواج سریع علم داده، صرفاً به‌کارگیری آن دیگر به شرکت‌ها مزیت رقابتی نمی‌دهد، بلکه در عوض نیاز به مجموعه داده‌های منحصر به فرد، فناوری‌های پیشرفته و/یا استعدادهای استثنایی است.
			
			\item[\textbf{--}] قوانین و مقررات جدیدی برای برخورد بهتر با قدرت روزافزون شرکت‌های بزرگ فناوری، و مسائلی مانند حریم خصوصی داده‌ها و توضیح‌پذیری هوش مصنوعی مورد نیاز است.
			
			\item[\textbf{--}] پژوهش کارآفرینی داده هنوز در دوران نوزادی خود است، بنابراین تحقیقات بسیار بیشتری برای درک بهتر اینکه چگونه کارآفرینان می‌توانند از کلان‌داده و هوش مصنوعی برای خلق ارزش جدید بهره‌برداری کنند، مورد نیاز است.
		\end{itemize}
		\vspace{0.3cm}
	\end{minipage}
}

% =================================================================
% بخش منابع (References)
% =================================================================

\vspace{1cm}

% تیتر منابع (به فارسی)
% =================================================================
% بخش منابع فصل ۲۲ (اصلاح شده: تمام & ها به \& تبدیل شدند)
% =================================================================

\vspace{1cm}

% تیتر منابع
% =================================================================
% بخش منابع فصل ۲۲ (اصلاح نهایی: رفع خطای &)
% =================================================================

\vspace{1cm}

% تیتر منابع
\section*{منابع}
\addcontentsline{toc}{section}{منابع}

% شروع محیط لاتین
\begin{latin}
	\begin{itemize}
		\setlength\itemsep{0.5em} 
		
		\item[] Attaran, M. (2017). The internet of things: Limitless opportunities for business and society. \textit{Journal of Strategic Innovation and Sustainability}, 12(1), 10–29.
		
		\item[] Bagchi, S., Siddiqui, M. B., Wood, P., \& Zhang, H. (2019). Dependability in edge computing. \textit{Communications of the ACM}, 63(1), 58–66.
		
		\item[] Bataineh, A. S., Mizouni, R., Bentahar, J., \& El Barachi, M. (2020). Toward monetizing personal data: A two-sided market analysis. \textit{Future Generation Computer Systems}, 111, 435–459.
		
		\item[] Brynjolfsson, E., \& McAfee, A. (2014). \textit{The second machine age: Work, progress, and prosperity in a time of brilliant technologies}. WW Norton \& Company.
		
		\item[] Calude, C. S., \& Longo, G. (2017). The deluge of spurious correlations in big data. \textit{Foundations of Science}, 22(3), 595–612.
		
		\item[] Chalmers, D., MacKenzie, N. G., \& Carter, S. (2020). Artificial intelligence and entrepreneurship: Implications for venture creation in the fourth industrial revolution. \textit{Entrepreneurship Theory and Practice}, 45, 1–26.
		
		\item[] Cockburn, I. M., Henderson, R., \& Stern, S. (2018). \textit{The impact of artificial intelligence on innovation}. National Bureau of Economic Research.
		
		\item[] Davenport, T. H., \& Ronanki, R. (2018). Artificial intelligence for the real world. \textit{Harvard Business Review}, 96(1), 108–116.
		
		\item[] Dubey, R., Gunasekeran, A., Childe, S. J., Bryde, D. J., Giannakis, M., Foropon, C., Roubaud, D., \& Hazen, B. T. (2020). Big data analytics and artificial intelligence pathway to operational performance under the effects of entrepreneurial orientation and environmental dynamism: A study of manufacturing organizations. \textit{International Journal of Production Economics}, 226, 107599.
		
		\item[] Ebert, C., Gallardo, G., Hernantes, J., \& Serrano, N. (2016). DevOps. \textit{IEEE Software}, 33(3), 94–100.
		
		\item[] European Commission. (2020). Proposal for a Regulation of the European Parliament and of the Council on a Single Market for Digital Services (Digital Services Act) and amending Directive 2000/31/EC. COM(2020) 825 final. Brussels: European Commission.
		
		\item[] Fabijan, A., Dmitriev, P., McFarland, C., Vermeer, L., Holmström Olsson, H., \& Bosch, J. (2018). Experimentation growth: Evolving trustworthy A/B testing capabilities in online software companies. \textit{Journal of Software: Evolution and Process}, 30(12), e2113.
		
		\item[] Ford, L. (1987). Artificial intelligence and software engineering: A tutorial introduction to their relationship. \textit{Artificial Intelligence Review}, 1, 255–273.
		
		\item[] Fountaine, T., McCarthy, B., \& Saleh, T. (2019). Building the AI-powered organization. \textit{Harvard Business Review}, 97(4), 62–73.
		
		\item[] Grieves, M. W. (2005). Product lifecycle management: The new paradigm for enterprises. \textit{International Journal of Product Development}, 2(1–2), 71–84.
		
		\item[] Huang, M., Rust, R., \& Maksimovic, V. (2019). The feeling economy: Managing in the next generation of artificial intelligence (AI). \textit{California Management Review}, 61(4), 43–65.
		
		\item[] Khan, W. Z., Ahmed, E., Hakak, S., Yaqoob, I., \& Ahmed, A. (2019). Edge computing: A survey. \textit{Future Generation Computer Systems}, 97, 219–235.
		
		\item[] Kohavi, R., Tang, D., Xu, Y., Hemkens, L. G., \& Ioannidis, J. P. (2020). Online randomized controlled experiments at scale: Lessons and extensions to medicine. \textit{Trials}, 21(1), 1–9.
		
		\item[] Müller, O., Fay, M., \& Vom Brocke, J. (2018). The effect of big data and analytics on firm performance: An econometric analysis considering industry characteristics. \textit{Journal of Management Information Systems}, 35(2), 488–509.
		
		\item[] Newman, J. M. (2019). Antitrust in digital markets. \textit{Vanderbilt Law Review}, 72(5), 1497–1561.
		
		\item[] Nogueira, A.F., Ribeiro, J.C., Zenha-Rela, M.A., \& Craske, A. (2018). Improving La Redoute’s CI/CD pipeline and DevOps processes by applying machine learning techniques. In: \textit{Proceedings of the 11th International Conference on the Quality of Information and Communications Technology (QUATIC)}, pp. 282–286.
		
		\item[] Obschonka, M., \& Audretsch, D. B. (2020). Artificial intelligence and big data in entrepreneurship: A new era has begun. \textit{Small Business Economics}, 55, 529–539.
		
		\item[] Parikh, R. B., Obermeyer, Z., \& Navathe, A. S. (2019). Regulation of predictive analytics in medicine. \textit{Science}, 363(6429), 810–812.
		
		\item[] Provost, F., \& Fawcett, T. (2013). Data science and its relationship to big data and data-driven decision making. \textit{Big Data}, 1(1), 51–59.
		
		\item[] Raisch, S., \& Krakowski, S. (2021). Artificial intelligence and management: The automationaugmentation paradox. \textit{Academy of Management Review}, 46(1), 192–210.
		
		\item[] Ransbotham, S., Kiron, D., Gerbert, P., \& Reeves, M. (2017). Reshaping business with artificial intelligence: Closing the gap between ambition and action. \textit{MIT Sloan Management Review}, 59(1).
		
		\item[] Rausch, T., Hummer, W., \& Muthusamy, V. (2020). An experimentation and analytics framework for large-scale {AI} operations platforms. In: \textit{2020 {USENIX} Conference on Operational Machine Learning (OpML20)}.
		
		\item[] Shafto, M., Conroy, M., Doyle, R., Glaessgen, E., Kemp, C., LeMoigne, J., \& Wang, L. (2012). Modeling, simulation, information technology and processing roadmap. \textit{National Aeornautics and Space Administration}, 32, 1–38.
		
		\item[] Shrestha, Y. R., Ben-Menahem, S. M., \& Von Krogh, G. (2019). Organizational decision-making structures in the age of artificial intelligence. \textit{California Management Review}, 61(4), 66–83.
		
		\item[] Tang, D., Agarwal, A., O’Brien, D., \& Meyer, M. (2010). Overlapping experiment infrastructure: More, better, faster experimentation. In: \textit{Proceedings of the 16th ACM SIGKDD International Conference on Knowledge Discovery and Data Mining}, pp. 17–26.
		
		\item[] Townsend, D. M., \& Hunt, R. A. (2019). Entrepreneurial action, creativity, and judgment in the age of artificial intelligence. \textit{Journal of Business Venturing Insights}, 11, e00126.
		
		\item[] U.S. House of Representatives. (2020). \textit{Investigation of competition in digital markets: Majority staff reports and recommendations}. U.S. House of Representatives.
		
		\item[] Varadarajan, R., Yadav, M. S., \& Shankar, V. (2008). First-mover advantage in an internet-enabled market environment: Conceptual framework and propositions. \textit{Journal of the Academy of Marketing Science}, 36(3), 293–308.
		
		\item[] Von Briel, F., Davidsson, P., \& Recker, J. (2018). Digital technologies as external enablers of new venture creation in the IT hardware sector. \textit{Entrepreneurship Theory and Practice}, 42(1), 47–69.
		
		\item[] Von Krogh, G. (2018). Artificial intelligence in organizations: New opportunities for phenomenonbased theorizing. \textit{Academy of Management Discoveries}, 4(4), 404–409.
		
		\item[] Wamba, S. F., Gunasekaran, A., Akter, S., Ren, S. J. F., Dubey, R., \& Childe, S. J. (2017). Big data analytics and firm performance: Effects of dynamic capabilities. \textit{Journal of Business Research}, 70, 356–365.
		
		\item[] Wiener, J. B. (2004). The regulation of technology, and the technology of regulation. \textit{Technology in Society}, 26(2–3), 483–500.
		
		\item[] Wixom, B. H., \& Ross, J. W. (2017). How to monetize your data. \textit{MIT Sloan Management Review}, 58(3), 10–13.
		
		\item[] Zuboff, S. (2019). \textit{The age of surveillance capitalism: The fight for a human future at the new frontier of power}. Profile Books.
		
	\end{itemize}
\end{latin}



