% =================================================================
% فایل اصلی فصل ۴ (chapter4.tex) - نسخه تمیز و نهایی
% =================================================================

\setcounter{chapter}{3} 

\newpage
\pagecolor{gray!10} 

\chapter{داده و جامعه} 
\label{ch:part4-intro}

\begin{center}
	\Large \textbf{مقدمه بخش چهارم} \\
	\vspace{0.3cm}
	\large نویسنده: آن لافار \lr{(Anne Lafarre)}
\end{center}

\vspace{0.5cm}

% متن مقدمه
دانشمندان داده اغلب به عنوان مهندسان آینده شناخته می‌شوند و امروزه اکثریت قریب به اتفاق نوآوری‌ها و پروژه‌های تحقیقاتی، داده‌محور هستند. هدف نهایی، خلق ارزش واقعی از داده‌ها و در نتیجه ایجاد بینش‌ها و تغییرات سودمند برای کسب‌وکار و جامعه است. برای مثال، به دست آوردن داده‌ها در مورد کیفیت هوا می‌تواند مبنایی برای اقدامات مؤثر در سیاست‌گذاری تغییرات اقلیمی و حذف آلودگی‌های خطرناک فراهم کند.

از زمان شروع همه‌گیری کووید-۱۹ در سال ۲۰۲۰، بسیاری از دولت‌ها و سایر احزاب با کارشناسان علم داده همکاری کرده‌اند تا برنامه‌ها و راه‌حل‌های داده‌محوری را توسعه دهند که به شناسایی الگوهای آلودگی، ردیابی تماس‌های بین‌فردی و تشخیص چهره کمک کند.

با این حال، این پروژه‌های علم داده نیازمند حس قوی مسئولیت‌پذیری حرفه‌ای، قانونی و اخلاقی هستند. در این بخش پایانی با عنوان «داده و جامعه»، مفاهیم حفاظت از داده‌ها، حریم خصوصی و حقوق مالکیت فکری به تفصیل مورد بحث قرار می‌گیرند.

% -----------------------------------------------------------------
% صفحه ۲: فهرست محتویات
% -----------------------------------------------------------------
\newpage
\pagecolor{white} 

\section*{محتویات این بخش}
\vspace{0.5cm}

{
	\small
	\begin{description}
		\setlength\itemsep{0.8em}
		\item[] \textbf{فصل ۱۷: قانون حفاظت از داده‌ها و علم داده مسئولانه} \dotfill رافائل گلرت
		\item[] \textbf{فصل ۱۸: دیدگاه‌هایی از قانون مالکیت فکری} \dotfill لیزا ون دونگن
		\item[] \textbf{فصل ۱۹: مسائل مربوط به مسئولیت و قرارداد در داده‌ها} \dotfill اریک تیجونگ
		\item[] \textbf{فصل ۲۰: اخلاق داده و علم داده: یک پیوند دشوار؟} \dotfill استر کیمولن و لینت تیلور
		\item[] \textbf{فصل ۲۱: طراحی نرم‌افزار حساس به ارزش} \dotfill پاولان کورنهوف
		\item[] \textbf{فصل ۲۲: علم داده برای کارآفرینی: مسیر پیش رو} \dotfill ویلم-جان ون دن هوول
	\end{description}
}

\newpage

% =================================================================
% فراخوانی زیرفصل‌ها
% =================================================================

% =================================================================
% فصل ۱۷: قانون حفاظت از داده‌ها و علم داده مسئولانه (نسخه اصلاح شده)
% =================================================================

\clearpage
\raggedbottom 

% تنظیم شماره‌گذاری فصل روی ۱۷
\setcounter{section}{17}
\setcounter{subsection}{0}
\renewcommand{\thesection}{17}
\renewcommand{\thesubsection}{17.\arabic{subsection}}

% -----------------------------------------------------------------
% عنوان و نویسنده
% -----------------------------------------------------------------
\noindent
\fcolorbox{black}{gray!15}{%
	\begin{minipage}{\dimexpr\linewidth-2\fboxsep-2\fboxrule}
		\vspace{0.5cm}
		\begin{center}
			\textbf{\huge قانون حفاظت از داده‌ها و علم داده مسئولانه}
			\vspace{0.4cm}
			
			\large
			\textit{رافائل گلرت} \\
			\lr{\textit{Raphaël Gellert}}
		\end{center}
		\vspace{0.5cm}
	\end{minipage}
}

\vspace{0.8cm}

% -----------------------------------------------------------------
% فهرست مطالب داخلی
% -----------------------------------------------------------------
\noindent
\textbf{\Large فهرست مطالب}
\vspace{0.3cm}

{ \small
	\noindent \textbf{۱۷.۱} \hspace{0.3cm} \textbf{مقدمه} \par \vspace{0.1cm}
	
	\noindent \textbf{۱۷.۲} \hspace{0.3cm} \textbf{چند کلمه در باب معنای حریم خصوصی و حفاظت از داده‌ها} \par \vspace{0.1cm}
	
	\noindent \textbf{۱۷.۳} \hspace{0.3cm} \textbf{دامنه مادی قانون حفاظت از داده‌ها: تعریف پردازش و داده‌های شخصی} \par 
	\hspace{0.8cm} ۱۷.۳.۱ تعریف پردازش \par
	\hspace{0.8cm} ۱۷.۳.۲ تعریف داده‌های شخصی \par
	\hspace{0.8cm} ۱۷.۳.۳ نتیجه‌گیری: داده‌های شخصی و داده‌های غیرشخصی \par \vspace{0.1cm}
	
	\noindent \textbf{۱۷.۴} \hspace{0.3cm} \textbf{دامنه شخصی حفاظت از داده‌ها: کنترل‌کننده و پردازشگر} \par
	\hspace{0.8cm} ۱۷.۴.۱ سه بازیگر اصلی حفاظت از داده‌ها \par
	\hspace{0.8cm} ۱۷.۴.۲ کنترل‌کنندگان داده \par
	\hspace{0.8cm} ۱۷.۴.۳ پردازشگران داده \par
	\hspace{0.8cm} ۱۷.۴.۴ موقعیت‌های مسئله‌ساز \par \vspace{0.1cm}
	
	\noindent \textbf{۱۷.۵} \hspace{0.3cm} \textbf{ماده ۶ \lr{GDPR}: نیاز به مبنای قانونی برای پردازش} \par
	\hspace{0.8cm} ۱۷.۵.۱ رضایت \par
	\hspace{0.8cm} ۱۷.۵.۲ قرارداد \par
	\hspace{0.8cm} ۱۷.۵.۳ منافع حیاتی موضوع داده \par
	\hspace{0.8cm} ۱۷.۵.۴ انجام وظیفه‌ای در جهت منافع عمومی \par
	\hspace{0.8cm} ۱۷.۵.۵ انطباق با یک تعهد قانونی \par
	\hspace{0.8cm} ۱۷.۵.۶ منافع مشروع کنترل‌کننده داده یا شخص ثالث \par \vspace{0.1cm}
	
	\noindent \textbf{۱۷.۶} \hspace{0.3cm} \textbf{ماده ۵ \lr{GDPR}: اصولی که باید در پردازش داده‌ها اعمال شوند} \par
	\hspace{0.8cm} ۱۷.۶.۱ اصل محدودیت هدف \par
	\hspace{0.8cm} ۱۷.۶.۲ کمینه‌سازی داده‌ها \par
	\hspace{0.8cm} ۱۷.۶.۳ محدودیت ذخیره‌سازی \par
	\hspace{0.8cm} ۱۷.۶.۴ تعهدات اضافی \par \vspace{0.1cm}
	
}

\vspace{0.8cm}

% -----------------------------------------------------------------
% اهداف یادگیری (Learning Objectives)
% -----------------------------------------------------------------
\noindent
\fcolorbox{black}{gray!15}{%
	\begin{minipage}{\dimexpr\linewidth-2\fboxsep-2\fboxrule}
		\vspace{0.3cm}
		\textbf{\large اهداف یادگیری}
		
		\vspace{0.2cm}
		\begin{itemize}
			\setlength\itemsep{0.5em}
			\item[\textbf{--}] درک تفاوت بین حریم خصوصی و حفاظت از داده‌ها
			\item[\textbf{--}] درک چگونگی عملکرد قانون حفاظت از داده‌ها
			\item[\textbf{--}] درک و توانایی تشخیص اینکه آیا یک داده، شخصی محسوب می‌شود یا خیر
			\item[\textbf{--}] درک و توانایی تعیین وظایف بازیگران تحت قانون حفاظت از داده‌ها
			\item[\textbf{--}] درک و توانایی انتخاب مبنای صحیح برای پردازش داده‌های شخصی
			\item[\textbf{--}] درک و توانایی به‌کارگیری صحیح اصولی که بر پردازش داده‌ها اعمال می‌شوند
		\end{itemize}
		\vspace{0.2cm}
	\end{minipage}
}

\vspace{0.8cm} % فاصله قبل از شروع متن اصلی



\subsection{مقدمه}
\label{sec:17-intro}

این فصل مقدمه‌ای بر قانون حفاظت از داده‌ها را برای دانشمندان داده فراهم می‌کند. علم داده بر جمع‌آوری و تحلیل داده‌ها متکی است. حفاظت از داده‌ها مجموعه‌ای از قوانین است که تعیین می‌کند چه نوع عملیاتی روی داده‌ها قابل انجام است و تحت چه شرایطی.

به همین دلیل، برای دانشمندان داده حیاتی است که دانش پایه‌ای از اصول اصلی قانون حفاظت از داده‌ها داشته باشند تا بتوانند علم داده را به شیوه‌ای «از نظر اجتماعی مسئولانه» انجام دهند. این فصل توضیحی کلی از اصول کلیدی را به گونه‌ای ارائه می‌دهد که به شما اجازه می‌دهد هنگام مواجهه با یک کاربرد علم داده، همان‌طور که مثال زیر نشان می‌دهد، از آن‌ها استفاده کنید.

\vspace{0.4cm}

% کادر سوالات کلیدی (مطابق تصویر)
\begin{center}
	\colorbox{gray!15}{%
		\begin{minipage}{0.9\linewidth}
			\vspace{0.2cm}
			\noindent
			$\blacktriangleright$ \textbf{سوالات کلیدی}
			\vspace{0.2cm}
			
			شرکت \lr{X} یک بازار آنلاین است. بنابراین به مقدار مشخصی از اطلاعات مشتری مانند آدرس، نام یا اطلاعات کارت اعتباری نیاز دارد. با این حال، به نظر می‌رسد که شرکت \lr{X} همچنین از این اطلاعات برای ایجاد «پروفایل‌های گرایش سیاسی» \lr{(political affinity profiles)} (یعنی پروفایل‌های مربوط به جهت‌گیری سیاسی مشتریان خود) استفاده می‌کند.
			
			این پروفایل‌ها سپس به احزاب سیاسی فروخته یا اجاره داده می‌شوند تا آن‌ها بتوانند تبلیغات هدفمند ارسال کنند. قوانین حاکم بر این نوع پردازش داده‌ها چیست؟ آیا این کار اصلاً قانونی است؟ $\blacktriangleleft$
			\vspace{0.2cm}
		\end{minipage}
	}
\end{center}

\vspace{0.4cm}

این فصل در چهار گام پیش می‌رود. اول، ملاحظات مقدماتی را در مورد آنچه منظور از قانون حفاظت از داده‌ها (اروپا) است و تفاوت آن با حق حریم خصوصی ارائه می‌دهد. دوم، به دامنه حفاظت از داده‌ها نگاه می‌کند. این شامل هم «دامنه مادی» (داده شخصی چیست) و هم «دامنه شخصی» (بازیگران چه کسانی هستند) می‌شود.

سوم، شرایطی را بررسی می‌کند که تحت آن شروع پردازش داده‌ها امکان‌پذیر است. در نهایت، اصولی را بررسی می‌کند که باید هنگام پردازش واقعی داده‌ها رعایت شوند.





% =================================================================
% متن بخش ۱۷.۲: چند کلمه در باب معنای حریم خصوصی و حفاظت از داده‌ها
% =================================================================

\subsection{چند کلمه در باب معنای حریم خصوصی و حفاظت از داده‌ها}
\label{sec:17-privacy-meaning}

حق حریم خصوصی یکی از حقوق کلیدی دولت‌های دموکراتیک (اروپایی) است. این یک حق پیچیده و چندوجهی است. این حق ابتدا به عنوان حقِ «رها شدن به حال خود» \lr{(right “to be let alone”)} مفهوم‌سازی شد که با مسائل صمیمیت، محرمانگی مکاتبات، حفاظت از محل سکونت و غیره مرتبط بود \lr{(see Gutwirth, 2002)}.

در اروپا، حق حریم خصوصی توسط دو نهاد فراملی محافظت می‌شود: اتحادیه اروپا \lr{(EU)} و شورای اروپا \lr{(CoE)}.\footnote{در حالی که حقوق بنیادین تنها بخش کوچکی از صلاحیت‌های اتحادیه اروپا هستند، شورای اروپا یک سازمان بین‌المللی است که در حفاظت از حقوق بنیادین تخصص دارد. این سازمان در حال حاضر ۴۷ کشور عضو دارد. این شامل تمام کشورهای عضو اتحادیه اروپا می‌شود، اما بسیاری دیگر را نیز در بر می‌گیرد؛ نگاه کنید به: \lr{https://www.coe.int/en/web/about-us/who-we-are}.}

حق حریم خصوصی فراتر رفته و مسائل بسیار بیشتری را پوشش می‌دهد، مانند حقِ انجام انتخاب‌های شخصیِ اساسی مثل نام فرد، گرایش جنسی، جنسیت، سلامت و هویت \lr{(see Gutwirth, 2002)}. به همین دلیل، حق حریم خصوصی در اروپا اکنون با «تعیین سرنوشت» \lr{(self-determination)} و «خودمختاری» \lr{(autonomy)} مرتبط است \lr{(see Gutwirth, 2002)}.

این فصل تنها بر یک جنبه از حق حریم خصوصی تمرکز دارد که گاهی اوقات به عنوان «حریم خصوصی داده» \lr{(data privacy)} نامیده می‌شود \lr{(see, Hoofnagle, Sloot, Zuiderveen Borgesius, 2019, p. 70)}؛ یعنی مسائل حریم خصوصی که هنگام پردازش داده‌های شخصی ما توسط کامپیوترها ایجاد می‌شود. این همان چیزی است که «حفاظت از داده‌ها» درباره آن است.

بنابراین، قانون حفاظت از داده‌ها می‌تواند به عنوان قانون/چارچوب حقوقی درک شود که تعیین می‌کند داده‌های شخصی ما چگونه باید پردازش شوند تا از نقض حق حریم خصوصی ما (و سایر حقوق بنیادین مانند عدم تبعیض) جلوگیری شود.

این فصل بر اتحادیه اروپا تمرکز دارد، زیرا اخیراً قانونی را تصویب کرده است که می‌توان آن را جامع‌ترین قانون حفاظت از داده‌ها نامید و مستقیماً در تمام کشورهای عضو اتحادیه اروپا قابل اجراست: «مقررات عمومی حفاظت از داده‌ها» یا همان \lr{GDPR}. توجه داشته باشید که حفاظت از داده‌ها نیز یک حق بنیادین اتحادیه اروپا است که در ماده ۸ منشور حقوق بنیادین اتحادیه اروپا درج شده است.

در نهایت، باید چند کلمه در مورد ایالات متحده گفته شود. اگرچه سیستم حقوقی ایالات متحده دارای قوانینی است که پردازش داده‌ها را تنظیم می‌کند، اما مفهوم «حفاظت از داده‌ها» را به رسمیت نمی‌شناسد. در آنجا، همه چیز تحت عنوان حریم خصوصی، حریم خصوصی اطلاعات، یا اخیراً حریم خصوصی مصرف‌کننده برچسب‌گذاری می‌شود، حتی اگر در مورد حفاظت از داده‌ها باشد \lr{(see Hoofnagle, Sloot, Zuiderveen Borgesius, 2019, p. 70)}.

مقررات \lr{GDPR} می‌تواند به عنوان یک قانون «جامع» \lr{(omnibus)} تصور شود. این بدان معناست که همه چیز در یک قانون گنجانده شده است که برای همه فعالیت‌ها و همه افراد اعمال می‌شود: نهادهای اداری، کسب‌وکارها، سایر طرف‌های خصوصی و غیره.

در مقابل، قوانین در سیستم ایالات متحده نمی‌توانند به عنوان قانون جامع در نظر گرفته شوند: در سطح فدرال، ایالات متحده قوانین معدودی دارد که فقط بخش‌های خاصی را مورد خطاب قرار می‌دهند، مانند «قانون قابلیت انتقال و پاسخگویی بیمه سلامت» \lr{(HIPAA)} یا «قانون گزارش‌دهی اعتباری منصفانه» \lr{(FCRA)}. فعالیت‌های تجاری عمدتاً توسط «کمیسیون تجارت فدرال» \lr{(FTC)} تنظیم می‌شوند که تنظیم‌کننده (نهاد نظارتی) حفاظت از مصرف‌کننده است. از سال ۱۹۹۵، این کمیسیون صلاحیت خود را برای تنظیم «شیوه‌های ناعادلانه و فریبکارانه» گسترش داده تا مسائل مربوط به پردازش داده‌های شخصی را نیز شامل شود \lr{(see, Gellman, 2019)}.

به‌طور کلی، منصفانه است که بگوییم سطح حفاظت در ایالات متحده بسیار پایین‌تر از اتحادیه اروپا است. فراتر از پیچیدگی و ناهماهنگی‌های چارچوب قانونی و فقدان یک نهاد نظارتی واقعی حفاظت از داده‌ها، حفاظت واقعی بسیار کمتر است. برخی از عناصر مهم عبارتند از تعریف محدودِ داده‌های شخصی (در مقایسه با تعریف اروپایی که در این فصل به طور گسترده بحث می‌شود) و سیستمی که عمدتاً بر اساس رضایت است (به‌اصطلاح «اطلاع‌رسانی و رضایت» \lr{notice and consent})، که در عمل حفاظت کمتری ارائه می‌دهد، به ویژه در مقایسه با چارچوب اروپایی \lr{(see, Gellman, 2019)}.

با این حال، تغییر در راه است: برای مثال، کالیفرنیا «قانون حریم خصوصی مصرف‌کننده کالیفرنیا» \lr{(CCPA)} را در سال ۲۰۱۸ تصویب کرده است که از \lr{GDPR} الهام گرفته است (اما فقط محدود به کالیفرنیا است). در سطح فدرال، «قانون حفاظت و امنیت داده‌های مصرف‌کننده ۲۰۲۰» \lr{(CDPSA)} در مارس ۲۰۲۰ مطرح شده است.

این فصل چهار جنبه زیر از \lr{GDPR} را در نظر می‌گیرد. اول، دامنه آن را، هم شخصی و هم مادی، بررسی خواهد کرد. یعنی قانون حفاظت از داده‌ها چه زمانی اعمال می‌شود و برای چه کسانی اعمال می‌شود؟ سپس به برخی از مهم‌ترین مقررات ماهوی نگاه خواهد کرد، یعنی اصول اصلی که در مواد ۵ و ۶ \lr{GDPR} درج شده‌اند. این‌ها تعیین می‌کنند که تحت چه شرایطی شروع پردازش داده‌های شخصی امکان‌پذیر است و چه اصولی باید هنگام انجام چنین پردازشی رعایت شوند.
% متن بخش ۱۷.۲ اینجا قرار می‌گیرد

% =================================================================
% متن بخش ۱۷.۳: دامنه مادی (با شماره‌گذاری اتوماتیک تا سطح ۴)
% =================================================================

% فعال‌سازی شماره‌گذاری تا سطح ۴ (برای نمایش ۱۷.۳.۲.۱)
\setcounter{secnumdepth}{4}
\renewcommand{\theparagraph}{\thesubsubsection.\arabic{paragraph}}

\subsection{دامنه مادی قانون حفاظت از داده‌ها: تعریف پردازش و داده‌های شخصی}
\label{sec:17-3-material-scope}

دامنه مادی را می‌توان ارجاع به «چه چیز» \lr{(what)} دانست: قانون حفاظت از داده‌ها بر چه چیزی اعمال می‌شود؟ پاسخ، پردازشِ (بخش ۱۷.۳.۱) داده‌های شخصی (بخش ۱۷.۳.۲) است.

\subsubsection{تعریف پردازش}
\label{sec:17-3-1-processing}

قانون حفاظت از داده‌ها بر «پردازش داده‌های شخصی که تماماً یا بخشی از آن به صورت خودکار انجام می‌شود» طبق ماده ۲(۱) مقررات \lr{GDPR} اعمال می‌شود. ابزارهای خودکار شامل کامپیوترها و هر نوع دستگاه دیجیتال است.

پردازش به نوبه خود به معنای «هرگونه عملیات یا مجموعه‌ای از عملیات است که بر روی داده‌های شخصی انجام می‌شود» (ماده ۴(۲)، \lr{GDPR}). این بدان معناست که چرخه عمر یک عملیات پردازش از لحظه جمع‌آوری داده‌ها آغاز می‌شود و زمانی پایان می‌یابد که داده‌ها نابود یا ناشناس‌سازی شوند (نگاه کنید به بخش ۱۷.۴.۲). در فاصله بین این دو لحظه (و شامل آن‌ها)، هر عملیاتی که روی داده‌های شخصیِ مورد نظر انجام شود، یک «پردازش» تلقی خواهد شد.

\subsubsection{تعریف داده‌های شخصی}
\label{sec:17-3-2-personal-data}

داده‌های شخصی در ماده ۴(۱) مقررات \lr{GDPR} به عنوان «هرگونه اطلاعات مربوط به یک شخص حقیقی شناسایی‌شده یا قابل شناسایی (موضوع داده یا \lr{Data Subject})» تعریف شده است. به زبان ساده‌تر، داده‌های شخصی، داده‌هایِ موضوعِ داده هستند که در حال پردازش می‌باشند (یعنی داده‌های شما).

همان‌طور که مشاهده می‌شود، این تعریف شامل چهار عنصر کلیدی است که در زیر به اختصار بررسی می‌شوند.

% سطح ۴: ۱۷.۳.۲.۱
\paragraph{«هرگونه اطلاعات»}
\label{sec:17-3-2-1-any-info}
\mbox{}\\
کاملاً روشن نیست چرا \lr{GDPR} در نظر می‌گیرد که داده شخصی، «اطلاعات» است (به جای داده). برای ساده‌سازی، هر دو اصطلاح در این فصل به عنوان مترادف در نظر گرفته می‌شوند.
طبق نظر «کارگروه حفاظت از داده ماده ۲۹» \lr{(Art. 29 WP)}، آنچه تحت عنوان اطلاعات در \lr{GDPR} واجد شرایط است، بسیار گسترده می‌باشد. مهم نیست که اطلاعات خصوصی باشد یا عمومی، درست باشد یا غلط، و ذهنی باشد یا عینی. علاوه بر این، تا زمانی که برای پردازش توسط ابزارهای خودکار مناسب باشد، هر فرمتی قابل قبول است (مثلاً بایت‌های دیجیتال، صوتی، ویدئویی، نقاشی).

% سطح ۴: ۱۷.۳.۲.۲
\paragraph{«مربوط به»}
\label{sec:17-3-2-2-relating-to}
\mbox{}\\
اینکه داده‌ها به موضوع داده «مربوط» باشند، به سادگی به این معناست که این داده‌ها باید «درباره» موضوع داده باشند. این می‌تواند به روش‌های مختلفی باشد. ساده‌ترین راه زمانی است که «محتوای» داده‌ها (به وضوح) به این موضوع داده مربوط باشد؛ برای مثال، زمانی که داده‌ها حاوی نام، آدرس و شماره تأمین اجتماعی یک شخص باشد.

با این حال، داده‌های شخصی همچنین می‌توانند به روش‌های پیچیده‌تری که فراتر از محتوای دقیق داده‌هاست، به موضوعات داده مربوط شوند. این حالت در دو وضعیت رخ می‌دهد: زمانی که داده‌ها از نظر «هدف» \lr{(purpose)} یا از نظر «نتیجه» \lr{(result)} مرتبط باشند.
داده‌ها زمانی از نظر **هدف** مرتبط هستند که با هدفِ ارزیابی، رفتار به شیوه‌ای خاص، یا تأثیرگذاری بر موضوع داده پردازش شوند.
در نهایت، داده‌ها می‌توانند از نظر **نتیجه** (یا تأثیر) به موضوع داده مربوط شوند، زمانی که داده‌ها به شیوه‌ای پردازش می‌شوند که در پایان راه، تأثیری بر موضوع داده داشته باشند.

% سطح ۴: ۱۷.۳.۲.۳
\paragraph{«شناسایی‌شده یا قابل شناسایی»}
\label{sec:17-3-2-3-identifiable}
\mbox{}\\
داده‌ها نه تنها باید به موضوع داده «مربوط» باشند، بلکه موضوع داده نیز باید «شناسایی‌شده» یا «قابل شناسایی» باشد.
مقررات \lr{GDPR} بین موضوعات داده‌ای که «شناسایی‌شده» هستند و آن‌هایی که «قابل شناسایی» هستند تمایز قائل می‌شود.

موضوع داده زمانی **شناسایی‌شده** است که کنترل‌کننده/پردازشگر داده از قبل در مجموعه داده‌های خود اطلاعاتی دارد که موضوع داده را شناسایی می‌کند (یعنی او را متمایز می‌کند). این می‌تواند نام یا شناسه (منحصر به فرد) دیگری مانند شماره تلفن همراه یا کد ملی باشد.

موضوع داده زمانی **قابل شناسایی** است که کنترل‌کننده اطلاعات مستقیمی ندارد، اما با این وجود قادر به شناسایی موضوع داده است (مثلاً با ترکیب داده‌ها). \lr{GDPR} استدلال می‌کند که شناسایی موضوع داده می‌تواند توسط کسانی که داده‌ها را پردازش می‌کنند یا هر شخص ثالث دیگری انجام شود.

معیار تعیین اینکه آیا یک موضوع داده قابل شناسایی است یا خیر، استفاده از تمام ابزارهایی است که «احتمالاً به‌طور معقول برای شناسایی استفاده می‌شوند». می‌توان بین ابزارهای «فنی» (افزودن اطلاعات به داده‌ها) و ابزارهای «سازمانی» (زمینه و انگیزه کنترل‌کننده) تمایز قائل شد.

% سطح ۴: ۱۷.۳.۲.۴
\paragraph{«شخص حقیقی (موضوع داده)»}
\label{sec:17-3-2-4-natural-person}
\mbox{}\\
شخص در عبارت «داده‌های شخصی»، یا شخصی که داده‌ها به او مربوط می‌شود، به عنوان «موضوع داده» شناخته می‌شود که باید یک «شخص حقیقی» باشد (ماده ۴(۱)، \lr{GDPR}). در اصل، این به معنای دو چیز است. اول، شخص باید زنده باشد. دوم، شخص نمی‌تواند یک «شخص حقوقی» باشد.

\subsubsection{نتیجه‌گیری: داده‌های شخصی و داده‌های غیرشخصی}
\label{sec:17-3-3-conclusion}

به عنوان راهی برای نتیجه‌گیری این مرور کلی بر مفهوم داده‌های شخصی، می‌توان موارد زیر را گفت. این یک مفهوم بسیار گسترده است که اکثر انواع اطلاعات را در بر می‌گیرد. همچنین مفهومی «وابسته به زمینه» است.

در این رابطه، آزمون شناسایی نشان می‌دهد که داده شخصی همچنین «احتمالی» \lr{(probabilistic)} است. با این حال، آستانه بسیار پایین است، که به این معنی است که در عمل، بسیار دشوار است که یک قطعه داده شخصی نباشد. اما زمانی که چنین باشد (شخصی نباشد)، چنین داده‌های غیرشخصی به عنوان «داده‌های ناشناس» \lr{(anonymous data)} نامیده می‌شوند و از دسترس قانون حفاظت از داده‌ها خارج هستند.

% =================================================================
% متن بخش ۱۷.۴: دامنه شخصی حفاظت از داده‌ها (ترجمه کامل و دقیق)
% =================================================================

% فعال‌سازی شماره‌گذاری تا سطح ۴ (برای نمایش صحیح ۱۷.۴.۴.۱)
\setcounter{secnumdepth}{4}
\renewcommand{\theparagraph}{\thesubsubsection.\arabic{paragraph}}

\subsection{دامنه شخصی حفاظت از داده‌ها: کنترل‌کننده و پردازشگر}
\label{sec:17-4-personal-scope}

\subsubsection{سه بازیگر اصلی حفاظت از داده‌ها}
\label{sec:17-4-1-three-actors}

در اصل، می‌توان استدلال کرد که سه بازیگر کلیدی در قانون حفاظت از داده‌ها وجود دارند:

\begin{itemize}
	\item \textbf{موضوع داده \lr{(The data subject)}} یک شخص حقیقی شناسایی‌شده یا قابل شناسایی است که داده‌های شخصی او در حال پردازش است.
	
	\item \textbf{کنترل‌کنندگان داده \lr{(Data controllers)}} (اشخاص حقیقی یا حقوقی) بازیگران اصلی و صاحبان وظیفه هستند. آن‌ها مسئول و پاسخگو برای انطباق با قانون حفاظت از داده‌ها می‌باشند.
	
	\item \textbf{پردازشگران داده \lr{(Data processors)}} اشخاص حقیقی یا حقوقی مجزایی هستند که داده‌های شخصی را «به نمایندگی از» کنترل‌کننده پردازش می‌کنند. اگرچه مسئولیت آن‌ها در قبال کنترل‌کننده داده است، اما \lr{GDPR} اکنون به مقامات نظارتی اداری اجازه می‌دهد تا مستقیماً آن‌ها را جریمه کنند و تحت شرایط خاصی، آن‌ها نیز می‌توانند در برابر موضوعات داده پاسخگو باشند \lr{(see Rodway \& Carey, 2018, p. 178)}.
\end{itemize}

\subsubsection{کنترل‌کنندگان داده}
\label{sec:17-4-2-controllers}

مقررات \lr{GDPR} یک کنترل‌کننده (داده) را به این صورت تعریف می‌کند: «شخص حقیقی یا حقوقی (...) که به تنهایی یا مشترکاً با دیگران، اهداف و ابزارهای پردازش داده‌های شخصی را تعیین می‌کند» (طبق ماده ۴(۷)، \lr{GDPR}).

یک شخص حقیقی یا حقوقی می‌تواند به یک فرد تنها، یک فرد خوداشتغال، یا شرکت‌هایی مانند بانک‌ها، شرکت‌های بیمه، شرکت‌های حقوقی، سوپرمارکت‌ها، مطب‌های پزشکی و موتورهای جستجوی اینترنتی اشاره داشته باشد \lr{(see Welfare \& Carey, 2018, p. 18)}. این بدان معناست که یک بازیگر اگر «هدف» یا «ابزار»، یا هر دو را تعیین کند، به عنوان کنترل‌کننده داده واجد شرایط خواهد بود.

هدف پردازش، دلیل یا هدف کلی است که چرا داده‌ها در وهله اول پردازش می‌شوند: این موضوع در ادامه این بخش با جزئیات بیشتر بررسی خواهد شد. مقررات \lr{GDPR} در مورد اینکه «ابزار» پردازش دقیقاً به چه معناست، ساکت است. طبق نظر «کارگروه حفاظت از داده ماده ۲۹» \lr{(Art. 29 WP)}، این ابزارها باید به عنوان «ابزارهای ضروری/اساسی» \lr{(essential means)} درک شوند \lr{(Art. 29 WP, 2010, p. 14)}.

این ابزارهای ضروری به حیاتی‌ترین و اساسی‌ترین انتخاب‌هایی اشاره دارند که باید انجام شوند. بنابراین شامل انتخاب‌های زیر می‌شوند: چه داده‌هایی و چه مقدار داده پردازش خواهد شد، برای چه مدت، چه کسی می‌تواند به داده‌ها دسترسی داشته باشد، چه نوع عملیات پردازشی انجام خواهد شد، چه تعداد موضوع داده تحت تأثیر قرار می‌گیرند و غیره \lr{(Art. 29 WP, 2010, p. 14)}.

ابزارهای غیرضروری به عنوان «ابزارهای سازمانی» \lr{(organisational means)} نامیده می‌شوند \lr{(Art. 29 WP, 2010, p. 14)}. آن‌ها شامل مواردی مانند انتخاب سخت‌افزار و نرم‌افزار و اینکه کدام کارمند شرکت با کامپیوترها کار خواهد کرد، می‌شوند \lr{(Art. 29 WP, 2010, p. 14)}. این ابزارها می‌توانند توسط کنترل‌کننده داده یا توسط پردازشگر داده تعیین شوند و تأثیری در تعیین نقش‌ها ندارند.

\subsubsection{پردازشگران داده}
\label{sec:17-4-3-processors}

یک پردازشگر داده «یک شخص حقیقی یا حقوقی (...) است که داده‌های شخصی را به نمایندگی از کنترل‌کننده پردازش می‌کند» (طبق ماده ۴(۸)، \lr{GDPR}). به عبارت دیگر، پردازشگر پردازش داده‌های شخصی را به حساب کنترل‌کننده انجام می‌دهد.

ایده در اینجا یکی از «تفویض اختیار» \lr{(delegation)} است: پردازشگر دستورالعمل‌های کنترل‌کننده را اجرا خواهد کرد \lr{(Art. 29 WP, 2010, p. 25)}. باید در نظر داشت که پردازشگر یک شخص «جداگانه» است. از نظر تئوری، این بدان معناست که تفویض پردازش به کارمند دیگری در همان شرکت، به عنوان پردازشگر داده واجد شرایط نیست \lr{(Art. 29 WP, 2010, p. 25)}.

این می‌تواند شامل یک پیمانکار فرعی برای مدیریت حقوق و دستمزد، ذخیره‌سازی داده‌ها، مدیریت \lr{IT}، میزبانی وب‌سایت، یک تأمین‌کننده رایانش ابری، یا یک مرکز محاسباتی باشد \lr{(Rodway \& Carey, 2018, p. 175; Voigt \& von dem Busche, 2017, p. 20)}.

باید توجه داشت که \lr{GDPR} ملزم می‌کند که تنها پردازشگرانی که «تضامین کافی» برای انطباق با \lr{GDPR} ارائه می‌دهند، می‌توانند به عنوان پردازشگر انتخاب شوند (ماده ۲۸(۱) \lr{GDPR}). علاوه بر این، رابطه بین یک کنترل‌کننده و یک پردازشگر باید توسط یک قرارداد اداره شود (ماده ۲۸(۳) \lr{GDPR}).

مورد اخیر (قرارداد) باید برخی از عناصر کلیدی پردازش را ذکر کند: ماهیت و هدف پردازش، نوع داده‌ها، دسته‌های موضوعات داده و غیره. علاوه بر این، باید شامل تعدادی از تعهدات برای پردازشگر باشد، مانند اتخاذ تدابیر امنیتی مناسب و شفاف بودن با کنترل‌کننده (ماده ۲۸(۳)، \lr{GDPR}).

\subsubsection{موقعیت‌های مسئله‌ساز}
\label{sec:17-4-4-problematic}

تعاریف مربوط به کنترل‌کننده و پردازشگر ممکن است سرراست به نظر برسند، اما کاربرد آن‌ها همیشه آسان نیست. می‌توان دو مسئله را برجسته کرد: منطقه خاکستری بین یک کنترل‌کننده و یک پردازشگر، و چگونگی برخورد با تعدد کنترل‌کنندگان.

\paragraph{کنترل‌کننده یا پردازشگر؟}
\label{sec:17-4-4-1-controller-processor}
\mbox{}\\
اگرچه پردازشگر تنها به نمایندگی از کنترل‌کننده داده پردازش می‌کند، اما ممکن است برخی مناطق خاکستری وجود داشته باشد. از آنجا که پردازشگر داده‌ها را پردازش می‌کند، ممکن است مجبور شود تعدادی انتخاب در مورد چگونگی انجام پردازش انجام دهد. این می‌تواند زمانی باشد که پردازشگر منابع بسیار بیشتری نسبت به کنترل‌کننده دارد، اما نه فقط در این مورد.

آیا این انتخاب‌ها صرفاً اجرای دستورالعمل‌های کنترل‌کننده هستند، یا این‌ها انتخاب‌های واقعی در مورد ابزارهای ضروری پردازش (که حق انحصاری کنترل‌کننده است) می‌باشند؟ سوال کلیدی در اینجا درجه «استقلال» \lr{(autonomy)} است: آیا پردازشگر همچنان به نمایندگی از کنترل‌کننده عمل می‌کرد، یا نفوذ خود را بر پردازش اعمال می‌کرد \lr{(Rodway \& Carey, 2018, p. 182; Voigt \& von dem Busche, 2017, p. 19)}؟

آیا اگر پردازشگر در غیاب دستورالعمل‌های کنترل‌کننده عمل می‌کرد، پردازش داده‌ها همچنان انجام می‌شد (یعنی آیا فضایی برای این نوع تصمیم‌گیری صلاحدیدی وجود دارد) \lr{(Voigt \& von dem Busche, 2017, p. 19)}؟ به عبارت دیگر، چه کسی قدرت تصمیم‌گیری را دارد؟ اگر پاسخ به این سوال مثبت است، پس پردازشگر باید به عنوان یک کنترل‌کننده نیز در نظر گرفته شود و ما با وضعیت «کنترل‌کنندگی مشترک» مواجه هستیم.

\paragraph{کنترل‌کنندگان متعدد}
\label{sec:17-4-4-2-multiple-controllers}
\mbox{}\\
کنترل‌کنندگان داده ابزار و هدف پردازش را «به تنهایی یا مشترکاً با دیگران» تعریف می‌کنند (طبق ماده ۴(۷)، \lr{GDPR}). مقررات \lr{GDPR} مفهوم «کنترل‌کنندگی مشترک» \lr{(joint controllership)} را برای اشاره به موقعیت‌هایی با چندین کنترل‌کننده معرفی کرد (ماده ۲۶ \lr{GDPR}).

با توجه به اینکه فرد با تعریفِ هدف یا ابزار پردازش کنترل‌کننده می‌شود، گونه‌شناسی گسترده‌ای از کنترل‌کنندگی مشترک وجود دارد. کنترل‌کنندگان مشترک می‌توانند رابطه نزدیکی داشته باشند و به همان هدف (و ابزارهای ضروری) پایبند باشند، هدف مشترکی نداشته باشند اما مشترکاً ابزارها را تعیین کنند، یا رابطه آزادتری با اشتراک‌گذاری تنها بخش‌هایی از هدف و/یا ابزارها داشته باشند \lr{(Art. 29 WP, 2010, pp. 17–23)}.

به عنوان کنترل‌کنندگان مشترک، همه بازیگران مسئول انطباق با \lr{GDPR} هستند (نگاه کنید به ماده ۲۶(۳)، \lr{GDPR}). با این حال، مورد اخیر ملزم می‌کند که آن‌ها مسئولیت‌های خود را تخصیص دهند (یعنی «چه کسی چه کاری انجام می‌دهد؟») (ماده ۲۶(۱)، \lr{GDPR}). مقررات \lr{GDPR} در مورد اینکه در عمل چنین مسئولیتی چگونه باید به اشتراک گذاشته شود، اختیار می‌دهد. تنها ملزم می‌کند که کنترل‌کنندگان توافق‌نامه‌ای در این مورد منعقد کنند و آن را شفاف سازند، به ویژه برای موضوعات داده (ماده ۲۶(۱)(۲)، \lr{GDPR}).

% =================================================================
% بخش ۱۷.۵: ماده ۶ GDPR
% =================================================================

\subsection{ماده ۶ \lr{GDPR}: نیاز به مبنای قانونی برای پردازش}
\label{sec:17-5-gdpr-art6}

اگر یک کنترل‌کننده داده بخواهد پردازش داده‌های شخصی را آغاز کند، باید به ماده ۶ مقررات \lr{GDPR} مراجعه کند. این ماده حاوی شش مبنای جایگزین است که پردازش داده‌ها می‌تواند بر اساس آن‌ها انجام شود. به عبارت دیگر، شروع پردازش داده‌های شخصی امکان‌پذیر نیست مگر اینکه یک مبنا برای پردازش انتخاب شده باشد. می‌توان یکی از مبانی زیر را انتخاب کرد:

\begin{enumerate}
	\setlength\itemsep{0.5em}
	\item[(الف)] موضوع داده به پردازش داده‌های شخصی خود «رضایت» داده است (نگاه کنید به بخش ۱۷.۵.۱ این فصل).
	\item[(ب)] پردازش برای «اجرای قراردادی» که موضوع داده طرف آن است، ضروری است (نگاه کنید به بخش ۱۷.۵.۲).
	\item[(ج)] پردازش برای انطباق با یک «تعهد قانونی» ضروری است (نگاه کنید به بخش ۱۷.۵.۵).
	\item[(د)] پردازش به منظور محافظت از «منافع حیاتی» موضوع داده ضروری است (نگاه کنید به بخش ۱۷.۵.۳).
	\item[(هـ)] پردازش برای انجام وظیفه‌ای که در جهت «منافع عمومی» انجام می‌شود، ضروری است (نگاه کنید به بخش ۱۷.۵.۴).
	\item[(و)] پردازش برای مقاصد «منافع مشروع» دنبال شده توسط کنترل‌کننده یا شخص ثالث ضروری است، مگر در مواردی که چنین منافعی توسط منافع یا حقوق و آزادی‌های بنیادین موضوع داده مغلوب شوند (نگاه کنید به بخش ۱۷.۵.۶).
\end{enumerate}

همان‌طور که مشاهده می‌شود، همه مبانی به جز «رضایت» نیاز دارند که پردازش برای مبنای مورد استناد «ضروری» \lr{(necessary)} باشد. الزام ضرورت دلالت بر این دارد که آن مبنا نمی‌تواند با ابزارهای دیگری که برای حقوق بنیادین موضوعات داده محدودیت کمتری دارند، محقق شود \lr{(see Art. 29 WP, 2014, p. 13)}.

\subsubsection{رضایت \lr{(Consent)}}
\label{sec:17-5-1-consent}

رضایت به نشانگرِ آزادانه، مشخص، آگاهانه و بدون ابهامِ خواسته موضوع داده اشاره دارد که توسط آن، او (حداکثر در زمان شروع پردازش) موافقت خود را با پردازش داده‌های شخصی‌اش اعلام می‌کند (طبق ماده ۴(۱۱)، \lr{GDPR}).

% جدول ۱۷.۱
\begin{table}[h!]
	\centering
	\caption{هدف نامشخص و مشخص \lr{(Source: Art. 29 WP, 2018b, p. 9)}}
	\vspace{0.2cm}
	\begin{tabular}{|p{0.45\linewidth}|p{0.45\linewidth}|}
		\hline
		\textbf{مثالی از یک هدف نامشخص} & \textbf{مثالی از یک هدف مشخص} \\
		\hline
		«بهبود تجربه کاربران» & «ما تاریخچه خرید شما را با اشخاص ثالث به اشتراک می‌گذاریم تا محتوای متناسب برای خریدهای آینده را به شما ارائه دهیم» \\
		\hline
	\end{tabular}
	\label{tab:17-1-unspecific-specific}
\end{table}

\paragraph{۱۷.۵.۱.۱ رضایت باید در رابطه با یک هدف مشخص داده شود}
یک هدف زمانی به اندازه کافی مشخص است که به حد کافی دقیق باشد تا موضوع داده بتواند تعیین کند چه نوع پردازشی شامل آن می‌شود و چه نوعی نمی‌شود. به عبارت دیگر، یک پردازش تنها در صورتی تحت پوشش رضایت قرار می‌گیرد که در آن هدف گنجانده شده باشد.
ایده این است که از به‌اصطلاح «رضایت سفید» \lr{(blank consent)} جلوگیری شود (یعنی مانند یک چک سفید امضا، ما رضایت داده‌ایم اما نمی‌دانیم دقیقاً به چه چیزی) \lr{(Art. 29 WP, 2018a, pp. 11–12)}. مثالی از هر دو هدف نامشخص و مشخص را می‌توان در جدول ۱۷.۱ یافت.

\paragraph{۱۷.۵.۱.۲ رضایت باید آگاهانه باشد}
کنترل‌کنندگان داده باید شفاف باشند تا موضوعات داده بتوانند درک کنند که به چه چیزی رضایت می‌دهند \lr{(Art. 29 WP, 2018a, p. 13)}. اصلِ رضایت آگاهانه دو جنبه دارد.
از یک سو، کنترل‌کننده داده را ملزم می‌کند که اطلاعات اضافی را در اختیار موضوع داده قرار دهد. نمونه‌هایی از اطلاعاتی که باید ارائه شوند شامل نام کنترل‌کننده داده، هدف پردازش، انواع داده‌ها و انواع پردازش است \lr{(Art. 29 WP, 2018a, p. 13)}.
از سوی دیگر، این اصل به «کیفیت» اطلاعات ارائه شده مربوط می‌شود. موضوع داده باید بتواند به راحتی آنچه را که کنترل‌کننده می‌گوید درک کند \lr{(Art. 29 WP, 2018a, pp. 13–14)}. این امر استفاده از «اصطلاحات تخصصی حقوقی» \lr{(legal jargon)} را که هنوز هم اغلب در شرایط و ضوابط یافت می‌شود، رد می‌کند. به همین ترتیب، اطلاعات باید «به راحتی قابل دسترسی» باشند (مثلاً آن را با فونت‌های بسیار ریز در انتهای شرایط و ضوابط ننویسید).

\paragraph{۱۷.۵.۱.۳ رضایت باید بدون ابهام باشد}
هدف این است که نباید هیچ شکی در مورد قصد موضوع داده برای رضایت دادن وجود داشته باشد \lr{(Art. 29 WP, 2018a, pp. 15–16)}. خودِ رضایت می‌تواند به هر شکلی داده شود، تا زمانی که موضوع داده فعالانه موافقت خود را اعلام کند. روش‌های مختلفی برای ارائه رضایت وجود دارد، از جمله کتبی، بیانیه شفاهی ضبط شده، تیک زدن جعبه، پارامترهای مرورگر و غیره.

\vspace{0.4cm}
% کادر مثال جعبه‌های از پیش تیک‌خورده
\begin{center}
	\colorbox{gray!15}{%
		\begin{minipage}{0.9\linewidth}
			\vspace{0.2cm}
			\noindent
			$\blacktriangleright$ \textbf{جعبه‌های از پیش تیک‌خورده \lr{(Pre-ticked Boxes)}}
			\vspace{0.2cm}
			
			جعبه‌های از پیش تیک‌خورده (که به عنوان \lr{opt-out} شناخته می‌شوند) روش معتبری برای ابراز رضایت نیستند، زیرا هیچ رضایت فعالی از طرف موضوع داده وجود ندارد. تنها جعبه‌های معتبر، \lr{opt-in} هستند.
			
			\textbf{مثال رضایت نامعتبر برای کوکی‌ها:} جعبه‌های «ترجیحات» و «آمار» از پیش تیک‌خورده هستند، مانند شکل ۱۷.۱. $\blacktriangleleft$
			\vspace{0.2cm}
		\end{minipage}
	}
\end{center}

% شکل ۱۷.۱ (جایگزین عکس با کادر لاتک)
\begin{figure}[h!]
	\centering
	\setlength{\fboxsep}{10pt}
	\fbox{
		\begin{minipage}{0.6\linewidth}
			\begin{latin}
				\textbf{This website uses cookies} \\[0.3cm]
				\small
				\textbf{[x] Preferences} \quad \textbf{[x] Statistics} \quad [ ] Marketing
			\end{latin}
		\end{minipage}
	}
	\caption{رضایت نامعتبر برای کوکی‌ها. منبع: شکل متعلق به نویسنده.}
	\label{fig:17-1-invalid-consent}
\end{figure}

\paragraph{۱۷.۵.۱.۴ رضایت باید آزادانه باشد}
برای اینکه رضایت آزادانه باشد، موضوع داده باید هنگام رضایت دادن یک انتخاب واقعی داشته باشد \lr{(see Art. 29 WP, 2018a, pp. 5–11)}. این در مواردی که موضوع داده مجبور به رضایت دادن است، در معرض ضرر و زیان است، یا هیچ گزینه واقعی ندارد، صدق نمی‌کند.
رضایت در مواردی که با «نابرابری‌های قدرت» مشخص می‌شوند (مانند اشتغال یا مقامات دولتی) اجباری تلقی خواهد شد.
رضایت زمانی در معرض «ضرر و زیان» است که امتناع از رضایت منجر به پیامدهای منفی مانند هزینه‌ها یا از دست دادن خدمات شود.
فقدان انتخاب واقعی به مسائل «شرطی بودن» یا «بسته‌بندی» \lr{(bundling)} اشاره دارد. این به موقعیت‌هایی اشاره دارد که رضایت برای عملیات پردازش با یک قرارداد برای انجام خدمات بسته‌بندی شده است، حتی اگر رضایت به خودی خود برای انجام خدمات ضروری نباشد.

\vspace{0.4cm}
% کادر مثال بسته‌بندی رضایت
\begin{center}
	\colorbox{gray!15}{%
		\begin{minipage}{0.9\linewidth}
			\vspace{0.2cm}
			\noindent
			$\blacktriangleright$ \textbf{بسته‌بندی رضایت \lr{(Bundling of Consent)}}
			\vspace{0.2cm}
			
			یک برنامه موبایل برای ویرایش عکس از کاربرانش می‌خواهد که موقعیت مکانی \lr{GPS} خود را برای استفاده از خدماتش فعال کنند. این برنامه همچنین می‌گوید که از داده‌های جمع‌آوری شده برای اهداف تبلیغات رفتاری استفاده خواهد کرد.
			
			نه موقعیت مکانی و نه تبلیغات رفتاری آنلاین برای ارائه خدمات ویرایش عکس ضروری نیستند. بنابراین، این نوع رضایت معتبر نیست.
			\vspace{0.1cm}
			\footnotesize{نکته: مثال بر اساس \lr{Art. 29 WP (2018a, p. 6)}.} $\blacktriangleleft$
			\vspace{0.2cm}
		\end{minipage}
	}
\end{center}

\paragraph{۱۷.۵.۱.۵ دسته‌های خاص داده‌ها: رضایت صریح}
برای دسته‌های خاصی از داده‌های حساس (مانند نژاد، عقاید سیاسی، داده‌های ژنتیکی)، یک رضایت معمولی کافی نیست و یک «رضایت صریح» مورد نیاز است (ماده ۹، \lr{GDPR}). رضایت باید به‌طور صریح در یک بیانیه کتبی تأیید شود (مثلاً: «من به پردازش داده‌ها برای هدف [x] رضایت می‌دهم»).

\subsubsection{قرارداد}
\label{sec:17-5-2-contract}

یک قرارداد بین یک موضوع داده و یک کنترل‌کننده داده می‌تواند در دو مورد به عنوان مبنایی برای پردازش داده‌های شخصی عمل کند.
اول، پردازش داده‌ها برای «اجرای قرارداد» ضروری است (جدول ۱۷.۲).
دوم، موردی است که پردازش در «مرحله پیش‌قراردادی» ضروری است (یعنی تا قرارداد بتواند ایجاد شود). این تنها در صورتی معتبر است که به درخواست موضوع داده باشد (جدول ۱۷.۳).

% جدول ۱۷.۲
\begin{table}[h!]
	\centering
	\caption{پردازش داده‌ها برای اجرای قرارداد ضروری است \lr{(Source: Voigt \& Von dem Busche, 2017, p. 102)}}
	\vspace{0.2cm}
	\small
	\begin{tabular}{|p{0.15\linewidth}|p{0.38\linewidth}|p{0.38\linewidth}|}
		\hline
		& \textbf{معتبر (Valid)} & \textbf{نامعتبر (Invalid)} \\
		\hline
		پردازش ضروری & نام و آدرس مشتری، نوع و تعداد کالا، روش پرداخت & نام والدین، سن همسر، سایر عادات خرید \\
		\hline
		اجرای قرارداد & تحویل محصولات به مشتری & وصول بدهی، مراجعه به دادگاه در صورت اختلاف \\
		\hline
	\end{tabular}
	\label{tab:17-2-contract-execution}
\end{table}

% جدول ۱۷.۳
\begin{table}[h!]
	\centering
	\caption{پردازش داده‌ها در مرحله پیش‌قراردادی ضروری است \lr{(Source: Voigt \& Von dem Busche, 2017, p. 102)}}
	\vspace{0.2cm}
	\small
	\begin{tabular}{|p{0.2\linewidth}|p{0.35\linewidth}|p{0.35\linewidth}|}
		\hline
		& \textbf{معتبر (Valid)} & \textbf{نامعتبر (Invalid)} \\
		\hline
		اقدامات پیش‌قراردادی (۱) & ارسال تبلیغات آنلاین برای مشتری که خواهان اطلاعات بیشتر است & \\
		\hline
		اقدامات پیش‌قراردادی (۲) & & بازاریابی مستقیم مبتنی بر پروفایل‌سازی بدون اطلاع موضوع داده \\
		\hline
	\end{tabular}
	\label{tab:17-3-pre-contractual}
\end{table}


% =================================================================
% ادامه بخش ۱۷.۵: از زیربخش ۱۷.۵.۳ تا ۱۷.۵.۶
% =================================================================

\subsubsection{منافع حیاتی موضوع داده}
\label{sec:17-5-3-vital-interests}

این مبنا — منافع حیاتی موضوع داده — حاشیه‌ای و باقی‌مانده است (یعنی زمانی که نمی‌توان به سایر مبانی استناد کرد) \lr{(Art. 29 WP, 2014, p. 20)}. این مورد موقعیت‌های مرگ و زندگیِ موضوع داده را هدف قرار می‌دهد، مانند مقاصد بشردوستانه، نظارت بر همه‌گیری‌ها، و بلایای طبیعی و انسانی. با افزایش گرمایش جهانی و مسائل پناهندگان، این مورد ممکن است در آینده اهمیت بیشتری پیدا کند.

\subsubsection{انجام وظیفه در جهت منافع عمومی یا اعمال اقتدار رسمی محول شده به کنترل‌کننده}
\label{sec:17-5-4-public-task}

نهادهای عمومی خدمات عمومی (مانند آموزش، حمل‌ونقل) ارائه می‌دهند. برای این به‌اصطلاح «وظایف منافع عمومی»، آن‌ها ممکن است نیاز به پردازش داده‌های شخصی داشته باشند. این می‌تواند برای یک وظیفه خاص (مانند راه‌اندازی یک طرح کارت شناسایی الکترونیکی جدید توسط دولت) یا به موجب صلاحیت عمومی آن‌ها باشد (برای مثال، مقامات مالیاتی برای انجام صحیح وظیفه خود، نیاز به پردازش اظهارنامه مالیاتی فرد دارند تا میزان مالیات پرداختی را تعیین کنند) \lr{(Art. 29 WP, 2014, pp. 21–23)}.

با افزایش خصوصی‌سازی خدمات عمومی، مفهوم اقتدار عمومی گسترش یافته تا شامل نهادهایی شود که تابع رژیم‌های ترکیبی حقوق خصوصی-عمومی هستند (مانند شرکت‌های راه‌آهن)، یا در موارد خاص نهادهای کاملاً خصوصی که همچنان یک وظیفه منافع عمومی را اعمال می‌کنند (مانند انجمن حرفه‌ای پزشکی) \lr{(see Art. 29 WP, 2014, p. 22)}.

مقامات عمومی اقتدار و صلاحیت خود را از قانون ملی (یا اتحادیه اروپا) می‌گیرند (یا حتی یک اقدام اداری، که مفهوم قانون نسبتاً بزرگ است). در واقع، یک نهاد عمومی تنها در صورتی می‌تواند وجود داشته باشد که قانونی وجود داشته باشد که آن را (یا صلاحیت خاص آن را) پیش‌بینی کند \lr{(Voigt \& von dem Busche, 2017, pp. 107–108)}.

چنین قانونی باید دارای ویژگی‌های خاصی باشد. به ویژه، قانون باید با قانون حفاظت از داده‌ها منطبق باشد و باید صلاحیت‌هایی را به مقام عمومی اعطا کند که متناسب با هدف دنبال شده باشد \lr{(Voigt \& von dem Busche, 2017, p. 108)}. بنابراین، قانونی که به مقامات مالیاتی قدرت می‌دهد تا حساب‌های رسانه‌های اجتماعی شهروندان را برای یافتن سرنخ‌های فرار مالیاتی «وب‌کاوی» \lr{(web scrape)} کنند، احتمالاً متناسب نیست.

\subsubsection{انطباق با یک تعهد قانونی که کنترل‌کننده تابع آن است}
\label{sec:17-5-5-legal-obligation}

قانون تعهداتی را بر همه ما تحمیل می‌کند. گاهی اوقات، برای انطباق با تعهدات قانونی، کنترل‌کنندگان داده مجبور به پردازش داده‌های شخصی خواهند بود.

این مبنا، که مانند مورد قبلی بر اساس قانون است، با این حال سخت‌گیرانه‌تر است. برای اینکه پردازش برای این مبنا ضروری باشد، کنترل‌کننده نباید هیچ انتخابی جز پردازش داده‌ها داشته باشد (یعنی هیچ اختیار یا صلاحدیدی نداشته باشد) \lr{(Art. 29 WP, 2014, p. 19)}.

این امر بر کیفیت قانونِ مورد بحث پیامدهایی دارد. علاوه بر برآورده کردن تمام الزامات دیده شده در مبنای قبلی، قانونی که منجر به یک تعهد قانونی برای پردازش داده‌های شخصی می‌شود، باید در مورد پردازش داده‌های شخصی که الزام می‌کند نیز به اندازه کافی شفاف باشد \lr{(Art. 29 WP, 2014, p. 19)}.

\subsubsection{منافع مشروع کنترل‌کننده داده یا شخص ثالث}
\label{sec:17-5-6-legitimate-interests}

آخرین مبنایی که کنترل‌کنندگان داده می‌توانند برای آغاز پردازش داده‌های شخصی به آن تکیه کنند، شامل «توازن منافع» است. سوالات مرتبط شامل موارد زیر است: «کدام منافع وزن بیشتری دارند؟» «منافع کنترل‌کننده داده (یا شخص ثالث) یا منافع موضوع داده؟»

بسته به پاسخ این سوال بسیار ذهنی و وابسته به زمینه، پردازش داده‌های شخصی ممکن (یا ناممکن) خواهد بود. این مبنا پیچیده است و نیاز به بررسی بیشتر دارد.

\paragraph{منافع کنترل‌کننده داده: یک نفع مشروع}
\label{sec:17-5-6-1-controller-interest}
\mbox{}\\
برای اینکه یک کنترل‌کننده داده نفعی داشته باشد، به این معنی است که او در پردازش ذینفع است یا منفعتی دارد \lr{(Art. 29 WP, 2014, p. 24)}. چنین نفعی باید به وضوح بیان شود (یعنی برای درک کردن روشن باشد) و باید واقعی و فعلی باشد (نه حدسی).

یعنی نفع باید با فعالیت‌های جاری کنترل‌کننده داده یا آن‌هایی که به طور واقع‌بینانه در آینده بسیار نزدیک انتظار می‌روند، مطابقت داشته باشد (به عبارت دیگر، نمی‌توان شروع به پردازش داده‌ها کرد چون شاید ۲ سال دیگر نفعی داشته باشند که آن را توجیه کند) \lr{(Art. 29 WP, 2014, p. 24)}.

این نفع همچنین می‌تواند متعلق به یک شخص ثالث باشد، که برای بسیاری از شرکت‌هایی که داده‌ها را به نمایندگی از مشتریان خود پردازش می‌کنند، حیاتی است \lr{(Voigt \& von dem Busche, 2017, p. 105)}.

با این حال، مهم‌تر از همه، نفع کنترل‌کننده داده باید «مشروع» باشد. این بدان معناست که باید قانونی باشد (یعنی مطابق با قانون): نه تنها با قانون حفاظت از داده‌ها، بلکه با قوانین به‌طور کلی (شامل قانون‌گذاری، احکام قضایی، کدهای رفتاری)، و فراتر از آن، با اخلاقیات و انتظارات اجتماعی از آنچه پردازش آن مشروع است \lr{(Art. 29 WP, 2014, p. 25)}.

توجه داشته باشید که این زمینه اجتماعی و ارزش‌های اجتماعی می‌توانند در طول زمان تغییر کنند \lr{(Art. 29 WP, 2014, p. 25)}. برای مثال، زمانی که استفاده گسترده از دوربین‌های مداربسته \lr{(CCTVs)} برای مقاصد نظارتی در شهرها در دهه ۱۹۹۰ آغاز شد، دور از آن بود که مشروع تلقی شود \lr{(see, e.g., Coleman \& McCahill, 2011, p. 146)}. با این حال، امروزه همه ما به دوربین‌های مداربسته عادت کرده‌ایم و بحث به سمت اشکال مزاحم‌تر نظارت مانند تشخیص چهره حرکت کرده است.\footnote{برای مثال نگاه کنید به: \lr{https://edps.europa.eu/press-publications/press-news/blog/facial-recognition-solution-search-problem\_en}، آخرین دسترسی ۲۹ ژوئن ۲۰۲۰.}

کارگروه ماده ۲۹ مثال‌های مختلفی از منافعی که مشروع یا نامشروع هستند ارائه می‌دهد \lr{(Art. 29 WP, 2014, p. 25, 63, 68)}:

\vspace{0.3cm}
\noindent
\textbf{نفع مشروع:}
\begin{itemize}
	\item اعمال حق آزادی بیان و/یا اطلاعات توسط روزنامه یا سازمان مردم‌نهاد \lr{(NGO)}
	\item بازاریابی مستقیم متعارف
	\item امنیت شبکه \lr{IT}
\end{itemize}

\vspace{0.3cm}
\noindent
\textbf{نفع نامشروع:}
\begin{itemize}
	\item نظارت بر کارمندان برای تأیید بهره‌وری
	\item ترکیب اطلاعات شخصی در سراسر خدمات وب
\end{itemize}

\paragraph{منافع یا حقوق بنیادین موضوع داده}
\label{sec:17-5-6-2-data-subject-interest}
\mbox{}\\
نفع موضوع داده شامل تمام حقوق و آزادی‌های بنیادین آن‌ها (مانند حریم خصوصی، عدم تبعیض، محاکمه عادلانه) می‌شود و نیازی نیست که برای به رسمیت شناخته شدن، «مشروع» باشد \lr{(Art. 29 WP, 2014, pp. 29–30)}.

بنابراین آستانه پایین‌تر است و حتی در مواردی که موضوع داده به‌طور بالقوه درگیر یک فعالیت غیرقانونی شده است نیز اعمال می‌شود. برای مثال، دانلود غیرقانونی محتوای دارای کپی‌رایت، به خودی خود نظارت بر ترافیک اینترنت موضوع داده را توجیه نمی‌کند \lr{(Art. 29 WP, 2014, pp. 29–30)}. توازن منافع مختلفِ درگیر همچنان باید انجام شود.

\paragraph{توازن منافع}
\label{sec:17-5-6-3-balancing}
\mbox{}\\
برای تعیین اینکه آیا نفع کنترل‌کننده داده به اندازه کافی مشروع است، باید در برابر نفع موضوع داده سنجیده (یا توازن) شود. توازن منافع تعیین می‌کند که کدام نفع وزن بیشتری دارد و بنابراین، آیا عملیات پردازش می‌تواند انجام شود یا خیر \lr{(Art. 29 WP, 2014, p. 30)}. توازن منافع به‌طور ملموس از طریق تعدادی از گام‌ها انجام می‌شود که در زیر توضیح داده شده‌اند.

\vspace{0.3cm}
\noindent
\textbf{گام ۱: توصیف منافع \lr{(Qualify the Interests)}}

اولین گام توصیف منافع است. آیا آن‌ها بسیار جدی و الزام‌آور هستند، یا صرفاً جزئی؟ تا آنجا که به کنترل‌کننده داده مربوط می‌شود، می‌توان به طبقه‌بندی نشان داده شده در جدول ۱۷.۴ مراجعه کرد.

تا آنجا که به موضوع داده مربوط می‌شود، نفع آن‌ها همیشه بالاست زیرا پردازش داده‌های شخصی بنا به تعریف شامل حقوق بنیادین آن‌ها می‌شود (نگاه کنید به مقدمه این فصل). به همین دلیل است که باید نگاهی به تأثیر بالقوه‌ای داشت که عملیات پردازش برنامه‌ریزی شده بر منافع و حقوق آن‌ها خواهد گذاشت (گام دوم).

\begin{table}[h!]
	\centering
	\caption{توصیف منافع \lr{(Source: Compiled by author, building upon Art. 29 WP, 2014, pp. 34–36)}}
	\vspace{0.2cm}
	\begin{tabular}{|l|l|l|}
		\hline
		\textbf{دسته‌بندی نفع} & \textbf{جدیت} & \textbf{مثال} \\
		\hline
		حق بنیادین & بسیار زیاد & روزنامه‌نگاری تحقیقی \\
		\hline
		منافع عمومی & متوسط & تحقیقات پزشکی \\
		\hline
		نفع شخصی & کم & سود خصوصی \\
		\hline
	\end{tabular}
	\label{tab:17-4-qualification-interests}
\end{table}

\vspace{0.3cm}
\noindent
\textbf{گام ۲: تأثیر(ات) بر موضوع داده}

یک تأثیر می‌تواند به عنوان «روش‌های مختلفی که در آن یک فرد ممکن است تحت تأثیر پردازش داده‌های شخصی خود قرار گیرد — چه مثبت و چه منفی» تعریف شود \lr{(Art. 29 WP, 2014 p. 37)}. تأثیر می‌تواند ماهیت متفاوتی داشته باشد. آن‌ها می‌توانند عاطفی/اخلاقی (مانند ترس، پریشانی، شهرت)، مادی (مانند زیان مالی، تبعیض شغلی یا قیمتی، فیزیکی)، سیاسی (اثر دلسردکننده، خودسانسوری) و غیره باشند \lr{(see, Art. 29 WP, 2014, p. 37)}.

به خودی خود، درک این تأثیرات ممکن است دشوار به نظر برسد. بنابراین می‌توان به تعدادی از عوامل نگاه کرد که ارزیابی آن‌ها را روان‌تر می‌کند.

\vspace{0.3cm}
\noindent
\textbf{گام ۳: عوامل ارزیابی تأثیرات}

برای تعیین بهتر اینکه تأثیر چیست، می‌توان به عوامل ریسک زیر نگاه کرد. می‌توان به ماهیت یا نوع داده‌های شخصیِ در حال پردازش نگاه کرد \lr{(Art. 29 WP, 2014, p. 38)}. هر چه داده‌ها حساس‌تر باشند، تأثیر بیشتر است. داده‌های حساس می‌توانند به دسته‌های خاص داده‌های مندرج در \lr{GDPR} (داده‌های مربوط به سلامت، وابستگی سیاسی و غیره) اشاره داشته باشند، اما همچنین می‌توانند در معنای عمومی حساس باشند (مانند داده‌های کودکان، داده‌های موقعیت مکانی دقیق). برعکس، برخی داده‌ها می‌توانند کمتر حساس تلقی شوند، مانند داده‌هایی که موضوع داده قبلاً آن‌ها را به صورت عمومی در دسترس قرار داده است (مثلاً پروفایل حرفه‌ای آنلاین).

عامل دیگر، نوع پردازشِ درگیر است \lr{(Art. 29 WP, 2014, p. 39)}. این شامل تنوعی از عوامل مانند تعداد موضوعات داده، مقدار داده‌ها، یا تنوع داده‌های پردازش شده است. همچنین شامل تعداد کنترل‌کنندگان و/یا پردازشگرانی است که داده‌ها با آن‌ها به اشتراک گذاشته می‌شود. در نهایت، چه نوع عملیات پردازشی روی داده‌ها انجام می‌شود؟ آیا مشمول عملیات ساده و نسبتاً خوش‌خیم (مانند جمع‌آوری و اشتراک‌گذاری) است، یا در پایگاه‌های داده با ابعاد بالا ادغام می‌شود (بنابراین با سایر داده‌ها ترکیب می‌شود) و بیشتر تحت تحلیل‌های پیشرفته قرار می‌گیرد؟

عامل دیگر، احتمال \lr{(likelihood)} است. از یک سو، یک تأثیر بسیار محتمل نشان‌دهنده تأثیر بالاست. از سوی دیگر، یک تأثیر بسیار نامطمئن نیز می‌تواند نشان‌دهنده تأثیر بالا باشد (چون ما سرنخ کمی داریم که آیا اتفاق می‌افتد یا خیر) \lr{(Art. 29 WP, 2014, p. 38)}.

در نهایت، آخرین نوع عامل، «انتظارات معقول» موضوع داده است \lr{(Art. 29 WP, 2014, p. 40)}. این یک مفهوم مهم در قانون حفاظت از داده‌ها است. این به آنچه یک موضوع داده می‌تواند به‌طور معقول در یک زمینه خاص درباره آنچه برای داده‌هایش اتفاق می‌افتد انتظار داشته باشد، اشاره دارد. یعنی، آیا اگر داده‌هایشان پردازش شود یا تحت یک عملیات پردازش خاص قرار گیرد، متعجب خواهند شد \lr{(Dehon \& Carey, 2018, p. 59)}؟

به عبارت دیگر، این به زمینه از دیدگاه موضوع داده اشاره دارد. برای تعیین انتظارات معقول موضوع داده، می‌توان به زیر-عوامل زیر نگاه کرد. رابطه، موازنه قدرت، بین موضوع داده و کنترل‌کننده داده چیست؟ آیا آن‌ها در یک رابطه استخدامی درگیر هستند؟ آیا کنترل‌کننده داده یک مقام عمومی است؟ آیا کنترل‌کننده داده خدماتی را در یک وضعیت شبه‌انحصاری ارائه می‌دهد (مثلاً سرویس شبکه‌های اجتماعی محبوب)، یا برعکس یک شرکت کوچک با قدرت چانه‌زنی بسیار کم است؟ آیا موضوع داده خود آسیب‌پذیر است (مثلاً کودک، بیمار روانی، سالمند)؟ همچنین می‌توان به تعهدات قانونی یا قراردادی موجود بین موضوع داده و کنترل‌کننده داده نگاه کرد: پزشکان یا وکلا مشمول تعهدات محرمانگی هستند، و یک قرارداد نیز ممکن است وظایف محرمانگی مشابهی را پیش‌بینی کند. به طور کلی، هر چه زمینه جمع‌آوری خاص‌تر و محدودتر باشد، انتظارات معقول موضوع داده محدودتر است \lr{(Art. 29 WP, 2014, pp. 40–41)}.

\vspace{0.3cm}
\noindent
\textbf{گام ۴: توازن موقت \lr{(Provisional Balance)}}

در این نقطه، می‌توان یک توازن اولیه بین نفع مشروع کنترل‌کننده داده و تأثیرات بر منافع و حقوق بنیادین موضوع داده ایجاد کرد \lr{(Art. 29 WP, 2014, p. 41)}. عملِ توازن کردن به خودی خود — مانند علم داده — بیشتر هنر است تا علم. هیچ «قاعده عینی» وجود ندارد که بتواند کنترل‌کننده داده را راهنمایی کند تا تعیین کند کدام عناصر توازن وزن سنگین‌تری دارند.

این یک تصمیم زمینه‌ای و موردی است که باید بر اساس عوامل توصیف شده در بالا اتخاذ شود. با توجه به عوامل مختلفِ درگیر، باید روشن باشد که همه تأثیرات وزن یکسانی ندارند. برخی موارد واضح هستند، در حالی که برخی دیگر ممکن است بر نوعی تصمیم‌گیری «قاعده سرانگشتی» \lr{(rule of thumb)} تکیه کنند، با علم به اینکه شخص دیگری ممکن است در صورت قرار گرفتن در همان موقعیت، راه حل مخالف را انتخاب کند. به همین دلیل ارائه توضیح تصمیم و نگهداری سوابق آن حیاتی است \lr{(Art. 29 WP, 2014, p. 43)}. در هر صورت، در صورت شک، توصیه می‌شود که به نفع موضوع داده توازن برقرار شود \lr{(Dehon \& Carey, 2018, p. 58)}.

\vspace{0.3cm}
\noindent
\textbf{گام ۵: پادمان‌های اضافی \lr{(Additional Safeguards)}}

همان‌طور که ذکر شد، توازن برقرار شده تنها موقتی است. این بدان دلیل است که اگر توازن به نفع موضوع داده متمایل شود، هنوز امکان بهبود وضعیت با توسل به به‌اصطلاح «پادمان‌ها» وجود دارد. پادمان‌ها می‌توانند به عنوان مکانیسم‌های قانونی اضافی درک شوند که حفاظت بیشتری برای موضوع داده فراهم می‌کنند، و با انجام این کار، ممکن است کمک کنند تا توازن به نفع کنترل‌کننده داده متمایل شود \lr{(Art. 29 WP, 2014, pp. 41–42)}. این پادمان‌های اضافی نباید به عنوان یک راه حل «معجزه‌آسا» دیده شوند. هر چه تأثیر سنگین‌تر باشد، به پادمان‌های بیشتری نیاز خواهید داشت، و این احتمال وجود دارد که آن‌ها نتوانند توازن را دوباره به نفع کنترل‌کننده داده متمایل کنند \lr{(Art. 29 WP, 2014, p. 42)}.

برخی از این پادمان‌ها هم‌اکنون بخشی از \lr{GDPR} هستند (و در ادامه بررسی خواهند شد). آن‌ها شامل یک کاربرد «تقویت‌شده» از مقررات موجود هستند: برای مثال، ارائه شفافیت بیشتر از حد معمول برای روشن کردن این موضوع برای موضوع داده که چرا مبنای نفع مشروع انتخاب شده است، چگونه تأثیرات ارزیابی و توازن شده‌اند و غیره. امکان دیگر، کاهش مقدار داده‌های جمع‌آوری شده فراتر از آنچه تجویز شده است برای کاهش تأثیر بر موضوع داده است. امکان دیگر، نامستعارسازی \lr{(pseudonymisation)} داده‌هاست \lr{(Art. 29 WP, 2014, p. 42)}.

سایر پادمان‌ها به عنوان بخشی از \lr{GDPR} نیستند. آن‌ها برای مثال شامل امکان گنجاندن یک مکانیسم \lr{opt-out} بی قید و شرط (موضوع داده عملیات پردازش را بدون هیچ شرطی پایان می‌دهد)، یا ایجاد بندهای محرمانگی (کنترل‌کننده داده، داده‌ها را با دیگران به اشتراک نخواهد گذاشت) هستند \lr{(Art. 29 WP, 2014, pp. 42–43)}.

\vspace{0.3cm}
\noindent
\textbf{گام ۶: توازن نهایی \lr{(Final Balance)}}

در این نقطه، یک توازن نهایی برقرار می‌شود. این از همان اصول توازن موقت پیروی می‌کند و تعیین خواهد کرد که آیا پادمان‌ها تأثیرات را به اندازه کافی کاهش داده‌اند تا توازن اکنون به نفع کنترل‌کننده داده متمایل شود یا خیر.

% =================================================================
% بخش ۱۷.۶: ماده ۵ GDPR
% =================================================================

\subsection{ماده ۵ \lr{GDPR}: اصولی که باید در پردازش داده‌ها اعمال شوند}
\label{sec:17-6-art5-principles}

زمانی که مبنای مناسبی برای پردازش پیدا شد، آنگاه شروع پردازش داده‌های شخصی امکان‌پذیر است. اینجاست که ماده ۵ \lr{GDPR} وارد عمل می‌شود. این ماده حاوی تعدادی از اصول است که اگر قرار است پردازش قانونی باشد، باید رعایت شوند.

\subsubsection{اصل محدودیت هدف \lr{(Purpose Limitation)}}
\label{sec:17-6-1-purpose-limitation}

ماده ۵(۱)(ب) \lr{GDPR} بیان می‌کند که داده‌ها باید تنها برای اهداف مشخص، صریح و مشروع جمع‌آوری شوند و نباید به هیچ وجهی که با آن اهداف ناسازگار است، پردازش شوند. این اصل را می‌توان به دو نوع الزام تجزیه کرد.

اولین نوع الزام مربوط به «هدف اولیه» است: باید مشخص، صریح و مشروع باشد («تعیین هدف»).
دومین نوع الزام مربوط به پردازش داده‌های جمع‌آوری شده قبلی برای یک «هدف جدید» است: این هدف جدید باید با هدف اولیه جمع‌آوری سازگار باشد («محدودیت هدف به معنای دقیق کلمه»).

\paragraph{۱۷.۶.۱.۱ تعیین هدف: چرا؟}
شروع عملیات پردازش با تعیین و مشخص کردن هدف آن ضروری است، زیرا در غیر این صورت به این معنی خواهد بود که ما می‌توانیم داده‌ها را بدون دلیل پردازش کنیم. پردازش همیشه باید با توجه به یک هدف، ضروری باشد \lr{(Art. 29 WP, 2013, p. 11)}. همچنین، تعدادی از الزامات دیگر بر این واقعیت تکیه دارند که هدف به خوبی تعریف شده باشد (در ادامه ببینید).

\paragraph{۱۷.۶.۱.۲ هدف مشخص \lr{(Specific Purpose)}}
اولین عنصر الزام تعیین هدف این است که هدف به وضوح و به‌طور مشخص شناسایی شود. این بدان معناست که هدف به گونه‌ای تعریف شود که به اندازه کافی خاص باشد، با سطح کافی از جزئیات، تا بتوان تعیین کرد چه عملیات پردازشی تحت آن قرار می‌گیرد \lr{(Art. 29 WP, 2013, p. 15)}. مثالی از هر دو هدف مبهم و مشخص را می‌توان در جدول ۱۷.۵ یافت.

مشخص کردن هدف ممکن است مستلزم ارائه اطلاعات اضافی مانند نوع عملیات پردازش، مدت زمان آن، یا نوع داده‌های مورد بحث باشد \lr{(Art. 29 WP, 2013, p. 16)}.

یک مسئله کلیدی، مسئله «اهداف چتری» \lr{(umbrella purposes)} است. این به موقعیت‌هایی اشاره دارد که در آن تعدادی از اهداف تحت یک هدفِ گسترده‌ترِ واحد گروه‌بندی می‌شوند. این کار اگر عملیات پردازش مختلف به هم مرتبط باشند می‌تواند منطقی باشد (اگرچه جزئیات کافی باید برای هر یک از آن‌ها ارائه شود). آنچه باید اجتناب شود، استفاده از یک هدف چتری برای توجیه اهداف مختلفی است که به هم مرتبط نیستند \lr{(Art. 29 WP, 2013, p. 16)}.

% جدول ۱۷.۵
\begin{table}[h!]
	\centering
	\caption{هدف مبهم و مشخص \lr{(Source: Art. 29 WP, 2018b, p. 9)}}
	\vspace{0.2cm}
	\begin{tabular}{|p{0.45\linewidth}|p{0.45\linewidth}|}
		\hline
		\textbf{هدف مبهم} & \textbf{هدف مشخص} \\
		\hline
		«بهبود تجربه کاربران» & «ما تاریخچه خرید شما را با اشخاص ثالث به اشتراک می‌گذاریم تا محتوای متناسب برای خریدهای آینده را به شما ارائه دهیم» \\
		\hline
	\end{tabular}
	\label{tab:17-5-vague-specific}
\end{table}

\paragraph{۱۷.۶.۱.۳ هدف صریح \lr{(Explicit Purpose)}}
صریح بودن هدف به این معنی است که باید به وضوح آشکار، توضیح داده شده، یا به شکلی قابل فهم بیان شود \lr{(Art. 29 WP, 2013, p. 17)}. در حالی که الزامِ مشخص بودن هدف از منظر کنترل‌کننده داده ساخته می‌شود (آن‌ها باید آن را مشخص کنند)، این الزام از منظر موضوع داده ساخته می‌شود: موضوع داده نباید در درک اینکه هدف چیست مشکلی داشته باشد \lr{(Art. 29 WP, 2013, p. 17)}. به همین دلیل، این الزام می‌تواند به دو زیر-الزام تجزیه شود.

اول از همه، هدف باید «به آسانی قابل درک» باشد \lr{(Art. 29 WP, 2013, p. 17)}. این الزام مربوط به کیفیت زبان استفاده شده است که باید توسط همه (تمام موضوعات داده بالقوه، مقامات حفاظت از داده و غیره) قابل درک باشد. کنترل‌کنندگان داده باید این واقعیت را در نظر بگیرند که ممکن است موضوعات داده متفاوتی با نیازهای مختلف داشته باشند (مثلاً کودکان، سالمندان، افراد با مهارت‌های سواد مختلف). به منظور اطمینان از اینکه زبان روشن و بدون ابهام است، کنترل‌کننده داده باید اطمینان حاصل کند که جزئیات کافی وجود دارد (اما نه بیش از حد) و زبان استفاده شده ساده و واضح است. مثال کاملِ «بد»، شرایط و ضوابطی است که بیش از حد طولانی هستند و بر اصطلاحات حقوقی پیچیده تکیه دارند \lr{(Art. 29 WP, 2013, p. 17)}.

دوم، هدف باید «به آسانی قابل دسترسی» باشد \lr{(Art. 29 WP, 2013, p. 18)}. این الزام مربوط به سهولتی است که موضوعات داده می‌توانند اطلاعات را پیدا کنند، که باید روشن و متمایز باشد. با در نظر گرفتن مثال شرایط و ضوابط یک سرویس، این بدان معناست که موضوع داده نباید در یافتن اطلاعاتی که هدف را صریح می‌کند (هایلایت شده، پررنگ، لینک مشخص و غیره) مشکلی داشته باشد \lr{(Art. 29 WP, 2013, p. 18, 51–55)}.

\paragraph{۱۷.۶.۱.۴ هدف مشروع \lr{(Legitimate Purpose)}}
آخرین عنصر الزام تعیین هدف این است که هدف باید مشروع باشد. این بدان معناست که هدف باید قانونی باشد (یعنی مطابق با قانون)، نه تنها با قانون حفاظت از داده‌ها، بلکه با قوانین به‌طور کلی (شامل قانون‌گذاری، احکام قضایی، کدهای رفتاری). فراتر از این، باید با اخلاقیات و هنجارهای اجتماعی که زمینه‌ای هستند و می‌توانند در طول زمان تغییر کنند، منطبق باشد \lr{(Art. 29 WP, 2013, pp. 19–20)}.

\paragraph{۱۷.۶.۱.۵ هدف متفاوت \lr{(Different Purpose)}}
این اصل مربوط به وضعیتی است که داده‌ها برای یک هدف جمع‌آوری می‌شوند، اما کنترل‌کننده داده سپس می‌خواهد از آن برای هدف متفاوتی استفاده کند (مثلاً هدف دیگری که در زمان جمع‌آوری در نظر گرفته نشده بود). به‌طور معمول، اگر هدف جدید در هدف اصلی گنجانده نشده باشد، این کار نباید امکان‌پذیر باشد («محدودیت هدف به معنای دقیق کلمه»).

با این حال، اصل محدودیت هدف می‌گوید که چنین پردازشِ بعدی تا زمانی که این هدف جدید با هدف جمع‌آوری «سازگار» باشد، امکان‌پذیر خواهد بود. این بدان معناست که کنترل‌کننده داده نیاز دارد تا یک «آزمون سازگاری» انجام دهد. ایده اصلی نگاه کردن به رابطه بین اهداف است. هر چه رابطه نزدیک‌تر باشد، شانس سازگار بودن اهداف بیشتر است \lr{(Art. 29 WP, 2013, p. 21)}.

آزمون سازگاری در تعدادی از گام‌ها انجام می‌شود (همچنین نگاه کنید به ماده ۶(۴) \lr{GDPR}). اولین گام تعیین این است که آیا پیوند آشکاری بین اهداف وجود دارد یا خیر. این مورد زمانی صادق است که همپوشانی بین اهداف وجود داشته باشد: برای مثال، اگر پردازش بعدی کم و بیش در اهداف اولیه مستتر بوده باشد، یا اگر پیوندی (حتی جزئی) بین اهداف وجود داشته باشد \lr{(Art. 29 WP, 2013, p. 22)}. اگر پیوند آشکاری وجود نداشته باشد، آزمون دقیق‌تری باید انجام شود. همان‌طور که در جدول ۱۷.۶ نشان داده شده است، این آزمون شبیه به آزمون توازنی است که تحت مبنای نفع مشروع انجام می‌شود (نگاه کنید به بخش ۱۷.۵.۶).

% جدول ۱۷.۶
\begin{table}[h!]
	\centering
	\caption{مقایسه آزمون‌های توازن \lr{(Source: Compiled by author)}}
	\vspace{0.2cm}
	\small
	\begin{tabular}{|p{0.48\linewidth}|p{0.48\linewidth}|}
		\hline
		\textbf{گام‌های توازن منافع (بخش ۱۷.۵.۶)} & \textbf{گام‌های ارزیابی سازگاری اهداف} \\
		\hline
		۱. ارزیابی تأثیر بر موضوع داده: \newline (الف) لیست تأثیر بر موضوع داده \newline (ب) عوامل ارزیابی تأثیر: \newline -- نوع داده \newline -- نوع پردازش \newline -- احتمال تأثیرات \newline -- انتظارات معقول موضوع داده 
		& 
		۱. ارزیابی سازگاری بین اهداف: \newline (الف) پیوند بین اهداف \newline -- آشکار: نیازی به ادامه نیست \newline -- غیرآشکار: نیاز به بررسی سایر گام‌ها \newline (ب) انتظارات معقول موضوع داده \newline (ج) لیست تأثیر بر موضوع داده \newline (د) عوامل ارزیابی تأثیر: \newline -- نوع داده \newline -- نوع پردازش \newline -- احتمال تأثیرات \\
		\hline
		۲. توازن موقت & ۲. توازن موقت \\
		\hline
		۳. پادمان‌های اضافی & ۳. پادمان‌های اضافی \\
		\hline
		۴. توازن نهایی & ۴. توازن نهایی \\
		\hline
	\end{tabular}
	\label{tab:17-6-balancing-tests}
\end{table}

ارزیابی سازگاری تقریباً مشابه آزمون توازن منافع است. علاوه بر تعیین وجود یک پیوند آشکار بین اهداف، ترتیب گام‌ها نیز در ارزیابی سازگاری کمی تغییر می‌کند. این به دلیل تغییر موضوع ارزیابی است: به جای ارزیابی تأثیر بر موضوعات داده، فرد باید سازگاری بین اهداف را ارزیابی کند.

یکی از راه‌های ارزیابی چنین سازگاری، نگاه کردن به انتظارات معقول موضوع داده در زمینه است \lr{(Art. 29 WP 2013, pp. 24–25)}. گام دوم نگاه کردن و لیست کردن تأثیرات بر موضوع داده خواهد بود \lr{(Art. 29 WP, 2013, pp. 25–26)}. این تأثیرات می‌توانند از طریق همان عوامل (یعنی نوع داده، نوع پردازش، احتمال تأثیر)، منهای انتظارات معقول موضوع داده (که قبلاً بررسی شد)، بهتر ارزیابی شوند \lr{(see, Art. 29 WP, 2013, p. 26)}.

در این نقطه، ارزیابی سازگاری انجام می‌شود و می‌توان یک توازن موقت نیز بین دو هدف برقرار کرد: آیا دو هدف به اندازه کافی سازگار هستند که پردازش بعدی بتواند انجام شود \lr{(Art. 29 WP, 2013, p. 26)}؟ دوباره به خاطر داشته باشید که چیزها در یک «طیف» قرار دارند و این یک تصمیم «قاعده سرانگشتی» است.

گام بعدی البته استفاده از پادمان‌های اضافی است که می‌تواند تأثیر بر موضوع داده را کاهش دهد. همان نکات مربوط به مبنای نفع مشروع اعمال می‌شود، با در نظر گرفتن اینکه آنچه مناسب‌ترین پادمان در نظر گرفته می‌شود، بسته به زمینه تغییر خواهد کرد \lr{(Art. 29 WP, 2013, pp. 26–27)}.

در این نقطه، یک توازن نهایی برقرار می‌شود. این از همان اصول توازن موقت پیروی می‌کند و تعیین خواهد کرد که آیا پادمان‌ها تأثیرات را به اندازه کافی کاهش داده‌اند تا توازن اکنون به نفع پردازش بعدی متمایل شود یا خیر.

اگر توازن به نفع پردازشِ بعدی متمایل شود، آنگاه پردازش می‌تواند انجام شود. در صورتی که توازن علیه پردازشِ بعدی باشد، دومی نمی‌تواند انجام شود \lr{(Art. 29 WP, 2013, p. 36)}. این بدان معناست که یک مبنای جدید برای پردازش باید پیدا شود.

\subsubsection{کمینه‌سازی داده‌ها \lr{(Data Minimisation)}}
\label{sec:17-6-2-data-minimisation}

ماده ۵(۱)(ج) \lr{GDPR} مقرر می‌دارد که پردازش داده‌های شخصی باید «کافی، مرتبط و محدود به آنچه ضروری است» در رابطه با اهداف پردازش باشد. به عبارت دیگر، کنترل‌کننده داده باید پردازش داده‌ها را تا حد امکان به گونه‌ای به حداقل برساند که همچنان آن‌ها را قادر به دستیابی به هدف پردازش سازد \lr{(Voigt \& von dem Busche, 2017, p. 90)}.

راه‌های مختلفی برای به حداقل رساندن پردازش داده‌ها وجود دارد. می‌توان مقدار داده‌های اولیه جمع‌آوری شده را محدود کرد، نوع داده‌ها را محدود کرد، تعداد موضوعات داده را محدود کرد، تعداد افرادی که به داده‌ها دسترسی دارند را محدود کرد، نوع عملیات پردازشی انجام شده روی داده‌ها را محدود کرد، و غیره \lr{(also see Carey, 2018a, pp. 35–36)}.

\subsubsection{محدودیت ذخیره‌سازی \lr{(Storage Limitation)}}
\label{sec:17-6-3-storage-limitation}

ماده ۵(۱)(هـ) \lr{GDPR} مقرر می‌دارد که داده‌های شخصی نباید طولانی‌تر از حد لازم برای هدف پردازش نگهداری شوند. این اصل را می‌توان به عنوان ادامه اصل کمینه‌سازی داده‌ها در مورد مدت زمان ذخیره‌سازی داده‌ها در نظر گرفت. نکته این است که به محض اینکه هدفی که داده‌ها برای آن جمع‌آوری شده‌اند محقق شد، داده‌ها نباید بیشتر نگه داشته شوند. آن‌ها می‌توانند حذف/نابود شوند یا ناشناس‌سازی شوند (در این مورد، مطمئن شوید که ناشناس‌سازی غیرقابل برگشت است).

پیش‌بینی اینکه چه مدت لازم است داده‌ها ذخیره شوند ممکن است دشوار باشد. برای جلوگیری از نگهداری داده‌ها «محض احتیاط» \lr{(just in case)}، توصیه می‌شود یک «سیاست ذخیره‌سازی» ایجاد کنید. این سیاست محدودیت‌های زمانی برای ذخیره‌سازی را تعیین می‌کند و این محدودیت‌ها تابع بازنگری دوره‌ای هستند (مقدمه ۳۹، \lr{GDPR}؛ همچنین نگاه کنید به \lr{Carey, 2018a, p. 38}).

\subsubsection{تعهدات اضافی}
\label{sec:17-6-4-additional-obligations}

به عنوان راهی برای نتیجه‌گیری این مرور کلی بر ماده ۵ \lr{GDPR}، می‌توان به طور خلاصه به اصول باقی‌مانده اشاره کرد.

\paragraph{۱۷.۶.۴.۱ دقت داده‌ها \lr{(Data Accuracy)}}
طبق ماده ۵(۱)(د) \lr{GDPR}، کنترل‌کنندگان داده باید اطمینان حاصل کنند که داده‌ها دقیق هستند و در صورت لزوم به‌روز نگه داشته می‌شوند. این برای اطمینان از این است که داده‌های در اختیار کنترل‌کننده داده به درستی واقعیتِ موضوع داده را منعکس می‌کنند \lr{(also see Voigt \& von dem Busche, 2017, p. 91)}.

\paragraph{۱۷.۶.۴.۲ قانونی بودن، انصاف و شفافیت}
طبق ماده ۵(۱)(الف) \lr{GDPR}، پردازش تنها در صورتی می‌تواند انجام شود که در معنای عمومی قانونی باشد و اگر با قانون حفاظت از داده‌ها منطبق باشد \lr{(see Carey, 2018a, p. 33)}. علاوه بر این، الزام انصاف مستلزم منصف بودن با موضوع داده است، از جمله با در نظر گرفتن انتظارات معقول آن‌ها در مورد آنچه پردازش مستلزم آن است \lr{(see Dehon \& Carey, 2018, p. 43)}.

اصل شفافیت دو جنبه دارد \lr{(Dehon \& Carey, 2018, p. 44)}. از یک سو، هدف آن اطمینان از این است که موضوع داده به اندازه کافی در مورد پردازش مطلع شده است و بنابراین نیاز به ارائه حداقل اطلاعات به آن‌ها در مورد هویت کنترل‌کننده داده، هدف پردازش، نوع داده‌های جمع‌آوری شده و غیره دارد. از سوی دیگر، مربوط به کیفیت اطلاعات است که باید هم به آسانی قابل درک و هم به آسانی قابل دسترسی باشد (همچنین نگاه کنید به بخش‌های ۱۷.۵.۱ و ۱۷.۶.۱ این فصل).

\paragraph{۱۷.۶.۴.۳ یکپارچگی و محرمانگی}
طبق ماده ۵(۱)(و) \lr{GDPR}، کنترل‌کنندگان داده باید سطح مناسبی از امنیت داده‌هایی را که پردازش می‌کنند تضمین کنند، که شامل حفاظت در برابر پردازش غیرمجاز یا غیرقانونی، و در برابر از دست دادن تصادفی، نابودی یا آسیب است \lr{(Voigt \& von dem Busche, 2017, p. 92)}.

% =================================================================
% نتیجه‌گیری
% =================================================================

\vspace{0.8cm}
\noindent
\fcolorbox{black}{gray!15}{%
	\begin{minipage}{\dimexpr\linewidth-2\fboxsep-2\fboxrule}
		\vspace{0.3cm}
		\begin{center}
			\textbf{\Large نتیجه‌گیری}
		\end{center}
		\vspace{0.2cm}
		
		برای اینکه از نظر اجتماعی مسئول باشیم، به نظر می‌رسد بسیار مهم است که علم داده با قانون حفاظت از داده‌ها منطبق باشد. برخلاف حقوق و ارزش‌های گسترده مانند حریم خصوصی که گاهی تعیین حدود آن‌ها دشوار است، حفاظت از داده‌ها دارای تعدادی قوانین و اصول بسیار مشخص است.
		
		این فصل نشان داده است که مفهوم داده‌های شخصی گسترده‌تر از آن چیزی است که معمولاً تصور می‌شود. همچنین نشان داده است که آنچه به عنوان کنترل‌کننده داده یا پردازشگر داده نامیده می‌شود، یک مفهوم فنی است که لزوماً با درک عامیانه‌ای که ممکن است از آن‌ها داشته باشیم مطابقت ندارد.
		
		همچنین مبانی اجازه دهنده پردازش داده‌های شخصی را به تفصیل بررسی کرده است. توجه ویژه‌ای باید به آنچه یک رضایت معتبر را تشکیل می‌دهد، مبذول شود. این کار آنقدر که ممکن است باور داشته باشیم آسان نیست. به همان اندازه، انجام توازن منافع تحت مبنای نفع مشروع، تمرین ظریفی باقی می‌ماند که می‌تواند مورد مناقشه قرار گیرد. چنین توازنی همچنین در تعیین اینکه چه چیزی یک پردازشِ بعدیِ قابل قبول برای یک هدف متفاوت را تشکیل می‌دهد، یافت می‌شود. این به ما یادآوری می‌کند که تعریف مناسب یک هدف، کلید قانون حفاظت از داده‌ها است زیرا اجازه می‌دهد تا ضرورت و تناسب عملیات پردازش در نظر گرفته شده ارزیابی شود، که خود کلید یک تمرین مسئولانه علم داده است.
		\vspace{0.3cm}
	\end{minipage}
}

% =================================================================
% پرسش‌ها و پاسخ‌ها (ترجمه شده)
% =================================================================
\vspace{1cm}
\section*{پرسش‌ها و پاسخ‌ها}
\addcontentsline{toc}{section}{پرسش‌ها و پاسخ‌ها}

\noindent \textbf{\large ؟ پرسش‌ها}
\begin{enumerate}
	\setlength\itemsep{0.5em}
	\item سه روشی که اطلاعات می‌تواند به موضوع داده مرتبط شود چیست؟
	\item دو معیاری که می‌تواند کسی را به عنوان کنترل‌کننده داده تعیین کند چیست؟
	\item چه الزامی برای تمام مبانی پردازش به جز «رضایت» اعمال می‌شود؟
	\item چرا مشخص کردن هدف عملیات پردازش مهم است؟
	\item ارتباط بین اصول «کمینه‌سازی داده‌ها» و «محدودیت ذخیره‌سازی» چیست؟
\end{enumerate}

\vspace{0.5cm}
\noindent \textbf{\large $\checkmark$ پاسخ‌ها}

\vspace{0.3cm}
\noindent \textbf{\textit{مثال ۱: سوالات کلیدی (پاسخ به کادر ابتدای فصل)}}

اگر شرکت \lr{X} تنها یک هدف پردازش داشته باشد، (فروش داده‌ها) به معنای «پردازش بیشتر برای هدفی متفاوت» است. بنابراین باید آزمون سازگاری انجام شود.
با توجه به فقدان پیوند آشکار بین اهداف، باید به موارد زیر نگاه کرد: انتظارات معقول موضوع داده (هیچ انتظار این‌چنینی وجود ندارد)، تأثیرات بر موضوع داده (زیاد: مداخله در ترجیحات رأی‌دهی و عقاید سیاسی)، عوامل ارزیابی تأثیر (داده‌های بسیار حساس، ایجاد پروفایل، انتقال به اشخاص ثالث) و پادمان‌ها (هیچ موردی ذکر نشده است). بنابراین، این پردازش \textbf{غیرقانونی} است.

\vspace{0.3cm}
\noindent \textbf{\textit{پاسخ سوالات بالا}}
\begin{enumerate}
	\setlength\itemsep{0.5em}
	\item اطلاعات می‌تواند از نظر محتوا، هدف، یا نتیجه (که به عنوان تأثیر نیز شناخته می‌شود) به موضوع داده مرتبط باشد.
	\item کنترل‌کننده داده بازیگری است که «هدف» و/یا «ابزارهای ضروری» پردازش را تعیین می‌کند. تحقق یکی از این شروط برای واجد شرایط شدن به عنوان کنترل‌کننده داده کافی است (که در این صورت به احتمال زیاد وضعیت کنترل‌کنندگی مشترک است).
	\item «ضرورت» پردازش باید تعیین شود.
	\item اگر هدفی وجود نداشته باشد (یا به خوبی تعریف نشده باشد)، بدین معناست که ما می‌توانیم داده‌ها را برای هر دلیلی که می‌خواهیم و بدون هیچ محدودیتی پردازش کنیم. همچنین، تعدادی از الزامات کلیدی برای ارزیابی کفایت، ارتباط یا ضرورت پردازش تنها در صورتی قابل رعایت هستند که هدف به خوبی تعریف شده باشد (مانند ضرورت، کمینه‌سازی داده‌ها، محدودیت ذخیره‌سازی).
	\item اصل محدودیت ذخیره‌سازی می‌تواند به عنوان کاربرد اصل کمینه‌سازی داده‌ها در مسئله «مدت زمان» ذخیره‌سازی داده‌ها درک شود.
\end{enumerate}

% =================================================================
% پیام‌های کلیدی (Take-Home Message) - ترجمه شده
% =================================================================
\vspace{0.8cm}
\noindent
\fcolorbox{black}{gray!15}{%
	\begin{minipage}{\dimexpr\linewidth-2\fboxsep-2\fboxrule}
		\vspace{0.3cm}
		\centering \textbf{\large پیام‌های کلیدی فصل}
		\vspace{0.2cm}
		\begin{itemize}
			\setlength\itemsep{0.5em}
			\item[\textbf{--}] داده‌های شخصی مفهوم گسترده‌ای است که اکثر داده‌های پردازش شده در فناوری‌های پردازش داده امروزی را در بر می‌گیرد.
			\item[\textbf{--}] تمایز بین یک کنترل‌کننده داده و یک پردازشگر داده می‌تواند دشوار و فریبنده باشد.
			\item[\textbf{--}] برای شروع پردازش داده‌ها، فرد بین شش مبنای مختلف حق انتخاب دارد؛ با این حال، حتماً «یک» مبنا باید انتخاب شود.
			\item[\textbf{--}] هنگام پردازش داده‌ها، تمام مفاد ماده ۵ مقررات \lr{GDPR} باید رعایت شوند.
		\end{itemize}
		\vspace{0.2cm}
	\end{minipage}
}

% =================================================================
% منابع (انگلیسی)
% =================================================================
\vspace{1cm}
\section*{منابع}
\addcontentsline{toc}{section}{منابع}

\begin{latin}
	\begin{itemize}
		\setlength\itemsep{0.6em}
		
		\item[] Art. 29 WP. (2013). Opinion 03/2013 on Purpose Limitation.
		
		\item[] Art. 29 WP. (2014). Opinion 06/2014 on the Notion of Legitimate Interests of the Data Controller under Article 7 of Directive 95/46/EC.
		
		\item[] Art. 29 WP. (2007). Opinion 4/2007 on the concept of personal data.
		
		\item[] Art. 29 WP. (2010). Opinion 1/2010 on the concepts of “controller” and “processor.”
		
		\item[] Art. 29 WP. (2018a). Article 29 Working Party Guidelines on consent under Regulation 2016/679.
		
		\item[] Art. 29 WP. (2018b). Guidelines on transparency under Regulation 2016/679.
		
		\item[] Carey, P. (2018a). Data protection principles. In P. Carey (Ed.), \textit{Data protection: A practical guide to UK and EU law} (5th ed., pp. 32–41). Oxford University Press.
		
		\item[] Coleman, R., \& McCahill, M. (2011). \textit{Surveillance \& crime}. Sage.
		
		\item[] Dehon, E., \& Carey, P. (2018). Fair, lawful, and transparent processing. In P. Carey (Ed.), \textit{Data protection: A practical guide to UK and EU law} (5th ed., pp. 42–65). Oxford University Press.
		
		\item[] European Parliament resolution of 16 February 2017 with recommendations to the Commission on Civil Law Rules on Robotics (2015/2103(INL)).
		
		\item[] Gellman, R. (2019). FAIR INFORMATION PRACTICES: A Basic History.
		
		\item[] Gutwirth, S. (2002). \textit{Privacy and the Information Age}. Rowman \& Littlefield.
		
		\item[] Hoofnagle, C. J., Sloot, B. van der., \& Zuiderveen Borgesius, F. (2019). ‘The European Union General Data Protection Regulation: What It Is and What It Means’. \textit{28 Information and Communications Technology Law} 65.
		
		\item[] Mourby, M., Mackey, E., Elliot, M., Gowans, H., Wallace, S. E., Bell, J., et al. (2018). Are “pseudonymised” data always personal data? Implications of the GDPR for administrative data research in the UK. \textit{Computer Law and Security Review}, 34(2), 222–233.
		
		\item[] Regulation (EU) 2016/679 of the European Parliament and of the Council of 27 April 2016 on the protection of natural persons with regard to the processing of personal data and on the free movement of such data, and repealing Directive 95/46/EC (General Data Protection Regulation), [2016], OJ L 119/1.
		
		\item[] Rodway, S., \& Carey, P. (2018). Outsourcing personal data processing. In P. Carey (Ed.), \textit{Data protection: A practical guide to UK and EU law} (5th ed., pp. 175–183). Oxford University Press.
		
		\item[] Voigt, P., \& von dem Busche, A. (2017). \textit{The EU General Data Protection Regulation (GDPR), a practical guide}. Springer.
		
		\item[] Welfare, D., \& Carey, P. (2018). Territorial scope and terminology. In P. Carey (Ed.), \textit{Data protection: A Practical guide to UK and EU law} (5th ed., pp. 1–31). Oxford University Press.
		
	\end{itemize}
\end{latin}
% =================================================================
% فصل ۱۸: دیدگاه‌هایی از قانون مالکیت فکری (نسخه نهایی)
% =================================================================

\raggedbottom 

% تنظیم شماره‌ها برای فصل ۱۸
\setcounter{section}{18}
\setcounter{subsection}{0}
\renewcommand{\thesection}{18}
\renewcommand{\thesubsection}{18.\arabic{subsection}}

% -----------------------------------------------------------------
% بلوک عنوان و نویسنده
% -----------------------------------------------------------------
\noindent
\fcolorbox{black}{gray!15}{%
	\begin{minipage}{\dimexpr\linewidth-2\fboxsep-2\fboxrule}
		\vspace{0.5cm}
		\begin{center}
			\textbf{\huge دیدگاه‌هایی از قانون مالکیت فکری}
			\vspace{0.4cm}
			
			\large
			\textit{لیزا ون دونگن} \\
			\lr{\textit{Lisa van Dongen}}
		\end{center}
		\vspace{0.5cm}
	\end{minipage}
}

\vspace{0.8cm}

% -----------------------------------------------------------------
% فهرست مطالب فصل ۱۸
% -----------------------------------------------------------------
\noindent
\textbf{\Large فهرست مطالب}
\vspace{0.3cm}

{ \small
	\noindent
	\textbf{۱۸.۱} \hspace{0.3cm} \textbf{مقدمه} \par
	\vspace{0.2cm}
	
	\noindent
	\textbf{۱۸.۲} \hspace{0.3cm} \textbf{برآورده کردن معیارها} \par
	\vspace{0.1cm}
	\hspace{0.8cm} ۱۸.۲.۱ \hspace{0.1cm} الزامات شکلی حق تکثیر \lr{(Copyright)} \par
	\hspace{0.8cm} ۱۸.۲.۲ \hspace{0.1cm} حق خاص پایگاه داده \lr{(Sui Generis)} \par
	\hspace{0.8cm} ۱۸.۲.۳ \hspace{0.1cm} حق اسرار تجاری \par
	\hspace{0.8cm} ۱۸.۲.۴ \hspace{0.1cm} خلاصه \par
	\vspace{0.2cm}
	
	\noindent
	\textbf{۱۸.۳} \hspace{0.3cm} \textbf{دامنه حفاظت} \par
	\vspace{0.1cm}
	\hspace{0.8cm} ۱۸.۳.۱ \hspace{0.1cm} حق تکثیر: موضوع مورد حمایت \par
	\hspace{0.8cm} ۱۸.۳.۲ \hspace{0.1cm} حفاظت خاص پایگاه داده \par
	\hspace{0.8cm} ۱۸.۳.۳ \hspace{0.1cm} حق اسرار تجاری \par
	\hspace{0.8cm} ۱۸.۳.۴ \hspace{0.1cm} خلاصه \par
	\vspace{0.2cm}
	
	\noindent
	\textbf{۱۸.۴} \hspace{0.3cm} \textbf{استثنائات و محدودیت‌ها} \par
	\vspace{0.1cm}
	\hspace{0.8cm} ۱۸.۴.۱ \hspace{0.1cm} محدودیت‌های حقوق \par
	\hspace{0.8cm} ۱۸.۴.۲ \hspace{0.1cm} استثنائات: وجه مشترک \par
	\hspace{0.8cm} ۱۸.۴.۳ \hspace{0.1cm} استثنائات خاص برای هر حق \par
	\vspace{0.2cm}
	
	\noindent
	\textbf{۱۸.۵} \hspace{0.3cm} \textbf{منابع جایگزین} \par
	\vspace{0.3cm}
	\noindent
	\textbf{منابع برای مطالعه بیشتر}
}

\vspace{0.8cm}

% -----------------------------------------------------------------
% اهداف یادگیری
% -----------------------------------------------------------------
\noindent
\fcolorbox{black}{gray!15}{%
	\begin{minipage}{\dimexpr\linewidth-2\fboxsep-2\fboxrule}
		\vspace{0.3cm}
		\textbf{\large اهداف یادگیری}
		
		\begin{itemize}
			\setlength\itemsep{0.5em}
			\item[\textbf{--}] حق تکثیر \lr{(copyright)}، حقوق خاص پایگاه داده \lr{(sui generis)} و اسرار تجاری شامل چه مواردی می‌شوند و چگونه می‌توان ذینفعان آن‌ها را تعیین کرد.
			
			\item[\textbf{--}] چه زمانی و چگونه استفاده از مجموعه داده‌های شخص ثالث توسط این حقوق محدود می‌شود و چه زمانی محدود نمی‌شود.
			
			\item[\textbf{--}] پتانسیل و محدودیت‌های منابع جایگزین برای تکمیل یا جایگزینی مجموعه داده‌های شخص ثالث، مانند حقوق قابلیت انتقال داده‌ها و اطلاعات بخش عمومی.
		\end{itemize}
		\vspace{0.2cm}
	\end{minipage}
}

\newpage

% =================================================================
% ۱۸.۱ مقدمه
% =================================================================
\subsection{مقدمه}
\label{sec:18-introduction}

اطلاعات بیشتر و بیشتری از طریق استفاده از دستگاه‌های هوشمند (مانند ترموستات هوشمند، تلفن هوشمند)، خدمات اینترنتی (مانند گوگل و فیس‌بوک)، حسگرها (مثلاً در اتومبیل‌ها، خانه‌های هوشمند و شهرها) و دوربین‌ها جمع‌آوری می‌شود. مجموعه داده‌های حاصل حاوی اطلاعات زیادی درباره افراد، و همچنین درباره جامعه به طور کلی هستند.

این مجموعه داده‌ها به ناظران خود اجازه می‌دهند تا مشکلات را شناسایی کرده و راه‌های رفع آن‌ها را بررسی کنند، اما همچنین فرصت‌ها را شناسایی کرده و راه‌های بهره‌برداری از آن‌ها را کاوش کنند. برای مثال، با مطالعه اطلاعات حاصل از حسگرهای اتومبیل‌ها، حسگرها و دوربین‌های متمرکز بر جاده‌ها و سیستم‌های چراغ راهنمایی، می‌توان علل تصادفات رانندگی را شناسایی کرد و راه‌حل‌هایی برای کاهش تعداد تصادفات در یک بلوک خاص پیشنهاد داد.

با این حال، دسترسی به چنین مجموعه داده‌هایی که توسط دیگران تولید شده‌اند، اغلب محدود است. گروه بزرگی از بازیگران وجود دارند که نمی‌خواهند دیگران از داده‌های «آن‌ها» استفاده کنند. یک عامل بسیار مهم که به چنین بازیگرانی کمک می‌کند تا دسترسی به مجموعه داده‌های خود را محدود کنند، «قانون مالکیت فکری» \lr{(intellectual property law)} است. دارنده حقوق مالکیت فکری بر روی یک مجموعه داده، توانایی محدود کردن دسترسی دیگران به (بخش‌هایی از) مجموعه داده خود و همچنین اعمال محدودیت‌هایی بر استفاده از آن را دارد.

برای درک چگونگی پیمایش در این حوزه حقوقی، ابتدا مهم است که درک کنیم حقوق مالکیت فکری چه هدفی را دنبال می‌کنند. همان‌طور که در «دستورالعمل اجرایی» \lr{(Enforcement Directive)} بیان شده است، دلیل اصلی زیربنایی در سیستم‌های فعلی قانون مالکیت فکری، ایجاد انگیزه (سرمایه‌گذاری) در نوآوری است. حقوق مالکیت فکری به عنوان حقوق مالکیت مصنوعی برای اصلاح برخی «شکست‌های بازار» ایجاد شده‌اند.

بازار را به عنوان زمینی پر از میوه تصور کنید. اگر همه آزاد باشند که بدون هیچ محدودیتی از زمین و میوه‌های آن استفاده کنند، احتمالاً بسیاری این کار را خواهند کرد. با این حال، بعید است که همه کسانی که از زمین استفاده می‌کنند، به صورت جداگانه در آن سرمایه‌گذاری کنند. این به دلیل عدم اطمینان از این موضوع است که آیا این سرمایه‌گذاری نتیجه‌ای برای آن‌ها خواهد داشت یا حتی به آن‌ها اجازه می‌دهد سرمایه خود را بازگردانند؛ زیرا همه آزادند بدون محدودیت از زمین استفاده کنند.

زمانی که عواید پیش‌بینی شده کمتر باشد، افراد کمتری مایل به سرمایه‌گذاری خواهند بود. علاوه بر این، هر سرمایه‌گذاری که انجام شود احتمالاً کوچکتر خواهد بود. اینجاست که حقوق مالکیت فکری وارد می‌شود. آن‌ها ابزارهایی برای اصلاح این «شکست بازار» هستند که با اعطای مجموعه‌ای از حقوق انحصاری برای مدت محدود به کسانی که در نوآوری سرمایه‌گذاری می‌کنند، پاداش می‌دهند. این حقوق ابزارهایی برای دارنده حق هستند تا دسترسی و استفاده از مالکیت فکری خود را به صورت قانونی محدود کند. این امر به دارنده حق اجازه می‌دهد تا قیمت‌های بالاتری را برای بازگشت سرمایه و کسب سود تعیین کند.

این فصل قصد دارد مقدمه‌ای بر مبانی حقوق مالکیت فکری در اتحادیه اروپا ارائه دهد. این فصل از ساده‌سازی‌هایی استفاده می‌کند و همیشه تصویر کاملی ارائه نمی‌دهد تا درک مطالب را به حداکثر برساند. چنین ساده‌سازی‌هایی عموماً ذکر می‌شوند و منابعی در مورد این موضوع برای کسانی که مایل به درک عمیق‌تر این مفهوم کمتر توسعه‌یافته هستند، در مراجع گنجانده شده است. بنابراین، این فصل نباید به عنوان جایگزینی برای مشاوره حقوقی یا به عنوان مبنایی برای بحث‌های آکادمیک استفاده شود.

علاوه بر این، در حالی که انواع مختلفی از حقوق مالکیت فکری وجود دارد، در اینجا تنها «حق تکثیر» \lr{(copyright)}، «حقوق خاص پایگاه داده» \lr{(sui generis database rights)} و «اسرار تجاری» \lr{(trade secrets)} مورد بحث قرار خواهند گرفت. تحت چارچوب حقوقی اتحادیه اروپا، موضوع و شرایط این حقوق مالکیت فکری ارتباط نزدیکی با داده‌ها و نرم‌افزار دارند، همان‌طور که در ادامه فصل مشخص خواهد شد.

سایر حقوق مانند حق اختراع \lr{(patents)} در حال حاضر نقش پیچیده‌تری در اتحادیه اروپا در زمینه داده‌ها و نرم‌افزار ایفا می‌کنند، از جمله به دلیل محدودیت‌ها در قابلیت ثبت اختراعِ موضوعاتی مانند روش‌های ریاضی و برنامه‌های کامپیوتری به عنوان چنین. چنین محدودیت‌هایی همچنین شروع به ایفای نقش بیشتری در مثلاً ایالات متحده کرده‌اند، همان‌طور که می‌توان از رویه قضایی دیوان عالی آن‌ها (یعنی در مورد مفهوم «ایده انتزاعی») بین سال‌های ۲۰۱۰ و ۲۰۱۴ استنباط کرد. بنابراین، این موضوع نیاز به توجه بیشتری دارد که در این نوشتار محدود نمی‌گنجد.

بنابراین، این فصل تنها بر این حقوق خاص مالکیت فکری از دیدگاه اتحادیه اروپا تمرکز دارد. سوالات بررسی شده در بخش‌های بعدی بر ایجاد این موضوع برای هر یک از این حقوق مالکیت فکری تمرکز خواهند کرد که چه زمانی اعمال می‌شوند (بخش ۱۸.۲)، و به دنبال آن، این موضوع برای استفاده شخص ثالث از داده‌ها چه معنایی دارد (بخش ۱۸.۳)، و همچنین محدودیت‌ها و استثنائات (بخش ۱۸.۴). این فصل با بحث در مورد راه‌های دسترسی قانونی به مجموعه داده‌های تحت پوشش یک یا چند مورد از این حقوق مالکیت فکری و منابع جایگزین پایان می‌بریم.

% =================================================================
% ۱۸.۲ برآورده کردن معیارها
% =================================================================
\subsection{برآورده کردن معیارها}
\label{sec:18-meeting-criteria}

% -----------------------------------------------------------------
% ۱۸.۲.۱ الزامات شکلی حق تکثیر
% -----------------------------------------------------------------
\subsubsection{الزامات شکلی حق تکثیر}

یک چیز ممکن است تحت حمایت قانون کپی‌رایت قرار گیرد اگر سه معیار تجمعی \lr{(cumulative criteria)} برای حفاظت کپی‌رایت را برآورده کند. طبق «کنوانسیون برن» \lr{(Berne Convention)}، این معیارها مستلزم آن هستند که اثر: (۱) یک «بیان/ابراز» \lr{(expression)} باشد، (۲) که «اصیل» \lr{(original)} باشد، (۳) در حوزه «ادبیات و هنر» باشد. در رژیم کپی‌رایت اتحادیه اروپا، عواملی مانند کار یا سرمایه‌گذاری بی‌ربط هستند.

سه عنصر از یک مجموعه داده وجود دارد که قادر به برآورده کردن این الزامات هستند:
\begin{itemize}
	\setlength\itemsep{0.5em}
	\item محتویات مجموعه داده
	\item انتخاب داده‌ها
	\item چیدمان/آرایش داده‌ها
\end{itemize}

اگر یک یا چند مورد از این عناصر الزامات شکلی را برآورده کنند، ممکن است حفاظت کپی‌رایت بر روی آن عناصر مجموعه داده وجود داشته باشد. در این صورت، محدودیت‌های قانونی برای استفاده از آن وجود خواهد داشت. بنابراین مهم است که این معیارها را درک کنیم تا بتوانیم احتمال حفاظت کپی‌رایت بر روی یک مجموعه داده را برای اطمینان از استفاده قانونی تعیین کنیم.

فرض اساسی معیار «بیان» \lr{(expression)} این است که حقایق و ایده‌ها خلق نمی‌شوند، بلکه کشف می‌شوند. این موضوع همچنین در پرونده \lr{Feist Publications, Inc., v. Rural Telephone Service Co.} تأیید شده است، که نشان می‌دهد ایالات متحده و اتحادیه اروپا به شیوه‌ای مشابه با این معیار برخورد می‌کنند. آنچه این موضوع برای کپی‌رایت معنا می‌دهد این است که کپی‌رایت از «آنچه گفته می‌شود» محافظت نمی‌کند، بلکه از «چگونگی بیان آن» محافظت می‌کند.

یک قاعده سرانگشتی خوب، نگاه کردن به آن به عنوان یک طیف است که در آن حقایق و ایده‌ها در یک طرف و بیان‌ها در طرف دیگر بر اساس خاص بودن \lr{(specificity)} قرار دارند. در اصل، هر چه یک حقیقت یا ایده خاص‌تر شود، عقربه عموماً به سمت بیان نزدیک‌تر می‌شود. استدلال پشت این موضوع این است که یک نویسنده می‌تواند یک حقیقت یا ایده را با انتخاب کلمات خود منتقل کند و بدین ترتیب چیزی فراتر و جدا از آن حقیقت یا ایده خلق کند. برای مثال، به تفاوت جزئیات در جملات جدول ۱۸.۱ نگاه کنید.

% جدول ۱۸.۱
\begin{table}[h!]
	\centering
	\caption{بیان \lr{(Expression)}}
	\vspace{0.2cm}
	\begin{tabular}{|p{0.45\linewidth}|p{0.45\linewidth}|}
		\hline
		\textbf{جمله نمونه} & \textbf{سطح جزئیات: واقعیت/ایده/بیان} \\
		\hline
		این خانه سبز است. & جزئیات بسیار کم و بسیار کلی \newline (واقعیت/ایده) \\
		\hline
		این خانه سه طبقه، سه طیف مختلف از رنگ سبز است. & جزئیات بیشتر، اما هنوز کاملاً کلی \newline (واقعیت/ایده، اما بیشتر به سمت بیان) \\
		\hline
		این محل سکونت سه طبقه، ترکیبی از طیف‌های سبز است، از جمله زیتونی، خزه‌ای و حتی برخی رگه‌های سبز متالیک در گوشه‌های پنجره‌ها و درهایش. & جزئیات زیاد و بسیار خاص \newline (بیان) \\
		\hline
	\end{tabular}
	\small{\newline \textbf{نکته جدول:} جدول متعلق به نویسنده است.}
	\label{tab:18-1-expression}
\end{table}

این الزام یک مانع احتمالی برای حفاظت کپی‌رایت بر روی یک مجموعه داده است. برای مثال، داده‌ها در چنین مجموعه داده‌هایی با هم ممکن است تصویر بسیار خاصی ایجاد کنند، اما اگر داده‌ها صرفاً به عنوان متغیرهایی در یک جدول نمایش داده شوند، داده‌ها فاقد «بیان» هستند.

در پرونده \lr{Football DataCo Ltd.}، عنصر دوم — اصالت \lr{(originality)} — به عنوان «حاشیه اختیار» برای انجام انتخاب‌های آزادانه و خلاقانه که مورد استفاده قرار گرفته، درک شد. به زبان ساده‌تر، این مستلزم آن است که خالق «مُهر شخصی» خود را بر روی آن زده باشد. البته این نباید به معنای تحت‌اللفظی گرفته شود. برای مثال، قرار دادن لوگوی خود روی چیزی آن را اصیل نمی‌کند.

سطح برآورده کردن این معیار در عمل خیلی بالا نیست. چنین انتخاب‌های خلاقانه‌ای می‌تواند به سادگیِ انتخاب نورپردازی، پس‌زمینه و زاویه برای گرفتن یک عکس، یا انتخاب کلمات در یک متن یا کد باشد. با این حال، مهم است تأکید شود که باید فضایی برای انجام چنین انتخاب‌هایی توسط خالق وجود داشته باشد.

برای مثال، یک عکس پاسپورت باید تعدادی از الزامات سخت‌گیرانه را برآورده کند. چنین تنظیمات از پیش تعیین شده‌ای بر فضایی که عکاس برای تصمیم‌گیری‌های خلاقانه خود دارد، تأثیر می‌گذارد. بنابراین برای یک عکس پاسپورت، بسیار بعید است که عکاس بتواند الزام اصالت را برآورده کند.

مثال دیگر الزامات عملکردی است. کد نرم‌افزار از زمان پیدایش «دستورالعمل نرم‌افزار» \lr{(Software Directive)} قادر به جذب حفاظت کپی‌رایت است، اما همان‌طور که در پرونده \lr{Bezpečnostní softwarová asociace} تأیید شد، بیان در کد اگر «توسط عملکرد فنی آن‌ها دیکته شده باشد»، نمی‌تواند به حد اصالت برسد.

بنابراین، شرایطی که نویسنده فضایی برای انجام انتخاب‌های خلاقانه داشته باشد، برای برآورده کردن الزام اصالت حیاتی است. علاوه بر این، در غیاب چنین الزاماتی، هنوز این مسئله وجود دارد که آیا انتخاب‌های خلاقانه واقعاً انجام شده‌اند یا خیر. انتخاب و/یا چیدمان داده‌ها در یک مجموعه داده می‌تواند، برای مثال، حداقل آستانه خلاقیت را برآورده کند، اما این انتخاب‌ها در عمل عموماً بر اساس مطلوبیت \lr{(utility)} انجام می‌شوند؛ انتخاب‌های انجام شده در انتخاب داده‌ها اغلب توسط کسب‌وکار اصلی یک شرکت تعیین می‌شوند و داده‌ها به دلایل عملی مانند ترتیب حروف الفبا یا تاریخ چیده می‌شوند.

آخرین معیار مستلزم آن است که اثر در حوزه «ادبیات و هنر» باشد. آنچه هنر یا ادبیات را تشکیل می‌دهد در رژیم کپی‌رایت بسیار گسترده درک می‌شود. برای مثال، ادبیات برای اهداف حفاظت کپی‌رایت می‌تواند اساساً شامل هر چیزی که شامل کلمه مکتوب است باشد. همان‌طور که در بالا ذکر شد، حتی می‌تواند کد نرم‌افزار را نیز پوشش دهد. این بدان معناست که داده‌ها — چه عددی و چه متنی — نیز در این دسته گسترده قرار می‌گیرند.

برخی دیگر از نمونه‌های آثار که ممکن است محافظت شوند عبارتند از کتاب‌ها، نقاشی‌ها، طرح‌ها، نقشه‌ها، معماری، مواد طراحی مقدماتی برای کد نرم‌افزار، فیلم‌ها، ترکیبات موسیقی، اشعار، توپوگرافی، آثار رقص‌پردازی و غیره: ماده ۲(۱) کنوانسیون برن شامل بیش از ۲۰ نمونه از انواع آثاری است که در حوزه ادبیات و هنر قرار می‌گیرند.

% =================================================================
% ۱۸.۲.۲ حق خاص پایگاه داده
% =================================================================
\subsubsection{حق خاص پایگاه داده}

اگر موادی مانند مجموعه داده‌ها و مواد طراحی مقدماتی برای کد نرم‌افزار بخشی از یک پایگاه داده باشند، استفاده از آن‌ها ممکن است توسط «حفاظت خاص پایگاه داده» محدود شود. به دلیل حفاظت محدودی که توسط رژیم کپی‌رایت در پایگاه‌های داده ارائه می‌شد، «دستورالعمل پایگاه داده» \lr{(Database Directive)} در سال ۱۹۹۶ برای تقویت بیشتر اقتصاد اطلاعات در اتحادیه اروپا تصویب شد. تا به امروز، این رژیم همچنان یک ابداع اروپایی است (برای مثال، هیچ معادل دقیقی در ایالات متحده ندارد).

حق خاص پایگاه داده بنابراین از پایگاه‌های داده‌ای محافظت می‌کند که فاقد اصالت هستند. با این حال، این بدان معنا نیست که اگر حفاظت کپی‌رایت بر روی محتوا، انتخاب و/یا چیدمان پایگاه داده وجود داشته باشد، نمی‌تواند حفاظت خاص پایگاه داده نیز وجود داشته باشد. این دو حق می‌توانند روی یک پایگاه داده واحد همزیستی داشته باشند.

یک مجموعه داده به احتمال زیاد تحت پوشش این حق قرار می‌گیرد اگر: (۱) یک «پایگاه داده» باشد که برای آن (۲) «سرمایه‌گذاری مرتبط» انجام شده است (۳) که آن سرمایه‌گذاری «قابل توجه» \lr{(substantial)} باشد.

برای اینکه یک مجموعه داده شرط اول — یعنی پایگاه داده بودن — را برآورده کند، ابتدا لازم است که مجموعه داده یک مجموعه یا تالیف از مواد باشد. چنین موادی شامل آثار کپی‌رایت شده، اعداد، حقایق و داده‌ها هستند، اما محدود به این دسته‌ها نمی‌شوند. سپس، چنین موادی باید سازماندهی، ذخیره و از طریق ابزارهای الکترونیکی یا غیرالکترونیکی قابل دسترسی باشند. این بدان معناست که یک سند مکتوب که تمام الزامات دیگر را برآورده می‌کند نیز می‌تواند به عنوان یک پایگاه داده واجد شرایط باشد. با این حال، برای یک پایگاه داده فیزیکی، لازم نیست که مواد به صورت فیزیکی به روشی سازمان‌یافته ذخیره شوند.

معیار دوم مستلزم آن است که یک سرمایه‌گذاری مرتبط انجام شده باشد. این بدان معناست که سرمایه‌گذاری باید در جمع‌آوری، تأیید و/یا ارائه داده‌ها برای پایگاه داده انجام شود. همان‌طور که در پرونده \lr{BHB v William Hill} روشن شد، سرمایه‌گذاری در دسته‌های دیگر مانند «خلق داده‌ها» برای برآورده کردن این معیار بی‌ربط است.

چنین سرمایه‌گذاری می‌تواند از طریق منابع مالی، منابع انسانی و منابع مادی انجام شود. سرمایه‌گذاری از طریق منابع انسانی می‌تواند، برای مثال، در تلاش یا زمان انجام شود. برای منابع مادی، سرمایه‌گذاری در تجهیزات برای ساخت پایگاه داده مانند سخت‌افزار و نرم‌افزار انجام می‌شود. البته، چنین نوع سرمایه‌گذاری‌هایی هزینه مالی نیز دارند. علاوه بر این، ورودی انسانی عموماً در کار با تجهیزات برای ساخت یک پایگاه داده مورد نیاز است. در واقعیت، ارتباط بین این سه نوع سرمایه‌گذاری اغلب ترکیبی از هر سه را با تأکید بر منابع مالی ایجاد می‌کند.

علاوه بر این، چنین سرمایه‌گذاری‌هایی نباید برای اهداف دیگری انجام شده باشند. برای مثال، رایانه‌هایی که برای ایجاد پایگاه داده استفاده می‌شوند اغلب صرفاً برای آن هدف خریداری نمی‌شوند. در آن صورت، سرمایه‌گذاری عموماً در ایجاد حفاظت خاص پایگاه داده به حساب نمی‌آید.

معیار آخر — اینکه سرمایه‌گذاری باید «قابل توجه» باشد — کمی مبهم‌تر است. دستورالعمل پایگاه داده راهنمایی قاطعی در مورد معنای این معیار یا چگونگی اعمال آن ارائه نمی‌دهد. رویه قضایی تاکنون بیشتر با مبالغ بالای سرمایه‌گذاری مالی سروکار داشته است، بنابراین این موارد نیز راهنمایی زیادی در مورد آستانه قابل توجه بودن ارائه نمی‌دهند. متأسفانه، سقف و کف دقیق این معیار نیز همچنان موضوع بحث‌های آکادمیک سنگین است، اما گنجاندن آن‌ها فراتر از اهداف این فصل خواهد بود.

این آستانه، برخلاف آنچه کلمه «قابل توجه» ممکن است پیشنهاد کند، نباید به عنوان «زیاد» تفسیر شود. در عوض، این معیار بهتر است به عنوان مستلزم سرمایه‌گذاری که «خیلی ناچیز نباشد» درک شود. این حدود در متن اصلی — نه زیاد، فقط نه خیلی ناچیز — عموماً در کشورهای عضو اتحادیه اروپا مانند آلمان پذیرفته شده‌اند. یک مثال روشن از چنین سرمایه‌گذاری ناچیزی، یک کارمند واحد از یک شرکت بزرگ است که تنها چند ساعت را به ساخت پایگاه داده اختصاص می‌دهد. مثالی از چیزی که واجد شرایط خواهد بود، سرمایه‌گذاری در تأیید مقدار زیادی داده با یک مجموعه داده دیگر است.

% =================================================================
% ۱۸.۲.۳ حق اسرار تجاری
% =================================================================
\subsubsection{حق اسرار تجاری}

طبق «دستورالعمل اسرار تجاری» \lr{(Trade Secret Directive)}، اگر مجموعه داده شامل (۱) اطلاعاتی باشد که در محافل مربوطه شناخته شده نیست، (۲) دارای ارزش تجاری باشد، و (۳) توسط شرکت مورد نظر مخفی نگه داشته شود، مجموعه داده ممکن است به عنوان یک راز تجاری محافظت شود.

معیار اول مستلزم آن است که اطلاعات مورد نظر به راحتی قابل دسترسی یا شناخته شده در محافل مربوطه نباشد. محافل مربوطه به افرادی اشاره دارد که عموماً با این نوع اطلاعات سروکار دارند، که به این معنی است که اگر موضوع محافظت شده شامل انواع مختلفی از اطلاعات باشد، محفل مربوطه ممکن است برای هر نوع اطلاعات متفاوت باشد. بنابراین، نمی‌تواند اطلاعات بی‌اهمیت یا نوعی را که از طریق تجربه شغلی عادی به دست می‌آید پوشش دهد. اطلاعاتی که می‌تواند تحت پوشش حق اسرار تجاری قرار گیرد حداقل شامل دانش فنی \lr{(know-how)}، اطلاعات تجاری یا اطلاعات فناوری است، اما ممکن است در قوانین داخلی گسترده‌تر تعریف شود.

دوم، اطلاعات باید دارای ارزش تجاری باشد. مهم نیست که این ارزش بالفعل باشد یا بالقوه. آنچه مهم است این است که اگر راز تجاری به خطر بیفتد، منافع دارنده حقِ راز تجاری — خواه ماهیت علمی، فنی، تجاری یا مالی داشته باشد — آسیب ببیند. بنابراین باید به دلیل مخفی بودن ارزش تجاری داشته باشد. اگر در صورت سوءاستفاده ارزش آن به طور منفی تحت تأثیر قرار نگیرد، برآورده شدن معیار دوم مورد تردید است.

در نهایت، دارنده حق راز تجاری باید تلاش‌های معقولی برای مخفی نگه داشتن اطلاعات انجام دهد. البته، این موضوع تابع شرایط پرونده است. در برخی موارد، ممکن است مخفی نگه داشتن اطلاعات دشوارتر باشد یا شرایط ممکن است اقدامات متفاوتی نسبت به موارد دیگر بطلبد.

این واقعیت که افراد زیادی می‌دانند، لزوماً به این معنی نیست که شرکت در تلاش خود برای برآورده کردن این معیار شکست خورده است. برای مثال، بسیاری از کارمندان ممکن است برای ساخت یک محصول به دانش (بخش‌هایی از) راز تجاری نیاز داشته باشند. تا زمانی که آن‌ها تحت تعهدات قراردادی رازداری باشند، مهم نیست که چه تعداد می‌دانند. همین امر برای توزیع‌کنندگانی که اطلاعات خاصی را تحت یک توافق‌نامه عدم افشا \lr{(NDA)} دریافت کرده‌اند تا بتوانند کار خود را انجام دهند، صادق است.

% =================================================================
% ۱۸.۲.۴ خلاصه
% =================================================================
\subsubsection{خلاصه}

الزامات شکلی هر یک از حقوق مالکیت فکری را می‌توان به سه مؤلفه اساسی تجزیه کرد. قرار دادن آن‌ها در کنار یکدیگر در یک جدول، تصویر زیر را ایجاد می‌کند (جدول ۱۸.۲).

% جدول ۱۸.۲
\begin{table}[h!]
	\centering
	\caption{الزامات شکلی \lr{(Source: Author's own table)}}
	\vspace{0.2cm}
	% تعریف ۳ ستون مساوی
	\begin{tabular}{|p{0.3\linewidth}|p{0.3\linewidth}|p{0.3\linewidth}|}
		\hline
		\textbf{حق تکثیر (Copyright)} & \textbf{حق خاص پایگاه داده (Sui generis)} & \textbf{حق اسرار تجاری (Trade secret)} \\
		\hline
		۱. بیان \lr{(Expression)} & ۱. پایگاه داده & ۱. عدم دسترسی آسان \\
		\hline
		۲. اصالت \lr{(Originality)} & ۲. سرمایه‌گذاری مرتبط & ۲. ارزش تجاری \\
		\hline
		۳. ادبیات و هنر & ۳. قابل توجه \lr{(Substantial)} & ۳. مخفی نگه داشته شود \\
		\hline
	\end{tabular}
	\label{tab:18-2-formal-requirements}
\end{table}

% =================================================================
% ۱۸.۳ دامنه حفاظت
% =================================================================
\subsection{دامنه حفاظت}
\label{sec:18-scope-protection}

% -----------------------------------------------------------------
% ۱۸.۳.۱ حق تکثیر: موضوع مورد حمایت
% -----------------------------------------------------------------
\subsubsection{حق تکثیر: موضوع مورد حمایت}

اگر یک مجموعه داده از طریق یک یا چند مورد از این مسیرها محافظت شود، هنوز محدودیتی در مورد اینکه این حقوق دقیقاً از چه چیزی و در برابر چه چیزی محافظت می‌کنند، وجود دارد.

زمانی که یک مجموعه داده یا کد نرم‌افزار الزامات حفاظت کپی‌رایت را برآورده می‌کند، این حفاظت تنها به «بیان اصیل» محدود می‌شود. این بدان معناست که حفاظت هرگز نمی‌تواند به مواردی مانند محتوای واقعی یا ایده‌ها گسترش یابد. علاوه بر این، اگر تنها انتخاب و/یا چیدمان یک مجموعه داده توسط کپی‌رایت محافظت شود و نه خود داده‌ها، بیان تنها در انتخاب و/یا چیدمان وجود دارد. برای کد نرم‌افزار، این بدان معناست که کپی‌رایت تنها می‌تواند بر روی کدی قرار گیرد که توسط عملکردهای فنی دیکته نشده باشد. بنابراین یک شخص ثالث می‌تواند از محتویات مجموعه داده یا چنین بخش‌های محافظت‌نشده‌ای از کد نرم‌افزار استفاده کند.

علاوه بر این، کپی‌رایت تنها از بیان اصیل در برابر انواع خاصی از استفاده توسط دیگران محافظت می‌کند. به عبارت دیگر، دارنده کپی‌رایت حقوق خاصی برای طرد کردن \lr{(exclude)} دارد. متفاوت از آنچه اصطلاح «کپی‌رایت» (حق تکثیر) نشان می‌دهد، این یک حق واحد نیست بلکه دسته‌ای از حقوق است. دسته حقوق شامل حقوق بهره‌برداری است که به عنوان حقوق اقتصادی شناخته می‌شوند.

چندین حق اقتصادی در «دستورالعمل جامعه اطلاعاتی» \lr{(InfoSoc Directive)} گنجانده شده است، اما تنها حق تکثیر \lr{(reproduction)} و حق در دسترس عموم قرار دادن \lr{(make public)} برای استفاده از داده‌ها و نرم‌افزار اهمیت ویژه‌ای دارند.

حق تکثیر مستلزم آن است که، در اصل، تنها دارنده کپی‌رایت حق دارد کپی‌هایی از اثر خود تهیه کند. علاوه بر این، مهم است توجه شود که تکثیر نباید دقیق باشد. گرفتن عکس از یک نقاشی نیز تکثیر اثر محسوب می‌شود. ابزارهای مورد استفاده برای کپی کردن برای این حق اهمیتی ندارند.

علاوه بر این، لازم نیست که کپی از کل اثر باشد. آنچه مهم است این است که باید به اندازه‌ای کپی شود که کار فکری و خلاقانه هنرمند را نمایش دهد. در پرونده \lr{Infopaq v Danske Dagblades Forening} مشخص شد که نمونه‌ای به کوچکی ۱۱ کلمه از مقالات روزنامه قادر به انجام این کار است. در نتیجه، قابل بحث است که بخش کوچکی از مجموعه داده یا کد نیز می‌تواند انتخاب‌های خلاقانه نویسنده را منتقل کند. اگر چنین باشد، در غیاب یک استثنای قابل اجرا، حتی استفاده از چنین گزیده‌های کوچکی نیاز به مجوز دارد.

دوم، حق در دسترس عموم قرار دادن است. برای مثال، فکر کنید به قرار دادن محتوای محافظت شده در یک وب‌سایت یا استفاده از هایپرلینک به محتوای محافظت شده. دقت کنید که این تا حدودی ساده‌سازی شده است. آنچه باید به عنوان در دسترس عموم قرار دادن درک شود و چه کسی باید به عنوان انجام‌دهنده این عمل درک شود، به دلیل تحولات قانونی و قضایی اخیر در سطح اتحادیه اروپا هنوز در حال تکامل است. در اکثر موارد، نوعی تکثیر لازم است تا بتوان آن را عمومی کرد. استثنائات قابل توجه در اینجا استفاده از هایپرلینک‌ها یا نمایش نسخه اصلی (مثلاً یک نقاشی در موزه) است.

آنچه این موضوع برای اشخاص ثالث معنا می‌دهد این است که آن‌ها نمی‌توانند بدون مجوز به طور قانونی در این استفاده‌ها از بیان اصیل مشارکت کنند. دارنده کپی‌رایت می‌تواند، برای مثال، از طریق لایسنس به دیگران اجازه دهد اثرش را تکثیر کنند. از آنجا که ۱۱ کلمه می‌تواند انتخاب‌های خلاقانه نویسنده را منتقل کند، الزام کسب مجوز خیلی سریع وارد عمل می‌شود.

در اصل، چنین مجوزی تنها می‌تواند از دارنده کپی‌رایت اخذ شود. دارنده حق عموماً یک شخص حقیقی است — نویسنده یا خالق. زمانی که اثری به سفارش خلق شده باشد، تخصیص کپی‌رایت بستگی به این دارد که چه کسی انتخاب‌های خلاقانه را انجام داده است.

در برخی موارد، انتخاب‌های خلاقانه ممکن است توسط چندین بازیگر انجام شده باشد که عموماً منجر به حقوق مشترک بر روی اثر می‌شود. با این حال، این موضوع در مورد خلق در حین استخدام متفاوت است. برای مثال، حقوق بهره‌برداری بر روی اثر اگر توسط کارمند در جریان استخدام و بنا به دستورات کارفرما خلق شده باشد، متعلق به کارفرما است. علاوه بر این، در مورد نرم‌افزار، اداره انتشارات در خلاصه دستورالعمل اجرایی خود روشن کرد که کشورهای عضو اتحادیه اروپا ممکن است مقرر کنند که اشخاص حقوقی یا نهادها نیز می‌توانند دارنده حق باشند. در برخی حوزه‌های قضایی، همه حقوق ممکن است همیشه از نویسنده به دیگری قابل انتقال نباشد.

% -----------------------------------------------------------------
% ۱۸.۳.۲ حفاظت خاص پایگاه داده
% -----------------------------------------------------------------
\subsubsection{حفاظت خاص پایگاه داده}

حق خاص پایگاه داده با در نظر گرفتن سرمایه‌گذار ایجاد شده است، بنابراین صرفاً سازنده واقعی بودن برای دارنده حق بودن کافی نیست. طبق دستورالعمل پایگاه داده، دارنده حق شخصی است که ابتکار عمل و ریسک سرمایه‌گذاری را بر عهده می‌گیرد. پیمانکاران فرعی و کار برای استخدام به صراحت از این تعریف مستثنی شده‌اند.

اگر پایگاه داده توسط یک کارمند ساخته شده باشد، تخصیص حقوق بستگی به معیارهای قانون ملی دارد. اگر چندین نفر یا نهاد در یک پایگاه داده مشارکت داشته باشند، ممکن است حقوق مشترک وجود داشته باشد. برخلاف کپی‌رایت، حق خاص پایگاه داده کاملاً قابل انتقال است.

مانند کپی‌رایت، حق خاص پایگاه داده یک حق واحد نیست. زمانی که یک پایگاه داده تحت پوشش حفاظت خاص پایگاه داده قرار می‌گیرد، دارنده حق حقوق انحصاری برای (۱) استخراج \lr{(extraction)} و (۲) استفاده مجدد \lr{(reutilization)} را دارد. این حقوق باید به شرح زیر درک شوند.

**استخراج** به انتقال پایگاه داده یا بخش قابل توجهی از آن اشاره دارد. این انتقال ممکن است دائمی یا موقت باشد. علاوه بر این، وسیله‌ای که از طریق آن منتقل می‌شود اهمیتی ندارد. همچنین بی‌ربط است که پایگاه داده به کجا منتقل می‌شود (نوع رسانه). آنچه مهم است این است که پایگاه داده یا بخش قابل توجهی از آن منتقل شود.

این بدان معناست که هر شخصی غیر از دارنده حق، در اصل نیاز به مجوز از دارنده حق دارد تا این عمل را به طور قانونی انجام دهد. با این حال، مجوز برای استخراج سیستماتیک بخش‌های ناچیز نیز مورد نیاز است. این موضوع در تعریف استخراج گنجانده شده است تا با «دوشیدن» \lr{(milking)} مبارزه شود. این فرآیندِ انتقال مکرر بخش‌های کوچک پایگاه داده است تا زمانی که کل پایگاه داده یا بخش قابل توجهی از آن منتقل شود.

نوع دیگر استفاده — **استفاده مجدد** — به در دسترس قرار دادن پایگاه داده یا بخش قابل توجهی از آن برای عموم اشاره دارد. این شامل توزیع یا اجاره کپی‌ها، انتقال آنلاین پایگاه داده و سایر انواع انتقالات است. هر روشی که در آن پایگاه داده عمومی شود، تحت این تعریف قرار می‌گیرد. در اصل، این حق به دارنده حق، حق انحصاری انجام استفاده مجدد ضمنی از (بخش قابل توجهی از) پایگاه داده را می‌دهد.

با این حال، درست مانند حق استخراج، حق استفاده مجدد نیز در برابر استفاده مجدد سیستماتیک از بخش‌های ناچیز محافظت می‌کند. باز هم، اگر این تعریف محدود به بخش‌های قابل توجه یا کل پایگاه داده بود، این فرصت را برای اشخاص ثالث فراهم می‌کرد که همچنان (بخش قابل توجهی از) پایگاه داده را ارتباط دهند، فقط هر بار بخش کوچکتری را. در نهایت، یک مورد آخر وجود دارد که در آن استفاده مجدد وجود دارد. این شامل استفاده از یک موتور جستجوی متا \lr{(meta search engine)} با عملکردهای خاص است.

یک موتور جستجوی متا، موتور جستجویی است که امکان جستجو در تعدادی از پایگاه‌های داده دیگر را فراهم می‌کند. عموماً این موتور کوئری جستجو را که توسط بازدیدکننده موتور جستجوی متا وارد شده، به سایر موتورهای جستجو منتقل می‌کند. چیزی از پایگاه‌های داده‌ای که در آن‌ها جستجو می‌کند کپی نمی‌کند، بلکه نتایج جستجو را نشان می‌دهد، از جمله نتایج سایر پایگاه‌های داده.

در پرونده \lr{Innoweb BV v Wegener} مشخص شد که چنین موتور جستجوی متایی احتمالاً (بخش قابل توجهی از) پایگاه داده را مورد استفاده مجدد قرار می‌دهد اگر سه عملکرد زیر وجود داشته باشد.

اول، فرم‌های جستجوی ارائه شده به کاربر نهایی توسط موتور جستجوی متا و پایگاه داده دیگر اساساً یکسان عمل می‌کنند. دوم، کوئری‌ها برای کاربر نهایی به صورت بلادرنگ به سایر موتورهای جستجو ترجمه می‌شوند. این بدان معناست که تمام اطلاعات پایگاه داده دیگر پس از شروع جستجو توسط کاربر نهایی موتور جستجوی متا، به صورت بلادرنگ جستجو می‌شود. سوم و در نهایت، نتایج همه با هم به ترتیبی ارائه می‌شوند که معیارهای مشابهی با معیارهای استفاده شده توسط پایگاه داده دیگر را منعکس می‌کند. برای این منظور، از قالب وب‌سایت خود موتور جستجوی متا در نمایش نتایج استفاده می‌شود و موارد تکراری را با هم به عنوان یک آیتم بلوک نشان می‌دهد.

برای تکرار، اگر یک موتور جستجوی متا که سایر پایگاه‌های داده را جستجو می‌کند به روش فوق عمل کند، اپراتور این موتور جستجوی متا احتمالاً درگیر استفاده مجدد از (بخش‌های قابل توجهی از) پایگاه داده دیگر است. البته، این بدان معنا نیست که اگر یک موتور جستجوی متا این ویژگی‌ها را نداشته باشد، با این وجود نمی‌تواند استفاده مجدد وجود داشته باشد.

برای هر دوی این حقوق، کلمه **«قابل توجه»** دوباره نقش بازی می‌کند. برای اهداف استخراج و استفاده مجدد، اصطلاح «قابل توجه» به حجم داده‌های یک پایگاه داده اشاره دارد، به طور خاص‌تر، حجم داده‌هایی که نسبت به کل پایگاه داده استخراج یا مورد استفاده مجدد قرار می‌گیرند \lr{(see BHB v William Hill)}. در اینجا پیوندی بین سرمایه‌گذاری و دو حق وجود دارد.

راه آسان برای برخورد با این موضوع، رویکرد کمی \lr{(quantitatively)} است. مثال زیر را در نظر بگیرید. سرمایه‌گذاری قابل توجهی در جمع‌آوری، تأیید و/یا ارائه داده‌ها انجام شده بود، اما تفاوت‌های معنی‌داری در سرمایه‌گذاری در سراسر داده‌ها وجود نداشت. یک شخص ثالث اکنون نیمی از داده‌های پایگاه داده را استخراج می‌کند. این بدان معناست که نیمی از سرمایه‌گذاری توسط بخش استخراج شده نمایندگی می‌شود. بخشی که استخراج شده است بنابراین احتمالاً قابل توجه است.

با این حال، اینکه استخراج یا استفاده مجدد قابل توجه است، می‌تواند به صورت کیفی \lr{(qualitatively)} نیز آزمایش شود. این کمی مبهم‌تر است. شرایط مثال ما تا حدودی تغییر می‌کند. اکنون، داده‌های خاصی در پایگاه داده وجود دارد که نیاز به سرمایه‌گذاری بسیار بیشتری در جمع‌آوری، تأیید و/یا ارائه آن‌ها نسبت به بقیه داده‌ها داشته است. داده‌های «گران‌تر» تنها بخش کوچکی از کل پایگاه داده هستند.

یک شخص ثالث اکنون تنها بخشی از پایگاه داده را که حاوی داده‌های «گران‌تر» است، مورد استفاده مجدد قرار می‌دهد. اگرچه داده‌های کمتری است، اما بخش بزرگتری از سرمایه‌گذاری را نشان می‌دهد. این بدان معناست که احتمالاً چنین استفاده مجددی توسط شخص ثالث از نظر کیفی قابل توجه خواهد بود. در هر دو مثال، شخص ثالث احتمالاً نمی‌تواند این اعمال را بدون مجوز از دارنده حق یا توسط قانون انجام دهد.

% -----------------------------------------------------------------
% ۱۸.۳.۳ حق اسرار تجاری
% -----------------------------------------------------------------
\subsubsection{حق اسرار تجاری}

دستورالعمل اسرار تجاری تصریح می‌کند که دارنده اسرار تجاری هر شخص حقیقی یا حقوقی است که به طور قانونی کنترل راز تجاری را در اختیار دارد. مانند حق خاص پایگاه داده، این حق می‌تواند کاملاً منتقل شود. حق اسرار تجاری در برابر اکتساب، استفاده و/یا افشای غیرقانونیِ موضوعِ محافظت شده محافظت می‌کند. این اعمال باید بسیار گسترده تفسیر شوند. هر عملی مغایر با رویه‌های تجاری صادقانه، دسترسی غیرمجاز و/یا تصاحب هر ماده‌ای که حاوی موضوع محافظت شده باشد، تحت اکتساب غیرقانونی قرار می‌گیرد.

همین امر برای موادی که اطلاعات راز تجاری می‌تواند از آن‌ها استخراج شود، صادق است. البته، اگر شخصی سپس اقدام به استفاده و/یا افشای راز تجاری کند، این نیز غیرقانونی خواهد بود. استفاده یا افشای موضوع محافظت شده در نقض یک وظیفه قراردادی — از جمله توافق‌نامه محرمانگی — یا هر وظیفه دیگری که محدودیت‌هایی بر آن اعمال تحمیل می‌کند نیز غیرقانونی است. علاوه بر این، استفاده غیرقانونی شامل تولید کالاهای ناقض، یا ارائه یا قرار دادن آن‌ها در بازار است. ذخیره‌سازی، واردات و صادرات کالاهای ناقض برای آن هدف نیز در این تعریف قرار می‌گیرند. یک کالا ناقض است اگر موضوع محافظت شده‌ای که به طور غیرقانونی اکتساب، استفاده یا افشا شده است، به روشی معنادار به (فرآیند تولید یا بازاریابی) یک محصول کمک کند.

حق اسرار تجاری مسلماً شکننده‌ترین حق مالکیت فکری است. زمانی که کپی‌رایت یا حق خاص پایگاه داده نقض شود، این حقوق مالکیت فکری به حیات خود ادامه می‌دهند. اما زمانی که داده‌های تحت پوشش یک حق اسرار تجاری به گونه‌ای مورد سوءاستفاده قرار گیرند که دیگر شرایط مربوط به محرمانه بودن خود را برآورده نکنند، حق منقضی می‌شود.

با این حال، مهم است تکرار شود که راز تجاری در برابر اعمال غیرمجاز محافظت می‌کند. مثال زیر را در نظر بگیرید. داده‌های تحت پوشش یک راز تجاری تحت یک توافق‌نامه عدم افشا در قبال پرداخت فاش می‌شوند. اگر وظایف ارائه‌دهنده و گیرنده — قراردادی و غیره — مانع این انتقال داده تحت شرایط نشوند، احتمالاً قانونی است. چنین افشایی احتمالاً راز تجاری را دست‌نخورده باقی می‌گذارد. معامله تحت یک توافق‌نامه عدم افشا لزوماً منجر به از دست دادن حق اسرار تجاری نمی‌شود. بنابراین قراردادهایی مانند قراردادهای استخدامی با بندهای محرمانگی و توافق‌نامه‌های عدم افشا ابزارهای حیاتی برای دارنده حق اسرار تجاری هستند.

% -----------------------------------------------------------------
% ۱۸.۳.۴ خلاصه
% -----------------------------------------------------------------
\subsubsection{خلاصه}

اگر یک مجموعه داده واجد شرایط کپی‌رایت، حفاظت خاص پایگاه داده و/یا حق اسرار تجاری باشد، حفاظت همچنان محدود به موضوع خاصی است. علاوه بر این، تنها در برابر اعمال غیرقانونی خاصی که توسط شخصی غیر از دارنده حق انجام شود، محافظت می‌شود (نگاه کنید به جدول ۱۸.۳). چنین اعمالی بدون مجوز ارائه شده توسط دارنده حق یا قانون (مثلاً استثنا) غیرقانونی هستند.

% جدول ۱۸.۳
\begin{table}[h!]
	\centering
	\caption{دامنه حفاظت \lr{(Source: Author's own table)}}
	\vspace{0.2cm}
	% تعریف ۳ ستون مساوی
	\begin{tabular}{|p{0.3\linewidth}|p{0.3\linewidth}|p{0.3\linewidth}|}
		\hline
		& \textbf{حق تکثیر (Copyright)} & \textbf{حق خاص پایگاه داده (Sui generis)} \\ 
		\hline
		\textbf{موضوع} & بیان اصیل & پایگاه داده \\
		\hline
		\textbf{حفاظت} & تکثیر و عمومی‌سازی & استخراج و استفاده مجدد \\
		\hline
	\end{tabular}
	\label{tab:18-3-scope}
\end{table}

% =================================================================
% ۱۸.۴ استثنائات و محدودیت‌ها
% =================================================================
\subsection{استثنائات و محدودیت‌ها}
\label{sec:18-exceptions-limitations}

% -----------------------------------------------------------------
% ۱۸.۴.۱ محدودیت‌های حقوق
% -----------------------------------------------------------------
\subsubsection{محدودیت‌های حقوق}

در برخی موارد، استفاده توسط شخص ثالث خارج از دامنه حق قرار می‌گیرد. محدودیت‌ها، همان‌طور که از نامشان پیداست، حفاظت را محدود می‌کنند. برای مثال، حقوق مالکیت فکری به طور نامحدود ادامه نمی‌یابد.

در اتحادیه اروپا، کپی‌رایت تا ۷۰ سال پس از مرگ نویسنده طبق «دستورالعمل مدت» \lr{(Term Directive)} ادامه می‌یابد. طبق دستورالعمل پایگاه داده، حفاظت خاص پایگاه داده به مدت ۱۵ سال از روز تکمیل پایگاه داده ادامه دارد، اما ساعت با هر تغییر و/یا سرمایه‌گذاری قابل توجه جدید دوباره شروع می‌شود.

حقوق اسرار تجاری در اینجا استثنا هستند: هیچ حداکثر مدتی برای حفاظت در دستورالعمل اسرار تجاری درج نشده است. حق اسرار تجاری تا زمانی که موضوع محافظت شده آن دیگر معیارها را برآورده نکند، ادامه خواهد داشت.

همان‌طور که قبلاً ذکر شد، کپی‌رایت به حقایق و ایده‌ها گسترش نمی‌یابد. علاوه بر این، حتی موضوعی که نه واقعیت است و نه ایده، زمانی که بخشی از بیان اصیل نباشد، می‌تواند خارج از دامنه حفاظت قرار گیرد. علاوه بر این، اصالت به این معنی است که انتخاب‌های خلاقانه توسط نویسنده انجام شده است، نه اینکه باید جدید باشد. این بدان معناست که در برابر «خلق مستقل» \lr{(independent creation)} محافظت نمی‌کند.

برای حق خاص پایگاه داده، حفاظت حول محور سرمایه‌گذاری می‌چرخد. اگر یک شخص ثالث به طور تصادفی بخش‌های ناچیزی را استخراج و/یا مورد استفاده مجدد قرار دهد، در اصل، این کار قانونی خواهد بود. با این حال، مرزهایی نیز در آنجا وجود دارد. در انجام این کار، دستورالعمل پایگاه داده شخص ثالث را ملزم می‌کند که مراقب باشد اعمالش با بهره‌برداری عادی از پایگاه داده توسط دارنده حق در تضاد نباشد یا به منافع او آسیب نامعقول نرساند. به طور خلاصه، اعمال شخص ثالث نباید به سرمایه‌گذاری «آسیب» برساند.

دامنه حفاظت ارائه شده توسط حق اسرار تجاری نیز محدودیت‌های خود را تحت دستورالعمل اسرار تجاری دارد. حق اسرار تجاری تنها در برابر اعمال غیرقانونی محافظت می‌کند. این بدان معناست که خلق یا کشف مستقل با حقوق اسرار تجاری تداخل ندارد.

علاوه بر این، «مهندسی معکوس» \lr{(reverse engineering)} پس از به دست آوردن قانونی یک محصول نیز حق اسرار تجاری را نقض نمی‌کند. این بدان معناست که برای یک شخص ثالث قانونی خواهد بود که محصولی را که توسط دارنده حق در بازار اتحادیه اروپا عرضه شده است بخرد و عملکرد آن را برای بهبود فرآیند تولید، محصول و/یا خدمات خود مطالعه کند. برای مثال، یک تولیدکننده خودرو می‌تواند سنسور خودرویی را که توسط رقیب در بازار عرضه شده است بخرد تا آن را مهندسی معکوس کند و از دانش به دست آمده برای بهبود سنسورهای خودروی خود استفاده کند.

در نهایت، چندین ارجاع به قابلیت انتقال حقوق اختصاص یافته به دارنده حق توسط این حقوق مالکیت فکری شده است. آنچه این بدان معناست این است که عموماً امکان «رزرو» یا انتقال قراردادی چنین حقوقی یا اجازه دادن به اعمال تحت شرایط خاص وجود دارد. دارندگان حق خودشان نیز می‌توانند به طور قراردادی حقوق خود را محدود کنند.

برای رزرو حقوق، برای مثال به وضعیت حقوق مشترک فکر کنید. ممکن است برای طرفین مفید باشد که به طور قراردادی تعیین کنند که مجوز همه دارندگان حق باید اخذ شود، نه فقط یکی. متناوباً، یک دارنده حق می‌تواند حق انحصاری را به یک توزیع‌کننده انحصاری منتقل کند تا حق مالکیت فکری را علیه ناقضان (ادعایی) اجرا کند و بدین ترتیب دستان خود را آزاد کند.

مثالی از اجازه دادن به اعمال تحت شرایط خاص را می‌توان در بسیاری از شرایط خدمات در صنعت بازی یافت. چنین شرایطی اغلب حاوی بندی است که به کاربران اجازه می‌دهد در اعمالی مانند پخش زنده بازی کردن خود شرکت کنند. مثال مناسب دیگر استفاده از یک آستانه است که به کاربران اجازه می‌دهد از مواد محافظت شده استفاده کنند تا زمانی که سودی بیش از تعداد مشخصی کسب نکنند یا به تعداد مشخصی از مشتریان نرسند (جدول ۱۸.۴).

% جدول ۱۸.۴
\begin{table}[h!]
	\centering
	\caption{محدودیت‌ها \lr{(Source: Author's own table)}}
	\vspace{0.2cm}
	\small
	\begin{tabular}{|p{0.22\linewidth}|p{0.22\linewidth}|p{0.22\linewidth}|p{0.22\linewidth}|}
		\hline
		& \textbf{حق تکثیر (Copyright)} & \textbf{حق خاص پایگاه داده (Sui generis)} & \textbf{حق اسرار تجاری (Trade secret)} \\
		\hline
		\textbf{حداکثر مدت} & ۷۰ سال پس از مرگ نویسنده & ۱۵ سال، اما قابل تمدید & — \\
		\hline
		\textbf{خارج از دامنه} & $\bullet$ حقایق \newline $\bullet$ ایده‌ها \newline $\bullet$ خلق مستقل & $\bullet$ استخراج بخش‌های ناچیز \newline $\bullet$ استفاده مجدد از بخش‌های ناچیز & $\bullet$ خلق مستقل \newline $\bullet$ کشف مستقل \newline $\bullet$ مهندسی معکوس \\
		\hline
		\textbf{محدودیت‌های قراردادی ممکن} & بله، روی حقوق بهره‌برداری\textsuperscript{الف} & بله & بله \\
		\hline
	\end{tabular}
	\vspace{0.1cm}
	\footnotesize{\newline \textsuperscript{الف} همان‌طور که قبلاً ذکر شد، دسته حقوق همیشه به طور کامل قابل انتقال نیست. با این حال، مهم است توجه شود که این عموماً در مورد حقوق بهره‌برداری صدق نمی‌کند.}
	\label{tab:18-4-limitations}
\end{table}

% -----------------------------------------------------------------
% ۱۸.۴.۲ استثنائات: وجه مشترک
% -----------------------------------------------------------------
\subsubsection{استثنائات: وجه مشترک}

اگر عملی تحت پوشش یک استثنا قرار گیرد، توسط قانون مجاز است. این بدان معناست که دارنده حق نمی‌تواند آن عمل را مجاز یا منع کند. طبق اسناد قانونی مانند کنوانسیون برن، موافقت‌نامه تریپس \lr{(TRIPS Agreement)} و دستورالعمل جامعه اطلاعاتی \lr{(InfoSoc Directive)}، استثنائات باید محدود به موارد خاص باشند و با بهره‌برداری عادی از اثر تداخل نداشته باشند و به منافع مشروع نویسنده آسیب نامعقول نرسانند. به طور کلی، این استثنائات در سراسر اتحادیه اروپا به نفع حفاظت بالای حقوق مالکیت فکری به صورت محدود اعمال می‌شوند.

استثنائات تا حدودی برای هر حق مالکیت فکری متفاوت است، اما وجه مشترکی وجود دارد. برای مثال، استثنا برای تدریس و تحقیق و اهداف امنیت عمومی یا یک رویه اداری یا قضایی هم در دستورالعمل جامعه اطلاعاتی (در مورد کپی‌رایت) و هم در دستورالعمل پایگاه داده وجود دارد.

در اولی، این‌ها استثنائاتی بر حق تکثیر دارنده حق هستند. در دومی، این استثنائات هم حق استخراج و هم حق استفاده مجدد را هدف قرار می‌دهند. با این حال، در رژیم خاص پایگاه داده، این استثنائات تنها توسط یک «کاربر قانونی» \lr{(lawful user)} قابل استناد هستند. برای مثال، دور زدن الزام اشتراک برای دسترسی به یک پایگاه داده غیرعمومی بدون مجوز را در نظر بگیرید. استخراج و/یا استفاده مجدد توسط چنین کاربری نمی‌تواند در دامنه این استثنائات قرار گیرد.

مثالی از تدریس و تحقیق می‌تواند نمایش کلیپ‌ها، (مواد طراحی مقدماتی برای) کد نرم‌افزار، متون کوچک، یا بخش‌هایی از یک پایگاه داده برای تصویرسازی به دانشجویان یا محققان باشد. برای واجد شرایط بودن، هر دو رژیم مستلزم آن هستند که اشخاص ثالث نباید چنین استفاده‌هایی را برای اهداف تجاری انجام دهند. در صورت امکان، منبع باید ذکر شود و استفاده نباید فراتر از آنچه برای هدف غیرتجاری دنبال شده لازم است، باشد.

برای استثنای اهداف امنیت عمومی یا یک رویه اداری یا قضایی، یک مثال می‌تواند کپی کردن یک اثر یا داده‌های خاص از یک پایگاه داده برای تأیید کالاهای وارداتی باشد. مثال دیگر می‌تواند گنجاندن چنین موادی در تصمیم کتبی یک پرونده دادگاه حول محور مسائل نقض کپی‌رایت و/یا حفاظت خاص پایگاه داده باشد. باز هم، چنین اعمالی نباید برای اهداف تجاری انجام شده باشند.

% -----------------------------------------------------------------
% ۱۸.۴.۳ استثنائات خاص برای هر حق
% -----------------------------------------------------------------
\subsubsection{استثنائات خاص برای هر حق}

رایج‌ترین و مرتبط‌ترین استثنائات خاص برای رژیم کپی‌رایت اتحادیه اروپا عبارتند از روزنامه‌نگاری، نقل قول برای نقد و بررسی، و کاریکاتور، پارودی و تقلید ادبی \lr{(pastiche)}. قانون اتحادیه اروپا، به طور خاص‌تر دستورالعمل جامعه اطلاعاتی، هیچ شرطی برای هیچ یک از این استثنائات ارائه نمی‌دهد. این بدان معناست که برای مثال، کشورهای عضو آزاد بودند که استثنائات را تنها به شرایط یا استفاده‌های خاص محدود کنند.

در نهایت — و شاید مهم‌تر از همه — استثنای «داده‌کاوی و متن‌کاوی» \lr{(text and data mining)} است که اخیراً در «دستورالعمل بازار واحد دیجیتال» \lr{(Digital Single Market Directive)} معرفی شده است. این مفهوم به بهترین وجه به عنوان هر تکنیک تحلیلی که خودکار است درک می‌شود. این تکنیک برای استخراج اطلاعات با تحلیل متن و داده‌ها به شکل دیجیتال استفاده می‌شود. این تمرین می‌تواند، برای مثال، برای کشف الگوها، روندها و همبستگی‌ها در یک مجموعه داده انجام شود.

دو نوع از این حق معرفی شده است، یکی متمرکز بر متن‌کاوی و داده‌کاوی برای اهداف علمی و دیگری عمومی. هر دو مستلزم آن هستند که دسترسی قانونی به آثاری که قرار است تحت متن‌کاوی و داده‌کاوی قرار گیرند، وجود داشته باشد. نوع اول اجازه ذخیره و نگهداری تکثیرهای آثار برای تحقیقات علمی را می‌دهد. با این حال، باید سطح مناسبی از امنیت در ذخیره‌سازی کپی‌های آثار وجود داشته باشد.

برای استثنای عمومی، پیش‌شرط این است که دارنده حق به صراحت استفاده از اثر خود را رزرو (ممنوع) نکرده باشد. در غیاب چنین رزروی، آثار می‌توانند تا زمانی که برای هدف دنبال شده با متن‌کاوی و داده‌کاوی لازم است، «کاوش»، نگهداری و ذخیره شوند.

تحت رژیم حقوق اسرار تجاری، مرتبط‌ترین استثنائات سه مورد زیر هستند. اول، اگر بتوان با موفقیت به آزادی بیان و حق دسترسی به اطلاعات استناد کرد، عمل ممکن است حق اسرار تجاری را نقض نکند.

علاوه بر این، اکتساب، استفاده یا افشای موضوع محافظت شده توسط یک حق اسرار تجاری در تعقیبِ آشکارسازی سوءرفتار، تخلف یا فعالیت غیرقانونی، مانند «افشاگری» \lr{(whistleblowing)}، نیز ممکن است قانونی باشد. علاوه بر این، مرتبط با یکی از محدودیت‌های ذکر شده قبلی، اگر وظایف مشروع آن‌ها به عنوان کارگران یا نمایندگان کارگران افشا را ایجاب کرده باشد، دارنده حق اسرار تجاری ممکن است نتواند علیه آن‌ها درخواست جبران خسارت کند (جدول ۱۸.۵).

% جدول ۱۸.۵
\begin{table}[h!]
	\centering
	\caption{خلاصه استثنائات \lr{(Source: Author's own table)}}
	\vspace{0.2cm}
	\small
	\begin{tabular}{|p{0.3\linewidth}|p{0.3\linewidth}|p{0.3\linewidth}|}
		\hline
		\textbf{حق تکثیر (Copyright)} & \textbf{حق خاص پایگاه داده (Sui generis)} & \textbf{حق اسرار تجاری (Trade secret)} \\
		\hline
		$\bullet$ تدریس و تحقیق \newline $\bullet$ اهداف امنیت عمومی و رویه اداری یا قضایی \newline $\bullet$ روزنامه‌نگاری \newline $\bullet$ نقل قول برای نقد و بررسی \newline $\bullet$ کاریکاتور، پارودی و تقلید ادبی \newline $\bullet$ متن‌کاوی و داده‌کاوی & $\bullet$ تدریس و تحقیق \newline $\bullet$ اهداف امنیت عمومی و/یا رویه اداری یا قضایی & $\bullet$ آزادی بیان و حق دسترسی به اطلاعات \newline $\bullet$ آشکارسازی سوءرفتار، تخلف یا فعالیت غیرقانونی \newline $\bullet$ وظایف مشروع کارگر(ان) (نمایندگان) \\
		\hline
	\end{tabular}
	\label{tab:18-5-exceptions}
\end{table}

% =================================================================
% ۱۸.۵ منابع جایگزین
% =================================================================
\subsection{منابع جایگزین}
\label{sec:18-alternative-sources}

این حقوق مالکیت فکری ممکن است از نظر دامنه و اهداف متفاوت باشند، اما کاملاً ممکن است که چندین مورد از آن‌ها برای (بخش‌هایی از) همان مجموعه داده یا کد قابل اجرا باشند. استثنائات این حقوق مختلف محدود به حقوق و اهداف خاص هستند. بنابراین، ممکن است عملی که تحت یک استثنا برای یک حق مالکیت فکری قرار می‌گیرد، به دلیل وجود حق دیگر مجاز نباشد. اگر شخص ثالثی نیاز به دسترسی به مجموعه داده‌هایی (تا حدی) تحت پوشش این حقوق داشته باشد، چندین گزینه برای به دست آوردن دسترسی قانونی وجود دارد.

ساده‌ترین گزینه، اخذ «مجوز» \lr{(license)} از دارنده حق برای استفاده از مجموعه داده‌های اوست. یک مجوز به صاحب امتیاز اجازه می‌دهد تا از موضوع محافظت شده مطابق با شرایط توافق شده، معمولاً در قبال پرداخت، استفاده کند. موضوع محافظت شده می‌تواند برای برخی یا تمام استفاده‌های تحت پوشش کپی‌رایت و/یا حفاظت خاص پایگاه داده مجوز داده شود، اما نویسنده یا سازنده مالک باقی می‌ماند. گزینه مشابه دیگر در اینجا ورود به یک توافق‌نامه موردی \lr{(ad hoc agreement)} یا شراکت با پرداخت مبلغ یا ارائه چیزی در ازای آن است.

متناوباً، گاهی اوقات امکان دسترسی به مجموعه داده‌های قابل مقایسه از طریق منابع دیگر، مانند «اطلاعات بخش عمومی» یا \lr{PSI} وجود دارد. این فرصتی بسیار جالب و مفید برای بررسی است زیرا دولت داده‌های زیادی در اختیار دارد — برای مثال به نقشه‌ها، تصمیمات دادگاه، داده‌های شرکت‌ها، آمار شهروندان و غیره فکر کنید — و ممکن است تعهدی برای انتشار آن داده‌ها و اجازه استفاده مجدد از آن‌ها داشته باشد (مانند قوانین آزادی اطلاعات)، اگرچه لزوماً رایگان نیست.

داده‌ها به احتمال زیاد زمانی تابع رژیم \lr{PSI} هستند که: (۱) مرتبط با اجرای فعالیت‌های دولتی باشند، (۲) هیچ حق مالکیت فکری متعلق به اشخاص ثالث بر روی آن‌ها وجود نداشته باشد، و (۳) داده‌ها به دلایل سیاست عمومی (از جمله حفاظت از داده‌ها) محرمانه نگه داشته نشوند.

بسته به مدل کسب‌وکار، گزینه دیگری که باید در نظر گرفت استفاده از نرم‌افزار یا داده‌های تابع طرح‌های مجوز باز («متن‌باز» یا \lr{Open Source}) است. استفاده از چنین داده‌ها یا نرم‌افزاری رایگان است، اما بسته به نوع مجوز، ممکن است انواع دیگری از محدودیت‌ها وجود داشته باشد. رایج‌ترین تقسیم‌بندی بین مجوزهای «سهل‌گیرانه» \lr{(permissive)} و مجوزهای «کپی‌لفت» (ضعیف یا قوی) \lr{(copyleft)} است.

این انواع مجوزها بهتر است به عنوان طیفی از کمترین محدودیت تا بیشترین محدودیت تصور شوند. هر دو نوع مجوز استفاده از موضوع را از نظر استفاده، تغییر و توزیع مجدد محدود نمی‌کنند، اما مجوزهای سهل‌گیرانه اجازه می‌دهند آثار مشتق شده اختصاصی (انحصاری) شوند در حالی که مجوزهای کپی‌لفت این اجازه را نمی‌دهند.

این بدان معناست که، برای مثال، یک شخص ثالث می‌تواند تغییراتی در موضوع تحت مجوز سهل‌گیرانه ایجاد کند و آن را تحت نوع متفاوتی از مجوز، مجوز دهد و توزیع کند. از طرف دیگر، یک مجوز کپی‌لفت ضعیف این اجازه را نمی‌دهد. چنین مجوزهایی حاوی بندی هستند که اختصاصی کردن مواد مشتق شده از موضوع آن یا مجوزدهی مجدد این مواد مشتق شده را ممنوع می‌کنند.

مجوزهای کپی‌لفت قوی علاوه بر این مستلزم آن هستند که موضوع آن نیز نتواند تحت مجوزی متفاوت از مجوز اصلی مجوز داده شود. این بدان معناست که اثری که تابع یک مجوز اختصاصی «عادی» است نمی‌تواند با اثر دیگری که تابع یک مجوز کپی‌لفت است ترکیب شود.

ارائه خدمات یا محصولات مکمل در بازار برای ایجاد یا به دست آوردن دسترسی به یک مجموعه داده مشابه نیز یک امکان است. برای مثال، یک شخص ثالث داده‌های مشابهی را می‌خواهد که توسط سنسورهای عرضه شده در بازار توسط یک رقیب تولید می‌شود. شخص ثالث می‌تواند تصمیم بگیرد نرم‌افزاری ارائه دهد که بتواند سنسورهای رقیب را راه اندازی کند یا سنسورهای رقیب ارائه دهد. گزینه دیگر در اینجا تبدیل مشتریان خودتان به جمع‌آوری‌کنندگان داده با واداشتن آن‌ها به اصلاح یا گزارش داده‌های خاص است. برای مثال، به گزارش افزودنی‌ها به یک نقشه یا تغییرات در یک خیابان فکر کنید.

در نهایت، اگر این مجموعه داده‌ها حاوی داده‌های شخصی هستند، می‌توانید از آن افراد بخواهید که از حق «قابلیت انتقال داده» \lr{(data portability)} خود از طریق تبلیغاتی برای مشتریان جدید یا موجودِ خدمات یا محصولات خودشان استفاده کنند. طبق مقررات عمومی حفاظت از داده‌ها \lr{(GDPR)}، این حق به اشخاص حقیقی فرصت می‌دهد تا داده‌های شخصی خود را از یک سرویس آنلاین به سرویس دیگر منتقل کنند.

الزاماتی که در اینجا باید برآورده شوند عبارتند از اینکه داده‌ها (۱) داده‌های شخصی باشند و (۲) توسط شخصی که داده‌های شخصی مربوط به اوست به کنترل‌کننده ارائه شده باشند. برای مثال، یک شرکت بیمه یا شهرداری می‌تواند در ازای انتقال داده‌های شخصی آن‌ها به شما، مزایایی مانند تخفیف در حق بیمه یا خدمات ارائه شده توسط شهرداری ارائه دهد.

به طور خلاصه، اگر حقوق مالکیت فکری بر روی یک مجموعه داده وجود داشته باشد و هیچ یک از استثنائات قابل اجرا نباشد، هنوز چندین راه برای به دست آوردن دسترسی قانونی وجود دارد. علاوه بر این، منابع جایگزین می‌توانند به عنوان منبع مکمل یا جایگزین برای مجموعه داده محافظت شده بررسی شوند.

% =================================================================
% نتیجه‌گیری (Conclusion)
% =================================================================
\vspace{0.8cm}
\noindent
\fcolorbox{black}{gray!15}{%
	\begin{minipage}{\dimexpr\linewidth-2\fboxsep-2\fboxrule}
		\vspace{0.3cm}
		\begin{center}
			\textbf{\Large نتیجه‌گیری}
		\end{center}
		\vspace{0.2cm}
		
		برای خلاصه کردن، هنگام برخورد با موضوعاتی مانند مجموعه داده‌ها و نرم‌افزار، مهم است که ابتدا تعیین کنید آیا حقوق مالکیت فکری ممکن است بر روی آن‌ها وجود داشته باشد یا خیر. اگر چنین باشد، استفاده از چنین موضوعاتی توسط اشخاص ثالث ممکن است محدود شود. اینکه کدام استفاده‌ها محدود شده‌اند و تحت چه شرایطی، بستگی به این دارد که کدام حق اعمال می‌شود و تا درجات مختلف، کدام رژیم (یعنی اتحادیه اروپا یا ایالات متحده) اعمال می‌شود.
		
		دوم، باید تعیین شود که دارنده حق کیست. اگر شما دارنده حق هستید، این بدان معناست که ممکن است بتوانید دسترسی و استفاده دیگران از موضوع محافظت شده را محدود کنید. اگر شخص دیگری است، چندین مسیر ممکن برای استفاده قانونی از موضوع محافظت شده آن طرف یا جایگزین‌هایی برای این موضوع وجود دارد، از اخذ رضایت از دارنده حق گرفته تا عمل در محدوده محدودیت‌ها یا استثنائات تا یافتن یا ایجاد منابع جایگزین.
		\vspace{0.3cm}
	\end{minipage}
}

% =================================================================
% پیام‌های کلیدی
% =================================================================
\vspace{0.8cm}
\noindent
\fcolorbox{black}{gray!15}{%
	\begin{minipage}{\dimexpr\linewidth-2\fboxsep-2\fboxrule}
		\vspace{0.3cm}
		\textbf{\large پیام‌های کلیدی}
		
		\begin{itemize}
			\setlength\itemsep{0.5em}
			
			\item[\textbf{--}] کپی‌رایت بر روی محتوا، انتخاب یا چیدمان یک مجموعه داده به دارنده حق، حق انحصاری تکثیر و در دسترس عموم قرار دادن مواد محافظت شده را می‌دهد.
			
			\item[\textbf{--}] حق خاص پایگاه داده \lr{(sui generis)} به دارنده حق، حق انحصاری استخراج و استفاده مجدد از بخش‌های قابل توجه پایگاه داده را می‌دهد.
			
			\item[\textbf{--}] حقوق اسرار تجاری بر روی داده‌ها از دارنده حق در برابر اکتساب، استفاده و افشای غیرقانونی مواد محافظت شده محافظت می‌کند.
			
			\item[\textbf{--}] یک شخص ثالث تنها در صورتی می‌تواند در استفاده قانونی از موضوع محافظت شده توسط این حقوق مشارکت کند که توسط دارنده حق یا توسط قانون (اگر استثنائات قابل اجرا باشند) مجاز شده باشد.
			
			\item[\textbf{--}] در غیاب مجوز، چندین راه وجود دارد که می‌توان به طور قانونی به (بخش‌هایی از) مجموعه داده یا نرم‌افزار یا به منابع قابل مقایسه دسترسی پیدا کرد.
		\end{itemize}
		\vspace{0.3cm}
	\end{minipage}
}

% =================================================================
% سوالات بحث و گفتگو
% =================================================================
\vspace{0.8cm}
\section*{سوالات بحث و گفتگو}
\addcontentsline{toc}{section}{سوالات بحث و گفتگو}

\begin{enumerate}
	\item چرا ما حقوق مالکیت فکری داریم؟
	\item تمایز بین ایده‌ها و بیان‌ها در قانون کپی‌رایت را چگونه توضیح می‌دهید؟
	\item لطفاً تمام انواع سرمایه‌گذاری‌های مرتبط برای حفاظت خاص پایگاه داده، از جمله ابزارهایی که از طریق آن‌ها چنین سرمایه‌گذاری‌هایی می‌تواند انجام شود را تعریف کنید.
	\item لطفاً به طور خلاصه وضعیت مهندسی معکوس تحت حفاظت اسرار تجاری را توضیح دهید.
\end{enumerate}

% =================================================================
% منابع برای مطالعه بیشتر
% =================================================================
\vspace{1cm}
\section*{منابع برای مطالعه بیشتر}
\addcontentsline{toc}{section}{منابع برای مطالعه بیشتر}

\begin{latin}
	\begin{itemize}
		\setlength\itemsep{0.5em}
		
		\item[] Agreement on Trade-Related Aspects of Intellectual Property Rights, Annex 1c of the Marrakesh Agreement Establishing the World Trade Organization, 1994.
		
		\item[] Berne Convention for the Protection of Literary and Artistic Works of 1886 as amended on September 28, 1979 (‘Berne Convention’).
		
		\item[] Case C-202/12 Innoweb BV v Wegener [2013] ECLI:EU:C:2013:850.
		
		\item[] Case C-203/02 British Horseracing Board v William Hill [2004] ECLI:EU:C:2004:695.
		
		\item[] Case C-393/09 Bezpečnostní softwarová asociace [2010] ECLI:EU:C:2010:816.
		
		\item[] Case C-5/08, Infopaq v Danske Dagblades Forening Case C-5/08, Infopaq v Danske Dagblades Forening [2009] ECLI:EU:C:2009:465.
		
		\item[] Convention on the Grant of European Patents of 1973 as amended last on 29 November 2000 (‘European Patent Convention’).
		
		\item[] Council Directive 93/98/EEC of 29 October 1993 harmonizing the term of protection of copyright and certain related rights [1993] OJ L290/9 (‘Term Directive’).
		
		\item[] Estelle Derclaye, ‘Database sui generis right: what is a substantial investment? A tentative definition’ (2005) IIC 36(1).
		
		\item[] Directive 2001/29/EC of the European Parliament and of the Council on the harmonisation of certain aspects of copyright and related rights in the information society [2001] OJ L167/10 (‘InfoSoc Directive’).
		
		\item[] Directive 2004/48/EC of the European Parliament and of the Council on the enforcement of intellectual property rights [2004] OJ L157/45 (‘Enforcement Directive’).
		
		\item[] Directive 2009/24/EC of the European Parliament and of the Council of 23 April 2009 on the legal protection of computer programs [2009] OJ L111/16 (‘Software Directive’).
		
		\item[] Directive (EU) 2016/943 of the European Parliament and of the Council on the protection of undisclosed know-how and business information (trade secrets) against their unlawful acquisition, use and disclosure [2016] OJ L 157/1 (‘Trade Secret Directive’).
		
		\item[] Directive 2019/790 of the European Parliament and of the Council on copyright and related rights in the Digital Single Market and amending Directives 96/9/EC and 2001/29/EC [2019] OJ L130/11 (‘Digital Single Market Directive’).
		
		\item[] Directive 96/6/EC of the European Parliament and of the Council of 11 March 1996 on the legal protection of databases [1996] OJ L77/20 (‘Database Directive’).
		
		\item[] Feist Publications, Inc., v. Rural Telephone Service Co., 499 U.S. 340 (1991).
		
		\item[] Husovec, M. (2019). How Europe Wants to Redefine Global Online Copyright Enforcement. TILEC Discussion Paper, 2019–2016.
		
		\item[] Publications Office in their Summary of Directive 2009/24/EC—the legal protection of computer programs, 23 January 2017.
		
		\item[] Regulation 2016/679 of the European Parliament and of the Council of 27 April 2016 on the protection of natural persons with regard to the processing of personal data and on the free movement of such data, and repealing Directive 95/46/EC [2016] OJ L 119/1 (‘General Data Protection Regulation’).
		
		\item[] Rosati, E. (2017). GS Media and its implications for the construction of the right of communication to the public within EU copyright architecture. \textit{Common Market Law Review}, 54(4), 1221–1242.
	\end{itemize}
\end{latin}
%% =================================================================
% فصل ۱۹: مسائل مسئولیت و قرارداد در خصوص داده‌ها
% =================================================================

\clearpage
\raggedbottom 

% تنظیم شماره‌گذاری فصل روی ۱۹
\setcounter{section}{19}
\setcounter{subsection}{0}
\renewcommand{\thesection}{19}
\renewcommand{\thesubsection}{19.\arabic{subsection}}

% تنظیم عمق شماره‌گذاری
\setcounter{secnumdepth}{4}
\renewcommand{\theparagraph}{\thesubsubsection.\arabic{paragraph}}

% -----------------------------------------------------------------
% عنوان و نویسنده
% -----------------------------------------------------------------
\noindent
\fcolorbox{black}{gray!15}{%
	\begin{minipage}{\dimexpr\linewidth-2\fboxsep-2\fboxrule}
		\vspace{0.5cm}
		\begin{center}
			\textbf{\huge مسائل مسئولیت و قرارداد در خصوص داده‌ها}
			\vspace{0.4cm}
			
			\large
			\textit{اریک تیونگ جین تای} \\
			\lr{\textit{Eric Tjong Tjin Tai}}
		\end{center}
		\vspace{0.5cm}
	\end{minipage}
}

\vspace{0.8cm}

% -----------------------------------------------------------------
% فهرست مطالب داخلی
% -----------------------------------------------------------------
\noindent
\textbf{\Large فهرست مطالب}
\vspace{0.3cm}

{ \small
	\noindent \textbf{۱۹.۱} \hspace{0.2cm} \textbf{مقدمه} \par \vspace{0.1cm}
	
	\noindent \textbf{۱۹.۲} \hspace{0.2cm} \textbf{ویژگی‌های عمومی حقوق خصوصی} \par \vspace{0.1cm}
	
	\noindent \textbf{۱۹.۳} \hspace{0.2cm} \textbf{داده چیست؟} \par \vspace{0.1cm}
	
	\noindent \textbf{۱۹.۴} \hspace{0.2cm} \textbf{قراردادها و داده‌ها} \par
	\hspace{0.8cm} ۱۹.۴.۱ تشکیل قراردادها \par
	\hspace{0.8cm} ۱۹.۴.۲ محتوای قراردادها \par
	\hspace{0.8cm} ۱۹.۴.۳ جبران خسارت‌های قراردادی \par \vspace{0.1cm}
	
	\noindent \textbf{۱۹.۵} \hspace{0.2cm} \textbf{حقوق مسئولیت مدنی ( شبه‌جرم) و داده‌ها} \par
	\hspace{0.8cm} ۱۹.۵.۱ مسئولیت مبتنی بر تقصیر \par
	\hspace{0.8cm} ۱۹.۵.۲ مسئولیت محض \par
	\hspace{0.8cm} ۱۹.۵.۳ رابطه‌ی سببی و دفاعیات \par
	\hspace{0.8cm} ۱۹.۵.۴ خسارات و سایر جبران‌ها در مسئولیت مدنی \par \vspace{0.1cm}
	
	\noindent \textbf{منابع}
}

\vspace{0.8cm}

% -----------------------------------------------------------------
% اهداف یادگیری
% -----------------------------------------------------------------
\noindent
\fcolorbox{black}{gray!15}{%
	\begin{minipage}{\dimexpr\linewidth-2\fboxsep-2\fboxrule}
		\vspace{0.3cm}
		\textbf{\large اهداف یادگیری}
		\vspace{0.2cm}
		\begin{itemize}
			\setlength\itemsep{0.5em}
			\item[\textbf{--}] درک چگونگی عملکرد قواعد حقوق خصوصی.
			\item[\textbf{--}] درک معانی مختلف «داده» در قانون.
			\item[\textbf{--}] ارزیابی یک قرارداد و شناسایی بندهای قراردادی مهم.
			\item[\textbf{--}] درک مفهوم زیان اقتصادی محض و ارتباط آن با داده‌ها.
			\item[\textbf{--}] آشنایی با مهم‌ترین مبانی مسئولیت، به ویژه موارد مرتبط با مسائل داده.
		\end{itemize}
		\vspace{0.2cm}
	\end{minipage}
}

\newpage

% =================================================================
% ۱۹.۱ مقدمه
% =================================================================
\subsection{مقدمه}
\label{sec:19-1-introduction}

این فصل مقدمه‌ای بر حوزه‌های خاصی از قانون را تا جایی که برای دانشمندان داده مرتبط است، فراهم می‌کند. اگر به عنوان یک دانشمند داده کار می‌کنید، ممکن است با سوالات حقوقی مواجه شوید. ممکن است نیاز به مذاکره در مورد یک قرارداد داشته باشید یا ممکن است نگران مسئولیت احتمالی باشید. یک مقدمه معمولی بر حقوق مانند آنچه توسط \lr{Ventura (2005)} یا \lr{Wacks (2015)} ارائه شده است، تنها کمک محدودی خواهد کرد، زیرا داده‌ها مشکلات حقوقی خاصی را ایجاد می‌کنند که ادبیات عمومی به آن‌ها پاسخ نمی‌دهد \lr{(Mak et al. 2018)}.

هدف این فصل تجهیز شما به دانش پایه حقوق قراردادها و حقوق مسئولیت مدنی است که باید شما را با اصول اساسی درگیر آشنا کند. همچنین، این فصل حاوی نکاتی برای اجتناب از دام‌های احتمالی است. از آنجا که این تنها یک مقدمه مختصر است، امکان ورود به قواعد دقیقی که ممکن است در پرونده‌های واقعی اعمال شوند وجود ندارد. در صورت تردید، با یک وکیل مشورت کنید.

ابتدا، با یک مثال مختصر و چند نکته کلی شروع خواهیم کرد. این با یک تحلیل حقوقی از اینکه داده چیست دنبال می‌شود. متعاقباً بحثی در مورد حقوق قراردادها با جزئیات بیشتر خواهد آمد. در نهایت، مسئولیت در حقوق شبه‌جرم (مسئولیت مدنی) مورد بحث قرار می‌گیرد.

\vspace{0.4cm}
% کادر مثال آلیس و باب
\begin{center}
	\colorbox{gray!15}{%
		\begin{minipage}{0.9\linewidth}
			\vspace{0.2cm}
			\noindent
			$\blacktriangleright$ \textbf{مثال}
			\vspace{0.2cm}
			
			آلیس کسب‌وکاری را راه‌اندازی کرده است که تحلیل پروفایل‌های مشتریان را برای شرکت‌های بزرگ ارائه می‌دهد. باب، آلیس را استخدام می‌کند تا داده‌های تجاری او را تحلیل کند. تحلیل توسط ایو \lr{(Eve)}، کارمند آلیس، انجام می‌شود. پس از تکمیل تحلیل، ایو به طور تصادفی پایگاه داده باب را حذف می‌کند و باب نسخه پشتیبان \lr{(backup)} نداشته است. آیا آلیس باید برای از دست رفتن پایگاه داده به باب خسارت بپردازد، و اگر چنین است، چقدر؟ $\blacktriangleleft$
			\vspace{0.2cm}
		\end{minipage}
	}
\end{center}

% =================================================================
% ۱۹.۲ ویژگی‌های عمومی حقوق خصوصی
% =================================================================
\subsection{ویژگی‌های عمومی حقوق خصوصی}
\label{sec:19-2-private-law}

همان‌طور که از مثال می‌بینید، سوالات مختلفی مطرح می‌شود. قانون، به ویژه حوزه‌ای که «حقوق خصوصی» \lr{(private law)} نامیده می‌شود، با این سوالات سروکار دارد. حقوق خصوصی بخشی از قانون است که ادعاها و روابط بین افراد خصوصی را پوشش می‌دهد: دو بخش عمده حقوق خصوصی، «حقوق قراردادها» و «حقوق مسئولیت مدنی» (که با مسئولیت سروکار دارد) هستند.

اگر ادعا یا اختلاف دیگری در مورد حقوق خصوصی دارید، می‌توانید در نهایت به دادگاه بروید تا تصمیمی در مورد اختلاف دریافت کنید. دادگاه ممکن است ادعای شما را بپذیرد که منجر به یک جبران خسارت می‌شود (نگاه کنید به بخش ۱۹.۴.۳).

در حقوق خصوصی، ما بین مواردی که قراردادی بین طرفین وجود دارد، یا جایی که ادعای مسئولیت وجود دارد در حالی که هیچ قراردادی بین قربانی و شخصی که ادعا می‌شود به اشتباه عمل کرده (مرتکب شبه‌جرم) وجود ندارد، تمایز قائل می‌شویم. نوع اول پرونده توسط حقوق قراردادها (بخش ۱۹.۴) و نوع دوم پرونده توسط حقوق مسئولیت مدنی یا مسئولیت ناشی از جرم (بخش ۱۹.۵) اداره می‌شود.

حقوق خصوصی، تا جایی که در اینجا مرتبط است، متشکل از قواعد (و استثنائات) است که تعیین می‌کند آیا یک پیامد حقوقی خاص وجود دارد یا خیر. یک تحلیل حقوقی معمولاً شامل یافتن قواعد حقوقی مربوطه و سپس ارزیابی این است که آیا این قواعد بر واقعیات پرونده اعمال می‌شوند و نتیجه چیست.

برای مثال، اگر ایجاب و قبول وجود داشته باشد، قرارداد تشکیل می‌شود. اما اگر عیبی در اراده مانند اشتباه وجود داشته باشد، قرارداد، اگرچه از نظر شکلی معتبر است، می‌تواند باطل شود (بخش ۱۹.۴.۱). علاوه بر این، اگر قرارداد معتبر باشد، اما تعهدی از قرارداد نقض شود (بخش ۱۹.۴.۳)، طلبکار (شخصی که حق دارد تعهد برایش اجرا شود) ممکن است ادعای خسارت کند (بخش ۱۹.۴.۳).

در شبه‌کد زبان برنامه‌نویسی، ساختار و رابطه بین چنین قواعدی می‌تواند به صورت زیر بیان شود (برای ارائه مثال):

\begin{latin}
	\begin{verbatim}
		if (offer & acceptance) {
			contract.valid()
		}
		if contract.mistake == TRUE {
			contract.invalid()
		}
		if contract.valid() {
			if contract.breached(case) {
				/* further conditions */
				creditor.money += contract.breached.damages(case)
			}
		}
	\end{verbatim}
\end{latin}

این نشان می‌دهد که چگونه بسته به چندین شرایط، نتیجه ممکن است این باشد که یک قرارداد معتبر یا نامعتبر است و اینکه طلبکار ممکن است حق دریافت خسارت داشته باشد. قواعد واقعی قانون بسیار پیچیده‌تر از آن چیزی هستند که این مثال نشان می‌دهد، اما حداقل ممکن است ایده‌ای از نحوه عملکرد قواعد حقوقی به شما بدهد.

برخی اطلاعات بیشتر برای جلوگیری از سوءتفاهم لازم است. اول از همه، حقوق خصوصی تعهداتی را بر افراد انسانی (اشخاص حقیقی) و همچنین بر شرکت‌ها و سایر سازمان‌ها تحمیل می‌کند. با چنین «اشخاص حقوقی» عموماً به همان روش افراد انسانی رفتار می‌شود: آن‌ها می‌توانند قرارداد منعقد کنند و ممکن است در مسئولیت مدنی مسئول شناخته شوند.

قواعد خاصی بر اشخاص حقوقی حاکم است که موضوع حقوق تجارت است و در اینجا به آن پرداخته نمی‌شود. در عمل، اشخاص حقوقی توسط نمایندگان و پرسنل مجاز (مانند مدیر عامل یا مدیر اجرایی) نمایندگی می‌شوند.

ثانیاً، باید آگاه باشید که حقوق خصوصی محلی و وابسته به زمان است: در درجه اول ملی است و ممکن است در طول زمان تغییر کند. می‌توانیم قیاسی با نحوه ایجاد برنامه‌های کامپیوتری در یک نسخه خاص از یک زبان برنامه‌نویسی خاص و سیستم عامل انجام دهیم. شما باید بدانید محیطی که برنامه در آن اجرا می‌شود چیست: حتی اگر یک برنامه ممکن است در چندین نسخه مختلف اجرا شود، تضمینی نیست که در نسخه‌ای که برای آن نوشته یا آزمایش نشده است اجرا شود. به طور مشابه، وکلا تنها می‌توانند پاسخ‌های دقیق در مورد قانون را در رابطه با سیستم حقوقی که اعمال می‌شود ارائه دهند.\footnote{این موضوع توسط آنچه حقوق بین‌الملل خصوصی نامیده می‌شود، که تعارض قوانین نیز نامیده می‌شود، اداره می‌شود.} با این حال، توصیف خطوط کلی قانون آنطور که در اکثر سیستم‌ها قابل اجراست، مشابه نحوه توصیف یک الگوریتم در شبه‌کد با انتزاع از جزئیات زبان‌های برنامه‌نویسی واقعی، امکان‌پذیر است. در این فصل، ما از چنین رویکرد انتزاعی به قانون استفاده می‌کنیم.

یک تمایز مهم که نمی‌توانیم از آن انتزاع کنیم، تمایز بین کشورهایی\footnote{یا به طور خاص‌تر، بخش‌هایی از یک کشور که سیستم یکسانی دارند: این‌ها همچنین حوزه‌های قضایی \lr{(jurisdictions)} نامیده می‌شوند.} است که دارای سیستم «کامنی‌لا» \lr{(common law)} هستند و کشورهایی که دارای سیستم «حقوق نوشته» \lr{(civil law)} هستند.

حوزه‌های قضایی کامن‌لا عبارتند از انگلستان و ولز (نه بریتانیا، زیرا اسکاتلند سیستم حقوقی متفاوتی دارد)، ایالات متحده و مستعمرات سابق انگلیس (اکثراً بخشی از کشورهای مشترک‌المنافع). اکثر کشورهای دیگر دارای سیستم‌های حقوق نوشته هستند:\footnote{همچنین برخی استثنائات وجود دارد که در هیچ یک از این دسته‌ها قرار نمی‌گیرند، به ویژه سیستم‌های ترکیبی مانند آفریقای جنوبی که دارای عناصری از کامن‌لا و حقوق نوشته و/یا سایر سیستم‌ها هستند.} آن‌ها یک کد، یک قانون مکتوب دارند که اکثر قواعد حقوق قراردادها و حقوق مسئولیت مدنی را جمع‌آوری می‌کند.

کامن‌لا ویژگی‌هایی دارد که به طور قابل توجهی از قواعد کشورهای حقوق نوشته منحرف می‌شود: ما چند مثال را در زیر بحث خواهیم کرد. به طور کلی، کامن‌لا بر تشریفات و معنای تحت‌اللفظی قراردادها تأکید دارد و طرفین را مسئول تدوین قرارداد برای بیان دقیق آنچه می‌خواهند می‌داند. سیستم‌های حقوق نوشته تمایل دارند بر قصد واقعی طرفین تأکید کنند و به دادگاه‌ها آزادی عمل بیشتری برای تفسیر قرارداد می‌دهند.

% =================================================================
% ۱۹.۳ داده چیست؟
% =================================================================
\subsection{داده چیست؟}
\label{sec:19-3-what-is-data}

قبل از اینکه بتوانیم جنبه‌های قراردادی داده‌ها و مسئولیت داده‌ها را بحث کنیم، باید اطمینان حاصل کنیم که می‌فهمیم داده واقعاً چیست، هم در واقعیت و هم در قانون.

به عنوان اولین رویکرد، ممکن است در نظر بگیرید که مردم واقعاً چگونه با داده‌ها کار می‌کنند. داده‌ها ممکن است به شکل اسناد واژه‌پرداز به عنوان پیوست ایمیل، فایل‌های موسیقی و عکس‌های دیجیتال آپلود شده در پایگاه‌های داده ابری استفاده شوند. از نظر فنی، همه این‌ها فایل‌های داده هستند. علاوه بر این، داده همچنین با اطلاعات موجود در چنین فایل‌هایی شناسایی می‌شود، مانند زمانی که از «داده‌های شخصی» صحبت می‌کنیم. در نهایت، عبارت «کلان‌داده» \lr{(big data)} مد شده است.

کلان‌داده نه آنقدر به یک پایگاه داده یا فایل مشخص، بلکه به یک سیستم مداوم اشاره دارد که به موجب آن داده‌ها به طور مداوم دریافت و پردازش می‌شوند. برای حفظ چنین سیستمی، یک سازمان نیاز به امکانات فنی (سیستم‌های مدیریت پایگاه داده، سرورها)، خدمات (تغذیه مداوم داده‌ها) و منابع انسانی (دانشمندان داده، کارکنان پشتیبانی \lr{IT}) دارد که همه آن‌ها نیاز به پشتیبانی حقوقی (قراردادهای مجوز، قراردادهای استخدام) دارند. این را می‌توان به صورت گرافیکی به شکل زیر در شکل ۱۹.۱ نمایش داد (که جریان‌های داده بین منابع مختلف و امکانات ذخیره‌سازی را نشان می‌دهد).

% استفاده از کادر جایگزین به جای TikZ برای جلوگیری از خطا
\begin{figure}[h!]
	\centering
	\fbox{
		\begin{minipage}{0.8\linewidth}
			\centering
			\vspace{0.2cm}
			\textbf{[نمودار جریان داده‌ها]}
			\vspace{0.2cm}
			
			\small
			این نمودار نشان می‌دهد که سرور مرکزی با عناصر زیر در ارتباط است:
			\begin{itemize}
				\item رابط کاربری (PC)
				\item تحلیل‌گران داده و کاربران
				\item پایگاه‌های داده (داخلی و خارجی)
				\item بسته‌های نرم‌افزاری استاندارد و سفارشی
				\item حسگرها (داخلی و خارجی)
			\end{itemize}
			\vspace{0.2cm}
		\end{minipage}
	}
	\caption{نمایش گرافیکی جریان داده‌ها. منبع: شکل متعلق به نویسنده.}
	\label{fig:19-1-graphical-representation}
\end{figure}

این سه شکل از داده، یعنی (۱) اطلاعات، (۲) فایل‌های داده، و (۳) کلان‌داده، منجر به مسائل حقوقی مختلفی می‌شوند. در ادامه، من بر دو شکل اول داده تمرکز خواهم کرد، زیرا کلان‌داده مسئله‌ای جداگانه است که نیاز به فضای بیشتری نسبت به آنچه در اینجا موجود است دارد. به طور خلاصه، کلان‌داده را می‌توان با نگاه کردن به هر عنصر به نوبه خود بهتر درمان کرد.

سوال بعدی این است که داده از نظر حقوقی چیست. داده چگونه می‌تواند توصیف شود؟ وقتی آن را بدانیم، همچنین می‌دانیم چگونه با داده در قراردادها برخورد کنیم، یا اینکه آیا داده می‌تواند منجر به مسئولیت شود.

در مورد کلان‌داده، آن چیزی خاص در قانون نیست. عناصر مختلف کلان‌داده می‌توانند به نوبه خود توصیف شوند. اطلاعات به خودی خود یک موضوع حقوقی نیست. اطلاعات ممکن است منجر به مسئولیت شود و امکان قرارداد بستن در مورد اطلاعات وجود دارد. ما برخی مثال‌ها را در زیر خواهیم دید. اما اطلاعات به خودی خود یک شیء به رسمیت شناخته شده قانونی نیست. این یک شیء ملموس نیست و به خودی خود توسط قانون محافظت نمی‌شود.

ممکن است اطلاعات تا حدی توسط یک حق مالکیت فکری محافظت شود زیرا دارای حق تکثیر یا موارد مشابه است، و قانون اسرار تجاری اطلاعات را به شرط داشتن ارزش تجاری و مخفی بودن فراهم می‌کند (نگاه کنید به فصل ۱۸ که به طور گسترده در مورد حقوق مالکیت فکری بحث می‌کند). اما اطلاعات به خودی خود محافظت نمی‌شود.

فایل‌های داده نیاز به توجه کمی بیشتر دارند. بسیاری از وکلا تمایل دارند فایل‌های داده را با اطلاعات برابر بدانند و به طور مشابه از اعطای حفاظت خاص به فایل‌های داده خودداری می‌کنند. در واقع، حقوق مالکیت فکری و حقوق پایگاه داده ممکن است خود پایگاه‌های داده را پوشش دهند، اما از فایل‌های داده به خودی خود محافظت نمی‌کنند: آن‌ها فقط کپی و توزیع شیء محافظت شده را ممنوع می‌کنند. اسرار تجاری اغلب داده‌های تجاری مرتبط را پوشش می‌دهند (نگاه کنید به فصل ۲۵).

با این حال، یک چیز توسط حقوق مالکیت فکری محافظت نمی‌شود. این کنترل واقعی بر روی یک فایل داده است. غیر وکلا کاملاً خوشحال هستند که در مورد داده صحبت کنند که دارایی کسی است یا کسی مالک داده است. وکلا در انجام این کار مردد هستند. این به دلیل آن است که داده فاقد ویژگی‌هایی است که برای اشیاء عادی مالکیت، به ویژه کالاهای ملموس مانند ماشین‌ها، کتاب‌ها و گلدان‌ها رایج است.

فایل‌های داده ناملموس هستند و فایل‌های داده انحصاری نیستند: شما می‌توانید یک کپی تهیه کنید و از این کپی بدون ایجاد مانع برای «مالک» فایل داده اصلی استفاده کنید.\footnote{وکلا از اصطلاح فنی «رقابت‌پذیر» \lr{(rivalrous)} برای نشان دادن ماهیت انحصاری مالکیت کالاهای ملموس استفاده می‌کنند.} در قانون، معمولاً تنها اشیاء ملموس توسط حقوق مالکیت محافظت می‌شوند که به شما حق بازگرداندن تصرف شیء را می‌دهند. ناملموس‌ها معمولاً تنها از دیدگاه مالکیت فکری درک می‌شوند. همان‌طور که در فصل ۱۸ دیدیم، حقوق مالکیت فکری مانند حقوق مالکیت عادی عمل نمی‌کنند زیرا آن‌ها تنها در برابر نقض محافظت می‌کنند؛ آن‌ها حقی برای به دست آوردن مجدد کنترل شیء خود (زیرا آن خلق غیرمادی است) نمی‌دهند. برای مثال، شما نمی‌توانید آهنگ \lr{Yesterday} را بدزدید. حتی اگر آن را سرقت ادبی کنید، موسیقی و متن ترانه برای دیگران در دسترس باقی می‌ماند. شما نمی‌توانید اطلاعات را به خودی خود کنترل کنید.

با این وجود، با فایل‌های داده، نوعی کنترل واقعی وجود دارد، صرفاً به این دلیل که شما از دسترسی دیگران به فایل داده جلوگیری می‌کنید. این شکل از کنترل ناملموس‌ها جدید است. در گذشته، شما فقط می‌توانستید اطلاعات را در مغز خود کنترل کنید، اما با فایل‌های داده، امکان کنترل اطلاعات خارجی وجود دارد. فایل داده شکل خاصی از اطلاعات است، درست همان‌طور که یک پرینت از داده‌ها شکل خاصی است (که توسط قانون مالکیت محافظت می‌شود، زیرا یک دسته کاغذ یک کالای ملموس است).

برابر دانستن فایل داده با اطلاعات به خودی خود نادرست است: یک فایل داده مزایایی را نسبت به اطلاعات صرفاً انتزاعی بدون توجه به فرم ارائه می‌دهد. این موضوع به راحتی آشکار می‌شود اگر عمل اسکن کردن یک کتاب کاغذی و انجام تشخیص متن روی اسکن را در نظر بگیریم: اگرچه فایل متنی حاصل نباید حاوی اطلاعات بیشتر یا متفاوتی نسبت به کتاب باشد، فایل متنی ممکن است برای اهدافی مانند تحلیل داده‌ها به روشی که کتاب کاغذی نیست، مفید باشد.

در حال حاضر، قانون در اکثر کشورها از کنترل بر فایل‌های داده تا حد معینی محافظت می‌کند، و در برخی موارد حتی به شما اجازه می‌دهد بازگرداندن فایل داده‌ای را که دزدیده شده است ادعا کنید. این حفاظتی مشابه حقوق مالکیت برای داده‌ها فراهم می‌کند. با این حال، این در مورد همه کشورها صدق نمی‌کند \lr{(Tjong Tjin Tai, 2018b)}. سایر حقوق مربوط به مالکیت (برای مثال حقوق وثیقه، مانند رهن و گرو) برای فایل‌های داده به سختی قابل تنظیم هستند.

با توجه به موقعیت نامطمئن فایل‌های داده در قانون، حفاظت از فایل‌های داده عمدتاً غیرمستقیم است: زیرا مداخله در داده‌ها یک شبه‌جرم است، یا زیرا شما تعهدات قراردادی برای رفتار با داده‌ها به شیوه صحیح را تنظیم کرده‌اید.



% =================================================================
% ۱۹.۴ قراردادها و داده‌ها
% =================================================================
\subsection{قراردادها و داده‌ها}
\label{sec:19-4-contracts-and-data}

قرارداد چیست؟ به زبان ساده، قرارداد توافقی بین دو نفر است که هر کدام تعهدات خاصی را در قبال طرف دیگر بر عهده می‌گیرند.\footnote{ممکن است بیش از دو طرف در یک قرارداد وجود داشته باشد؛ ما در مورد آن‌ها بحث نخواهیم کرد.} مثال کلاسیک، قرارداد بیع (فروش) است: خریدار باید قیمت خرید را بپردازد، و فروشنده موظف است شیء فروخته شده را به خریدار ارائه دهد.

هنگام در نظر گرفتن یک قرارداد، مهم است که بین کاغذ (یا فایل دیجیتال) که اثبات قرارداد را فراهم می‌کند و رابطه حقوقی قراردادی بین طرفین (که ناشی از عمل امضا است، که پذیرش را ثابت می‌کند) تمایز قائل شوید. کاغذ امضا شده نیز «قرارداد» نامیده می‌شود. با این حال، اگر کاغذ گم شود، خودِ قرارداد (رابطه حقوقی) همچنان وجود خواهد داشت و معتبر خواهد بود. در اینجا، ما بر رابطه حقوقی حاصل تمرکز می‌کنیم. مقدمات عمومی عبارتند از \lr{(Bix 2012, Cartwright 2013, Smits 2014)}.

موقعیت‌هایی وجود دارد که بلافاصله مشخص نیست آیا قراردادی منعقد شده است یا خیر. اینکه آیا واقعاً قراردادی وجود دارد ممکن است به قانون قابل اجرا و سایر شرایط نیز بستگی داشته باشد. مثالی از این مورد، شرایط و ضوابط \lr{(T\&C)} است که بر دسترسی و استفاده از یک وب‌سایت حاکم است: مسلماً، این یک قرارداد است زیرا به نظر می‌رسد شما را ملزم به رعایت این شرایط در حین دسترسی به وب‌سایت می‌کند، که در ازای آن دسترسی دریافت می‌کنید. اما \lr{T\&C} همچنین می‌تواند به عنوان صرفاً شرایطی که تحت آن به شما اجازه دسترسی داده می‌شود تفسیر شود، که تعهدات اضافی را فراهم نمی‌کند.

تمایز بین دو دیدگاه زمانی روشن می‌شود که، برای مثال، \lr{T\&C} جریمه‌ای را در صورت ارسال نقد منفی از وب‌سایت در جای دیگر تعیین کند. در تفسیر اول، این می‌تواند یک شرط الزام‌آور باشد؛ در شکل دوم اینطور نیست، زیرا شرط دسترسی نیست بلکه یک تعهد اضافی است که اگر قراردادی وجود نداشته باشد نمی‌تواند تحمیل شود. ما در مورد این مسائل بیشتر بحث نخواهیم کرد زیرا این‌ها منجر به بحث‌های حقوقی پیچیده می‌شوند و به راحتی قابل توضیح یا حل نیستند.

برای داده‌ها، «مجوز» \lr{(license)} از اهمیت ویژه‌ای برخوردار است. این در درجه اول به اجازه استفاده از مالکیت فکریِ کسی اشاره دارد اما با تعمیم برای اجازه استفاده از داده‌ها نیز استفاده می‌شود. یک مجوز می‌تواند بخشی از یک توافق‌نامه گسترده‌تر باشد، و می‌تواند موضوع یک «فروش» باشد، برای مثال زمانی که شما برنامه‌ای را برای تلفن هوشمند خود «می‌خرید». چنین قراردادی شامل حق دریافت فایل برنامه و یک مجوز (حق استفاده) برای برنامه است.

شما به هر دوی این عناصر نیاز دارید. ممکن است مجوزی داشته باشید در حالی که فایل را گم کرده‌اید (چون تلفن هوشمند جدیدی دارید)، که در این صورت از حق خود برای دریافت مجدد فایل برنامه استفاده خواهید کرد. برعکس، اگر کپی غیرقانونی از فایل برنامه را به دست آورده باشید، طبیعتاً مجاز به استفاده از آن نیستید: شما به اجازه، یعنی مجوز، نیاز دارید.

\subsubsection{تشکیل قراردادها}
\label{sec:19-4-1-formation}

قراردادها با «ایجاب» \lr{(offer)} و «قبول» \lr{(acceptance)} منعقد می‌شوند. یک طرف ایجابی برای یک قرارداد می‌دهد، و طرف دیگر ایجاب را می‌پذیرد.\footnote{لازم نیست که طرفین یک ایجاب صریح و روشن ارائه دهند که متعاقباً به عنوان یک عمل جداگانه پذیرفته شود؛ کافی است که طرفین به توافق متقابل در مورد قرارداد برسند، برای مثال با پیش‌نویس کردن قرارداد با هم و متعاقباً امضای قرارداد.}

در اکثر موارد، هیچ الزام شکلی وجود ندارد: قرارداد می‌تواند به صورت کتبی، دیجیتالی یا شفاهی منعقد شود؛ رضایت به قرارداد می‌تواند از اقدامات (مانند بالا بردن دست) یا حتی (در موقعیت‌های خاص) از سکوت (رضایت ضمنی) استنتاج شود. با این حال، برای برخی از انواع قراردادها، تشریفات اضافی وجود دارد. مثالی از این مورد توافق‌نامه‌های پیش از ازدواج است که عموماً نیاز به نوعی کمک حقوقی (توسط سردفتر یا سایر ارائه‌دهندگان خدمات حقوقی) دارد تا اطمینان حاصل شود که رضایت آزادانه و آگاهانه داده شده است.

در کامن‌لا، یک الزام اضافی برای یک قرارداد معتبر وجود دارد: «عوض» \lr{(consideration)}. این بدان معناست که باید نوعی اجرای متقابل در ازای آن وجود داشته باشد. اگر یک طرف به طور یک‌جانبه تعهداتی را بر عهده بگیرد، بدون دریافت چیزی در عوض، این به عنوان یک قرارداد معتبر در نظر گرفته نمی‌شود.

الزام عوض مسلماً می‌تواند منجر به مشکلاتی در مورد نرم‌افزار متن‌باز شود. چنین نرم‌افزاری به صورت رایگان ارائه می‌شود، اما ممکن است طرفی را که از نرم‌افزار استفاده می‌کند ملزم به پذیرش محدودیت‌های خاصی در استفاده از نرم‌افزار کند، به ویژه ممکن است تعهدی برای عمومی کردن هرگونه تغییری که در نرم‌افزار متن‌باز ایجاد می‌شود تحت همان مجوز متن‌باز تحمیل کند.\footnote{این نسخه به اصطلاح کپی‌لفت \lr{(copyleft)} از مجوزهای متن‌باز است. نسخه‌های دیگری نیز وجود دارد.} با این حال، دادگاه‌ها پذیرفته‌اند که استفاده از نرم‌افزار متن‌باز نیز شامل نوعی عوض به معنای انطباق با الزامات مجوز است \lr{(Jacobsen v. Katzer, 2008)}. بنابراین، پذیرش مجوز متن‌باز ممکن است یک قرارداد معتبر تحت کامن‌لا ایجاد کند.

اگرچه قراردادها عموماً تشریفات دیگری ندارند، ممکن است تعهدات اطلاعاتی وجود داشته باشد که یک طرف را ملزم به اطلاع‌رسانی کافی به طرف دیگر هنگام ارائه ایجاب برای قرارداد کند. نمونه‌هایی از آن هویت و محل کسب‌وکار هنگام قرارداد بستن در اینترنت است (برای موارد اروپایی).

یک مسئله خاص در حقوق قراردادها امکان این است که قراردادی که معتبر به نظر می‌رسد بعداً معلوم شود که واقعاً نامعتبر است. دو دلیل اصلی این است که «عیب در اراده» وجود دارد (رضایت به روش مناسبی شکل نگرفته است، مثلاً با تهدید به خشونت) یا اینکه قرارداد «نظم عمومی» را نقض می‌کند (مانند قرارداد با یک آدمکش). قراردادهای خلاف نظم عمومی باطل و بی‌اثر هستند (هیچ اثر الزام‌آوری ندارند؛ گویی هرگز منعقد نشده‌اند).

در عمل، مهم‌ترین عیب در اراده در صورت سوءتفاهم بین طرفین ایجاد می‌شود. در سیستم‌های حقوق نوشته، این ممکن است ناشی از «اشتباه» \lr{(mistake)} باشد (یک سوءتفاهم بین طرفین در مورد موضوع قرارداد، برای مثال اینکه آیا یک تلفن هوشمند کاملاً نو است یا یک مدل بازسازی شده). این بدان معناست که طرفین باید اطلاعات مرتبط را به ابتکار خود فاش کنند، و اگر طرفی در انجام این کار کوتاهی کند، طرف دیگر ممکن است به اشتباه استناد کند.

در کشورهای کامن‌لا، ساختار اصلی «اظهار خلاف واقع» \lr{(misrepresentation)} است: یکی از طرفین در قرارداد اظهار کرده (یعنی بیان کرده که چیزی چنین است) که رایانه نو است و بازسازی شده نیست. اگر معلوم شود که این درست نیست، طرف دیگر بر اساس اظهار خلاف واقع ادعایی دارد. این بدان معناست که طرفین باید ابتکار عمل را برای پرسش در مورد ویژگی‌هایی که مهم می‌دانند به دست بگیرند. پیامد اشتباه و اظهار خلاف واقع این است که طرف زیان‌دیده می‌تواند قرارداد را فسخ کند، که این اثر را دارد که طرفین به وضعیت قبل از انعقاد قرارداد بازگردانده می‌شوند و هر آنچه قبلاً اجرا شده است باید بازگردانده (اعاده) شود.

\subsubsection{محتویات قراردادها}
\label{sec:19-4-2-content}

در یک قرارداد، می‌توانید مشخص کنید که طرفین باید چه کاری انجام دهند و ممکن است چه انتظاری از یکدیگر داشته باشند. یک قرارداد معمولاً شامل چندین ماده یا بند است که این تعهدات و پیش‌فرض‌ها را توصیف می‌کند و سایر مسائل را نیز تنظیم می‌کند. تفسیر این بندها تمایل دارد بر متن بندها تمرکز کند (زیرا معمولاً فرض می‌شود که طرفین عمداً فرمول‌بندی خاصی را انتخاب کرده‌اند)، اما معنای تحت‌اللفظی ممکن است بر اساس نیات طرفین یا سایر شرایط نیز اصلاح شود.\footnote{این امر به ویژه در کشورهای حقوق نوشته صدق می‌کند.}

یک تمایز مهم بین «تعهدات به وسیله» \lr{(obligations of means)} و «تعهدات به نتیجه» \lr{(obligations of result)} است. اگر موافقت کنید که تحلیل داده انجام دهید، ممکن است قرارداد ببندید که نتیجه منجر به صرفه‌جویی ۱۰٪ از هزینه‌های جاری شود. با این حال، شما احتمالاً مایل به انجام این کار نخواهید بود، زیرا هیچ راهی وجود ندارد که بتوانید اطمینان حاصل کنید که این اتفاق می‌افتد. در آن صورت، ترجیح می‌دهید آن را به عنوان تعهد به وسیله بیان کنید: شما از بهترین تلاش‌های خود برای انجام یک تحلیل پیشرفته استفاده خواهید کرد. این شبیه به پزشکی است که تنها باید از مراقبت و مهارت، دقت حرفه‌ای، برای تلاش جهت بهبود وضعیت بیمار استفاده کند، بدون تضمین اینکه بیمار سلامت خود را به طور کامل باز خواهد یافت. برای دادن اطمینان به طرف دیگر، می‌توانید در مورد برخی تعهدات فرعی عینی‌تر توافق کنید. برای مثال، توافق می‌کنید که از روش تحلیل خاصی استفاده کنید و حداقل سه تحلیل یا گزارش مختلف ارائه دهید.

یک قرارداد همچنین می‌تواند به عنوان موضوع، «فروش» یا مجوزدهی داده‌ها را داشته باشد.\footnote{دقیق‌تر بگوییم، داده‌ها نمی‌توانند فروخته شوند زیرا فروش تنها شامل اشیاء ملموس می‌شود. با این حال، به نظر می‌رسد قانون در حال توسعه است تا معنای «فروش» را به قراردادهای مربوط به محتوای دیجیتال نیز گسترش دهد.} قرارداد می‌تواند شامل بندهایی در مورد کیفیت داده‌ها باشد (منبع چیست، تا چه سطح جزئیات/چند بیت، چه دوره‌ای، آیا مداوم است، یا آیا آپ‌تایم ۹۹.۹۹٪ وجود دارد). ممکن است مربوط به یک مجموعه داده ثابت (مانند دمای مجموعه‌ای از حسگرها در طول یک سال) یا قراردادی در مورد تغذیه مداوم داده باشد.

مهم است درک کنید که مجوز داده ممکن است این اثر را داشته باشد که دارنده مجوز استفاده دائمی از داده‌ها یا نوعی تغییر داده‌ها را داشته باشد. اگر داده‌ها، برای مثال، برای آموزش یک الگوریتم استفاده شوند، به نوعی در الگوریتم گنجانده شده‌اند و نمی‌توانند از آن حذف شوند. هنگام تنظیم پیش‌نویس مجوز موقت برای چنین استفاده‌ای، منطقی است که تصریح شود که طرفین در مورد این تثبیت قطعی داده‌ها توافق دارند، در حالی که قرارداد می‌بندند که پس از مدت مجوز، دارنده مجوز داده‌های خام را حذف کند.

علاوه بر این، اگر یک مجوز ابدی و نامحدود باشد (با حق اعطای مجوزهای فرعی)، این در واقع به معنای تبدیل دارنده مجوز به «مالک» نسخه کپی داده‌های خودش است. حتی اگر شما «مالک» داده‌های خود باقی بمانید (مگر اینکه مجوز انحصاری ارائه دهید، که در واقع دارنده مجوز را مالک جدید واقعی می‌کند)، نمی‌توانید دارنده مجوز را از انجام هر کاری که شما می‌توانید با داده‌ها انجام دهید، بازدارید. این یکی از راه‌هایی است که شرکت‌ها ممکن است اسماً بگویند که شما «مالک» باقی می‌مانید در حالی که عملاً از طریق یک مجوز گسترده، اختیاراتی شبیه به مالک به دست می‌آورند.

علاوه بر این، قراردادها اغلب حاوی بندهای زیادی هستند که نوع جبران خسارت‌های موجود برای طرفین را تنظیم می‌کنند. این بندها به ویژه در کامن‌لا مهم هستند، زیرا قوانین انگلستان و ایالات متحده تنها برای انواع خاصی از بندهای قراردادی، به اصطلاح ضمانت‌ها \lr{(warranties)} و شروط \lr{(terms)}، جبران خسارت ارائه می‌دهند،\footnote{اصطلاحات دقیق در حقوق انگلستان با حقوق ایالات متحده متفاوت است.} در حالی که قرارداد باید تصریح کند که کدام جبران خسارت‌ها برای این بندها اعمال می‌شود. در حقوق نوشته، نیازی نیست که اینقدر دقیق باشید: دادگاه حقوق نوشته هر تعهدی را که در قرارداد گنجانده شده است، بدون نیاز به فرم دقیق اجرا خواهد کرد.

یک دسته مهم از بندها، «انتخاب قانون» و «انتخاب دادگاه» است. یک قرارداد می‌تواند (و اغلب می‌کند) مشخص کند که کدام قانون اعمال می‌شود و به کدام دادگاه می‌توانید مراجعه کنید. همچنین می‌تواند حاوی بندی باشد که می‌گوید شما اصلاً نمی‌توانید به دادگاه بروید بلکه (برای مثال) باید پرونده را داوری کنید. انتخاب قانون می‌تواند این اثر را داشته باشد که شما نوعی از حفاظت را که تحت سیستم خودتان داشتید از دست بدهید. برخی انواع دیگر بندها در زیر (بخش ۱۹.۴.۲) مورد بحث قرار می‌گیرند زیرا مربوط به جبران خسارت‌ها هستند.

قانون برخی بررسی‌ها را بر روی محتوای قرارداد از طریق دکترین‌های «شروط ناعادلانه» و «رویه‌های تجاری ناعادلانه» فراهم می‌کند. به زبان ساده، این‌ها شروط به وضوح نامعقول و از نظر اجتماعی غیرقابل قبول را مجاز نمی‌دانند و ممکن است این پیامد را داشته باشند که شروطی که به نظر منطقی می‌رسد ضمنی باشند، به عنوان بخشی از قرارداد در نظر گرفته شوند.\footnote{در کامن‌لا، این تا حدی با دکترین شروط ضمنی همپوشانی دارد.} برای مثال، یک قرارداد فروش فرض می‌شود که دلالت بر این دارد که شیء فروخته شده برای هدفی که معمولاً دارد مناسب است.\footnote{در ایالات متحده، کد تجاری یکنواخت بخش ۲-۳۱۵ ضمانت ضمنیِ تناسب برای یک هدف خاص را کدگذاری می‌کند.}

در حالی که این شکل مهمی از کنترل برای اطمینان از معقول بودن قراردادها است، قواعد حقوقی کاملاً کلی هستند. آن‌ها برای روش‌های خاصی که قراردادهای شامل داده ممکن است نامعقول یا ناعادلانه باشند به‌روز نشده‌اند، و همیشه مشخص نیست که آیا می‌توان آن‌ها را برای تناسب با جامعه داده تفسیر مجدد کرد.

دکترین شروط ناعادلانه مربوط به دکترین کلی‌تری در کشورهای حقوق نوشته است، یعنی اینکه قراردادها فرض می‌شود شامل وظیفه‌ای برای تفسیر و اجرای قرارداد بر اساس «حسن نیت» \lr{(good faith)} هستند: عمل کردن مانند یک فرد معقول، و نه صرفاً تمرکز بر متن تحت‌اللفظی قرارداد. این ابزار همچنین می‌تواند برای پر کردن شکاف‌های قرارداد استفاده شود. یک نقطه ضعف استفاده از حسن نیت این است که می‌تواند منجر به انحراف دادگاه‌ها از نیات اصلی طرفین شود. در کامن‌لا، حسن نیت به طور کلی به عنوان یک ابزار پذیرفته نشده است، و بحث‌هایی در مورد اینکه آیا می‌توان از آن برای اصلاح قراردادها استفاده کرد وجود دارد. با این وجود، دادگاه‌های ایالات متحده نیز پذیرفته‌اند که در هر قراردادی، میثاق ضمنی حسن نیت و معامله منصفانه وجود دارد.

\subsubsection{جبران خسارت‌های قراردادی}
\label{sec:19-4-3-remedies}

اگر قرارداد به درستی اجرا نشود (که یعنی اگر طرف دیگر به طور کامل آنطور که در قرارداد وعده داده بود عمل نکند، یعنی او یک یا چند مورد از تعهدات خود را نقض کند)، به اصطلاح «نقض قرارداد» \lr{(breach of contract)} رخ داده است. پس از نقض، طرفی که تعهد به او مدیون بوده است (ما معمولاً این طرف را طلبکار می‌نامیم) ممکن است به یک «جبران خسارت» \lr{(remedy)} استناد کند. جبران خسارت پاسخ قانونی است که دادگاه به طلبکار ارائه می‌دهد، که هدف آن تشویق بدهکار (طرفی که تعهد بر او بوده است) به اجرای واقعی تعهدش است.

طلبکار معمولاً باید ثابت کند که بدهکار تعهد را به درستی انجام نداده است. در مورد تعهد به نتیجه این آسان است، او فقط باید نشان دهد که نتیجه محقق نشده است. در مورد تعهد به وسیله این دشوارتر است: او باید نشان دهد که بدهکار مراقبت و مهارتی را که لازم بود ارائه نداده است، که معمولاً تنها به طور غیرمستقیم قابل تعیین است.

\paragraph{۱۹.۴.۳.۱ پیش‌نیازهای استناد به جبران خسارت}
\label{sec:19-4-3-1-prerequisites}
\mbox{}\\
قبل از اینکه بتوانید ادعای جبران خسارت کنید، اکثر سیستم‌های حقوقی ابتدا نیاز دارند که بدهکار در «قصور» \lr{(default)} باشد. این بدان معناست که او قطعاً تعهد خود را انجام نداده است، و تقصیر اوست که این کار را نکرده است. این دلالت بر دو الزام دارد: اخطار قصور و فقدان عذر موجه.

\begin{enumerate}
	\item[(الف)] \textbf{بدهکار معمولاً باید شانس دومی برای انجام تعهد خود داشته باشد.}
	
	ممکن است او از عدم اجرا آگاه نبوده و در صورت شکایت شما با کمال میل وضعیت را اصلاح کند. برای مثال، اگر یک تلفن هوشمند از آمازون سفارش دهید، فروشنده نمی‌داند اگر شما بسته را دریافت نکرده‌اید. اگر شکایت کنید، او باید فرصتی برای جبران وضعیت با ارسال تلفن دیگر داشته باشد. بنابراین، بسیاری از سیستم‌های حقوقی از شما می‌خواهند که ابتدا یک «اخطار قصور» \lr{(notice of default)} ارسال کنید: بیانیه‌ای روشن مبنی بر اینکه نقض تعهد قراردادی وجود داشته است، و مهلتی که در آن می‌خواهید نقض برطرف شود. تنها پس از انقضای مهلت، بدهکار در قصور است و می‌توانید از دادگاه درخواست جبران خسارت کنید.
	
	در برخی موارد، زمانی که روشن است که بدهکار به فرصتی برای جبران نقض پاسخ نخواهد داد، یا زمانی که قرارداد خود یک مهلت قطعی را مشخص کرده است (لحظه‌ای که تعهد قطعاً باید در آن انجام شود، مانند تاریخ تحویل کیک عروسی)، ارسال اخطار قصور لازم نیست. در عوض، بدهکار به محض گذشتن مهلت و عدم انجام تعهد، فوراً در قصور تعهد خود قرار می‌گیرد.
	
	\item[(ب)] \textbf{حتی اگر تعهدی آنطور که وعده داده شده انجام نشود، ممکن است بدهکار عذر موجهی داشته باشد.}
	
	علت واقعی نقض ممکن است یک عامل خارجی باشد. برای مثال، بسته‌ای تحویل داده نشد زیرا کل ساختمان به دلیل حمله تروریستی غیرقابل دسترسی بود. در چنین موردی، رسماً نقض قرارداد وجود دارد، اما چون حمله عذر موجهی را فراهم می‌کند، بدهکار مقصر نیست و طلبکار ممکن است جبران خسارتی نداشته باشد. سوال این است: چگونه تعیین می‌کنید که آیا عذر موجهی وجود دارد؟
	
	این معمولاً با نگاه کردن به علت واقعی عدم اجرا یا نقض، و در نظر گرفتن اینکه آیا این علت «قابل انتساب» به اوست انجام می‌شود: آیا او برای آن علت مقصر است، یا مسئولیت اوست. اگر او مقصر نباشد، چنین علتی «فورس ماژور» \lr{(force majeure)} در نظر گرفته می‌شود. این موضوع روشنی نیست. برای مثال، اگر تولیدکننده‌ای نتواند تراشه‌های کامپیوتری را به دلیل اعتصاب در کارخانه‌اش به شما عرضه کند، می‌توانید این را فورس ماژور در نظر بگیرید. با این حال، اگر اعتصاب به این دلیل شروع شد که مدیران با پرسنل خود بدرفتاری کردند، می‌توانید استدلال کنید که خودشان مقصر اعتصاب هستند.
	
	برای اجتناب از این نوع بحث‌ها، قراردادها اغلب حاوی یک «بند فورس ماژور» هستند که رویدادهایی را که فورس ماژور تلقی می‌شوند فهرست می‌کند. چنین لیستی ممکن است حصری باشد یا ممکن است تنها تمثیلی باشد (نشان دادن نوع مواردی که فورس ماژور را تشکیل می‌دهند، و اجازه دادن به اینکه سایر موارد مشابه نیز فورس ماژور را تشکیل دهند).
\end{enumerate}

در سیستم‌های حقوق نوشته، عموماً تمام تعهدات قراردادی (همانطور که از تفسیر بندهای قرارداد ناشی می‌شود) در صورت عدم اجرای صحیح تعهد، منجر به جبران خسارت می‌شوند. در سیستم‌های کامن‌لا، عمدتاً انگلستان و ایالات متحده، تنها بندهای قراردادی خاص ممکن است منجر به نقض قرارداد و جبران خسارت شوند.

در انگلستان، تمایز بین اظهارات \lr{(representations)} (که منجر به دعوای اظهار خلاف واقع می‌شود) و شروط \lr{(terms)} (که منجر به نقض قرارداد می‌شود) است. چنین شروطی بیشتر به شروط اصلی \lr{(conditions)} (که منجر به فسخ می‌شود)، ضمانت‌ها \lr{(warranties)} (که علت خسارت هستند)، و شروط میانی \lr{(intermediate terms)} (که ممکن است منجر به فسخ و/یا خسارت شوند) تقسیم می‌شوند.

در ایالات متحده، اظهارات نیز منجر به دعوای اظهار خلاف واقع می‌شوند. با این حال، در حالی که در حقوق انگلستان بندی که اظهار نیست تمایل دارد به طور خودکار شرطی باشد که منجر به جبران خسارت می‌شود، در حقوق ایالات متحده، بندی که اظهار نیست به طور خودکار منجر به جبران خسارت نمی‌شود. تنها به اصطلاح «ضمانت‌ها» به طلبکار اجازه می‌دهند تا به جبران خسارت استناد کند. قرارداد اغلب نوع جبران خسارت‌هایی را که به نقض ضمانت‌های خاص متصل هستند مشخص می‌کند. در سیستم‌های حقوق نوشته، نیازی نیست که وارد جزئیات شوید که کدام جبران خسارت‌ها اعمال می‌شوند و به چه روشی. با این حال، هنوز هم مفید است که این موضوع را تحت حقوق نوشته نیز تصریح کنید، زیرا این امر ممکن است از بحث‌های پیچیده در صورت بروز نقض جلوگیری کند.

بنابراین، چهار الزام احتمالی قبل از اینکه بتوانید به جبران خسارت استناد کنید وجود دارد:
\begin{itemize}
	\item نقض یک تعهد قراردادی
	\item قصور \lr{(Default)}
	\item قابلیت انتساب (بدون عذر، بدون فورس ماژور)
	\item در کامن‌لا: تعهد در بندی قرار داده شده است که اجازه جبران خسارت می‌دهد (مانند ضمانت یا شرط)
\end{itemize}

\paragraph{۱۹.۴.۳.۲ جبران خسارت‌های موجود}
\label{sec:19-4-3-2-available-remedies}
\mbox{}\\
به طور کلی، سه جبران خسارت قراردادی وجود دارد:
\begin{itemize}
	\item اجرای عین تعهد \lr{(Specific performance)}
	\item اعطای خسارت \lr{(Award of damages)}
	\item فسخ \lr{(Termination)}
\end{itemize}

این جبران خسارت‌ها را می‌توان از دادگاه درخواست کرد. فسخ ممکن است، بسته به حوزه قضایی، خارج از دادگاه نیز در دسترس باشد (فسخ با اخطار).

\textbf{اجرای عین تعهد} به این معناست که دادگاه به بدهکار دستور می‌دهد تا تعهد خود را انجام دهد. برای مثال، شما قرارداد بسته‌اید تا یک پایگاه داده بسازید، اما از انجام آن امتناع می‌کنید، و دادگاه متعاقباً به شما دستور می‌دهد تا آنچه را وعده داده‌اید انجام دهید. در سیستم‌های حقوق نوشته، این جبران خسارت اولیه است. در کامن‌لا، اجرای عین تعهد همیشه در دسترس نیست: بستگی به نوع تعهد دارد. دلیل این نگرش ممکن است این باشد که در کامن‌لا اجرای عین تعهد در اصل توسط دادگاه نظارت می‌شود (که برای دادگاه وقت‌گیر است)، زیرا نقض اجرای عین تعهد به معنای «تمرد از دستور دادگاه» است. در سیستم‌های حقوق نوشته، اجرای عین تعهد تعهدی به طلبکار است و اغلب از طریق یک «جریمه» خصوصی اجرا می‌شود که طلبکار می‌تواند بدون زحمت دادن به دادگاه به آن استناد کند.\footnote{در حقوق فرانسه، این «astreinte» است، نوعی جریمه که زمانی که تعهد پس از یک دوره معین هنوز انجام نشود، قابل پرداخت است.}

\textbf{اعطای خسارت} جبران خسارت اولیه در اکثر موارد است، چه در حقوق نوشته و چه در کامن‌لا. این بدان معناست که دادگاه به طلبکار خسارت اعطا می‌کند، یعنی بدهکار باید خسارت را به طلبکار بپردازد. خسارت \lr{(damages)} به معنای مبلغی پول است، با هدف جبران پیامدهای منفی نقض برای طلبکار. توجه داشته باشید: خسارات \lr{(damages)} مبلغی پول است؛ خسارت \lr{(damage)} (بدون s) به معنای زیان یا آسیب واقعی است. خسارات برای جبران خسارت در نظر گرفته شده‌اند. وکلا چندین نوع خسارت را متمایز می‌کنند.

یک تمایز مهم در اشکال خسارت بین موارد زیر است:
\begin{enumerate}
	\item صدمه شخصی \lr{(personal injury)}
	\item خسارت به اموال \lr{(property damage)}
	\item زیان اقتصادی محض \lr{(pure economic loss)}
\end{enumerate}

\textbf{صدمه شخصی} به معنای خسارتی است که از آسیب به بدن یا سلامت یک فرد ناشی می‌شود، مانند هزینه‌های پزشکی و از دست دادن درآمد (به دلیل صدمه).

\textbf{خسارت به اموال} آسیب به اموال فیزیکی و پیامدهای این آسیب است. نمونه‌ها کاهش ارزش خودروی آسیب‌دیده، هزینه‌های تعمیر خودرو، و همچنین هزینه حمل‌ونقل جایگزین در زمانی است که از خودرو نمی‌توان استفاده کرد.

\textbf{زیان اقتصادی محض} شامل تمام انواع دیگر خسارات می‌شود: این به معنای زیان‌هایی است که نه از صدمه شخصی ناشی می‌شوند و نه از خسارت به اموال. مثالی از آن یک بیانیه تهمت‌آمیز است: این با صدمه یا خسارت به اموال شروع نمی‌شود، بلکه با یک آسیب غیرمادی شروع می‌شود. نمونه‌ها از دست دادن سود، زمان تلف شده و زیان‌های بازار سهام هستند. همچنین خسارت به یک پایگاه داده یا فایل داده زیان اقتصادی محض است. بنابراین، یک زیان اقتصادی (مانند از دست دادن درآمد) تنها در صورتی زیان اقتصادی محض است که پیامد صدمه یا خسارت به اموال نباشد؛ در غیر این صورت، تحت یکی از دو شکل دیگر خسارت قرار می‌گیرد.

اهمیت این تمایز این است که دو نوع اول خسارت معمولاً قابل جبران هستند، از جمله زیان‌های تبعی ناشی از صدمه یا خسارت اولیه. در مقابل آن، برخی حوزه‌های قضایی (به ویژه کامن‌لا) اجازه جبران زیان اقتصادی محض برای نقض قرارداد را نمی‌دهند. بنابراین، حتی اگر شما ممکن است رسماً جبران خسارتی برای نقض قرارداد داشته باشید، در واقعیت، ممکن است زیان شما با اعطای خسارت جبران نشود، زیرا زیان، زیان اقتصادی محض محسوب می‌شود. اگر خواهان جبران برای این نوع زیان‌ها هستید، باید بندی برای «خسارات مقطوع» \lr{(liquidated damages)} برای نقض‌های خاص اضافه کنید (در ادامه ببینید).

مثالی که ممکن است تفاوت بین این نوع زیان‌ها را نشان دهد: ارائه‌دهنده خدمات \lr{IT} شما به طور تصادفی پایگاه داده شما را حذف می‌کند. از آنجا که حذف ناشی از خسارت به اموال یا صدمه شخصی نیست، این زیان اقتصادی محض محسوب می‌شود و ممکن است در انگلستان قابل جبران نباشد. اگر ارائه‌دهنده خدمات اتاق سرور شما را آتش می‌زد، که به موجب آن پایگاه داده و تمام نسخه‌های پشتیبان (روی هارد دیسک‌ها) از بین می‌رفت، از دست رفتن پایگاه داده زیان اقتصادی تبعی ناشی از خسارت به اموال می‌بود. این زیان قابل جبران می‌بود. بنابراین توصیف بستگی به خود زیان ندارد، بلکه به روشی که در آن زیان رخ داده است (به عنوان پیامد خسارت به اموال یا صدمه شخصی، یا نه) بستگی دارد.

بندهای محدودیت یا بندهای معافیت بندهایی هستند که میزان یا نوع خسارتی را که بدهکار باید در صورت نقض جبران کند، محدود می‌کنند. مثالی که اغلب در مجوزهای نرم‌افزار یافت می‌شود این است که تنها خسارت ناشی از صدمه شخصی یا خسارت فیزیکی به اموال جبران می‌شود.\footnote{به عبارت دیگر، جبران زیان اقتصادی محض صراحتاً مستثنی شده است.} از آنجا که نرم‌افزار معمولاً باعث صدمه شخصی یا خسارت به اموال نمی‌شود، نتیجه این است که تولیدکننده نرم‌افزار معمولاً مجبور به پرداخت هیچ خسارتی نخواهد بود (مگر اینکه بند ناعادلانه باشد). یک بند خسارت همچنین ممکن است میزان خسارتی را که ممکن است اعطا شود محدود کند، برای مثال به حداکثر ۱۰,۰۰۰ یورو.

بندهای خسارات مقطوع بندهایی هستند که میزان خسارات اعطایی برای انواع خاصی از نقض را تثبیت می‌کنند. مزیت این است که هر دو طرف از قبل می‌دانند چه مقدار خسارت برای نقض‌های خاص باید پرداخت شود، که به موجب آن طرفین نیازی به بحث‌های طولانی در مورد مثلاً ارزش یک پروژه اتوماسیون ناموفق ندارند. اگر مبلغ بسیار بیشتر از زیان واقعی متحمل شده باشد، چنین بندی به منزله «بند جریمه» \lr{(penalty clause)} است.\footnote{بندهای جریمه معمولاً مجاز هستند، اما برخی حوزه‌های قضایی آن‌ها را مجاز نمی‌دانند. در سایر حوزه‌های قضایی، اگر آن‌ها به طور نامتناسبی بالا باشند، ممکن است توسط دادگاه تعدیل شوند.}

بندهای محدودیت و بندهای خسارات مقطوع می‌توانند به منزله یک شرط ناعادلانه باشند، در این صورت مجاز نیستند (بخش ۱۹.۴.۲). برای مثال، معمولاً مجاز نیست که مسئولیت مرگ یا صدمه شخصی را به طور کامل سلب کنید.

علاوه بر این، خسارت تنها در صورتی جبران می‌شود که خیلی دور از نقض نباشد. در حقوق انگلستان، الزام «قابلیت پیش‌بینی» وجود دارد: خسارت باید قابل پیش‌بینی باشد. نقض قرارداد ممکن است برای مثال نیاز به جبران هزینه‌های جایگزینی داشته باشد، اما نه جبران هزینه‌های روان‌درمانی کارمندی که احساس می‌کند شخصاً توسط نقض بی‌احترامی شده است. این میزان مسئولیت را محدود می‌کند.

\textbf{فسخ} به این معناست که قرارداد به دلیل نقض به پایان می‌رسد. نتیجه این است که هیچ تعهد دیگری بین طرفین وجود ندارد، به جز تعهدات لازم برای خاتمه دادن به قرارداد. فسخ ممکن است این اثر را داشته باشد که اجرای قراردادی باید بازگردانده شود: هر آنچه انجام شده یا داده شده است خنثی می‌شود.

این ممکن است برای مثال به این معنی باشد که پایگاه داده‌ای که ایجاد شده و به یک شرکت تحویل داده شده است باید به توسعه‌دهنده بازگردانده شود، اما پرداختی که توسعه‌دهنده دریافت کرده است نیز ممکن است مجبور باشد بازگردانده شود.\footnote{برای مثال، بخش ۶:۲۷۱ کد مدنی هلند.} یک پیامد همچنین می‌تواند این باشد که شرکت ممکن است پایگاه داده را نگه دارد اما باید مبلغ مشخصی پول برای آن به توسعه‌دهنده بپردازد. همان‌طور که این مثال روشن می‌کند، احتمالات زیادی وجود دارد. مفید است که مهم‌ترین پیامدهای فسخ را با جزئیات تنظیم کنید تا اطمینان حاصل شود که پس از فسخ در موقعیتی که می‌خواهید خواهید بود.

انواع دیگر فسخ وجود دارد که نیاز به نقض ندارد (برای مثال، بندی که اجازه فسخ می‌دهد اگر طرف قرارداد شما توسط رقیب شما تصاحب شود).

اکنون می‌توانیم به مثال بخش ۱۹.۱ بازگردیم: حذف پایگاه داده به وضوح به عنوان بخشی از قرارداد در نظر گرفته نشده است، در حالی که قابل تصور است که آلیس باید اقدامات احتیاطی را در برابر چنین اتفاق ناگواری انجام می‌داد. بنابراین، ممکن است گفته شود که این نقض قرارداد را تشکیل می‌دهد. اگر کامن‌لا اعمال شود، همچنین ممکن است لازم باشد که ضمانت یا شرط دیگری وجود داشته باشد که زمینه‌ای برای نقض قرارداد در صورت حذف پایگاه داده فراهم کند. اگرچه اشتباه توسط ایو انجام شده است، آلیس به عنوان کارفرمای او مسئول است زیرا نمی‌تواند خود را با اشاره به تقصیر کارمند تبرئه کند (این فورس ماژور محسوب نمی‌شود).

با این حال، زیان، زیان اقتصادی محض است که تحت کامن‌لا ممکن است بر اساس نقض قرارداد قابل بازیابی نباشد. این ممکن است متفاوت باشد اگر قرارداد حاوی بند خسارات مقطوع باشد. در یک حوزه قضایی حقوق نوشته، زیان ممکن است قابل جبران باشد، و خسارات ممکن است به ارزش بازار پایگاه داده یا هزینه‌های (باز)تولید پایگاه داده ارزیابی شود (نگاه کنید به بخش ۶.۴).



% =================================================================
% ۱۹.۵ حقوق مسئولیت مدنی (شبه‌جرم) و داده‌ها
% =================================================================
\subsection{حقوق مسئولیت مدنی (شبه‌جرم) و داده‌ها}
\label{sec:19-5-tort-law-and-data}

اگر قراردادی بین دو نفر وجود نداشته باشد، مسئولیت باید بر اساس حقوق مسئولیت مدنی \lr{(tort law)} باشد.\footnote{در سیستم‌های حقوق نوشته، نام‌های جایگزینی مانند حقوق مسئولیت ناشی از جرم \lr{(delictual liability)} یافت می‌شود. ادبیات عمومی در مورد حقوق مسئولیت مدنی: \lr{Van Dam 2013, Tjong Tjin Tai 2022}.} یک شبه‌جرم، به زبان ساده، برای موارد خاص توصیف می‌کند که تحت چه شرایطی یک شخص مسئول است. مثالی از آن افترا \lr{(defamation)} است: این شبه‌جرم تنظیم می‌کند که چه زمانی کسی برای بیان اظهارات افتراآمیز مسئول است. انواع مختلفی از شبه‌جرم‌ها وجود دارد. در حقوق مسئولیت مدنی، ما بین دو شکل از مسئولیت تمایز قائل می‌شویم: مسئولیت مبتنی بر تقصیر و مسئولیت محض.

\subsubsection{مسئولیت مبتنی بر تقصیر \lr{(Fault Liability)}}
\label{sec:19-5-1-fault-liability}

ایده کلی مسئولیت مبتنی بر تقصیر این است که کسی (وکلا این شخص را «مرتکب شبه‌جرم» یا \lr{tortfeasor} می‌نامند) مسئول است اگر مرتکب تقصیری شده باشد: او کاری انجام داده است که از نظر قانونی نباید انجام می‌داد، و این تقصیر باعث آسیب به قربانی شده است. بنابراین، سه شرط وجود دارد:
\begin{itemize}
	\item رفتار نادرست (تقصیر)
	\item آسیب \lr{(Harm)}
	\item رابطه سببی بین تقصیر و آسیب
\end{itemize}

آسیب به معنای نوعی عدم مزیت، حادثه یا زیان است. این ممکن است یک حادثه فیزیکی، آسیب به کالا، آسیب به شهرت و از دست دادن حریم خصوصی باشد. رابطه سببی در بخش ۱۹.۵.۳ بیشتر مورد بحث قرار می‌گیرد.

اکثر شبه‌جرم‌ها با نوع تقصیری که پوشش می‌دهند تعریف می‌شوند، اما نوع آسیب نیز ممکن است مرتبط باشد. مثالی از آن دوباره افترا است: این مورد برای نوع خاصی از اظهارات اعمال می‌شود، اما همچنین نیاز به وجود آسیب به شهرت قربانی دارد. سیستم‌های حقوقی روش‌های مختلفی برای تعیین اینکه آیا رفتار خاصی (که ممکن است شامل فعل و همچنین ترک فعل باشد)\footnote{در سیستم‌های کامن‌لا، ترک فعل‌های محض (که ناشی از یک عمل مسامحه‌آمیز قبلی نیستند) ممکن است منجر به مسئولیت نشوند.} نادرست است یا خیر، دارند.

مهم‌ترین شبه‌جرم، «غفلت» یا «مسامحه» \lr{(negligence)} است.\footnote{این نام در کامن‌لا است؛ در سیستم‌های حقوق نوشته، نام‌های دیگری یافت می‌شود، اما غفلت ممکن است برای نشان دادن شبه‌جرم محلی در ترجمه انگلیسی استفاده شود.} دقیق‌تر بگوییم، این یک شبه‌جرم در کامن‌لا است، اما سیستم‌های حقوق نوشته نیز قواعدی دارند که مسئولیت را در مواردی که تحت پوشش شبه‌جرم غفلت قرار می‌گیرند، فراهم می‌کنند. غفلت به این معناست که شما مراقبت یا دقت کافی را نسبت به منافع قربانی رعایت نکرده‌اید: این مستلزم نقض «وظیفه مراقبت» \lr{(duty of care)} است.

غفلت می‌تواند برای مثال زمانی اعمال شود که شما الگوریتمی برای یک خودروی خودران توسعه داده‌اید بدون اینکه در طول توسعه مراقبت کافی برای جلوگیری از اشتباهات در شناسایی موانع انجام داده باشید، و خودرو عابری را کشت زیرا فکر می‌کرد یک جعبه مقوایی است. غفلت مفید است زیرا یک هنجار باز \lr{(open-ended norm)} ارائه می‌دهد که می‌تواند برای تحولات جدید اعمال شود. نقطه ضعف این است که کاربرد واقعی ممکن است نامشخص باشد: چه زمانی رفتار مسامحه‌آمیز است؟ آزمون معمول این است که آیا یک «فرد معقول» \lr{(reasonable person)} به همان روش عمل می‌کرد یا خیر.

شبه‌جرم‌های زیادی غیر از غفلت وجود دارند که ممکن است برای داده‌ها اعمال شوند. نمونه‌ها نقض حریم خصوصی و افترا هستند. این‌ها توسط قواعد و شرایط خاصی اداره می‌شوند؛ این مرور کلی جای پرداختن به جزئیات نیست. علاوه بر این، بسیاری از سوءاستفاده‌ها یا مداخلات نادرست در داده‌ها و رایانه‌ها جرم محسوب می‌شوند.\footnote{به ویژه به دلیل تأثیر کنوانسیون‌های مختلف در مورد جرایم سایبری.} نقض حقوق کیفری معمولاً به عنوان یک شبه‌جرم نیز تحت عنوان شبه‌جرم «نقض وظیفه قانونی» \lr{(breach of a statutory duty)} در نظر گرفته می‌شود.

\subsubsection{مسئولیت محض \lr{(Strict Liability)}}
\label{sec:19-5-2-strict-liability}

مسئولیت محض به این معناست که شما مسئول هستید حتی اگر شخصاً مرتکب تقصیری نشده باشید. نمونه‌ها عبارتند از:
\begin{itemize}
	\item مسئولیت کارفرما برای شبه‌جرم‌های ارتکابی توسط کارمند
	\item مسئولیت مالک خودرو برای حوادث شامل خودرو
	\item مسئولیت محصولِ تولیدکننده صنعتیِ محصولات ملموس
\end{itemize}

اکثر حوزه‌های قضایی این سه شکل از مسئولیت محض را به رسمیت می‌شناسند. برای اعمال این اشکال از مسئولیت محض، حداقل موارد زیر مورد نیاز است:
\begin{enumerate}
	\item رابطه خاصی بین شخصی که مسئول شناخته می‌شود و شیء یا شخصی که باعث خسارت شده است (رابطه استخدامی، مالکیت، کنترل).
	\item شکل خاصی از رفتار یا فعالیت شیء یا شخص (عمل شبه‌جرمی، تحقق یک خطر خاص).
\end{enumerate}

در بسیاری از کشورها، اشکال دیگری از مسئولیت محض نیز وجود دارد، مانند مسئولیت برای کودکان، حیوانات، اشیاء خطرناک و فعالیت‌های خطرناک. با این حال، این اشکال همه جا یا به یک اندازه پذیرفته نشده‌اند. مزیت مسئولیت محض این است که حفاظت بیشتری برای قربانیان فراهم می‌کند، که می‌توانند جبران خسارت را از طرفی که واقعاً از فعالیت پرخطر یا استخدام شخص سود می‌برد (یا تصمیم به ورود به آن گرفته است) تأمین کنند. در ادبیات، مسئولیت محض اغلب به عنوان مدلی برای تنظیم مسئولیت ربات‌ها و الگوریتم‌ها پیشنهاد می‌شود \lr{(Tjong Tjin Tai 2018a, and references therein)}.

برای اطمینان از اینکه مسئولیت محض بیش از حد گسترده نمی‌شود، چندین محدودیت وجود دارد. به ویژه، محدودیت‌های عمومی علیت و ارزیابی خسارات، و همچنین دفاعیات اعمال می‌شوند (بخش‌های ۱۹.۵.۳ و ۱۹.۵.۴).

\subsubsection{رابطه سببی و دفاعیات}
\label{sec:19-5-3-causality-defenses}

یک شرط برای دریافت جبران خسارت برای یک شبه‌جرم این است که آسیبی وجود داشته باشد که توسط تقصیر ایجاد شده باشد: «علیت» \lr{(causality)}. رابطه سببی بین تقصیر و آسیب ابتدا با نگاه کردن به رابطه سببی واقعی (علیت واقعی) ارزیابی می‌شود: آیا اگر تقصیر رخ نمی‌داد، خسارت نیز رخ می‌داد؟ این به اصطلاح «آزمونِ اگر-نبود» \lr{(but-for test)} (یا \lr{conditio sine qua non} در سیستم‌های حقوق نوشته) یک شرط لازم برای فرض یک شبه‌جرم است.

پس از آن، یک رابطه سببی دوم مورد نیاز است: «علیت حقوقی» \lr{(legal causality)}. آسیب و خسارت نباید خیلی دور باشد. این آزمونی مشابه با مسئولیت قراردادی است: قابلیت پیش‌بینی یا دوری عمومی \lr{(remoteness)} ممکن است به عنوان معیار استفاده شود. جزئیات بستگی به شبه‌جرم و سیستم حقوقی دارد. دوری کمک می‌کند تا قرار گرفتن احتمالی در معرض مسئولیت محدود شود. برای مثال، اگر شما اشتباهی در به‌روزرسانی یک روتین متن‌باز مرتکب شوید که در نهایت باعث اختلال در عملکرد سرورها در سراسر جهان شود، مسئولیت شما ممکن است محدود شود اگر این پیامدها خیلی دور تشخیص داده شوند.

اگر چندین مرتکب شبه‌جرم وجود داشته باشد، علیت معمولاً «تضامنی» \lr{(joint and several)} است: مرتکبین شبه‌جرم به صورت فردی و همچنین جمعی مسئول هستند. هر کدام ممکن است به تنهایی برای کل مبلغ خسارت مورد شکایت قرار گیرند؛ پس از آن، مرتکبی که خسارت را پرداخت کرده است ممکن است سهمی را از سایر مرتکبین دریافت کند.

اگر دفاعیه‌ای اعمال شود، مسئولیت کاهش می‌یابد یا محدود می‌شود. یک دفاع مهم «غفلت مشارکتی» \lr{(contributory negligence)} است: اگر قربانی کاری انجام داده باشد که همچنین در وقوع آسیب یا میزان خسارت نقش داشته باشد، دادگاه ممکن است اعطای خسارت را به نسبتی که قربانی در خسارت نقش داشته است کاهش دهد.

سایر دفاعیات ممکن است مانع مسئولیت شوند. مثالی از آن دفاع فورس ماژور در مسئولیت محض است.\footnote{این تا حدودی شبیه به فورس ماژور در حقوق قراردادها است (بخش ۱۹.۴.۳.۱).} اگر برای مثال حادثه‌ای نه توسط خودرو یا راننده آن، بلکه به دلیل برخورد کامیونی از عقب به خودرو و در نتیجه هل دادن آن به سمت خودروی جلویی ایجاد شده باشد، این ممکن است مالک خودرو را از مسئولیت تبرئه کند.

\subsubsection{خسارات و سایر جبران‌ها در مسئولیت مدنی}
\label{sec:19-5-4-damages-tort}

اگر شرایط شبه‌جرم محقق شود، قربانی حق دریافت جبران خسارت دارد. مفهوم کلی جبران خسارت در شبه‌جرم شبیه به جبران خسارت در قرارداد است (بحث شده در بخش ۱۹.۴.۳)، اما الزامات متفاوت است و همه جبران خسارت‌ها در هر دو موقعیت اعمال نمی‌شوند.

جبران خسارت اصلی برای یک شبه‌جرم، اعطای خسارت است. سایر جبران خسارت‌های مهم ممکن است در قالب حکم دادگاه \lr{(court order)} داده شوند. احکام دادگاه اعلامیه‌هایی توسط دادگاه هستند. انواع مختلفی از احکام وجود دارد؛ احکامی که به عنوان ارائه دهنده جبران خسارت مهم هستند آن‌هایی هستند که به طرفی دستور می‌دهند عمل یا اعمال خاصی را انجام دهد، یا از رفتار خاصی خودداری کند. مثالی از آن حکم منع کننده \lr{(restraining order)} است که رفتار خاصی را ممنوع می‌کند. نوع خاصی از حکم، «دستور موقت» \lr{(injunction)} است که انجام یک عمل خاص را ممنوع یا دستور می‌دهد.\footnote{تعاریف دقیق احکام و دستورات موقت ممکن است بین سیستم‌های حقوقی مختلف متفاوت باشد.}

اعطای خسارت به این معناست که زیانی که شبه‌جرم ایجاد کرده است باید توسط مرتکب جبران شود. دادگاه خسارت را ارزیابی می‌کند. با این حال، چندین پیچیدگی وجود دارد.

اول از همه، برای بسیاری از شبه‌جرم‌ها، همه انواع خسارت جبران نمی‌شوند. به ویژه، زیان اقتصادی محض اغلب جبران نمی‌شود.\footnote{برای مثال، شبه‌جرم غفلت در بسیاری از موارد اجازه جبران زیان اقتصادی محض را نمی‌دهد، اگرچه چنین جبرانی همیشه مستثنی نیست.} از آنجا که پرونده‌های مربوط به داده‌ها اغلب تنها باعث زیان اقتصادی محض می‌شوند، چنین شبه‌جرم‌هایی ممکن است عملاً جبران خسارت مناسبی نداشته باشند.

شکل‌های جایگزین خسارت وجود دارد که اشکال معمول خسارت را تکمیل می‌کند. یک مثال خاص «خسارات تنبیهی» \lr{(punitive damages)} است (اعطای مبلغی پول که از خسارت واقعی فراتر می‌رود: این به عنوان مجازات مرتکب شبه‌جرم تحمیل می‌شود تا به عنوان بازدارنده عمل کند).

آسیب‌هایی که ممکن است از داده‌ها ناشی شوند چیست؟ به طور خیلی کلی، می‌توانیم مسائل زیر را در نظر بگیریم:
\begin{itemize}
	\item فقدان کیفیت
	\item خطا در تحلیل
	\item از دست دادن داده‌ها
	\item نشت داده‌ها/از دست دادن حریم خصوصی
\end{itemize}

در مورد موضوع اول، اطلاعات بسیار کمی در این مورد وجود دارد \lr{(Zeno-Zencovich, 2018)}. در کلان‌داده، کیفیت به عنوان یک امر مسلم فرض نمی‌شود؛ بلکه ایده کلان‌داده اغلب کار با آنچه دارید، آلوده یا نه، با این فرض است که لکه‌های جزئی از نظر آماری خنثی می‌شوند. با این حال، داده‌هایی که به درستی ساختار نیافته‌اند یا به نحو دیگری نادرست هستند ممکن است باعث پیامدهای منفی شوند: کاربر ممکن است مجبور شود تلاش قابل توجهی برای نرمال‌سازی یا بازسازی داده‌ها صرف کند، و ممکن است نتیجه‌گیری‌های نادرستی بگیرد که منجر به تصمیمات بد شود.

خارج از قرارداد، دلیل کمی برای فرض اینکه شما دعوی برای فقدان کیفیت دارید وجود دارد، زیرا شما برای داده‌ها پولی پرداخت نکرده‌اید. اگر داده‌ها را در دسترس قرار می‌دهید، می‌تواند مفید باشد که تصریح کنید هیچ تضمینی در مورد کیفیت داده‌ها ارائه نمی‌دهید. اگر داده‌ها بر اساس قرارداد ارائه شوند، پیامدهای منفی در تئوری می‌تواند جبران شود، اما این‌ها معمولاً زیان اقتصادی محض هستند و ممکن است خارج از جبران خسارت قرار گیرند. علاوه بر این، اکثر قراردادها حاوی بندهایی در مورد چنین خسارتی هستند، که یا قواعد روشن خسارات مقطوع یا حتی بندهای استثنا/محدودیت را ارائه می‌دهند.

در مورد خطا در تحلیل: این احتمالاً یک مسئله قراردادی خواهد بود. قواعد قرارداد اعمال خواهد شد، به ویژه بندهای خسارات مقطوع یا بندهای محدودیت.

برای از دست دادن داده‌ها، ایده کلی این خواهد بود که از دست دادن ارزش داده‌ها می‌تواند جبران شود، یا احتمالاً هزینه بازسازی پایگاه داده.

برای نشت داده‌ها و از دست دادن حریم خصوصی، اغلب هیچ جبران خسارت مؤثری وجود ندارد \lr{(Peters 2014, Varuhas 2018)}. مثال یک وب‌سایت اجتماعی را در نظر بگیرید که هک شده است، که به موجب آن تمام داده‌های خصوصی شما در معرض هکرها قرار گرفته است. در حالی که این ممکن است تجاوز جدی به حریم خصوصی باشد، هیچ زیان مادی از این نقض وجود ندارد. حریم خصوصی به خودی خود ارزش روشنی ندارد؛ در بهترین حالت، ممکن است مقدار نمادین یا اسمی خسارت برای آسیب غیرمادی، یا در موارد نادر فجیع خسارات تنبیهی اعطا شود.

در حالی که داده‌های خصوصی شما می‌تواند برای هک کردن سایر حساب‌ها و متعاقباً ایجاد زیان مادی (مانند سرقت پول) استفاده شود، این معمولاً زیان اقتصادی محض خواهد بود. حتی اگر چنین زیانی در سیستم حقوقی شما قابل جبران باشد، اثبات رابطه سببی بین نشت داده و زیان دشوار خواهد بود. تنها در مورد نقض حریم خصوصی افراد مشهور ممکن است امکان اعطای خسارت قابل توجه وجود داشته باشد. بنابراین، در مورد شبه‌جرم‌های مربوط به داده‌ها، به دست آوردن خسارات قابل توجه ممکن است دشوار باشد.

ممکن است آموزنده باشد که دوباره به مثال بخش ۱۹.۱ نگاه کنیم، این بار از دیدگاه حقوق مسئولیت مدنی. فرض کنید که ایو این بار پایگاه داده شخص ثالثی (چارلی) را که اتفاقاً روی همان سرورِ پایگاه داده مشتری ذخیره شده بود، حذف کرده است. حذف پایگاه داده توسط ایو احتمالاً یک عمل مسامحه‌آمیز (غفلت) را تشکیل می‌دهد. آلیس، به عنوان کارفرما، مسئولیت نیابتی \lr{(vicariously liable)} برای غفلت ایو دارد. از آنجا که زیان، زیان اقتصادی محض است (ناشی از خسارت به اموال یا صدمه شخصی نیست)، ممکن است در کامن‌لا قابل بازیابی نباشد. در سیستم‌های حقوق نوشته، زیان ممکن است جبران شود. می‌توان استدلال کرد که غفلت مشارکتی از طرف چارلی وجود دارد: اگر پایگاه داده آنقدر مهم است، او نیز می‌توانست نسخه پشتیبان تهیه کند. در یک موقعیت قراردادی، اگر قرارداد بدهکار (آلیس) را ملزم به تهیه نسخه پشتیبان کند، این دفاع ممکن است کارساز نباشد.




% =================================================================
% نتیجه‌گیری (Conclusion)
% =================================================================
\vspace{0.8cm}
\noindent
\fcolorbox{black}{gray!15}{%
	\begin{minipage}{\dimexpr\linewidth-2\fboxsep-2\fboxrule}
		\vspace{0.3cm}
		\begin{center}
			\textbf{\Large نتیجه‌گیری}
		\end{center}
		\vspace{0.2cm}
		
		مسائل متعددی وجود دارند که هنگام قرارداد بستن در مورد داده‌ها و هنگام در نظر گرفتن مسئولیت برای داده‌ها نیاز به توجه دارند. علاوه بر نکات حقوقی عمومی ( که در این فصل معرفی شدند)، پیچیدگی‌های خاصی نیز وجود دارد که عمدتاً نتیجه ماهیت ناملموس داده‌هاست. 
		
		باید درک کرد که «داده» یک مفهوم ثابت نیست و ممکن است عناصری از زیرساخت‌های اطراف را نیز در بر بگیرد. توجه ویژه‌ای در مورد جبران خسارت محدودی که ممکن است به دست آید و انواع گسترده‌تر آسیب‌هایی که ممکن است توسط داده‌ها ایجاد شود، مورد نیاز است.
		\vspace{0.3cm}
	\end{minipage}
}

% =================================================================
% پیام‌های کلیدی (Take-Home Messages)
% =================================================================
\vspace{0.8cm}
\noindent
\fcolorbox{black}{gray!15}{%
	\begin{minipage}{\dimexpr\linewidth-2\fboxsep-2\fboxrule}
		\vspace{0.3cm}
		\centering \textbf{\large پیام‌های کلیدی فصل}
		\vspace{0.2cm}
		\begin{itemize}
			\setlength\itemsep{0.5em}
			\item[\textbf{--}] داده یک مفهوم به خوبی تعریف شده در حقوق نیست. ممکن است به فایل‌های داده، اطلاعات موجود در فایل‌ها یا کلان‌داده (مجموعه‌ای از عناصر مختلف) اشاره داشته باشد. فایل‌های داده به عنوان یک نهاد در قانون شناخته نمی‌شوند؛ اطلاعات تنها توسط حقوق مالکیت فکری (از جمله قانون اسرار تجاری) محافظت می‌شود.
			\item[\textbf{--}] در قراردادها، صراحت در مورد انتظارات و پیامدهای عدم تحقق آن‌ها بسیار مهم است. این امر در مورد اظهارات، ضمانت‌ها و تعهدات قراردادی به طور کلی صدق می‌کند.
			\item[\textbf{--}] آسیب ممکن است شامل صدمه شخصی، خسارت به اموال یا زیان اقتصادی محض باشد. زیان اقتصادی محض در بسیاری از موارد به دلیل محدودیت‌های حقوق قراردادها جبران نمی‌شود. در مورد داده‌ها، اغلب فقط زیان اقتصادی محض وجود دارد؛ لذا ممکن است هیچ خسارتی دریافت نکنید.
			\item[\textbf{--}] مسئولیت برای داده‌ها مستلزم وجود یک شبه‌جرم (مسئولیت مدنی) قابل اعمال است که با داده‌ها سروکار دارد و نیاز به عمل نادرست، آسیب و رابطه سببی دارد.
			\item[\textbf{--}] شما ممکن است نه تنها برای اعمال نادرست مستقیم، بلکه برای اعمال شخص دیگر (مسئولیت نیابتی) یا پیامدهای اشیائی که تحت کنترل دارید، مسئول باشید.
		\end{itemize}
		\vspace{0.2cm}
	\end{minipage}
}

% =================================================================
% پرسش‌ها و پاسخ‌ها
% =================================================================
\vspace{1cm}
\section*{پرسش‌ها و پاسخ‌ها}
\addcontentsline{toc}{section}{پرسش‌ها و پاسخ‌ها}

\noindent \textbf{\large ؟ پرسش‌ها}
\begin{enumerate}
	\setlength\itemsep{0.5em}
	\item شروط قراردادی ناعادلانه (سوءاستفاده‌گرانه) به چه روشی تنظیم می‌شوند؟
	\item الزامات استناد به جبران خسارت برای نقض قرارداد چیست؟
	\item زیان اقتصادی محض \lr{(Pure economic loss)} چیست؟
	\item چگونه می‌توان ریسک مسئولیت قراردادی را محدود کرد؟
	\item الزامات اساسی مسئولیت مدنی (شبه‌جرم) چیست؟
\end{enumerate}

\vspace{0.5cm}
\noindent \textbf{\large $\checkmark$ پاسخ‌ها}
\begin{enumerate}
	\setlength\itemsep{0.5em}
	\item از طریق کنترل شروط ناعادلانه یا رویه‌های تجاری ناعادلانه، یا از طریق اصل حسن نیت.
	\item نقض یک تعهد قراردادی، قصور \lr{(default)}، و قابلیت انتسابِ علتِ نقض. در کامن‌لا همچنین: نقض یک ضمانت \lr{(warranty)}.
	\item زیان‌هایی که پیامد صدمه فیزیکی یا خسارت به اموال نیستند.
	\item با محدود کردن دامنه تعهدات/ضمانت‌ها، استفاده از بندهای محدودیت/معافیت، داشتن بند فورس ماژور گسترده و بندهای خسارت مقطوع.
	\item تقصیر، علیت (واقعی و حقوقی) و آسیب.
\end{enumerate}

% =================================================================
% مراجع (References) - اصلاح شده برای رفع خطاها
% =================================================================
\vspace{1cm}
\section*{منابع}
\addcontentsline{toc}{section}{منابع}

\begin{latin}
	\begin{itemize}
		\setlength\itemsep{0.6em}
		
		\item[] Bix, B. H. (2012). \textit{Contract law. Rules, theory, and context}. Cambridge University Press.
		
		\item[] Cartwright, J. (2013). \textit{Contract law. An introduction to the English law of contract for the civil lawyer}. Hart, Oxford.
		
		\item[] Mak, V., Tjong Tjin Tai, T. F. E., \& Berlee, A. (Eds.). (2018). \textit{Research handbook in data science and law}. Elgar Publishing.
		
		\item[] Peters, R. M. (2014). So you’ve been notified, now what? the problem with current data-breach notification laws. \textit{Arizona Law Review}, 56(4), 1171–1202.
		
		\item[] Smits, J. M. (2014). \textit{Contract law: A comparative introduction}. Elgar.
		
		\item[] Tjong Tjin Tai, T. F. E. (2018a). Liability for (semi-)autonomous systems, in: Mak et al. (2018), pp. 55–82.
		
		\item[] Tjong Tjin Tai, T. F. E. (2018b). Data ownership and consumer protection. \textit{Journal of European Consumer and Market Law}, 7(4), 136–140.
		
		\item[] Tjong Tjin Tai, T. F. E. (2022). \textit{Tort Law: A Comparative Introduction}. Elgar.
		
		\item[] US Federal Court of Appeals, \textit{Jacobsen v. Katzer}, 535 F.3d 1373 (Fed. Cir. 2008).
		
		\item[] van Dam, C. C. (2013). \textit{European tort law} (2nd ed.). Oxford University Press.
		
		\item[] Varuhas, J. N. E. (2018). Varieties of damages for breach of privacy. In J. N. E. Varuhas \& N. A. Moreham (Eds.), \textit{Remedies for breach of privacy}. Hart Publishing.
		
		\item[] Ventura, J. (2005). \textit{Law For dummies} (2nd ed.). Wiley.
		
		\item[] Wacks, R. (2015). \textit{Law: A very short introduction}. Oxford University Press.
		
		\item[] Zeno-Zencovich, V. (2018). Liability for data loss, in: Mak et al. (2018), pp. 39–54.
		
	\end{itemize}
\end{latin}
% =================================================================
% فایل chapter4_20.tex (نسخه نهایی و کامل با اصلاح شماره‌گذاری)
% =================================================================

\clearpage % شروع از صفحه جدید

% -----------------------------------------------------------------
% تنظیمات حیاتی شماره‌گذاری (حل مشکل ۱.۲۰ -> ۲۰.۱)
% -----------------------------------------------------------------
\setcounter{section}{20} 
\setcounter{subsection}{0} 
\renewcommand{\thesubsection}{20.\arabic{subsection}} 

% -----------------------------------------------------------------
% تیتر اصلی فصل ۲۰
% -----------------------------------------------------------------
\vspace*{0.5cm}
\begin{flushright}
	{\huge \textbf{۲۰. \hspace{0.2cm} اخلاق داده و علم داده: یک پیوند دشوار؟}}
	\addcontentsline{toc}{section}{۲۰. اخلاق داده و علم داده: یک پیوند دشوار؟}
\end{flushright}

\vspace{1cm}

\begin{center}
	\large \textbf{استر کیمولن و لینت تیلور} \\
	\lr{(Esther Keymolen \& Linnet Taylor)}
\end{center}

\vspace{1cm}

% -----------------------------------------------------------------
% فهرست محتویات فصل
% -----------------------------------------------------------------
\noindent \textbf{\large محتویات فصل}
\begin{description}
	\setlength\itemsep{0em}
	\item[] ۲۰.۱ مقدمه \dotfill \pageref{sec:20-intro}
	\item[] ۲۰.۲ اخلاق داده در محیط آکادمیک \dotfill \pageref{sec:20-academia}
	\begin{itemize}
		\item[] ۲۰.۲.۱ نظریه‌های اخلاقی \dotfill \pageref{subsec:20-moral-theories}
		\item[] ۲۰.۲.۲ پیامدگرایی \dotfill \pageref{subsec:20-consequentialism}
		\item[] ۲۰.۲.۳ اخلاق وظیفه‌گرا \dotfill \pageref{subsec:20-deontological}
		\item[] ۲۰.۲.۴ اخلاق فضیلت \dotfill \pageref{subsec:20-virtue}
		\item[] ۲۰.۲.۵ کانون توجه اخلاق داده آکادمیک \dotfill \pageref{subsec:20-focus}
	\end{itemize}
	\item[] ۲۰.۳ اخلاق داده در حوزه تجاری \dotfill \pageref{sec:20-commercial}
	\begin{itemize}
		\item[] ۲۰.۳.۱ سطح فناورانه \dotfill \pageref{subsec:20-tech-level}
		\item[] ۲۰.۳.۲ سطح فردی \dotfill \pageref{subsec:20-individual-level}
		\item[] ۲۰.۳.۳ سطح سازمانی \dotfill \pageref{subsec:20-org-level}
	\end{itemize}
	\item[] ۲۰.۴ قانون و اخلاق داده \dotfill \pageref{sec:20-law}
	\item[] ۲۰.۵ اخلاق داده و علم داده: آیا این همراهی پایدار است؟ \dotfill \pageref{sec:20-long-run}
	\item[] مراجع \dotfill \pageref{sec:20-references}
\end{description}

\newpage

% =================================================================
% کادر اهداف یادگیری
% =================================================================
\noindent
\fcolorbox{black}{gray!10}{%
	\begin{minipage}{\dimexpr\linewidth-2\fboxsep-2\fboxrule}
		\vspace{0.2cm}
		\textbf{\large اهداف یادگیری}
		\vspace{0.2cm}
		در پایان این فصل، خواننده باید بتواند:
		\begin{itemize}
			\item تمایز میان سه نظریه اخلاقی کلیدی (پیامدگرایی، اخلاق وظیفه‌گرا و اخلاق فضیلت) را تشخیص دهد.
			\item مثال‌هایی ارائه دهد که چگونه می‌توان از نظریه‌های اخلاقی برای تحلیل فعالیت‌های علم داده استفاده کرد.
			\item تفاوت‌های اخلاق داده در محیط آکادمیک و محیط تجاری را مقایسه کند.
			\item تعامل میان قانون و اخلاق را در حوزه علم داده تبیین کند.
			\item ارزیابی نقادانه‌ای از کاربردهای مختلف اخلاق داده داشته باشد.
		\end{itemize}
		\vspace{0.1cm}
	\end{minipage}
}
\vspace{0.5cm}

% =================================================================
% ۲۰.۱ مقدمه
% =================================================================
\subsection{مقدمه}
\label{sec:20-intro}

در دو دهه گذشته، شرکت‌های داده‌محور نحوه عملکرد جامعه را دگرگون کرده‌اند. شهروندان به‌طور فزاینده‌ای از پلتفرم‌های رسانه‌های اجتماعی برای ایجاد و حفظ روابط اجتماعی خود استفاده می‌کنند؛ دولت‌ها مداخلات خود را بر اساس ابزارهای الگوریتمی آماده بنا می‌نهند؛ شهرها با جمع‌آوری انواع داده‌های حسگری از ساکنان خود (با همکاری نزدیک با بخش خصوصی) به سمت «هوشمند» شدن حرکت می‌کنند؛ و در حوزه‌های مختلف—از بهداشت تا حقوق—ابتکاراتی برای پیاده‌سازی ابزارهای هوش مصنوعی \lr{(AI)} جهت بهبود فرآیند تصمیم‌گیری وجود دارد. به معنای واقعی کلمه، آسمان دیگر محدودیت نیست، چرا که ایلان ماسک \lr{(Elon Musk)}، کارآفرین حوزه فناوری، در حال کار بر روی امکان انتقال تعداد زیادی از افراد به مریخ به عنوان پاسخی به بحران اقلیمی است \lr{(New York Times, 2019)}.

همراه با این امیدهای بزرگ به آنچه فناوری و به‌ویژه نوآوری‌های داده‌محور برای ما به ارمغان می‌آورند، ما همزمان با حوادث بزرگ و تأثیرگذاری مواجه هستیم که توسط همین فناوری‌ها و نوآوری‌ها ایجاد شده‌اند. از کمبریج آنالیتیکا \lr{(Cambridge Analytica)} که در انتخابات سراسر جهان مداخله می‌کند \lr{(New Statesman, 2018; Observer, 2020)} تا شرکت \lr{Clearview} که رسانه‌های اجتماعی را برای تغذیه برنامه تشخیص چهره خود می‌کاود \lr{(Wired, 2020)}، و از رفتار نامناسب آمازون با کارمندانش \lr{(Guardian, 2020)} تا پیشنهادات جستجوی نژادپرستانه گوگل \lr{(Wired, 2018)}، مشخص می‌شود که شرکت‌های فناوری با پذیرش مسئولیت تأثیرات اجتماعی و سیاسی خود دست‌ و پنجه نرم می‌کنند.

در برابر این پس‌زمینه، تعجبی ندارد که جامعه مدنی در سراسر جهان از شرکت‌های داده‌محور خواسته است تا مسئولیت خود را جدی بگیرند و برای عادلانه‌تر، شفاف‌تر، پاسخگوتر و قابل‌اعتمادتر شدن تلاش کنند (تنها برای نام بردن از چند هدفی که تعیین شده است). اما دقیقاً چه انتظارات اخلاقی می‌توانیم از این شرکت‌ها داشته باشیم و شرکت‌ها برای بهبود خود چه باید بکنند؟ با این آرمان‌ها برای همسوسازی بیشتر شرکت‌های داده‌محور و همچنین محصولات و خدمات آن‌ها با ارزش‌های محوری یک جامعه دموکراتیک و شکوفا، \gls{data-ethics} وارد صحنه می‌شود.

همان‌طور که در این فصل خواهیم دید، \gls{data-ethics} یک مفهوم مبهم \lr{(fuzzy concept)} است که می‌تواند برای افراد مختلف معانی متفاوتی داشته باشد. بنابراین، بخش بزرگی از این فصل به توضیح \gls{data-ethics} از زوایای مختلف اختصاص دارد. ما با مقدمه‌ای کوتاه بر \gls{data-ethics} به عنوان یک رشته دانشگاهی آغاز می‌کنیم و نشان می‌دهیم که چگونه برخی از این دیدگاه‌های آکادمیک در بحث‌های مربوط به \gls{data-science} و هوش مصنوعی نفوذ می‌کنند. سپس، تمرکز خواهیم کرد بر اینکه چگونه \gls{data-ethics} به عنوان یک استراتژی تنظیمی توسط شرکت‌های داده‌محور مطرح شده است تا پاسخی به حوادث و چالش‌های توصیف‌شده در بالا تدوین کنند.

در ادامه، توضیح خواهیم داد که \gls{data-ethics} چگونه با \gls{law} (قانون) ارتباط دارد (که همچنین با سایر فصل‌های این ماژول ارتباط نزدیکی دارد). ما به ریسک «اخلاق‌شویی» \lr{(ethics washing)} خواهیم پرداخت؛ حالتی که در آن اگر \gls{data-ethics} در این بافت کارآفرینی به‌درستی نهادینه نشود، ممکن است به عنوان راه فراری از مقررات قانونی مورد استفاده قرار گیرد. ما این فصل را با تأملی بر رابطه آینده \gls{data-ethics} و \gls{data-science} به پایان می‌بریم و تعدادی سؤال بحث‌برانگیز برای تحریک بحث‌های بیشتر ارائه می‌دهیم.

% =================================================================
% ۲۰.۲ اخلاق داده در محیط آکادمیک
% =================================================================
\raggedbottom
\subsection*{۲۰.۲ \gls{data-ethics} در \gls{academia}}
\addcontentsline{toc}{subsection}{۲۰.۲ \gls{data-ethics} در \gls{academia}}
\label{sec:20-academia}

بیایید ابتدا به بخش «اخلاق» در \gls{data-ethics} نگاه کنیم. اخلاق شاخه‌ای از فلسفه است که حول این پرسش می‌چرخد: «چگونه باید عمل کرد؟» این رشته به روشی سیستماتیک، دلایل و استانداردهای زیربنای اعمال ما را مطالعه می‌کند و بررسی می‌کند که چه چیزی اعمال ما را از نظر اخلاقی درست یا غلط، و خوب یا بد می‌سازد \lr{(Timmons, 2012, p. 4)}. «از نظر اخلاقی» در اینجا یک صفت مهم است، زیرا دامنه نوع اعمالی که به آن‌ها علاقه‌مند هستیم را تعیین می‌کند. یک مثال:

\vspace{0.3cm}
\noindent \textbf{شما باید سوپ خود را با قاشق بخورید.}

این جمله دستوری \lr{(prescriptive)} است. به شما می‌گوید چگونه عمل کنید، احتمالاً بر اساس دلایل کارایی و آداب معاشرت. این جمله هنجاری \lr{(normative)} است، به این معنا که به شما می‌گوید عمل مناسب در یک موقعیت خاص چیست (استفاده از قاشق). با این حال، اگرچه این یک جمله هنجاری است، اما لزوماً یک جمله اخلاقی نیست. اینکه سوپ را با قاشق بخورید یا نه، بر ارزش‌های کلیدی مانند کرامت انسانی، آزادی یا رفاه تأثیر نمی‌گذارد. عموماً، این کار با شانس زندگی خوب شما یا دیگران تداخلی ندارد. خوردن سوپ با قاشق یک وظیفه اخلاقی نیست؛ بلکه بیشتر مسئله داشتن آداب معاشرت خوب است. بنابراین، اینکه آیا باید سوپ را با قاشق بخورید یا خیر، سوالی نیست که اخلاق نگران آن باشد.

\vspace{0.3cm}
\noindent \textbf{شما نباید دروغ بگویید.}

مشابه مورد قبلی، این جمله نیز دستوری است و به شما می‌گوید عمل درست چیست (دروغ نگفتن). با این حال، برخلاف جمله «سوپ با قاشق»، این جمله بدیهتاً با برخی از جنبه‌های اصلی زندگی یک فرد تماس دارد. این جمله فوراً با ارزش‌های کلیدی مانند آزادی، رفاه و کرامت انسانی شما تعامل دارد. اینکه به این جمله پایبند باشید یا خیر، تأثیر قابل‌توجهی بر شانس زندگی خوب یک فرد دارد. بنابراین، این یک جمله هنجاری است که شامل یک «بایدِ اخلاقی» \lr{(moral ought)} می‌شود. نقض این گزاره هنجاری، مسئله رفتار بد (بی‌ادبی) نیست، بلکه رفتاری شدیداً غیراخلاقی یا نامشروع محسوب می‌شود.

% =================================================================
% ۲۰.۲.۱ نظریه‌های اخلاقی
% =================================================================
\subsubsection*{۲۰.۲.۱ \glspl{moral-theory}}
\addcontentsline{toc}{subsubsection}{۲۰.۲.۱ \glspl{moral-theory}}
\label{subsec:20-moral-theories}

ما ممکن است به‌طور شهودی بگوییم که دروغ گفتن غیراخلاقی است، اما آیا می‌توانیم توضیحی منسجم و عقلانی نیز ارائه دهیم که چرا اشتباه است؟ چندین \gls{moral-theory} توسعه یافته‌اند تا توضیح دهند چه چیزی یک عمل خاص را از نظر اخلاقی قابل‌قبول (یا غیرقابل‌قبول) می‌سازد. ما به‌طور بسیار مختصر به سه نظریه اخلاقی تأثیرگذار خواهیم پرداخت: \gls{consequentialism}، \gls{deontological-ethics} و \gls{virtue-ethics}. برای هر نظریه، ما مثالی اضافه کرده‌ایم که نشان می‌دهد چگونه این دیدگاه‌های خاص می‌توانند در درک مسائل \gls{data-science} نقش ایفا کنند.

% =================================================================
% ۲۰.۲.۲ پیامدگرایی
% =================================================================
\subsubsection*{۲۰.۲.۲ \gls{consequentialism}}
\addcontentsline{toc}{subsubsection}{۲۰.۲.۲ \gls{consequentialism}}
\label{subsec:20-consequentialism}

\gls{consequentialism} (پیامدگرایی) بر این باور است که برای قضاوت اخلاقی در مورد اعمال خاص، فرد باید صرفاً بر پیامدهای آن اعمال تمرکز کند. به عبارت دیگر، این نظریه «تجسم شهودِ پایه‌ای است که آنچه بهترین یا درست است، هر آن چیزی است که جهان را در آینده به بهترین شکل درآورد» \lr{(Sinnott-Armstrong, 2019)}. آنچه اهمیت دارد، تأثیر اعمال ماست.

در قرن هجدهم، جرمی بنتام \lr{(Jeremy Bentham)} ([1789] 1996) یک نظریه پیامدگرای مشهور به نام فایده‌گرایی \lr{(utilitarianism)} را توسعه داد. در فایده‌گرایی کلاسیک، یک عمل زمانی از نظر اخلاقی درست تلقی می‌شود که «خیر» را به حداکثر برساند. بنتام ادعا می‌کند که تنها ارزشی که فی‌نفسه خوب است، «لذت» است. بنابراین، این ارزش ذاتی باید راهنمای تمام اعمال ما باشد. بدین ترتیب، از دیدگاه فایده‌گرایی، وظیفه اخلاقی ما به حداکثر رساندن خیر، یا همان به حداکثر رساندن لذت است.

برای تعیین اینکه آیا یک عمل به‌طور مؤثر بیشترین لذت یا خوشبختی را برای بیشترین تعداد افراد به ارمغان می‌آورد یا خیر، بنتام پیشنهاد می‌کند که از یک «ترازنامه اخلاقی» \lr{(moral balance sheet)} استفاده شود؛ نوعی تحلیل هزینه-فایده که سعی دارد محاسبه کند هنگام انجام یک عمل، چقدر خوشبختی/ناراحتی می‌توان انتظار داشت. یک ترازنامه باید امکان مقایسه اعمال ممکنِ مختلف را فراهم کند و تعیین نماید کدام‌یک بیشترین خوشبختی را به ارمغان می‌آورد و بنابراین، از دیدگاه فایده‌گرایی، بهترین عمل است. بنابراین، اگر دروغ گفتن رفاه را به حداکثر برساند—یا به تعبیر بنتام، خوشبختی را—آن‌گاه از نظر اخلاقی کار درستی است.

\vspace{0.5cm}
\noindent
\fcolorbox{black}{gray!10}{%
	\begin{minipage}{\dimexpr\linewidth-2\fboxsep-2\fboxrule}
		\vspace{0.2cm}
		\textbf{$\blacktriangleright$ پیامدگرایی در اقدامات \gls{data-science}}
		\vspace{0.2cm}
		
		به منظور مهار پیامدهای ویرانگر همه‌گیری کووید-۱۹ \lr{(Covid-19)}، بسیاری از راه‌حل‌های داده‌محور پیشنهاد و توسعه یافته‌اند؛ مانند اپلیکیشن‌های گوشی هوشمند برای امکان ردیابی تماس، ارائه گواهی مصونیت، یا تنظیم دسترسی به ساختمان‌ها و خدمات خاص. همچنین، کاربردهای هوش مصنوعی مانند تشخیص چهره به عنوان راه‌حلی برای شناسایی افرادی که در مجاورت افراد آلوده قرار گرفته‌اند، مطرح می‌شوند. یک نگرانی تکرار شونده این است که این راه‌حل‌های داده‌محور حریم خصوصی شهروندان را نقض می‌کنند. برای مثال، مردم ممکن است کنترل داده‌های شخصی خود را از دست بدهند و بدون رضایت خود شناسایی و دسته‌بندی شوند.
		
		هوان تون-تات \lr{(Hoan Ton-That)}، هم‌بنیان‌گذار شرکت تشخیص چهره \lr{Clearview AI}، توضیح می‌دهد که این نوع راه‌حل‌ها «کمی از حریم خصوصی ما را می‌گیرند» \lr{(NBC News, 2020)} اما برای حل مشکل بزرگ سلامتی که بر ما تحمیل شده، مورد نیاز هستند. به عبارت دیگر، در اینجا خیر بر شر می‌چربد. بله، ما بخشی از حریم خصوصی خود را از دست می‌دهیم، اما در نهایت وضعیت بهتری خواهیم داشت زیرا در این شرایط سلامتی مهم‌تر است.
		
		این شیوه تفکر ذاتاً پیامدگراست. چه عملی رفاه را به حداکثر می‌رساند: «حفاظت از سلامت» یا «حفاظت از حریم خصوصی»؟ با قاب‌بندی مسئله به این شکل، تمرکز بر پیامدهای عمل قرار می‌گیرد و یک تحلیل هزینه-فایده را تحریک می‌کند. این دیدگاه فرض می‌کند که به طریقی می‌توان کمی‌سازی کرد که چقدر «خیر» از طریق حفاظت از سلامت و چقدر از طریق حفاظت از حریم خصوصی به دست می‌آوریم. اگر حفاظت از سلامت با معرفی این فناوری‌های غیردوستدار حریم خصوصی، رفاه کلی ما را به حداکثر برساند، آنگاه وظیفه اخلاقی ما انجام این کار است، حتی اگر در این فرآیند برخی افراد از نقض حریم خصوصی خود رنج ببرند.
		\vspace{0.1cm}
	\end{minipage}
}
\vspace{0.5cm}

% =================================================================
% ۲۰.۲.۳ اخلاق وظیفه‌گرا
% =================================================================
\subsubsection*{۲۰.۲.۳ \gls{deontological-ethics}}
\addcontentsline{toc}{subsubsection}{۲۰.۲.۳ \gls{deontological-ethics}}
\label{subsec:20-deontological}

در \gls{deontological-ethics}، یک عمل زمانی از نظر اخلاقی درست تلقی می‌شود که شما بر اساس وظیفه اخلاقی خود عمل کنید. یک روایت تأثیرگذار از اخلاق وظیفه‌گرا، اخلاق کانتی است که توسط فیلسوف آلمانی قرن هجدهم، ایمانوئل کانت \lr{(Immanuel Kant)} ([1785] 1997) توسعه یافت. تمرکز در اخلاق کانتی بر پیامد عمل نیست (آن‌طور که در مورد \gls{consequentialism} صدق می‌کند)، بلکه بر دلیلِ درگیر شدن در یک عمل خاص است.

طبق نظر کانت، یک عمل غیراخلاقی، عملی است که مغایر با عقل باشد. یک عمل درستِ اخلاقی، عملی است که با قانون اخلاقی یا آنچه او «امر مطلق» \lr{(categorical imperative)} می‌نامید، مطابقت داشته باشد. دو صورت‌بندی از امر مطلق عبارتند از:
\begin{enumerate}
	\item «تنها بر پایه آن قاعده‌ای \lr{(maxim)} عمل کن که در عین حال بخواهی که آن قاعده به یک قانون جهان‌شمول تبدیل شود.»
	\item «چنان عمل کن که انسانیت را، چه در شخص خود و چه در شخص دیگران، همواره و در عین حال به عنوان یک غایت (هدف) ببینی، نه صرفاً به عنوان یک وسیله (ابزار).»
\end{enumerate}
هنگام در نظر گرفتن یک مسیر عمل، فرد باید بررسی کند که آیا دلایل او برای عملش—یا آنچه کانت «قاعده» \lr{(maxim)} می‌نامد—با امر مطلق مطابقت دارد یا خیر.

اگر صورت‌بندی اول را در نظر بگیریم، روشن می‌شود که دروغ گفتن از نظر اخلاقی غیرقابل‌قبول است. ما نمی‌توانیم دنیایی را تصور کنیم که در آن همه قاعده دروغ گفتن را داشته باشند، زیرا دنیایی برای تصور کردن وجود نخواهد داشت که در آن همه دروغ بگویند و من همچنان قادر باشم بر اساس قاعده خود عمل کنم. اما همچنین، صورت‌بندی دوم نشان می‌دهد چرا از دیدگاه کانتی این کار از نظر اخلاقی اشتباه است. دروغ گفتن به افراد ذاتاً به معنای ناکامی در رفتار با آن‌ها به عنوان یک غایت است—یعنی ناکامی در احترام به ظرفیت عقلانی آن‌ها—و بنابراین ناکامی در احترام به انسانیت آن‌هاست.

\vspace{0.5cm}
\noindent
\fcolorbox{black}{gray!10}{%
	\begin{minipage}{\dimexpr\linewidth-2\fboxsep-2\fboxrule}
		\vspace{0.2cm}
		\textbf{$\blacktriangleright$ اخلاق وظیفه‌گرا در اقدامات \gls{data-science}}
		\vspace{0.2cm}
		
		در سال ۲۰۲۰، در یک حکم تاریخی، دادگاه منطقه‌ای در لاهه (هلند) به این قضاوت رسید که استفاده از \lr{SyRI}—یک سیستم پروفایل‌سازی ریسک کلاهبرداری رفاهی که توسط نهادهای دولتی ایجاد شده بود—غیرقانونی است. این سیستم منابع داده‌ای مختلف—از داده‌های درآمد و مالکیت خانه تا داده‌های مصرف آب و انرژی—را برای امتیازدهی به افراد تحلیل می‌کرد. بالاترین امتیاز منجر به برچسب «شایسته تحقیق» می‌شد \lr{(BVV, 2018)}. این تحقیق می‌توانست برای مثال به شکل بازدید از منزل برای بررسی مشروعیت مزایای دریافتی یا یک جلسه رسیدگی به عنوان مقدمه‌ای برای اعمال تحریم‌ها باشد.
		
		دادگاه در حکم خود، به‌ویژه بر عدم شفافیت سیستم مورد استفاده تأکید کرد. شهروندان نمی‌دانند که در حال امتیازدهی شدن هستند، برای عموم روشن نیست که دولت چگونه به نتیجه خاصی رسیده است، و شهروندان نمی‌توانند تا حد معقولی پیگیری کنند که چه اتفاقی با داده‌هایشان می‌افتد.
		
		اهمیتی که دادگاه برای شفافیت قائل است را می‌توان از دیدگاه کانتی توضیح داد. با ارائه نکردن توضیحی به شهروندان در مورد اینکه سیستم چگونه به یک تصمیم خاص می‌رسد، آن‌ها از فرصت تأمل در مورد آن محروم می‌شوند. شهروندان نمی‌توانند تصمیم بگیرند که آیا با نتیجه موافق هستند یا خیر. عملکرد مبهم سیستم همچنین امکان اعتراض به تصمیم را برای آن‌ها دشوار می‌کند، زیرا نمی‌توانند گام‌هایی را که منجر به نتیجه شده دنبال کنند. در مجموع، این سیستم عقلانیت انسان‌ها را ارج نمی‌نهد و با آن‌ها صرفاً به عنوان وسیله‌ای در عملکرد سیستم رفتار می‌کند.
		\vspace{0.1cm}
	\end{minipage}
}
\vspace{0.5cm}

% =================================================================
% ۲۰.۲.۴ اخلاق فضیلت
% =================================================================
\subsubsection*{۲۰.۲.۴ \gls{virtue-ethics}}
\addcontentsline{toc}{subsubsection}{۲۰.۲.۴ \gls{virtue-ethics}}
\label{subsec:20-virtue}

\gls{virtue-ethics} (اخلاق فضیلت)، برخلاف دو نوع قبلی نظریه‌های اخلاقی، بر یک قانون اخلاقی یا بر پیامدهای اعمال تمرکز نمی‌کند، بلکه بر ویژگی‌های شخصیتی یا فضایل یک فرد تمرکز دارد. طرفداران اخلاق فضیلت بر جزئیات یک موقعیت خاص تمرکز می‌کنند و اعمال فرد را با تقاضاهای زمینه خاصی که در آن عمل می‌کند، تنظیم می‌کنند. در این دیدگاه، اخلاق قبل از هر چیز درباره توسعه «خرد عملی» \lr{(practical wisdom)} است: توانایی تعیین آنچه از نظر اخلاقی مورد نیاز است، حتی اگر مربوط به یک موقعیت جدید یا غیرمعمول باشد که قوانین عمومی نمی‌توانند به راحتی اعمال شوند.

اخلاق فضیلت درباره «آدم خوبی بودن» است. این نظریه بر توسعه فضایل لازم برای داشتن یک زندگی خوب و شکوفا متمرکز است. زمانی که چیزی از نظر اخلاقی مهم در خطر است، فرد باید این سوال را بپرسد: «یک فرد بافضیلت در این موقعیت چه می‌کرد؟» و از الگوی او پیروی کند.

فیلسوف یونانی، ارسطو \lr{(Aristotle)} (قرن چهارم پیش از میلاد، ۱۹۸۴)، یکی از پدران بنیان‌گذار اخلاق فضیلت است. او نظریه «حد وسط» \lr{(golden mean)} را توسعه داد. ارسطو معتقد است فضایل مورد نیاز برای یک زندگی شکوفا، در میانه دو حد افراط و تفریط قرار دارند. برای مثال، شجاعت در میانه بزدلی (یک حد افراط) و بی‌پروایی (حد افراط دیگر) قرار دارد. فرار از همه خطرات بزدلانه است، در حالی که ریسک کردن بیش از حد، بی‌پروایی است.

برای تبدیل شدن به یک فرد بافضیلت، صرفاً داشتن این فضایل کافی نیست؛ فرد باید آن‌ها را به عمل درآورد. «الگوهای اخلاقی» \lr{(Moral exemplars)}—افرادی که قبلاً این فضایل را توسعه داده‌اند و بافضیلت عمل می‌کنند—در این زمینه مهم هستند زیرا راه پیش رو را به ما نشان می‌دهند. تمرین نیز جنبه مهمی از اخلاق فضیلت است. ما تلاش می‌کنیم، شکست می‌خوریم و از اشتباهات خود در مسیر بافضیلت‌تر شدن درس می‌گیریم. برای رسیدن به یک ارزیابی و عمل اخلاقی صحیح، شما باید ویژگی‌های شخصیتی لازم را توسعه دهید و از خود بپرسید: «آیا یک فرد بافضیلت دروغ می‌گوید؟»

\vspace{0.5cm}
\noindent
\hspace{0.8cm}
\fcolorbox{black}{gray!10}{%
	\begin{minipage}{\dimexpr\linewidth-1.2cm-2\fboxsep-2\fboxrule}
		\vspace{0.2cm}
		\textbf{$\blacktriangleright$ اخلاق فضیلت در اقدامات \gls{data-science}}
		\vspace{0.2cm}
		
		در سال ۲۰۱۸، کارمندان گوگل علیه مشارکت شرکتشان در «پروژه میوِن» \lr{(Project Maven)} اعتراض کردند؛ پروژه‌ای که بر بهبود تحلیل تصاویر ویدئویی ضبط شده توسط پهپادها با همکاری وزارت دفاع ایالات متحده متمرکز بود \lr{(Hicks, 2018)}. کارمندان نگران بودند که این فناوری برای مقاصد مرگبار استفاده شود، برای مثال با انتخاب اهداف انسانی برای حملات هوایی.
		
		کارمندان گوگل ابتدا نگرانی‌های خود را در داخل شرکت مطرح کردند. وقتی این کار به پاسخ رضایت‌بخشی منجر نشد، اعتراض آن‌ها صریح‌تر شد. حدود ۴۰۰۰ کارمند گوگل طوماری را امضا کردند و خواستار تغییر سیاست گوگل و تعهد به عدم مشارکت در ساخت فناوری‌های جنگی شدند. برخی از آن‌ها در اعتراض به فعالیت‌های گوگل استعفا دادند. اعتراضات کارمندان گوگل در نهایت منجر شد که گوگل این پروژه را پایان دهد.
		
		از دیدگاه \gls{virtue-ethics}، کارمندان معترض گوگل را می‌توان «الگوهای اخلاقی» در نظر گرفت. در این موقعیت بسیار حساس، آن‌ها توانستند تشخیص دهند چه چیزی در خطر است و مسئولیت خود را پذیرفتند. آن‌ها جزئیات موقعیت را در نظر گرفتند و اقدامات خود را بر اساس آن تنظیم کردند. با این کار، آن‌ها ویژگی‌های شخصیتی خاصی را به عمل درآوردند. آن‌ها با گفتن حقیقت به قدرت، «شجاعت»؛ با همکاری برای تنظیم نامه، «روحیه‌ی همکاری»؛ و با قربانیان احتمالی این فناوری، «شفقت» نشان دادند.
		\vspace{0.1cm}
	\end{minipage}
}
\vspace{0.5cm}

\noindent
صرف واقعیت بحث در مورد سه نظریه اخلاقی، نشان‌دهنده این است که همیشه یک اجماع روشن در مورد اینکه چه چیزی یک عمل را از نظر اخلاقی درست یا غلط، و خوب یا بد می‌سازد، وجود ندارد. با این حال، این نباید منجر به نوعی نسبی‌گرایی «هر چه پیش آید خوش آید» \lr{("anything goes" relativism)} شود. برعکس، این امر باید دانشمندان داده را ترغیب کند تا زمان و تلاش زیادی را صرف توضیح دلایل و اصولی کنند که تصمیماتشان را بر آن بنا می‌کنند. آن‌ها باید فعالانه به دنبال دیگران باشند تا تأملات اخلاقی خود را با آن‌ها به اشتراک بگذارند و ببینند آیا استدلال‌های مخالف ارزشمندی وجود دارد که آن‌ها در نظر نگرفته‌اند (همچنین نگاه کنید به \lr{Leonelli, 2016}).

متأسفانه نقد و بررسی انتقادی این نظریه‌های اخلاقی فراتر از دامنه این فصل است، اما امیدواریم شما هنگام مرور این بررسی کوتاه، خودتان به برخی سؤالات انتقادی فکر کرده باشید. برای مثال: چگونه می‌توانیم به‌طور عینی عملیاتی کنیم و بسنجیم که یک عمل چقدر خوشبختی به بار می‌آورد؟ آیا فکر می‌کنم قابل‌قبول است که برخی افراد رنج ببرند اگر اکثریت سود ببرند؟ آیا همیشه ممکن است در نظر بگیریم که آیا دلایل عمل من قابلیت جهان‌شمول شدن دارند؟ اگر باید به عنوان یک فرد بافضیلت عمل کنم، چگونه می‌توانم یک الگوی اخلاقی را در زندگی روزمره شناسایی کنم؟

همان‌طور که در ادامه این فصل خواهیم دید، اخلاق و تأمل اخلاقی همانند پیروی از یک دستورپخت \lr{(recipe)} نیستند. این کار تضمین نمی‌کند که شما در نهایت به یک غذای اخلاقی عالی برسید. این تلاشی بی‌پایان \lr{(open-ended endeavor)} است که نیازمند توجه مداوم است.

% =================================================================
% ۲۰.۲.۵ کانون توجه اخلاق داده آکادمیک
% =================================================================
\subsubsection*{۲۰.۲.۵ کانون توجه \gls{data-ethics} آکادمیک}
\addcontentsline{toc}{subsubsection}{۲۰.۲.۵ کانون توجه \gls{data-ethics} آکادمیک}
\label{subsec:20-focus}

بیایید اکنون به \gls{data-ethics}—یا گاهی اوقات اخلاق هوش مصنوعی (\lr{AI ethics})—به عنوان زیرمجموعه‌ای خاص از اخلاق بپردازیم. در حوزه آکادمیک، \gls{data-ethics} بدین صورت تعریف می‌شود:

\begin{quote}
	«شاخه‌ای جدید از اخلاق که مشکلات اخلاقی مربوط به **داده‌ها** (شامل تولید، ضبط، مدیریت، پردازش، انتشار، اشتراک‌گذاری و استفاده)، **الگوریتم‌ها** (شامل هوش مصنوعی، عامل‌های مصنوعی، یادگیری ماشین و ربات‌ها) و **رویه‌های متناظر** (شامل نوآوری مسئولانه، برنامه‌نویسی، هک کردن و کدهای حرفه‌ای) را مطالعه و ارزیابی می‌کند تا راه‌حل‌های اخلاقی خوب (مانند رفتارهای درست یا ارزش‌های درست) را فرموله و پشتیبانی نماید.» \lr{(Floridi \& Taddeo, 2016, p. 1)}
\end{quote}

در این حوزه نوظهور، دانشگاهیان با پیشینه‌های مختلف—از اخلاق‌شناسان تمام‌عیار گرفته تا مورخان، قوم‌نگاران، پژوهشگران حقوقی، دانشمندان علوم کامپیوتر و ریاضی‌دانان—بر چالش‌های خاصی که توسط اقدامات \gls{data-science} ایجاد شده، تمرکز می‌کنند. نظریه‌های اخلاقی که در بالا به‌طور مختصر بحث کردیم، سوخت برخی از این مطالعات اخلاق داده را تأمین می‌کنند. برای مثال، می‌توان پرسش‌های اخلاقی ناشی از معرفی خودروهای خودران را از طریق لنز یک نظریه اخلاقی یا ترکیبی از آن‌ها تحلیل کرد \lr{(Nyholm, 2018)}. با این حال، خوب است توجه داشته باشیم که گاهی اوقات این نظریه‌های اخلاقی تنها به‌طور ضمنی در پس‌زمینه کار می‌کنند یا به‌سادگی غایب هستند؛ برای مثال، زمانی که پژوهش ماهیت توصیفی‌تری دارد یا یک ارزش خاص—مانند انصاف یا عدالت—یا رویکرد مبتنی بر حقوق بشر را به عنوان نقطه شروع می‌گیرد.

یک مثال از این پیش‌فرضِ پس‌زمینه‌ای نظریه اخلاقی، پروژه «ماشین اخلاقی» \lr{(Moral Machine)} است (فصل ۲۱ را ببینید) که توسط محققان \lr{MIT} ایجاد و بعداً در نشریه \lr{Nature} منتشر شد \lr{(Awad et al., 2018)}. این پروژه شامل جمع‌سپاری \lr{(crowdsourcing)} آنلاین نظرات ۴۰ میلیون نفر درباره تصمیم‌گیری توسط ماشین‌ها در رابطه با سوالات اخلاقی مختلف بود که با مثال خودروی خودران آغاز می‌شد. این پروژه از یک آزمایش فکری مشهور به نام «مسئله تراموا» \lr{(trolley problem)} سرچشمه گرفت که در آن یک ناظر باید انتخاب کند که آیا یک نفر را نجات دهد یا فرد دیگری (یا گروهی) را از برخورد تراموای فراری. این مسئله به‌طور سنتی به عنوان مسئله «انجام دادن در مقابل اجازه دادن به آسیب» تحلیل شده است \lr{(Woollard \& Howard-Snyder, 2002)}، اما پروژه \lr{MIT} با افزودن موضوع خودروهای بدون راننده، بُعد متفاوتی به استفاده از این مسئله بخشید. اکنون، آزمایش به جای اینکه فرد را به عنوان «یک ناظر» قرار دهد و به تفکر انتزاعی وا دارد، از شرکت‌کننده می‌خواهد که آزمایش فکری را به دنیای واقعی ترجمه کند و آن را به‌طور خاص بر توسعه یک کالای مصرفی، یعنی نوع خاصی از خودرو، اعمال نماید.

این قاب‌بندی، مفروضات متعددی را اضافه می‌کند که در نسخه اصلی وجود نداشت: اینکه حضور خودروها در جاده‌ها هم اجتناب‌ناپذیر و هم ضروری است، اینکه مردم باید تعداد مشخصی از مرگ‌ومیرها را به عنوان نتیجه بپذیرند، و اینکه تنها عنصر انتخاب اخلاقیِ موجود، شکل دادن به نحوه وقوع آن مرگ‌ها از طریق تصمیم‌گیری میان خطای خودکار و خطای راننده است.

یک رویکرد جایگزین ممکن است دامنه پرسش اخلاقی را گسترش دهد تا بپرسد چه ارزش‌هایی باید در شکل‌دهی به سیاست کلی حمل‌ونقل مرکزی باشند. برای مثال، ما ممکن است بپرسیم که آیا سیاست حمل‌ونقل باید بر اشکال عمومی حمل‌ونقل تمرکز کند یا رانندگی با خودرو را به عنوان شیوه اصلی سفر تشویق نماید؛ آیا تمام مرگ‌ومیرهای مرتبط با صنعت خودرو (شامل آلودگی و سهم بخش انرژی در تغییرات اقلیمی)، و نه فقط تصادفات رانندگی، باید هنگام تصمیم‌گیری در مورد نحوه سفر مردم در نظر گرفته شوند؛ یا اینکه آیا یک چشم‌انداز فردی به جای جمعی، موجه‌ترین رویکرد برای این مسئله است. محققان با قرار دادن تنها یک سطح از مسئله به عنوان دوراهی اخلاقی، به‌طور ضمنی سایر مسیرهای ممکن برای تحقیق را مسدود می‌کنند.

در تعریف \gls{data-ethics} از فلوریدی و تادئو \lr{(Floridi \& Taddeo)}، تأکید از یک سو بر جنبه‌های فناورانه (پردازش داده و الگوریتم‌ها) و از سوی دیگر بر رویه‌هایی است که این جنبه‌های فناورانه در آن تعبیه شده‌اند. در مورد اول، اخلاق داده ممکن است به چالش‌های مربوط به استفاده از داده (مانند جمع‌آوری، تحلیل و انتشار داده که می‌تواند حریم خصوصی افراد را نقض کند) یا استفاده از مدل‌ها (مانند مدلی که ممکن است افراد را به‌اشتباه طبقه‌بندی کرده و باعث آسیب شود) بپردازد \lr{(Saltz \& Dewar, 2019, p. 206)}.

در مورد دوم، سوالات اخلاقی مربوط به رویه‌ها می‌تواند در هر دو سطح فردی و سازمانی مطرح شود. در سطح فردی، پرسش‌هایی مانند «یک کارمند حوزه فناوری یا یک دانشمند داده چه فضایلی را باید پرورش دهد؟» \lr{(Vallor, 2016)} یا «چگونه می‌توان یک دانشمند داده پاسخگو بود؟» ممکن است ایجاد شود. در سطح سازمانی، سوالات مربوط به چگونگی ساختار یک شرکت می‌تواند مرتبط باشد، برای مثال: «چه ارزش‌هایی در کد رفتاری یک شرکت فناوری حضور دارند؟» \lr{(Hagendorff, 2020)} یا «آیا مدل کسب‌وکاره یک شرکت داده‌محور با ارزش‌های کلیدی مانند کرامت انسانی و آزادی بیان شهروندان همسو است؟».

این تدقیق بیشتر در تعریف ارائه‌شده توسط فلوریدی و تادئو بدین معناست که روی‌هم‌رفته، \gls{data-ethics} بر چالش‌های اخلاقی تمرکز دارد که در سه سطح متفاوت اما مرتبط پدید می‌آیند: سطح فناورانه، فردی و سازمانی.

% =================================================================
% ۲۰.۳ اخلاق داده در حوزه تجاری
% =================================================================
\raggedbottom
\subsection*{۲۰.۳ \gls{data-ethics} در \gls{commercial-domain}}
\addcontentsline{toc}{subsection}{۲۰.۳ \gls{data-ethics} در \gls{commercial-domain}}
\label{sec:20-commercial}

\gls{data-ethics} تنها محدود به \gls{academia} (محیط آکادمیک) نیست؛ بلکه در \gls{commercial-domain} (حوزه تجاری) نیز شتاب گرفته است. تحت فشار تمامی حوادثی که آشکار می‌شوند—از نشت داده‌ها و اخبار جعلی گرفته تا دستکاری مردم در فضای آنلاین—و واکنش‌های عمومی محبوبی که این حوادث ایجاد کرده‌اند، شرکت‌های فناوری و همچنین افراد (شامل دانشمندان داده) که در صنعت فناوری کار می‌کنند، به دنبال راه‌هایی برای بهبود عملیات خود هستند، به امید آنکه با این حوادث مقابله کنند.

این یک باور دیرینه است که اگر اعتماد به یک شرکت یا سرویس کاهش یابد، مردم سعی می‌کنند از استفاده از آن اجتناب کنند. این امر هم برای کسب‌وکارها و هم برای کل جامعه که به‌طور فزاینده‌ای به زیرساخت‌های داده‌محور وابسته شده است، زیان‌بار خواهد بود. با این حال، اعتماد بدون «قابلیت اعتماد» \lr{(trustworthiness)}، پوسته‌ای توخالی است. متقاعد کردن مشتریان برای اعتماد به محصول یا سرویس شما با سرمایه‌گذاری روی رابط‌های کاربری پرزرق‌وبرق و محصولات با کاربری آسان، کار ساده‌ای است. اما اگر در پشت این رابط کاربری، طرف‌های تجاری سعی در دستکاری مردم داشته باشند و داده‌ها نشت کنند یا فروخته شوند، تنها مسئله زمان است که سوءرفتار بعدی علنی شود و اعتماد فروبپاشد \lr{(Keymolen, 2016, 2017)}.

اگر شرکت‌های داده‌محور و \gls{data-science} به عنوان یک حرفه بخواهند در بلندمدت باقی بمانند، نیاز دارند که (بیشتر) قابل‌اعتماد شوند. \gls{data-ethics} به عنوان یکی از ابزارهای مهم برای قابل‌اعتمادتر شدن و برای ایجاد، بازیابی و حفظ اعتماد مصرف‌کننده در نظر گرفته شده است \lr{(Hasselbalch \& Tranberg, 2016)}.

% =================================================================
% ۲۰.۳.۱ سطح فناورانه
% =================================================================
\subsubsection*{۲۰.۳.۱ \gls{tech-level}}
\addcontentsline{toc}{subsubsection}{۲۰.۳.۱ \gls{tech-level}}
\label{subsec:20-tech-level}

اخلاق داده شرکتی اشکال گوناگونی به خود می‌گیرد. مشابه حوزه آکادمیک، در حوزه تجاری نیز ابتکارات \gls{data-ethics} سطوح فناورانه، فردی و سازمانی را در بر می‌گیرند.

در سطح فناورانه، هم شرکت‌های چندملیتی و هم استارتاپ‌ها اصول طراحی اخلاقی را تدوین کرده‌اند تا اطمینان حاصل کنند که محصولات و خدماتشان بر پایه ارزش‌های کلیدی مانند شفافیت، پاسخگویی، انصاف و عدم تبعیض بنا شده است. در نگاه اول، این‌ها ارزش‌هایی هستند که همه ما به‌راحتی می‌توانیم بر سر آن‌ها توافق کنیم (به هر حال، چه کسی می‌تواند مخالف انصاف یا عدم تبعیض باشد؟). با این حال، زمانی که فرد تلاش می‌کند تعریفی از این ارزش‌ها ارائه دهد، چه رسد به اینکه بخواهد آن‌ها را در یک محصول یا خدمت پیاده‌سازی کند، وضعیت به‌سرعت پیچیده می‌شود \lr{(Mittelstadt, 2019)}.

برای مثال، بحث‌های زیادی در مورد نحوه عملیاتی کردن «انصاف» در هوش مصنوعی وجود دارد \lr{(Suresh \& Guttag, 2019)}. آیا انصاف باید در سطح گروهی (برابری گروهی) تعریف شود یا در سطح فردی (برابری فردی) \lr{(Chouldechova, 2017)}؟ آیا انصاف عمدتاً باید درباره نتایج منصفانه باشد یا رویه‌های منصفانه \lr{(Grgić-Hlača, 2018)}؟ آیا ممکن است هر دو را داشت؟ همسو با تنوعی که پیش‌تر در مرور کوتاه نظریه‌های اخلاقی با آن مواجه شدیم، در اینجا نیز، در اخلاق داده تجاریِ عمل‌گرا و واقع‌بینانه، می‌بینیم که پاسخ روشنی به این پرسش که معنا و محتوای این ارزش‌ها چه باید باشد، وجود ندارد \lr{(Jobin et al., 2019)}. زمانی که فلسفه سیاسیِ انصاف با رویکردهای علوم کامپیوتر به انصاف تماس پیدا می‌کند، به نظر می‌رسد یک تضاد حل‌نشدی رخ می‌دهد، زیرا «اغلب، یک رویکرد متناسب با زمینه به انصاف که واقعاً جوهره نکات فلسفی مربوطه را تسخیر کند، ممکن است منوط به عواملی باشد که معمولاً در داده‌های موجود یافت نمی‌شوند» \lr{(Binns, 2018, p. 9)}.

در مسئله تعریف ارزش‌ها و تعبیه آن‌ها در طراحی و استفاده فناورانه، اخلاق آکادمیک ممکن است به کمک بیاید، چرا که در اینجا رویکردهای مختلفی برای تعریف و ترجمه ارزش‌ها در فناوری‌ها به‌شیوه‌ای سیستماتیک توسعه یافته‌اند. این رویکردها می‌توانند برای شرکت‌هایی که با عملیاتی کردن اصول طراحی خود دست‌وپنج نرم می‌کنند، مفید باشند. یکی از این رویکردها، «طراحی حساس به ارزش» \lr{(Value Sensitive Design)} \lr{(Simon, 2017; Chen \& Zhu, 2019)} است که در فصل ۲۱ به آن پرداخته خواهد شد، اما روش‌های دیگری نیز توسعه یافته‌اند، مانند رویکرد «ارزش‌هایی که اهمیت دارند» \lr{(VtM)} \lr{(Smits et al., 2019)}.

آنچه اغلب در این رویکردها مورد تأکید قرار می‌گیرد این است که چون ارزش‌ها مفاهیم صریح و قطعی نیستند و برای ذینفعان مختلف معانی متفاوتی دارند، دریافت ورودی از این ذینفعان برای درک این تفاسیر متفاوت از اهمیت بالایی برخوردار است. در واقع، از همان ابتدا، هنگام تعریف مسئله‌ای که فعالیت‌های \gls{data-science} را هدایت می‌کند، تعامل با جوامعی که تحت تأثیر ابزار یا خدمتی که شما توسعه می‌دهید قرار می‌گیرند، حیاتی است. صورت‌بندی مسئله توسط آن‌ها باید پیشرو باشد؛ فناوری باید پیروی کند.

همان‌طور که در فصل ۲۱ به‌طور مفصل‌تر توضیح داده خواهد شد، فناوری یک ابزار خنثی نیست. فناوری می‌تواند روابط قدرت موجود را تحکیم کند یا درهم بشکند، می‌تواند عاملیت کاربران نهایی را تقویت کند یا اقدامات آن‌ها را خنثی سازد، و می‌تواند دسترسی به خدمات را فراهم کند یا افراد را طرد نماید. شما نمی‌توانید تمام این پیامدهای ممکن را به‌تنهایی پیش‌بینی کنید. حتی زمانی که ذینفعان را شامل می‌کنید، همیشه پیامدهایی وجود خواهد داشت که پیش‌بینی نکرده بودید. بنابراین، مشورت با خبرگانِ حوزه یا بخشی که در آن کار می‌کنید نیز مهم است. چنین مشورتی فراتر از یک مرور ادبیات صرف است. توصیه می‌شود با خبرگانی تعامل کنید که دانش و تجربه عملی در حوزه کاربردی شما دارند. همان‌طور که مشخص است، زندگی واقعی همیشه بسیار آشفته‌تر و پیچیده‌تر از آن چیزی است که داده‌ها می‌توانند به شما بگویند. این رویکرد تعاملی شانس بهتری به شما می‌دهد تا درک غنی‌تری از زمینه‌ای که در آن کار می‌کنید توسعه دهید و معنای ارزش‌هایی را که می‌خواهید محصول یا خدمتتان پشتیبانی کند، دریابید.

تلاش برای توسعه روش‌هایی جهت بنا نهادن توسعه فناوری بر پایه ارزش‌ها، تنها محدود به حوزه آکادمیک نیست. شرکت‌ها، سازمان‌های عمومی (یا در همکاری‌های عمومی-خصوصی) و همچنین بازیگران دولتی نیز در حال ارائه استراتژی‌هایی برای پیاده‌سازی پربار ارزش‌ها هستند. این‌ها شامل دستورالعمل‌ها و چک‌لیست‌ها (مثلاً \lr{RSS and IFoA, 2019})، پرسشنامه‌ها، وب‌سایت‌های تعاملی، مطالعات موردی، چارچوب‌ها (مثلاً \lr{PartnershipOnAI, 2019})، بوم‌های ارزش، ارزیابی‌های تأثیر و بسیاری موارد دیگر است. انگیزه‌های متفاوتی برای توسعه این اسناد اخلاق علم داده و هوش مصنوعی قابل تشخیص است: از انگیزه مسئولیت اجتماعی و استفاده از آن به عنوان ابزاری برای تغییر گرفته تا دیدن آن به عنوان یک مزیت رقابتی برای شناخته شدن به عنوان «اخلاقی» \lr{(Hasselbalch \& Tranberg, 2016; Schiff et al., 2020)}.

در حالی که از یک سو امیدوارکننده است که استراتژی‌های بسیار زیادی در حال توسعه هستند، کثرت فعلی استراتژی‌ها تصمیم‌گیری در مورد اینکه کدام‌یک بهترین تناسب را دارد دشوار می‌سازد. اگرچه عواملی وجود دارند که می‌توانند نشان‌دهنده موفقیت این اسناد اخلاق هوش مصنوعی باشند—مانند تعامل با قانون، اختصاصی بودن، گستره دسترسی، قابلیت اجرا، و تکرار و پیگیری \lr{(Schiff et al., 2020, p. 156–157)}—هنوز هیچ تحقیق تطبیقی گسترده‌ای انجام نشده است تا اثربخشی هر یک از این رویکردها را تعیین کند.

% =================================================================
% ۲۰.۳.۲ سطح فردی
% =================================================================
\subsubsection*{۲۰.۳.۲ \gls{individual-level}}
\addcontentsline{toc}{subsubsection}{۲۰.۳.۲ \gls{individual-level}}
\label{subsec:20-individual-level}

تحقیقات تجربی نشان می‌دهد که متخصصان هوش مصنوعی (مانند دانشمندان داده) خود را تا حدی از نظر اخلاقی مسئول تأثیرات اجتماعی برنامه‌های خود می‌دانند. با این حال، آن‌ها همچنین ابراز می‌کنند که عاملیت آن‌ها تا حد زیادی توسط نیروهای قدرتمند شرکتی و دولتی محدود شده است \lr{(Orr \& Davis, 2020)}. در واقع، ما شاهد علاقه فزاینده‌ای به استانداردهای اخلاقی و حرفه‌ای برای هدایت و راهبری اقدامات دانشمندان داده هستیم. کدهای رفتاری \lr{(Codes of conduct)}—اسنادی که در آن‌ها سازمان‌ها (مانند شرکت‌ها یا انجمن‌های حرفه‌ای) دستورالعمل‌هایی برای رفتار حرفه‌ای تعیین می‌کنند—در حال توسعه هستند تا متخصصان این حوزه را راهنمایی کنند، آگاهی اخلاقی را افزایش دهند و بحث‌های اخلاقی را در میان همکاران و در درون شرکت تحریک نمایند \lr{(Van de Poel \& Royakkers, 2011, p. 32–42)}.

انواع مختلفی از کدهای رفتاری وجود دارد. **کدهای آرمانی** \lr{(Aspirational codes)} خطاب به دنیای بیرون هستند و بیانگر آن چیزی هستند که یک شرکت پایبند آن است. برای مثال، مایکروسافت \lr{(Microsoft)} (2020) اصول هوش مصنوعی مسئولانه‌ای را توسعه داده است که فعالیت‌های تجاری آن‌ها را هدایت می‌کند. **کدهای مشورتی** \lr{(Advisory codes)} بر متخصصان متمرکز هستند و هدفشان کمک به آن‌ها در تصمیم‌گیری‌های اخلاقی در کارشان است. برای مثال، یک کد حرفه‌ای برای دانشمندان داده توسعه یافته است تا آن‌ها را در کارشان راهنمایی کند \lr{(Oxford-Munich, 2020)}. همچنین **کدهای انضباطی** \lr{(Disciplinary codes)} وجود دارند. این کدها قوانین پایه‌ای را تعیین می‌کنند تا اطمینان حاصل شود که اقدامات کارمندان با استانداردهای خاصی مطابقت دارد و عمدتاً بر عملکرد داخلی شرکت متمرکز هستند \lr{(Van de Poel \& Royakkers, 2011, p. 32–42)}.

مهم است توجه داشته باشیم که به‌طور کلی، کدهای رفتاری شکلی از «خودتنظیمی» \lr{(self-regulation)} هستند. آن‌ها عموماً توسط قانون الزامی یا اجرا نمی‌شوند. آن‌ها وضعیت قانونی روشنی ندارند. این بدان معناست که مشتریان نمی‌توانند زمانی که معتقدند یک شرکت یا کارمند شرکت کد رفتاری خود را نقض کرده است، به دادگاه مراجعه کنند. علاوه بر این، شرکت‌ها هیچ مسئولیت قانونی برای در دسترس قرار دادن کدهای رفتاری داخلی خود برای کاربران ندارند، که این امر تأکید می‌کند چنین کدهایی برای عمل کردن به عنوان مقررات رسمی طراحی نشده‌اند.

در کنار این کدهای رفتاری، تحول مهم دیگری نیز در سطح فردی اخلاق داده در حال رخ دادن است که می‌توان آن را «اخلاق داده از درون» \lr{(data ethics from within)} نامید. این به کارگران حوزه فناوری اشاره دارد که با اعتراض علیه شرکت‌های خود زمانی که معتقدند به شیوه‌ای غیراخلاقی عمل می‌کنند، حقیقت را به قدرت می‌گویند. پروژه میوِن \lr{(Maven project)} که در بالا در بخش ۲۰.۲.۴ به‌طور مختصر بحث شد، نمونه‌ای از چنین «اخلاق داده از درون» است. جنبش‌های کارگران منتقدِ فناوری نور را می‌بینند و خواستار دانستن این هستند که فناوری‌ای که روی آن کار می‌کنند در واقع قرار است چه هدفی را دنبال کند. آن‌ها طومار امضا می‌کنند، به مدیران اجرایی شرکت‌های خود اعتراض می‌کنند و گاهی حتی شغل خود را ترک می‌کنند تا بر شرکت‌ها و جامعه برای مداخله فشار بیاورند. به‌ویژه در بازار کاری که تقاضای بالایی برای دانشمندان داده، دانشمندان علوم کامپیوتر و سایر متخصصان فنی کلیدی وجود دارد، نفوذ و قدرت این کارمندانِ فردیِ حوزه فناوری نباید دست‌کم گرفته شود، قطعاً نه اگر آن‌ها راهی برای سازماندهی خود و بیان جمعی نگرانی‌ها و اعتراض به اقدامات مشکوک شرکتی پیدا کنند.

% =================================================================
% ۲۰.۳.۳ سطح سازمانی
% =================================================================
\subsubsection*{۲۰.۳.۳ \gls{org-level}}
\addcontentsline{toc}{subsubsection}{۲۰.۳.۳ \gls{org-level}}
\label{subsec:20-org-level}

با تحریک روند \gls{data-ethics} تجاری، شرکت‌ها همچنین در حال سرمایه‌گذاری در انواع نوآوری‌های سازمانی هستند. آن‌ها کمیته‌های اخلاق \lr{(ethics boards)} را برای بررسی شکایات از درون و بیرون شرکت راه‌اندازی می‌کنند. اخلاق‌شناسان استخدام می‌شوند تا تیم‌های طراحی را غنی کنند، و جوامع اخلاقی نصب می‌شوند تا فعالیت‌های شرکت‌ها را پایش کرده و در مورد مسائل مرتبط با اخلاق مشاوره دهند.

در یک وضعیت ایده‌آل، سطوح فناورانه، فردی و سازمانیِ اخلاق داده تجاری در هم قفل می‌شوند. \gls{data-ethics} یک «فرآیند مشارکتی» \lr{(collaborative process)} است و همواره «در حال تغییر» \lr{(in flux)} می‌باشد \lr{(Orr \& Davis, 2020, p. 13)}. کدهای رفتاری و اصول طراحی به گونه‌ای عملیاتی می‌شوند که بتوانند در واقعیت، به شیوه‌ای معنادار، اقدامات کارمندان را هدایت کنند. کارمندان منتقد به عنوان دارایی شرکت شناخته می‌شوند و ساختار شرکت به گونه‌ای است که مشارکت‌های آن‌ها را در فرآیند تصمیم‌گیری شامل شود. به‌طور ایده‌آل، شرکت‌ها طوری تنظیم می‌شوند که هم در برابر کارمندان خود و هم در برابر دنیای بیرون، و به‌طور خاص در برابر جوامعی که محصولات و خدماتشان بر آن‌ها تأثیر می‌گذارد، پاسخگو باشند.

با این حال، همان‌طور که قبلاً اشاره کردیم، تمام این ابتکارات \gls{data-ethics} داوطلبانه هستند. هر شرکتی می‌تواند لیست اصول طراحی اخلاقی خود را در وب‌سایتش منتشر کند، دانشمندان داده بافضیلت را با بیانیه‌های اخلاقی پرشور فریب دهد تا به شرکت بپیوندند، و خود را به دنیای بیرون به عنوان شرکتی اخلاقی و پایدار که منافع کاربران نهایی را در قلب خود دارد معرفی کند، در حالی که در واقعیت ممکن است هیچ اهمیتی به این موضوع ندهند.

بنابراین حیاتی است که تمام این ابتکارات با مکانیزم‌های اجرایی سازمانی در قالب پاسخگویی ساختاری در درون و بیرون شرکت همراه باشند، مانند الزامات گزارش‌دهی و حسابرسیِ آن گزارش‌ها؛ در غیر این صورت، آن‌ها صرفاً در حد شعار و ظاهرسازی \lr{(paying lip service)} برای \gls{data-ethics} باقی می‌مانند. تحقیقات نشان می‌دهد که کار بیشتری باید در این زمینه انجام شود \lr{(Hagendorff, 2020)}. برای مثال، از بیش از ۱۶۰ دستورالعمل اخلاق هوش مصنوعی که جمع‌آوری شد، تنها ۱۰ مورد دارای مکانیزم‌های اجرایی مناسب بودند. علاوه بر این، این کدهای رفتاری و اصول طراحی اخلاقی نسبتاً مبهم و انتزاعی باقی مانده‌اند که پیاده‌سازی واقعی آن‌ها را دشوار می‌سازد. این مسئله این پرسش را مطرح می‌کند: دستورالعمل‌هایی که «نه می‌توانند اعمال شوند و نه اجرا»، آیا «مضرتر از نداشتن هیچ دستورالعمل اخلاقی نیستند»؟

% =================================================================
% ۲۰.۴ قانون و اخلاق داده
% =================================================================
\raggedbottom
\subsection*{۲۰.۴ \gls{law} و \gls{data-ethics}}
\addcontentsline{toc}{subsection}{۲۰.۴ \gls{law} و \gls{data-ethics}}
\label{sec:20-law}

در حوزه تجاری، زمانی که \gls{data-ethics} برای اجتناب از مقررات قانونی استفاده شود، تبدیل به «اخلاق‌شویی» \lr{(ethics washing)} می‌شود \lr{(Wagner, 2018)}. اخلاق‌شویی فرآیندی است که در آن یک شرکت رفتاری اخلاقی از خود نشان می‌دهد تا انتقادات نسبت به رویه‌های زیان‌بار را منحرف کند و بدین ترتیب شهرت خود را «تطهیر» نماید، بدون آنکه مدل کسب‌وکار خود را تغییر دهد. یک مثال از این مورد، شرکت تحلیل داده و نظارت آمریکایی «پالانتیر» \lr{(Palantir)} است که کنفرانس‌های حقوق حریم خصوصی را حمایت مالی می‌کرد، در حالی که همزمان سیستم‌های نظارتی مورد استفاده برای جداسازی خانواده‌های مهاجر در ایالات متحده را توسعه می‌داد \lr{(Guardian, 2019)}.

در این حالت، \gls{data-ethics} به عنوان یک استراتژی خودتنظیمی، نه برای بهبود پاسخگویی، بلکه برای اجتناب از اقدامات تنظیمی سخت‌گیرانه‌ترِ بالا-به-پایین به کار گرفته می‌شود. یک استراتژی اصیل \gls{data-ethics} تنها زمانی می‌تواند توسعه یابد که جایگاه خود را در رابطه با \gls{law} بشناسد. اخلاق، به گفته هیلدبرانت \lr{(Hildebrandt, 2020, p. 297)}:

\begin{quote}
	«هم بیش از قانون است و هم کمتر از آن: بیش از آن است زیرا بسیاری از دغدغه‌های اخلاقی توسط قانون پوشش داده نمی‌شوند؛ و کمتر از آن است زیرا خروجی ملاحظات اخلاقی لزوماً به هنجارهای قانونی تبدیل نمی‌شوند و بنابراین از طریق قانون قابل اجرا نیستند.»
\end{quote}

بنابراین، زمانی که این را بر \gls{data-ethics} تجاری اعمال می‌کنیم، درمی‌یابیم که \gls{data-ethics} می‌تواند بیش از قانون باشد، زیرا می‌تواند استانداردهایی را تعیین کند که توسط قانون الزامی نشده‌اند و در موقعیت‌هایی که مستقیماً توسط قانون پوشش داده نمی‌شوند، راهنمایی ارائه دهد. برای مثال، دانشمندان داده می‌توانند انتخاب کنند که با داده‌های ناشناس به عنوان داده‌های شخصی رفتار کنند و به مقررات عمومی حفاظت از داده‌ها (\lr{GDPR}، که در فصل‌های ۱۷ و ۲۱ نیز بحث شده) پایبند باشند، زیرا آن‌ها تقویت حریم خصوصی را مهم می‌دانند، نه به این دلیل که قانوناً ملزم به آن هستند. در آن صورت، آن‌ها فراتر از انتظارات قانون از خود عمل می‌کنند و وارد قلمرو \gls{data-ethics} می‌شوند.

قانون یک «فضای عمل» \lr{(action space)} فراهم می‌کند که شرکت‌ها می‌توانند در آن رویه‌های اخلاق داده را توسعه دهند؛ با این حال، این رویه‌های اخلاق داده نمی‌توانند و نباید جایگزین قانون شوند، زیرا مهم‌تر از همه، قانون نوعی «فصل‌الخطاب» یا قطعیت \lr{(closure)} فراهم می‌کند که اخلاق نمی‌تواند \lr{(Hildebrandt, 2020)}. در یک جامعه دموکراتیکِ با عملکرد صحیح، شفاف بودن و قابل پیش‌بینی بودن اینکه چه نوع رفتارهایی قابل‌قبول هستند و کدام‌یک نیستند، از اهمیت بالایی برخوردار است \lr{(Tamanaha, 2004, 2007)}.

برای مثال، این ایده که «شما نباید به کسی دروغ بگویید» برای یک جامعه دموکراتیکِ شکوفا آن‌قدر اهمیت دارد که از قلمرو اخلاقی به یک هنجار قانونی تبدیل شده است. مثلاً هنگام شهادت در دادگاه، شما باید قول بدهید که حقیقت را بگویید و ارتکاب کلاهبرداری (که خود نوعی دروغگویی است) خلاف قانون و بنابراین قابل مجازات است.

آنچه یک هنجار قانونی را از یک هنجار اخلاقی متمایز می‌کند، این است که قانون هم «قابل پیش‌بینی» و هم «قابل اجرا» \lr{(enforceable)} است. به عبارت دیگر، شما از پیش می‌دانید چه عملی مناسب تلقی می‌شود و اگر به قانون پایبند نباشید چه انتظاری باید داشته باشید. این نوعی از وضوح و قدرت است که اخلاق نمی‌تواند فراهم کند. ما پیش‌تر در مقدمه اخلاق آکادمیک دیدیم که نظریه‌های اخلاقی متفاوتی وجود دارند که منطق‌های متفاوتی برای آنچه «عمل خوب» محسوب می‌شود، ارائه می‌دهند. این امر اخلاق را در ماهیت خود باز-پایان \lr{(open-ended)} می‌سازد و امکان پرسشگری و تأمل مداوم در تصمیمات و اعمالمان را فراهم می‌کند.

در ایده‌آل‌ترین وضعیت، رویه‌های \gls{data-ethics} اعمال ما را در درونِ فضای عملِ فراهم‌شده توسط قانون آگاه می‌سازند و ما را تشویق می‌کنند تا از «ذهنیت تیک زدن» \lr{(checkbox mentality)} فراتر رویم تا رویه‌های علم داده‌ای را توسعه دهیم که در آن مسئولیت و پاسخگویی نهادینه شده باشد. در این شرایط ایده‌آل، \gls{data-ethics} دانشمندان داده را قادر می‌سازد تا شایستگی‌های اخلاقی خود را توسعه و پرورش دهند، و شرکت‌ها را تسهیل می‌کند تا رویکردهای اخلاق داده‌ی متناسبی را توسعه دهند که با پروفایل و سازمان شرکتشان همخوانی داشته باشد.

با این حال، در چارچوب زمانی فعلی ما، به نظر می‌رسد فناوری‌های به‌سرعت در حال تحول، توانایی قانون را برای فراهم کردن و اجرای این فضای عملِ ضروری به چالش می‌کشند. برای مثال، پیشرفت‌ها در هوش مصنوعی ممکن است منجر به سوالات پیچیده‌ای در مورد توضیح‌پذیری، انصاف و پاسخگویی شود که ما را ملزم به بازاندیشی و تفسیر مجدد هنجارهای قانونی موجود می‌کند. این فرآیندهای دموکراتیک، با این حال، زمان‌بر هستند.

در نگاه اول، ممکن است جذاب به نظر برسد که اجازه دهیم \gls{data-ethics}، که بسیار چابک‌تر و منعطف‌تر از قانون است، به سرعت آن خلأ را پر کند. در این صورت، \gls{data-ethics} دیگر راهی برای توسعه رویه‌های علم داده در درون این فضای عمل نیست، بلکه تبدیل به فراهم‌کننده اصلی آن فضای عمل می‌شود.

اما از آنجا که \gls{data-ethics} فاقد قدرت اجراست و انعطاف‌پذیری آن با پیش‌بینی‌پذیری‌ای که ما از یک فضای عمل دموکراتیک انتظار داریم همخوانی ندارد—مردم تصمیم می‌گیرند چه مسیر اخلاقی را از طریق فرآیند تأمل دنبال کنند، نه با پیروی از قوانین وضع‌شده در قانون، به‌طوری که ممکن است بسته به زمینه، پاسخ «درست» متفاوتی برای یک سوال داده‌شده وجود داشته باشد—اخلاق هرگز نمی‌تواند قطعیتِ لازم را فراهم کند. در نتیجه، یک فضای عمل که بر پایه \gls{data-ethics} تجاری بنا شده باشد، منجر به یک «چهل‌تکه‌ی پراکنده» \lr{(scattered patchwork)} از فضاهای عمل متفاوت و سیال خواهد شد که در آن شرکت‌ها می‌توانند قوانین بازی خود را تعیین کنند و سایر بازیگران (هم شهروندان و هم دولت‌ها) فقط باید با آن همراهی کنند. این منجر به عدم تعادل قدرتی می‌شود که در واقع به جای محافظت از ارزش‌های کلیدی مانند برابری، انصاف و عدالت، آن‌ها را تضعیف خواهد کرد.

این سوءاستفاده شدید و حتی سوءرفتار با \gls{data-ethics}، حوزه سیاست‌گذاری را به این پرسش واداشته است که آیا \gls{data-ethics} به عنوان یک درمان، در واقع بدتر از خودِ بیماری نیست؟ این امر حتی منجر به آنچه بیِتی \lr{(Bietti, 2020)} به عنوان «اخلاق‌کوبی» \lr{(ethics bashing)} اشاره می‌کند، شده است: حمله به کل حوزه اخلاق به دلیل این موارد سوءاستفاده.

اخلاق می‌تواند فرآیندهای دموکراتیکی را که منجر به یک فضای عمل به‌روزرسانی‌شده برای رویه‌های علم داده می‌شود، آگاه سازد، اگر و تنها اگر اخلاق در معنای وسیع کلمه درک شود و نه در معنای محدودِ صرفاً \gls{data-ethics} تجاری (همچنین نگاه کنید به \lr{Taylor \& Dencik, 2020}). برای مثال، این چشم‌انداز وسیع مستلزم آن است که قطعاً نقشی برای اخلاق‌شناسان—و همچنین سایر دانشگاهیان، مانند فیلسوفان، دانشمندان علوم اجتماعی و محققان مطالعات علم و فناوری \lr{(STS)}—وجود دارد تا دانش خود را با مخاطبان وسیع‌تری به اشتراک بگذارند و بحث‌های عمومی و سیاسی را آگاه سازند. اخلاق می‌تواند برای مثال «به عنوان یک چشم‌انداز فرا-سطحی عمل کند تا هرگونه اختلاف نظر مربوط به حکمرانی فناوری را در نظر بگیرد» و «لایه‌ای از تفکر اصولی دقیق را به بحث‌های مملو از ارزش اضافه کند» \lr{(Bietti, 2020, p. 5)}.

«اخلاق از درون» نیز باید بخشی از این بحث عمومی باشد، زیرا ارزش زیادی در تجربه اول‌شخص و دانشِ کارمندان حوزه فناوری—مانند دانشمندان داده—که در واقع در حال ساخت فناوری‌های داده‌محور هستند، نهفته است. و نکته آخر اما نه کم‌اهمیت، جوامعی که ممکن است تحت تأثیر این رویه‌های علم داده قرار گیرند باید شنیده شوند و مورد مشورت قرار گیرند تا به فضای عملی برسیم که نه تنها منافع شرکت‌های داده‌محور را حفظ کند، بلکه در درجه اول و پیش از همه، منافع شهروندان را در قلب خود داشته باشد \lr{(Taylor \& Dencik, 2020)}. ناگفته پیداست که سازماندهی چنین فرآیندهای دموکراتیکی سخت و دشوار است. فراهم کردن یک زمین بازی هموار که در آن تمام این صداها واقعاً شنیده شوند، یک چالش اجتماعی و سیاسی است که نیازمند سرعتی آهسته‌تر از آن چیزی است که برخی از تکنولوژی-خوش‌بین‌ها (از جمله شرکت‌های داده‌محور) احتمالاً به آن امیدوارند.

در مجموع، از آنجا که \gls{data-science} (هم به عنوان یک حوزه پژوهشی و هم فعالیت تجاری) هم‌اکنون تأثیر بنیادینی بر سازماندهی جامعه دارد و خواهد داشت، اکنون زمان آن است که این قدرت را نقادانه ارزیابی کنیم و اطمینان حاصل کنیم که کنترل و توازن‌های کافی \lr{(checks and balances)} در جای خود قرار دارند تا تضمین شود که این علم واقعاً به یک جامعه شکوفا کمک خواهد کرد، نه اینکه آن را از میان ببرد. \gls{data-ethics} قطعاً می‌تواند به توسعه دانشمندان داده و کسب‌وکارهای داده‌محور مسئول‌تر و پاسخگوتر کمک کند، اما باید همواره در درون فضای عملِ فراهم‌شده توسط \gls{law} قرار گیرد.

\vfill

% =================================================================
% ۲۰.۵ اخلاق داده و علم داده: آیا این همراهی پایدار است؟
% =================================================================
\subsection*{۲۰.۵ \gls{data-ethics} و \gls{data-science}: آیا این همراهی پایدار است؟}
\addcontentsline{toc}{subsection}{۲۰.۵ \gls{data-ethics} و \gls{data-science}: آیا این همراهی پایدار است؟}
\label{sec:20-long-run}

این فصل «اخلاق داده و علم داده: یک ازدواج ناآرام؟» نام‌گذاری شده است. تا کنون، شما باید قادر باشید درک کنید که این «ازدواج ناآرام» به چه چیزی اشاره دارد. از یک سو، آشکارا روشن است که \gls{data-science} و کسب‌وکارهای داده‌محور تأثیر اخلاقی قابل‌توجهی بر جامعه ما دارند. آن‌ها جنبه‌های کلیدی زندگی روزمره را واسطه‌گری می‌کنند: از آموزش تا روابط اجتماعی، از تعامل ما با دولت تا شیوه‌ای که اخبارمان را دریافت می‌کنیم. در نتیجه، منطقی به نظر می‌رسد که ما \gls{data-science} و کسب‌وکارهایی را که پیش می‌برد، به گونه‌ای طراحی و سازماندهی کنیم که منعکس‌کننده ارزش‌های اخلاقی‌ای باشد که برایمان اهمیت دارد. \gls{data-ethics} می‌تواند در انجام این کار کمک کند. در نگاه اول، \gls{data-science} و \gls{data-ethics} یک زوج کامل هستند.

با این حال، در عین حال، ما همچنین تثبیت کردیم که \gls{data-ethics} یک چیز همگن (یکدست) نیست. این یک حوزه پژوهشی در اخلاق آکادمیک و یک استراتژی تجاری است، و گاهی این دو با هم می‌آیند. می‌تواند بر افراد، فناوری و سازمانِ یک کسب‌وکار متمرکز باشد. می‌تواند برای تفسیر فضای عملی که قانون فراهم می‌کند استفاده شود، و می‌تواند برای دور زدن مقررات قانونی مورد سوءاستفاده قرار گیرد.

آنچه \gls{data-ethics} را—به‌ویژه در حوزه تجاری—بسیار جذاب می‌کند، انعطاف‌پذیری آن است! شانسِ انجامِ «کار درست!» معلوم می‌شود که بزرگترین نقطه ضعف آن نیز هست. همان‌طور که مشخص است، \gls{data-ethics} زمانی که برای فراهم کردن قطعیت (فصل‌الخطاب) به کار گرفته می‌شود، به شدت دچار بار اضافی می‌گردد، زیرا ماهیت باز-پایان آن در واقع از چنین کاربردی پشتیبانی نمی‌کند.

شاید \gls{data-science}، که سر از پا نمی‌شناخت، بدون درک کامل از اینکه \gls{data-ethics} واقعاً درباره چیست، با شتاب وارد این رابطه شد. به هر حال، \gls{data-ethics} و همچنین \gls{data-science} هر دو هنوز نسبتاً جوان هستند. با این حال، در تاریک‌ترین سناریو، \gls{data-ethics} در دستان کسب‌وکارهای داده‌محور تبدیل به وسیله‌ای برای رفتار عمدی و مخرب می‌شود، زمانی که برای مانع‌تراشی و دور انداختن مقررات قانونی استفاده گردد.

روی‌هم‌رفته، کاملاً روشن می‌شود که \gls{data-ethics} و \gls{data-science} هنوز به جایگاهی آرام نرسیده‌اند. بنابراین، برای سال‌های پیش رو چه انتظاری می‌توانیم داشته باشیم؟ به عنوان راهی برای نتیجه‌گیری این فصل، ما به‌طور مختصر به سه تحولی که انتظار داریم این رابطه به خود بگیرد، خواهیم پرداخت.

\vfill

% =================================================================
% نتیجه‌گیری (داخل کادر خاکستری)
% =================================================================
\vspace{0.5cm}
\noindent
\fcolorbox{black}{gray!10}{%
	\begin{minipage}{\dimexpr\linewidth-2\fboxsep-2\fboxrule}
		\vspace{0.2cm}
		\begin{center}
			\textbf{\Large نتیجه‌گیری: سه تحول مورد انتظار}
		\end{center}
		\vspace{0.3cm}
		
		\textbf{۱. مقررات جدید} \\
		برخی مسائل مهم‌تر از آن هستند که به صلاحدید شرکت‌ها واگذار شوند. تا زمانی که \gls{data-science} بر داده‌هایی استوار است که عمدتاً از جامعه تأمین می‌شوند و خروجی‌های آن بر جامعه تأثیر می‌گذارد—که هر دو از ویژگی‌های ذاتی این رشته و حرفه هستند—\gls{data-science} باید به‌طور مؤثری تنظیم (رگولاتوری) شود تا اطمینان حاصل گردد که برای جامعه سودمند است. این شامل اثرات اجتماعی، سیاسی و اقتصادی آن می‌شود: بحث درباره اینکه این سودمندی شامل چه چیزهایی است، به همان اندازه که وظیفه شرکت‌هاست، وظیفه جامعه نیز می‌باشد.
		
		مقررات، این ارتباط هنجاری بین کسب‌وکار و جامعه را شکل می‌دهد، اما اثربخشی مقررات موجود در علم داده و فناوری داده به‌طور کلی، در حال حاضر به دلیل عدم انسجام بین شیوه‌های عملی و بدنه‌های قانونی محدود شده است. یک رویکرد یکپارچه به مقررات، قوانین مصرف‌کننده و خصوصی و همچنین حفاظت از داده‌ها را در نظر می‌گیرد و به حقوق بشر و مقررات قابل اجرای بین‌المللی برای علم داده‌ای که در مرزها و کشورهای مختلف انجام می‌شود، متصل می‌گردد. پاسخگویی باید بعدی بین‌المللی پیدا کند، همان‌طور که شکست قانون، مقررات و سیاست در مقابله با تخلفات شرکت‌هایی مانند «کمبریج آنالیتیکا» \lr{(Cambridge Analytica)} نشان‌دهنده این نیاز است.
		
		\vspace{0.3cm}
		\hrule
		\vspace{0.3cm}
		
		\textbf{۲. دانشمندان داده مسئول و پاسخگو} \\
		می‌توان انتظار داشت که \gls{data-science} به عنوان یک حرفه بلوغ یابد و با گذشت زمان، الزامات حرفه‌ای برای آن ایجاد شود. این الزامات نه تنها به استانداردهای فنی کار، بلکه به تأثیر اجتماعی آن نیز اشاره خواهند داشت. این بدین معناست که دانشمندان داده باید راه‌هایی برای تعامل با عاملیت اجتماعی و همچنین سیاسی خود بیابند. این امر فراتر از «نیت‌های خوب» می‌رود و شامل «ارزیابی‌های دقیق» و نیاز به تصریح «تعهدات سیاسی» خواهد بود \lr{(Green, 2018, p. 45)}.
		
		این موضوع همچنین منجر به تمرکز بیشتر بر مسئولیت شخصی و همچنین پاسخگویی خواهد شد. در چنین سیستمی، زمانی که راه‌حل‌های داده‌محور نتوانند به استانداردهای خاصی پایبند باشند یا ثابت شود که استانداردهای حرفه‌ای خاصی در فرآیند طراحی رعایت نشده‌اند، متخصصان در سطح شخصی—و نه صرفاً در سطح شرکتی—پاسخگو خواهند بود. علاوه بر این، پیوند دادن پاسخگویی به مسئولیت ضروری است. رویکرد فعلی علوم کامپیوتر به «داده‌های مسئولانه» ماهیتی عمدتاً فنی دارد و بر الزامات محدود و رسمی تمرکز می‌کند تا تقاضاهای پیچیده‌ترِ پاسخگویی دموکراتیک و قانونی. به این ترتیب، این رویکرد در حال حاضر برای اطمینان از سودرسانی یا جبران خسارت در جایی که حقوق یا اصول نقض می‌شوند، ناکافی است.
		
		\vspace{0.3cm}
		\hrule
		\vspace{0.3cm}
		
		\textbf{۳. تقاضای اجتماعی: کسب‌وکارهای داده‌محور ۲.۰} \\
		انتظار می‌رود که در سال‌های آینده، آگاهی اجتماعی از نفوذ و مشارکت‌های \gls{data-science} بیشتر رشد کند. این می‌تواند منجر به افزایش شهروندان منتقدی شود که از نهادهای دولتی خود مقررات سخت‌گیرانه‌تر و اجرای قوی‌تر را مطالبه کنند. علاوه بر این، شهروندان در نقش خود به عنوان مصرف‌کننده، می‌توانند شرکت‌های داده‌محور را تحت فشار قرار دهند تا در رویه‌های تجاری مسئولانه سرمایه‌گذاری کنند. بیان نگرانی‌ها و مطالبه خدمات جدید ممکن است در واقع فشار بزرگی برای اصلاح کسب‌وکارها باشد.
		
		با این حال، اگر مشتریان و کارمندان اعتماد خود را از دست بدهند، ممکن است تصمیم بگیرند که «با پاهایشان رأی دهند» \lr{(Hirschmann, 1970)} (یعنی ترک کردن سرویس) و شرکت و خدماتی را که ارائه می‌دهد رها کنند و به دنبال جایگزین‌های منصفانه‌تر و دوستدار حریم خصوصی باشند. به خودی خود، این تقاضای رو به رشد برای رویه‌های \gls{data-science} اخلاقی‌تر احتمالاً برای اصلاح و/یا خاتمه دادن به مدل‌های کسب‌وکاره مشکوکی مانند رویکرد «پرداخت با داده‌های شما» یا رویه‌های غیرشفاف دلالان داده کافی نخواهد بود.
		
		با این حال، این امر می‌تواند توسعه مدل‌های کسب‌وکاره علم داده پایدارتر را تحریک کند و از مدل استارتاپی «سیلیکون‌ولی» که صرفاً بر «رشد سریع» متمرکز است، دور شود. شرکت‌های جدیدی که از شرِ ذینتِ «سریع حرکت کن و چیزها را بشکن» خلاص شوند، می‌توانند در واقع از همان ابتدا کسب‌وکاره خود را ساختاردهی کنند و محصولات و خدماتشان را با یک ذینتِ \gls{data-ethics} اصیل و روشن توسعه دهند. این کمک خواهد کرد تا اطمینان حاصل شود که \gls{data-ethics} به یک شریک برابر در این رابطه تبدیل می‌شود و صرفاً به عنوان یک وصله ناجورِ پرزرق‌وبرق پایان نمی‌یابد.
		\vspace{0.2cm}
	\end{minipage}
}

\vspace{2\baselineskip}

% =================================================================
% نکات بحث و گفتگو
% =================================================================
\noindent
\fcolorbox{black}{gray!10}{%
	\begin{minipage}{\dimexpr\linewidth-2\fboxsep-2\fboxrule}
		\vspace{0.2cm}
		\begin{center}
			\textbf{\Large نکات بحث و گفتگو}
		\end{center}
		\vspace{0.2cm}
		
		\begin{enumerate}
			\item \textbf{مسائل اخلاقی اصلی:} \\
			به نظر شما مسائل اخلاقی اصلی ناشی از \gls{data-science} چیست؟ در نظر بگیرید که چگونه علم داده قدرت و نفوذ خود را در رابطه با جامعه اعمال می‌کند، تأثیرات آن چگونه توزیع می‌شود، و جامعه مدنی چه نوع نفوذی می‌تواند بر کار دانشمندان داده اعمال کند.
			
			\item \textbf{فضایل دانشمند داده:} \\
			فکر می‌کنید مهم‌ترین فضایلی که یک دانشمند داده باید در خود پرورش دهد چیست؟ نظریه‌های اخلاقی مطرح شده در این فصل را مرور کنید و بررسی نمایید که چگونه انتخاب‌های دانشمندان داده می‌تواند منجر به سودرسانی (خیررسانی) یا پتانسیل آسیب شود.
			
			\item \textbf{اخلاق به عنوان مزیت رقابتی:} \\
			آیا \gls{data-ethics} می‌تواند یک مزیت رقابتی برای یک شرکت باشد؟ معایب یا مزایای بالقوه برای شرکت‌هایی را ارزیابی کنید که کار علم داده‌ای انجام می‌دهند که به عنوان محصول جانبی، به جامعه آسیب می‌رساند، و راه‌هایی که درگیر نشدن در چنین کارهایی ممکن است بر شرکت‌ها تأثیر بگذارد.
		\end{enumerate}
		\vspace{0.2cm}
	\end{minipage}
}

\vfill
\clearpage

% =================================================================
% پیام‌های کلیدی
% =================================================================
\vspace{0.5cm}

\section*{پیام‌های کلیدی}

\begin{itemize}
	\item \textbf{اخلاق} شاخه‌ای از فلسفه است که حول این پرسش می‌چرخد: «چگونه باید عمل کرد؟» به روشی سیستماتیک، این رشته دلایل و استانداردهای زیربنایی اعمال ما را مطالعه می‌کند و بررسی می‌کند که چه چیزی اعمال ما را از نظر اخلاقی درست یا غلط، و خوب یا بد می‌سازد.
	
	\item \textbf{\gls{data-ethics}} شاخه‌ای از اخلاق آکادمیک و همچنین یک استراتژی تجاری است که شرکت‌های داده‌محور توسعه می‌دهند تا با تأثیرات اجتماعی محصولات و خدمات خود مقابله کنند.
	
	\item آنچه یک \textbf{هنجار قانونی} را از یک هنجار اخلاقی متمایز می‌کند، این است که قانون «قابل پیش‌بینی» و «قابل اجرا» است. به عبارت دیگر، شما از پیش می‌دانید چه عملی مناسب تلقی می‌شود و اگر به قانون پایبند نباشید چه انتظاری باید داشته باشید. این نوعی از وضوح و قدرت است که اخلاق نمی‌تواند فراهم کند.
	
	\item زمانی که \gls{data-ethics} برای دور زدن توسعه یا اجرای مقررات استفاده شود، به آن \textbf{«اخلاق‌شویی»} \lr{(ethics washing)} گفته می‌شود.
\end{itemize}

\vfill
\clearpage

% =================================================================
% منابع
% =================================================================
\clearpage
\section*{منابع}
\addcontentsline{toc}{section}{منابع}
\label{sec:20-references}

% تنظیمات شکستن لینک‌ها
\def\UrlBreaks{\do\/\do-\do\.\do\_\do\=\do\&\do\?\do\a\do\b\do\c\do\d\do\e\do\f\do\g\do\h\do\i\do\j\do\k\do\l\do\m\do\n\do\o\do\p\do\q\do\r\do\s\do\t\do\u\do\v\do\w\do\x\do\y\do\z\do\0\do\1\do\2\do\3\do\4\do\5\do\6\do\7\do\8\do\9}

\begin{latin}
	\setlength{\parindent}{0pt}
	\setlength{\parskip}{0.3cm}
	\urlstyle{same} 
	
	Aristotle. (1984). \textit{The complete works of Aristotle: Revised Oxford edition}. Edited by Jonathan Barnes. Princeton University Press.
	
	Awad, E., Dsouza, S., Kim, R., Schulz, J., Henrich, J., Shariff, A., Bonnefon, J.-F., \& Rahwan, I. (2018). The Moral Machine experiment. \textit{Nature}, 563(7729):59–64.
	
	Bentham, J. ([1789] 1996). \textit{The collected works of Jeremy Bentham: An introduction to the principles of morals and legislation}. Clarendon Press.
	
	Bietti, E. (2020). From ethics washing to ethics bashing: a view on tech ethics from within moral philosophy. In: \textit{Proceedings of the 2020 Conference on Fairness, Accountability, and Transparency}.
	
	Binns, R. (2018). Fairness in machine learning: Lessons from political philosophy. In: \textit{Proceedings of the 2018 Conference on Fairness, Accountability and Transparency} (pp. 149–159).
	
	BVV. (2018). Wat is Syri. Retrieved from \url{https://bijvoorbaatverdacht.nl/wat-is-syri/}.
	
	Chen, B., \& Zhu, H. (2019). Towards value-sensitive learning analytics design. In: \textit{Proceedings of the 9th International Conference on Learning Analytics \& Knowledge}.
	
	Chouldechova, A. (2017). Fair prediction with disparate impact: A study of bias in recidivism prediction instruments. \textit{Big Data}, 5(2), 153–163.
	
	Floridi, L., \& Taddeo, M. (2016). What is data ethics? \textit{Philosophical Transactions of the Royal Society A: Mathematical, Physical and Engineering Sciences}, 374(2083), 20160360. \url{https://doi.org/10.1098/rsta.2016.0360}
	
	Green, Ben. (2018). Data science as political action: grounding data science in a politics of justice. arXiv preprint arXiv:1811.03435.
	
	Grgić-Hlača, N. (2018). Muhammad Bilal Zafar, Krishna P Gummadi, and Adrian.
	
	Guardian. (2019). Palantir has no place at Berkeley: They help tear immigrant families apart. \textit{The Guardian}. Retrieved from \url{http://www.theguardian.com/commentisfree/2019/may/31/palantirberkeley-immigrant-families-apart}.
	
	Guardian. (2020). Hundreds of Amazon warehouse workers to call in sick in coronavirus protest. \textit{The Guardian}. Retrieved from \url{http://www.theguardian.com/technology/2020/apr/20/amazonwarehouse-workers-sickout-coronavirus}.
	
	Hagendorff, T. (2020). The ethics of AI ethics: An evaluation of guidelines. \textit{Mind. Machines}: 1-22.
	
	Hasselbalch, G., \& Tranberg, P. (2016). \textit{Data ethics: The new competitive advantage}. Publishare.
	
	Hicks, Mar. (2018, November 9). The long history behind the Google Walkout. \textit{The Verge}. Retrieved from \url{https://www.theverge.com/2018/11/9/18078664/google-walkout-history-tech-strikes-labororganizing}.
	
	Hildebrandt, M. (2020). \textit{Law for computer scientists and other folk}. Oxford University Press.
	
	Hirschmann, A. O. (1970). \textit{Exit, voice, and loyalty: Responses to decline in firms, organizations, and states}. Harvard University Press.
	
	Jobin, A., Ienca, M., \& Vayena, E. (2019). The global landscape of AI ethics guidelines. \textit{Nature Machine Intelligence}, 1(9), 389–399.
	
	Kant, I. ([1785] 1997). \textit{Groundwork of the metaphysics of morals}. Translated by Mary Gregor. Cambridge: Cambridge University Press.
	
	Keymolen, E. (2016). \textit{Trust on the line. A philosophical exploration of trust in the networked era}. Wolf Legal Publisher.
	
	Keymolen, E. (2017). Trust in the Networked Era: When Phones Become Hotel Keys. \textit{Techné: Research in Philosophy and Technology}, 22(1), 51–75.
	
	Leonelli, S. (2016). Locating ethics in data science: responsibility and accountability in global and distributed knowledge production systems. \textit{Philosophical Transactions of the Royal Society A: Mathematical, Physical and Engineering Sciences}, 374(2083), 20160122.
	
	Microsoft. (2020). Responsible AI principles. Retrieved from \url{https://www.microsoft.com/en-us/ai/responsible-ai?activetab=pivot1:primaryr6}
	
	Mittelstadt, B. (2019). AI Ethics—Too Principled to Fail? arXiv preprint arXiv:1906.06668.
	
	NBC News. (2020). Controversial tech company pitches facial recognition to track COVID-19. Retrieved from \url{https://www.nbcnews.com/now/video/controversial-tech-company-pitches-facialrecognition-to-track-covid-19-82638917537}.
	
	New Statesman. (2018). Cambridge Analytica and the digital war in Africa. Retrieved from \url{https://www.newstatesman.com/world/2018/03/cambridge-analytica-facebook-elections-africa-kenya}.
	
	New York Times. (2019). SpaceX Unveils Silvery Vision to Mars: ‘It’s Basically an I.C.B.M. That Lands.’ \textit{The New York Times}. Retrieved from \url{https://www.nytimes.com/2019/09/29/science/elonmusk-spacex-starship.html}.
	
	Nyholm, S. (2018). The ethics of crashes with self-driving cars: A roadmap, I. \textit{Philosophy Compass}, 13(7), e12507.
	
	Orr, W., \& Davis, J. L. (2020). Attributions of ethical responsibility by Artificial Intelligence practitioners. \textit{Information, Communication \& Society}: 1–17.
	
	Observer. (2020, January 4). Fresh Cambridge Analytica leak ‘shows global manipulation is out of control.’ \textit{The Observer}. Retrieved from \url{https://www.theguardian.com/uk-news/2020/jan/04/cambridge-analytica-data-leak-global-election-manipulation}.
	
	Oxford-Munich. (2020). Code of conduct. Retrieved from \url{http://www.code-of-ethics.org/code-ofconduct/}.
	
	PartnershipOnAI. (2019). Human-AI collaboration framework and case studies. Retrieved from \url{https://www.partnershiponai.org/wp-content/uploads/2019/09/CPAIS-Framework-and-CaseStudies-9-23.pdf}.
	
	RSS, and IFoA. (2019). A Guide for Ethical Data Science. Retrieved from \url{https://www.actuaries.org.uk/system/files/field/document/An\%20Ethical\%20Charter\%20for\%20Date\%20Science\%20WEB\%20FINAL.PDF}.
	
	Saltz, J. S., \& Dewar, N. (2019). Data science ethical considerations: a systematic literature review and proposed project framework. \textit{Ethics and Information Technology}, 21(3), 197–208.
	
	Schiff, D., Biddle, J., Borenstein, J., \& Laas, K. (2020). What’s next for AI ethics, policy, and governance? A global overview. In: \textit{Proceedings of the AAAI/ACM Conference on AI, Ethics, and Society}.
	
	Simon, J. (2017). Value-sensitive design and responsible research and innovation. In S. O. Hansson (Ed.), \textit{The ethics of technology: Methods and approaches} (pp. 219–236). Rowman \& Littlefield.
	
	Sinnott-Armstrong, W. (2019). Consequentialism. In E. N. Zalta (Ed.), \textit{The Stanford Encyclopedia of Philosophy} (Summer 2019). Stanford University. Retrieved from \url{https://plato.stanford.edu/archives/sum2019/entries/consequentialism/}.
	
	Smits, M., Bredie, B., van Goor, H., \& Verbeek, P.-P. (2019). Values that matter: Mediation theory and Design for Values. In: \textit{Academy for Design Innovation Management Conference 2019: Research perspectives in the era of Transformations}.
	
	Suresh, H., \& Guttag, J. V. (2019). A framework for understanding unintended consequences of machine learning. arXiv preprint arXiv:1901.10002.
	
	Tamanaha, B. Z. (2007). \textit{A concise guide to the rule of law}. FLORENCE WORKSHOP ON THE RULE OF LAW.
	
	Tamanaha, B. Z. (2004). \textit{On the rule of law: History, politics}. Cambridge University Press.
	
	Taylor, L., \& Dencik, L. (2020). Constructing commercial data ethics. \textit{Technology and Regulation}: 1–10.
	
	Timmons, M. (2012). \textit{Moral theory: An introduction}. Rowman \& Littlefield Publishers.
	
	Vallor, S. (2016). \textit{Technology and the Virtues. A philosophical guide to a future worth wanting}. Oxford University Press.
	
	Van de Poel, I. B. O., \& Royakkers, L. (2011). \textit{Ethics, technology, and engineering: An introduction}. Wiley.
	
	Wagner, B. (2018). Ethics as an escape from regulation: From ethics-washing to ethics-shopping. Being profiling. \textit{Cogitas ergo sum}: 84–90.
	
	Wired. (2018). When It Comes to Gorillas, Google Photos Remains Blind. \textit{Wired}. Retrieved from \url{https://www.wired.com/story/when-it-comes-to-gorillas-google-photos-remains-blind/}.
	
	Wired. (2020). Scraping the Web Is a Powerful Tool. Clearview AI Abused It. \textit{Wired}. Retrieved from \url{https://www.wired.com/story/clearview-ai-scraping-web/}.
	
	Woollard, F., \& Howard-Snyder, F. (2002). Doing vs. Allowing Harm. In: Zalta, E.N., \& Nodelman, U. (eds). \textit{The Stanford Encyclopedia of Philosophy} (Winter 2022 Edition).
	
\end{latin}



% =================================================================
% فایل chapter4_21.tex (اصلاح نهایی و رفع خطای Label تکراری)
% =================================================================

\clearpage

% -----------------------------------------------------------------
% تنظیمات شماره‌گذاری
% -----------------------------------------------------------------
\setcounter{section}{21} 
\setcounter{subsection}{0} 
\renewcommand{\thesubsection}{21.\arabic{subsection}} 

% -----------------------------------------------------------------
% تیتر اصلی
% -----------------------------------------------------------------
\vspace*{0.5cm} 
\begin{flushright} 
	{\huge \textbf{۲۱. \hspace{0.2cm} طراحی نرم‌افزار حساس به ارزش}}
	\addcontentsline{toc}{section}{۲۱. طراحی نرم‌افزار حساس به ارزش}
\end{flushright}

% -----------------------------------------------------------------
% نام نویسنده
% -----------------------------------------------------------------
\vspace{1.5cm} 

\begin{center}
	\Large \textbf{پائولان کورنهوف} \\
	\vspace{0.3cm} 
	\large \lr{(Paulan Korenhof)}
\end{center}

\vspace{1.5cm} 

% -----------------------------------------------------------------
% فهرست مطالب بخش (با ارجاعات اصلاح شده)
% -----------------------------------------------------------------
\noindent
\textbf{\large محتویات بخش}
\vspace{0.5cm}

\noindent
\textbf{۲۱.۱ مقدمه} \dotfill \pageref{sec:21-intro-main} \par
\vspace{0.3cm}

\noindent
\textbf{۲۱.۲ خوب، بد، و «هرگز خنثی»} \dotfill \pageref{sec:21-good-bad-main} \par
\vspace{0.1cm}
\noindent \hspace{0.8cm} ۲۱.۲.۱ عدم خنثی بودن \dotfill \par
\noindent \hspace{0.8cm} ۲۱.۲.۲ تأثیر در سطح خرد \dotfill \par
\noindent \hspace{0.8cm} ۲۱.۲.۳ تأثیر در سطح کلان \dotfill \par
\noindent \hspace{0.8cm} ۲۱.۲.۴ در مجموع \dotfill \par
\vspace{0.3cm}

\noindent
\textbf{۲۱.۳ به‌کارگیریِ «هرگز خنثی»} \dotfill \pageref{sec:21-employing-main} \par
\vspace{0.1cm}
\noindent \hspace{0.8cm} ۲۱.۳.۱ چالشی برای طراحان \dotfill \par
\noindent \hspace{0.8cm} ۲۱.۳.۲ طراحی حساس به ارزش \lr{(VSD)} \dotfill \par
\noindent \hspace{0.8cm} ۲۱.۳.۳ ارزش‌ها \dotfill \par
\noindent \hspace{0.8cm} ۲۱.۳.۴ ارزش‌های قانونی و طراحی \dotfill \par
\vspace{0.3cm}

\noindent
\textbf{منابع} \dotfill \pageref{sec:21-references-main} \par

\vspace{1cm}

% -----------------------------------------------------------------
% کادر اهداف یادگیری
% -----------------------------------------------------------------
\noindent
\fcolorbox{black}{gray!10}{%
	\begin{minipage}{\dimexpr\linewidth-2\fboxsep-2\fboxrule}
		\vspace{0.2cm}
		\begin{center}
			\textbf{\large اهداف یادگیری}
		\end{center}
		\vspace{0.2cm}
		\begin{itemize}
			\item درک اینکه چرا فناوری یک ابزار خنثی نیست.
			\item توانایی تشخیص تأثیر غیرخنثیِ یک فناوری خاص.
			\item تفکر درباره چگونگی تعبیه ارزش‌های اخلاقی در طراحی نرم‌افزار.
		\end{itemize}
		\vspace{0.2cm}
	\end{minipage}
}
\vspace{1cm}

% =================================================================
% شروع متن اصلی
% =================================================================
\clearpage

\subsection{مقدمه} 
\label{sec:21-intro-main}

نرم‌افزار تقریباً در همه جنبه‌های زندگی روزمره ما دخیل است: در دسکتاپ‌ها، لپ‌تاپ‌ها و گوشی‌های هوشمند ما وجود دارد، اما ما همچنین شاهد پیاده‌سازی آن در طیف فزاینده‌ای از اقلام پرمصرف مانند خودروها، دوچرخه‌ها، مسواک‌های برقی، اجاق‌گازها، دستگاه‌های تناسب اندام و اسباب‌بازی‌ها هستیم. ما از نرم‌افزار برای پرداخت، ارتباط، خرید آنلاین، برنامه‌ریزی مسیر، برنامه‌ریزی حمل‌ونقل عمومی، تماشای سریال و فیلم، تصمیم‌گیری در مورد اینکه کدام بیمه را بگیریم، به کدام حزب سیاسی رأی دهیم، سفارش غذا، چک کردن سلامتی‌مان و غیره استفاده می‌کنیم.

و این فقط ما به عنوان شهروندان خصوصی نیستیم که با کمال میل از نرم‌افزار برای بسیاری از امور در فعالیت‌های روزانه خود استفاده می‌کنیم. نهادهای دولتی و کسب‌وکارها نیز برای انجام بسیاری از فرآیندهای خود به‌شدت به نرم‌افزار متکی هستند. آن‌ها گاهی اوقات حتی فرآیندهای تصمیم‌گیری خاصی را کاملاً خودکار می‌کنند، مانند تصمیم‌گیری در مورد اینکه آیا کسی باید برای سرعت غیرمجاز جریمه شود، آیا به کسی باید وام یا کارت اعتباری داده شود، یا اینکه آیا کسی یک متقاضی شغلی امیدوارکننده است یا خیر.

در حالی که استفاده از نرم‌افزار به عنوان ابزاری برای کمک به ما در انواع وظایف مزایای زیادی دارد، اما یک نکته مهم (یا یک «گیر») وجود دارد. آن نکته در مورد برنامه‌های نرم‌افزاری این است که مانند تمام فناوری‌ها، آن‌ها **ذاتاً خنثی نیستند**: هر فناوری دارای یک سوگیری خاص است، شیوه‌ای خاص که در آن احتمالاً بر اعمال ما، انتخاب‌های ما، ادراک ما و نحوه تفسیر ما از جهان اطرافمان تأثیر می‌گذارد. در همین حال، تأثیر نرم‌افزار بر زندگی افراد می‌تواند بسیار زیاد باشد، به‌ویژه که به‌طور فزاینده‌ای عناصر بیشتری از زندگی ما به این برنامه‌ها وابسته و با آن‌ها درهم‌تنیده می‌شوند.

هدف این فصل جلب توجه خوانندگان به این ماهیت غیرخنثیِ فناوری و تشویق آن‌ها به تلاش برای بهره‌برداری از این عدم خنثی بودن به شیوه‌ای سودمند است. این فصل با بحث در مورد اینکه چرا فناوری هرگز یک ابزار خنثی نیست، آغاز می‌شود. با کمک مثال‌های مختلف مربوط به نرم‌افزار، تأثیر فناوری بر عناصر مختلف زندگی انسان مورد بحث قرار می‌گیرد. با توجه به ماهیت ذاتاً غیرخنثیِ فناوری و تأثیرات مشکل‌ساز بالقوه آن، مهم است که بفهمیم چگونه می‌توانیم از ثمرات فناوری بهره‌مند شویم و در عین حال آسیب‌های احتمالی آن را کاهش دهیم.

بنابراین، نیمه دوم فصل استدلال می‌کند که به‌طور ایده‌آل، ما باید از همان ابتدای طراحی فناوری، فعالانه با این عدم خنثی بودن برخورد کنیم. برای کمک به طراحان در این امر، این فصل ایده‌های اصلی زیربنای «طراحی حساس به ارزش» \lr{(VSD)} را معرفی می‌کند. از آنجا که ارائه یک دفترچه راهنمای کامل برای طراحی نرم‌افزار حساس به ارزش در اینجا ممکن نیست، هدف این فصل ارائه «خوراک فکری» کافی به خوانندگان است تا بتوانند خودشان در این سفرِ پیگیری گام بردارند.

% =================================================================
% بخش ۲۱.۲
% =================================================================

\raggedbottom 

\subsection*{۲۱.۲ خوب، بد، و «هرگز خنثی»}
\addcontentsline{toc}{subsection}{۲۱.۲ خوب، بد، و «هرگز خنثی»}
\label{sec:21-good-bad-main}

در این بخش، ما عمیقاً به موضوع عدم خنثی بودن فناوری خواهیم پرداخت. ابتدا، پیشینه دیدگاه عدم خنثی بودن مورد بحث قرار خواهد گرفت. پس از آن، این عدم خنثی بودن با جزئیات بیشتر و با رویکرد به فناوری از دو دیدگاه سطح خرد \lr{(micro-level)} و سطح کلان \lr{(macro-level)} توضیح داده خواهد شد.

\subsubsection*{۲۱.۲.۱ عدم خنثی بودن}
\addcontentsline{toc}{subsubsection}{۲۱.۲.۱ عدم خنثی بودن}
\label{sec:21-non-neutrality}

اهمیت فناوری برای زندگی انسان را به سختی می‌توان نادیده گرفت: جامعه و زندگی آن‌گونه که امروز می‌شناسیم، بدون توسعه و استفاده از فناوری وجود نداشت. فناوری به ما اجازه می‌دهد تا به اهداف خاصی دست یابیم، کارهایی را انجام دهیم و چیزهایی را درک کنیم که بدون استفاده از فناوری قادر به انجام آن‌ها نبودیم، و جهان را به روش‌های جدیدی برای ما آشکار می‌کند.

برای مثال، ما می‌توانیم تک‌سلول‌های بدن را از طریق میکروسکوپ ببینیم، با افرادی در آن سوی کره زمین از طریق تلفن مشورت کنیم، یا با دستگاه اکو به درون بدن نگاه کنیم. با فراهم کردن چنین تجربیات، اقدامات و ادراکات جدیدی از جهان، فناوری ما را قادر می‌سازد تا به روش‌های جدیدی با جهان ارتباط برقرار کنیم و بر تفسیر ما از جهان اطرافمان، و همچنین بر اعمال و قراردادهای اجتماعی ما تأثیر می‌گذارد \lr{(Kiran \& Verbeek, 2010; Verbeek, 2011)}.

به دلیل تأثیر شکل‌دهنده فناوری بر ادراک، تجربه، اعمال، اهداف و درک ما، فناوری از نقشِ صرفاً ابزار بودن فراتر می‌رود. در قرن گذشته، فیلسوفان تکنولوژی استدلال کردند که فناوری **ذاتاً خنثی نیست**: فناوری‌ها می‌توانند جهان را به روش‌های جدیدی برای ما آشکار کنند؛ انتخاب‌ها و امکانات جدیدی برای عمل ایجاد کنند؛ هویت‌های اجتماعی، روابط قدرت، و موقعیت‌های شمول و طرد را ایجاد نمایند؛ و بر ما، انتخاب‌های ما، فرهنگ ما و جهان‌بینی ما تأثیر بگذارند \lr{(see, e.g., Heidegger, 1954; Ihde, 1983; Latour, 1993; Feenberg, 2002; Verbeek, 2005)}. به دلیل این عدم خنثی بودن، فناوری دارای یک تأثیر هنجاری بر رابطه بین انسان‌ها و جهان آن‌هاست \lr{(Hildebrandt, 2015)}.

تحلیل تأثیر و معنای فناوری برای وجود انسان، منجر به پیدایش مکاتب فکری مختلفی در فلسفه تکنولوژی شد. به جای اینکه خواننده را درگیر بحث بین ایده‌های گوناگون کنیم، برای هدف این فصل ارزشمندتر خواهد بود که دو جهت‌گیری اصلیِ این دیدگاه‌ها را به صورت ساده‌سازی شده در نظر بگیریم و آن‌ها را مکمل یکدیگر بدانیم.

به زبان ساده، فناوری بر نحوه تعامل انسان‌ها با جهان تأثیر می‌گذارد که از یک **سطح خرد** (فردی و تجربی) \lr{(see, e.g., Ihde, 1983; Verbeek, 2005)}، تا یک **سطح کلان** (اجتماعی و انتزاعی) \lr{(see, e.g., Stiegler, 1998; Feenberg, 2002)} امتداد دارد.

در حالی که در نظر گرفتن تأثیر فناوری در سطح کلان برای درک دامنه و عمق تأثیر آن ضروری است، تحلیل متمرکز بر سطح خرد می‌تواند برای ردیابی مشکلات به ویژگی‌های ملموس فناوری بسیار مفید باشد. با این حال، من استدلال می‌کنم که سطح خرد و کلانِ تأثیر را نمی‌توان کاملاً از هم جدا کرد، زیرا مکانیسم‌های خرد به تأثیر کلان شکل می‌دهند و بالعکس. با وجود این، برای شفافیت ساختاری، با تمرکز بر سطح خرد شروع می‌کنم و سپس به سطح کلان می‌روم—اما خوانندگان باید توجه داشته باشند که این دو به هم پیوسته هستند.

\subsubsection*{۲۱.۲.۲ تأثیر در سطح خرد}
\addcontentsline{toc}{subsubsection}{۲۱.۲.۲ تأثیر در سطح خرد}
\label{sec:21-micro-level}

در سطح خرد، فناوری با اجازه دادن به ما برای تجربه جهان با واسطه‌گری فناوری، بر ادراک، اعمال، رویه‌ها و اهداف انسانی تأثیر می‌گذارد \lr{(see, e.g., Ihde, 1983; Verbeek, 2005)}.

برای مثال، یک دماسنج می‌تواند دمای بدن ما را نشان دهد. اگر دماسنج عدد $38.5^\circ C$ را نشان دهد، ما احتمالاً نتیجه می‌گیریم که تب داریم، حتی اگر احساس بیماری نکنیم. با گفتن اینکه ما در واقع بیمار هستیم (در حالی که ممکن است احساس خوبی داشته باشیم)، فناوری بر نحوه درک ما از سلامتی‌مان تأثیر می‌گذارد. با انجام این کار، فناوری رابطه ما با جهان (در این مورد، بدن انسان) را «هم‌شکل» \lr{(co-shapes)} می‌دهد.

فناوری بر ادراک ما، اعمال ما و حتی نحوه تفکر و حافظه ما تأثیر می‌گذارد. مورد آخر به وضوح توسط تحقیقات در مورد تأثیرات موتورهای جستجو بر حافظه نشان داده شد: وقتی مردم می‌دانند که می‌توانند برای اطلاعات به موتور جستجو تکیه کنند، تمایل دارند به جای محتوا، «مکان» و «چگونگی» یافتن آن را به خاطر بسپارند \lr{(Sparrow et al., 2011, p. 778)}.

هنگامی که فناوری رابطه ما با جهان را میانجی‌گری می‌کند، عموماً تمرکز خاصی دارد: اغلب جنبه‌های خاصی از واقعیت را آشکار و برجسته می‌کند، در حالی که عناصر دیگر پنهان یا نادیده گرفته می‌شوند \lr{(Verbeek, 2005, p. 131)}. به تماس تلفنی فکر کنید: صدای تماس‌گیرنده برجسته می‌شود، در حالی که بقیه وجودِ فرد پنهان می‌ماند. فناوری بدین ترتیب رابطه خاصی بین انسان و جهان برقرار می‌کند؛ رابطه‌ای که جهت‌دار است. بنابراین می‌توانیم بگوییم که فناوری دارای نوعی «جهت‌مندی» \lr{(directionality)} است \lr{(Verbeek, 2005, p. 115)}.

این جهت‌مندی در طراحی مادی فناوری تعبیه شده است و یک «موضع» \lr{(stance)} خاص می‌گیرد: می‌تواند «پیشنهاد دهد، فعال کند، درخواست کند، ترغیب کند، تشویق کند و برخی اعمال را ممنوع یا ترویج کند» \lr{(Lazzarato \& Jordan, 2014, p. 30)}. یک طراح عموماً با القای ویژگی‌های خاص، قصد دارد جهت‌مندی خاصی به فناوری بدهد. مثلاً فروشگاه‌های آنلاین معمولاً ثبت سفارش را بدون پذیرش «شرایط و ضوابط» غیرممکن می‌سازند. با این حال، نفوذ طراح محدود است و فناوری وجود مستقلی دارد \lr{(Chabot, 2013, p. 15)} و ممکن است اثرات ناخواسته‌ای داشته باشد. اما چون طراحان ویژگی‌های مادی را تعیین می‌کنند، نقش محوری در شکل‌دهی به جهت‌مندی غیرخنثی دارند.

جهت‌مندی نرم‌افزار به ویژگی‌های فناوری و انتخاب‌های طراح بستگی دارد. دانش و محدودیت‌های توسعه‌دهنده پس‌زمینه طراحی را تشکیل می‌دهند \lr{(Kitchin, 2017, p. 18)}. این در یک فرآیند دو مرحله‌ای شکل می‌گیرد: توسعه‌دهنده باید (۱) وظیفه را تفسیر کند و (۲) آن را به کد ترجمه نماید. در این فرآیند، فرضیات و سوگیری‌های طراح در نرم‌افزار گنجانده می‌شود \lr{(Friedman \& Nissenbaum, 1996; Goldman, 2008)}. بنابراین، رویه‌های کدنویسی شده ناگزیر «آکنده از ارزش» هستند \lr{(see, e.g., Brey \& Søraker, 2009; Mittelstadt et al., 2016)}.

\vspace{0.4cm}

\noindent
\fcolorbox{black}{gray!10}{%
	\begin{minipage}{\dimexpr\linewidth-2\fboxsep-2\fboxrule}
		\vspace{0.2cm}
		\textbf{مثال ۱} \\
		\textit{تصور کنید شرکتی شما را استخدام می‌کند تا یک آگهی استخدام برای راننده کامیون پخش کنید و تعدادی نامزد بالقوه را انتخاب نمایید. شما تصمیم می‌گیرید یک فرم آنلاین برای روند درخواست شغل ایجاد کنید. برای درخواست، متقاضیان باید نام و تاریخ تولد خود را پر کنند، یک رزومه آپلود کنند، یک گزینه را برای زن یا مرد بودن تیک بزنند، و گزینه‌ای را مبنی بر رضایت به پردازش داده‌های شخصی‌شان تیک بزنند. برای جلوگیری از فراموش کردن موارد توسط افراد، تمام فیلدها را اجباری می‌کنید.}
		
		\par \vspace{0.2cm}
		
		در حالی که چنین فرمی ساده به نظر می‌رسد، در همین کاربرد کوچک نرم‌افزار «جهت‌مندی» خاصی دارد که ممکن است مشکل‌ساز باشد.
		اولاً، آنلاین بودن فرم بلافاصله فرآیند را دیجیتالی می‌کند و ممکن است افراد با مهارت دیجیتال کم را حذف کند.
		
		دوماً، اجباری بودن فرم افراد را مجبور به افشای اطلاعات (یا دروغ گفتن) و شناسایی خود بر اساس گزینه‌های فرم می‌کند. فرم دیدگاهی خاص از جهان دارد: عناصر خاصی را مهم می‌داند و دیدگاه جنسیتی دوگانه (باینری) را بیان می‌کند. متقاضیان ممکن است این «تطبیق هویت با جعبه‌های نرم‌افزار» را مشکل‌ساز بدانند (مثلاً نگرانی حریم خصوصی برای تاریخ تولد، یا عدم تمایل به تعیین جنسیت). اما انتخاب‌های آن‌ها محدود به گزینه‌های فرم آنلاین است.
		\vspace{0.2cm}
	\end{minipage}
}
\vspace{0.4cm}

شکل دادن به نرم‌افزار می‌تواند به ویژه هنگام طراحی نرم‌افزاری که نیاز به تولید تصمیمات بر اساس قوانین خاص دارد (مثل جریمه‌های خودکار سرعت یا یارانه مراقبت از کودک) دشوار باشد. در چنین مواردی، برنامه‌نویس باید قوانین حقوقی یا سیاست‌گذاری را به کد ترجمه کند و ممکن است با سوالاتی مواجه شود که دقیقاً چه زمانی یک مورد خاص تحت تعاریف قانون قرار می‌گیرد. با شکل دادن به چنین مرزهایی در کد، برنامه‌نویس مفاهیم قانونی را پر می‌کند و عملاً یک قاعده سیاستی را تثبیت می‌کند.

نقش انتخاب‌های طراحی در نرم‌افزار را به سختی می‌توان نادیده گرفت: کد کنترل می‌کند کاربران چه کاری می‌توانند و چه کاری نمی‌توانند انجام دهند. این همان چیزی است که لسیگ با بیانیه معروف خود «کد قانون است» منظور داشت \lr{(Lessig, 2006)}. با این حال، همان‌طور که پارایزر توضیح می‌دهد، کد نرم‌افزار با قدرت بیشتری نسبت به قانون رفتار کاربر را کنترل می‌کند، حداقل در ابتدا:

\begin{quote}
	\small 
	«اگر کد قانون است، مهندسان نرم‌افزار و خوره‌های کامپیوتر کسانی هستند که آن را می‌نویسند. و این نوع عجیبی از قانون است که بدون هیچ سیستم قضایی ایجاد شده و فوراً اجرا می‌شود. حتی با وجود قوانین ضد خرابکاری، در دنیای فیزیکی شما هنوز می‌توانید سنگی را به پنجره فروشگاهی که دوست ندارید پرتاب کنید. اما اگر خرابکاری بخشی از طراحی یک دنیای آنلاین نباشد، به سادگی غیرممکن است. سعی کنید سنگی را به یک ویترین مجازی پرتاب کنید؛ شما فقط یک پیام خطا دریافت می‌کنید» \lr{(Pariser, 2011, pp. 96–97)}.
\end{quote}

بنابراین توسعه‌دهنده از طریق معماری نرم‌افزار قدرت زیادی بر کاربران دارد. در این میان، **رابط کاربری** \lr{(interface)} نقش کلیدی ایفا می‌کند؛ از یک سو با پیشنهاد دادن به کاربر که نرم‌افزار چه کاری انجام می‌دهد و از سوی دیگر به عنوان قلمرو تعامل. رابط کاربری ادراک کاربران را شکل می‌دهد، دانش کار با نرم‌افزار را فراهم می‌کند و اقدامات آن‌ها را محدود می‌کند. معمولاً رابط گرافیکی \lr{(GUI)} کد منبع را پنهان می‌کند و عملکرد واقعی نرم‌افزار را غیرشفاف می‌سازد. علاوه بر این، رابط کاربری می‌تواند برای دستکاری یا «تلنگر» \lr{(nudging)} کاربران طراحی شود \lr{(see, e.g., Fogg, 1999; Thaler \& Sunstein, 2009)}. مثلاً استفاده از دکمه سبز بزرگ برای «پذیرش همه کوکی‌ها» در مقابل دکمه قرمز کوچک برای تنظیمات دیگر (به فصل گلرت در همین کتاب مراجعه کنید).

بسته به رابط کاربری، میزان بینش و انتخاب‌های کاربران تغییر می‌کند. این موضوع بر **خودمختاری** \lr{(autonomy)} کاربران تأثیر می‌گذارد: توانایی آن‌ها برای خود-حکمرانی که مستلزم آزادی تصمیم‌گیری آگاهانه است.\footnote{تعریف دقیق اینکه خودمختاری شامل چه چیزی است، بسته به دیدگاه اجتماعی و سیاسی متفاوت است. برای اهداف این فصل، من این مفهوم را نسبتاً باز نگه داشتم تا در رابطه با طراحی نرم‌افزار قابل استفاده باشد.}

برای مثال، برخی فروشگاه‌ها کاربران را ملزم به انتخاب جنسیت «زن» یا «مرد» می‌کنند، برخی گزینه «ترجیح می‌دهم نگویم» دارند و برخی اصلاً آن را اجباری نمی‌کنند. هرچه کاربران آزادانه‌تر انتخاب کنند، خودمختاری بیشتری دارند. کاهش خودمختاری و اجبار کاربران به مسیرهای خاص، می‌تواند آن‌ها را از وظیفه‌شان بیگانه کند. بنابراین دی مول و ون دن برگ استدلال می‌کنند: «آگاهی از، و بینش نسبت به "ماهیت متنی/اسکریپتی" مصنوع، و داشتن توانایی تأثیرگذاری بر آن، برای کاربران در پرتوِ واگذاری خودمختاری‌شان حیاتی است» \lr{(De Mul \& van den Berg, 2011, pp. 59–60)}.

\subsubsection*{۲۱.۲.۳ تأثیر در سطح کلان}
\addcontentsline{toc}{subsubsection}{۲۱.۲.۳ تأثیر در سطح کلان}
\label{sec:21-macro-level}

فناوری نه تنها بر فرآیندها، رویه‌ها و ادراک در سطح فردی تأثیر می‌گذارد، بلکه زندگی ما را در **سطح کلان** نیز تحت تأثیر قرار می‌دهد: سازماندهی و تراکنش‌های اجتماعی، نهادها، عاملیت دولتی، سیاست، علم، روابط بین افراد و حتی هویت ما را تحت نفوذ قرار داده و حتی شکل می‌دهد \lr{(Stiegler, 1998)}.

به‌ویژه در سطح عملکرد شرکت‌ها و نهادها، و همچنین کار افراد درون آن‌ها، استفاده از نرم‌افزار عمیقاً بر فرآیندها و خروجی‌ها تأثیر می‌گذارد که این امر به نوبه خود می‌تواند بر افراد خارج از سازمان و حتی کل جامعه تأثیر بگذارد.

برای مثال، استفاده از نرم‌افزارهای تصمیم‌گیری خودکار، مانند صدور خودکار جریمه برای سرعت غیرمجاز را در نظر بگیرید. این مستلزم تغییری در قدرت تصمیم‌گیری است، که در آن «تصمیم‌گیرنده» اصلی از یک عامل انسانی (که یاد گرفته بود دانش خود از قوانین حقوقی و سیاست‌ها را برای انجام یک ارزیابی بافتی/زمینه‌ای به کار گیرد)، به نرم‌افزاری تغییر می‌کند که قوانین را به شدت و بدون انعطاف اعمال می‌کند:

\begin{quote}
	\small 
	«تصمیمات در الگوریتم‌هایی پیش‌برنامه‌ریزی شده‌اند که همان تدابیر و قوانین را صرف‌نظر از شخص یا بافت اعمال می‌کنند (مثلاً، یک دوربین سرعت‌سنج به بافت [شرایط محیطی] اهمیت نمی‌دهد). مسئولیت تصمیمات اتخاذ شده، در این موارد، از "دیوان‌سالارانِ سطح خیابان" به "دیوان‌سالارانِ سطح سیستم"، مانند مدیران و کارشناسان کامپیوتر، منتقل شده است که تصمیم می‌گیرند چگونه چارچوب‌های سیاستی و قانونی را به الگوریتم‌ها و درخت‌های تصمیم تبدیل کنند» \lr{(Noorman, 2020)}.
\end{quote}

با تغییر قدرت تصمیم‌گیری، چنین نرم‌افزارهایی عموماً فضای اختیار فردی را کاهش می‌دهند و منجر به ایجاد نیروی کاری می‌شوند که تصمیمات را در یک فرآیند تولید یکنواخت، که تأثیر کمی بر آن دارند، تولید انبوه می‌کنند \lr{(Giritli Nygren, 2009; Wihlborg et al., 2016)}. به این ترتیب، نرم‌افزار «روابط، مسئولیت‌ها و صلاحیت‌ها را بازتعریف می‌کند» \lr{(Wihlborg et al., 2016, p. 2903)}.

علاوه بر این، هنگامی که دانش فنی \lr{(know-how)} در نرم‌افزار تعبیه می‌شود، نیاز عملیِ عوامل انسانی به داشتن همان دانش فنی کاهش می‌یابد و گاهی حتی ناپدید می‌شود: یک کلیک روی دکمه می‌تواند برای فراهم کردن آنچه کاربران نیاز دارند کافی باشد. مثالی از این مورد، بانکی است که در آن «مشاوران مشتری نرخ‌های بهره از پیش تعیین‌شده را از سیستم IT برای اعتبار مشتریانشان دریافت می‌کنند، اما نمی‌دانند این نرخ بهره چگونه محاسبه می‌شود یا چه چیزی آن را توجیه می‌کند» \lr{(Spiekermann, 2015, p. 12)}.

با واگذاری دانش فنی به نرم‌افزار، عوامل انسانی و جامعه به طور کلی، به‌طور فزاینده‌ای برای بسیاری از فرآیندهای خود به نرم‌افزار وابسته می‌شوند. استیگلر \lr{(Stiegler)} بنابراین استدلال می‌کند که فناوری به نوعی یک زهر است که هم‌زمان پادزهرِ خودش نیز هست—یک «فارماکون» \lr{(pharmakon)} \lr{(Stiegler, 2012)}: در حالی که نرم‌افزار به انسان‌ها اجازه می‌دهد دانش و چگونگی انجام کارهای خاص را فراموش کنند (زهر)، هم‌زمان با انجام آن اقدامات برای آن‌ها، این از دست دادنِ دانش فنی را جبران می‌کند (پادزهر).

برای مثال به شماره‌های تلفن فکر کنید. در دوران پیش از تلفن همراه، شماره‌های تلفن در تلفن ذخیره نمی‌شدند. این معمولاً بدان معنا بود که شما به‌طور خودکار شماره‌های خانواده و دوستان نزدیک را حفظ می‌کردید زیرا باید مرتباً شماره را تایپ می‌کردید و همچنین، صرف کمی تلاش برای به خاطر سپردن یک شماره سریع‌تر از جستجوی آن در دفترچه تلفن بود. با این حال، با تلفن‌های هوشمند، این نیاز به به خاطر سپردن عملاً زائد شد و تایپ کردن شماره غیرضروری است: فناوری این کار را برای ما انجام می‌دهد. نتیجه این است که ما بسیار کمتر احتمال دارد شماره تلفن‌ها را به خاطر بسپاریم مگر اینکه فعالانه برای حفظ آن‌ها تلاش کنیم. اثر این موضوع زمانی به‌طور دردناکی مشخص می‌شود که دسترسی به لیست مخاطبین در تلفن خود را از دست بدهید.

هرچه بیشتر به نرم‌افزارهای خاص وابسته شویم، قدرت آن‌ها بر ما بیشتر می‌شود. ما می‌توانیم این را به وضوح در استفاده از موتورهای جستجو ببینیم. به دلیل فراوانی منابع اطلاعاتی در وب، ما برای یافتن اطلاعات آنلاین به شدت به استفاده از آن‌ها وابسته شده‌ایم. به این ترتیب، این موقعیت محوریِ موتورهای جستجو، آن‌ها را با قدرت قابل توجهی بر ارتباط بین کاربران و ارائه‌دهندگان محتوا مجهز می‌کند: موتورهای جستجو «لنزهای توجه هستند؛ آن‌ها دنیای آنلاین را در کانون توجه قرار می‌دهند. آن‌ها می‌توانند تغییر مسیر دهند، آشکار کنند، بزرگ‌نمایی کنند و تحریف نمایند. آن‌ها قدرت عظیمی برای کمک کردن و پنهان کردن دارند» \lr{(Grimmelmann, 2010, p. 435)}. حذف شدن از لیست نتایج جستجوی یک موتور جستجو می‌تواند محتوا را برای بخش قابل توجهی از کاربران وب تقریباً نامرئی کند—با تمام عواقب ناشی از آن برای صاحب محتوای وب و همچنین برای کاربران جستجوگر.

قدرت نرم‌افزار با اعتمادی که مردم به فناوری برای انجام صحیح وظایفش دارند، تقویت می‌شود: مردم تمایل دارند دارای یک «سوگیری اتوماسیون» \lr{(automation bias)} باشند که به موجب آن به خروجی نرم‌افزار بیشتر از ارزیابی خود اعتماد می‌کنند \lr{(see, e.g., Skitka et al., 1999)}. به این ترتیب، آن‌ها ممکن است برای ارزیابی خود یا برای تصمیم‌گیری سریع، بیش از حد به نرم‌افزار تکیه کنند \lr{(Skitka et al., 2000)}. ترکیب تمایل انسان به سوگیری اتوماسیون با یک رابط کاربری غیرشفاف که عینیت یا بی‌طرفیِ عملیات نرم‌افزار را القا می‌کند (در حالی که نرم‌افزار در واقع ناگزیر است حاوی برخی سوگیری‌های عمدی یا غیرعمدی و شاید حتی برخی خطاها باشد)، دستورالعملی برای فاجعه است.

مقیاس پردازشِ فراهم شده توسط نرم‌افزار می‌تواند تأثیر سوگیری‌های آن را به سطحی در گستره‌ی جامعه بزرگ‌نمایی کند. نرم‌افزار ممکن است «تأثیرات بسیار عظیم‌ترِ سوگیری‌های سطح سیستم و نقاط کور را عادی‌سازی کند» \lr{(Gandy, 2010, p. 33)}.

برای مثال رسانه‌های اجتماعی را در نظر بگیرید. این نوع نرم‌افزار با ایجاد استانداردهای جدیدی از آنچه «نرمال» در نظر گرفته می‌شود، منجر به تغییراتی در فرهنگ وب شد \lr{(Van Dijck, 2013; Wittkower, 2014)}. یکی از تغییرات ایجاد شده توسط رسانه‌های اجتماعی، تغییر از ارتباطات آنلاین نسبتاً ناشناس به ارتباطات الگویی است که در آن «افراد به‌طور فزاینده‌ای شناخته شده‌اند و در واقع با میل خود بسیاری از اطلاعات شخصی‌شان را آنلاین به اشتراک می‌گذارند» \lr{(Sparrow et al., 2005, p. 283)}.

نقش محوری در اینجا توسط **تنظیمات پیش‌فرض** \lr{(default settings)} نرم‌افزار ایفا می‌شود. تنظیمات پیش‌فرض استانداردی را برای استفاده از آن تعیین می‌کنند و کاربران با ترجیحات متفاوت را ملزم می‌کنند تا برای تنظیم مجدد پیش‌فرض، زمان و تلاش صرف کنند \lr{(see, e.g., van den Berg \& Leenes, 2013; Acquisti et al., 2015)}. تنظیمات پیش‌فرض بدین ترتیب دیدگاه خاصی از جهان، یک «هنجار»، را با توجه به استفاده از آن بیان می‌کنند.

برای مثال، در ابتدا در فیس‌بوک، تنظیم پیش‌فرض یک حساب کاربری این بود که تمام اطلاعات و پست‌های کاربر به‌صورت عمومی در دسترس باشند. با این تنظیمات پیش‌فرض، فیس‌بوک پیشنهاد می‌کرد که استاندارد این است که به عنوان یک شخصِ حقیقیِ خاص، برای مخاطبانی بالقوه در سراسر جهان در دسترس، قابل دسترسی و قابل شناسایی باشید. علاوه بر این، کاربران تمایل دارند تنظیمات پیش‌فرض را بپذیرند، زیرا «راحت است، و مردم اغلب تنظیمات پیش‌فرض را به عنوان توصیه‌های ضمنی تفسیر می‌کنند» \lr{(Acquisti et al., 2015, p. 512)}. بنابراین تنظیمات پیش‌فرض به‌شدت بر رفتار و هنجارهای کاربر تأثیر می‌گذارد.

در نهایت، خروجی نرم‌افزار می‌تواند بر زندگی افرادی که کاربرانِ نرم‌افزار نیستند—و همچنین بر کل جامعه—تأثیر بگذارد. برای مثال با قرار دادن گروه‌های خاصی از مردم در وضعیت نامطلوب. بیایید با تمرکز بر برنامه‌های تصمیم‌گیری خودکار، مانند آن‌هایی که جریمه سرعت صادر می‌کنند، افراد را به عنوان ریسک کلاهبرداری علامت‌گذاری می‌کنند، یا مبلغی را که افراد باید برای بیمه خود بپردازند محاسبه می‌کنند، کمی عمیق‌تر به این موضوع بپردازیم.

در این موارد، افرادی که کاربران اولیه نرم‌افزار نیستند، در معرض خروجی (تصمیم) تولید شده توسط نرم‌افزار قرار می‌گیرند. با این حال، از آنجا که شفافیت خروجی نرم‌افزار وابسته به آن چیزی است که در نرم‌افزار برنامه‌نویسی شده تا به عنوان خروجی نشان داده شود، مردم ممکن است پروفایل‌بندی شوند و در معرض تصمیمی قرار گیرند که چگونگی، چرایی و چیستی آن برایشان روشن نیست. به این ترتیب، برای آن‌ها دشوار است که بفهمند آیا خطایی رخ داده است یا خیر، و اگر بله، کجا و چگونه.

کمبود بینش در مورد آنچه در نرم‌افزار اتفاق می‌افتد می‌تواند به‌ویژه در مورد نرم‌افزارهای تصمیم‌گیری خودکار مورد استفاده توسط آژانس‌های دولتی مشکل‌ساز باشد، زیرا این آژانس‌ها وظیفه دارند در تصمیمات خود شفاف باشند و برای آن‌ها دلیل بیاورند. علاوه بر این، فقدان شفافیت و عدم دسترسی به همان نرم‌افزار، به چالش کشیدنِ مؤثرِ یک تصمیمِ تولید شده به صورت خودکار را برای مردم دشوار می‌کند.

این امر با ایجاد «نابرابری در سلاح‌ها» \lr{(inequality of arms)} بین یک شهروند عادی و آژانس—که عموماً از قبل دارای موقعیت قدرت است زیرا مردم برای یک چیز یا چیز دیگر به آژانس وابسته هستند—منجر به عدم توازن قدرت می‌شود. این‌ها تنها برخی از مسائل مربوط به نرم‌افزارهای تصمیم‌گیری خودکار هستند. تأثیر نرم‌افزارهای تصمیم‌گیری خودکار بر زندگی و جهان ما بحثی بسیار گسترده است که در اینجا نمی‌گنجد.

\subsubsection*{۲۱.۲.۴ در مجموع}
\addcontentsline{toc}{subsubsection}{۲۱.۲.۴ در مجموع}
\label{sec:21-in-sum}

این بخش بحث کرد که چرا نرم‌افزار، مانند تمام فناوری‌ها، یک ابزار خنثی نیست. نرم‌افزار دارای «جهت‌مندی» \lr{(directionality)} خاصی است که به احتمال زیاد بر شیوه‌ای که ما کار می‌کنیم، تصمیم می‌گیریم و تعامل داریم، تأثیر می‌گذارد و حتی آن را تغییر می‌دهد.

تأثیر آن می‌تواند تا عمق جامعه و به‌ویژه به زندگی افراد کشیده شود.\footnote{تحت فشار نهادهای عمومی و قانون‌گذاری اروپا، فیس‌بوک در نهایت تنظیمات پیش‌فرض خود را به مخاطبان محدود تغییر داد و با آن استانداردی تا حدودی دوست‌دارِ حریم خصوصی‌تر تعیین کرد.} سوال اکنون این است که چگونه باید با این عدم خنثی بودن برخورد کنیم.

% =================================================================
% بخش ۲۱.۳ (با لیبل اصلاح شده برای رفع خطا)
% =================================================================

\raggedbottom 

\subsection*{۲۱.۳ به‌کارگیریِ «هرگز خنثی»}
\addcontentsline{toc}{subsection}{۲۱.۳ به‌کارگیریِ «هرگز خنثی»}
\label{sec:21-employing-main} % <--- اصلاح شد: نام لیبل تغییر کرد

این بخش رویکردی را در مورد نحوه برخورد با عدم خنثی بودن فناوری ارائه می‌دهد. این بخش با استدلال برای یک رویکرد پیش‌دستانه \lr{(proactive)} نسبت به ارزش‌ها در طراحی فناوری آغاز خواهد شد. برای ارائه دستگیره‌هایی در مورد چگونگی شروع، این بخش سپس طرح کلی از «طراحی حساس به ارزش» را ارائه می‌دهد. در نهایت، با توجه به تمرکز این فصل بر نرم‌افزار، این بخش نگاهی به ارزش‌های ترویج شده توسط مقررات عمومی حفاظت از داده‌ها \lr{(GDPR)} در هنگام پردازش داده‌های شخصی می‌اندازد.

\subsubsection*{۲۱.۳.۱ چالشی برای طراحان}
\addcontentsline{toc}{subsubsection}{۲۱.۳.۱ چالشی برای طراحان}
\label{sec:21-challenge}

در حالی که فناوری لزوماً خوب یا بد نیست، اما هرگز خنثی نیست. بنابراین طراحی فناوری نقشی محوری دارد: در این مرحله از فرآیند، بخش قابل توجهی از آنچه یک فناوری خاص انجام می‌دهد و انجام نمی‌دهد (یعنی جهت‌مندی آن) تعیین می‌شود. بنابراین دی مول و ون دن برگ اشاره می‌کنند که علی‌رغم نفوذ شدید فناوری بر ما و جهان ما، «مسئولیت آن جهان و آنچه در آن اتفاق می‌افتد هنوز در دست انسان‌هاست و نه در دست فناوری‌ها. از آنجا که انسان‌ها معماران، طراحان و کاربران فناوری‌ها هستند، به همین دلیل مسئول مخلوقات خود و خروجیِ مخلوقاتشان هستند» \lr{(De Mul \& van den Berg, 2011, p. 46)}.

فناوری توسط ما طراحی می‌شود و در بسیاری از موارد، ما قادر خواهیم بود فناوری را به گونه‌ای طراحی کنیم که بتوانیم تأثیرات مشکل‌ساز آن را کاهش دهیم یا حتی از آن‌ها جلوگیری کنیم.

بنابراین راهی برای مقابله با عدم خنثی بودنِ ذاتیِ فناوری، طراحیِ آگاهانه‌ی فناوری به شیوه‌ای است که از ارزش‌های اجتماعی یا اخلاقی خاصی مانند آزادی، ایمنی و حریم خصوصی حمایت یا آن‌ها را ترویج کند. بنابراین، ما باید در همان فرآیند طراحی بپرسیم که تأثیر بالقوه یک نرم‌افزار چه می‌تواند باشد، و اگر بخواهیم ارزش‌های خاصی را ترویج کنیم (در حالی که از درج سوگیری‌های مشکل‌ساز در فناوری جلوگیری می‌کنیم)، این نرم‌افزار چگونه باید کار کند.

البته، همه اثرات آینده و استفاده‌های ناخواسته قابل پیش‌بینی نیستند (به‌ویژه که زندگی واقعی پیچیده و درهم‌ریخته است، نکته‌ای که کیمولن و تیلور در این کتاب به آن اشاره کرده‌اند)، و از همه چیز نمی‌توان پیشگیری کرد. با این حال، شروع خوب این است که از مراحل اولیه طراحی، آگاهانه ارزش‌های خاصی را پیاده‌سازی کنیم و سعی کنیم نسبت به ارزش‌هایی که ناخودآگاه در فناوری می‌سازیم، آگاه شویم. با این کار، «نوآوری فناورانه می‌تواند به نوآوری مسئولانه تبدیل شود» \lr{(van den Hoven et al., 2015, p. 3)}.

این موضوع مسئولیت فعالی را بر دوش مهندسان می‌گذارد. تمرکز آگاهانه بر ارزش‌های درج شده در طراحی می‌تواند کمک کند تا اطمینان حاصل شود که فناوری نیازهای اجتماعی را برآورده می‌کند و به کاهش خطر اثرات ناخواسته یا مضر کمک می‌کند. این همچنین برای طراحان و مهندسان سودمند است، زیرا ممکن است از آسیب به شهرت آن‌ها (زمانی که مردم فناوری را غیرقابل اعتماد یا مضر بدانند) جلوگیری کند.

علاوه بر این، در برخی موارد، طراحی فناوری به گونه‌ای که ارزش‌های خاصی را ترویج کند، حتی توسط قانون الزامی شده است. قانون مهمی در این زمینه \lr{GDPR} است (برای اطلاعات بیشتر در مورد این مقررات به فصل گلرت در این کتاب مراجعه کنید)، که از عواملی که داده‌های شخصی را پردازش می‌کنند می‌خواهد تا در «حریم خصوصی در طراحی» \lr{(privacy by design)} مشارکت کنند (ماده ۲۵، \lr{GDPR}، که بعداً به آن بازخواهم گشت).

\subsubsection*{۲۱.۳.۲ طراحی حساس به ارزش}
\addcontentsline{toc}{subsubsection}{۲۱.۳.۲ طراحی حساس به ارزش}
\label{sec:21-vsd}

یکی از راه‌هایی که می‌توانیم آگاهانه تلاش کنیم با عدم خنثی بودنِ (مشکل‌ساز یا سودمند) فناوری برخورد کنیم، درگیر شدن در شیوه‌ای از طراحی است که «حساس به ارزش» باشد. رویکردهای متعددی برای در نظر گرفتن صریح ارزش‌های انسانی هنگام طراحی فناوری توسعه یافته‌اند. این رویکردها «حداقل در چهار ادعای کلیدی مشترک هستند: ارزش‌ها می‌توانند در فناوری بیان و تعبیه شوند؛ فناوری‌ها تأثیرات واقعی و گاهی غیرواضح بر کسانی دارند که به‌طور مستقیم و غیرمستقیم تحت تأثیر قرار می‌گیرند؛ تفکر صریح در مورد ارزش‌هایی که در طراحی فنی منتقل می‌شوند از نظر اخلاقی مهم است؛ و ملاحظات ارزشی باید در اوایل فرآیند طراحی فنی آشکار شوند» \lr{(Friedman et al., 2017, p. 65)}. مشهورترینِ این رویکردها، «طراحی حساس به ارزش» \lr{(VSD)} است (برای مرور گسترده، نگاه کنید به \lr{Friedman \& Hendry, 2019}).

ایده کلی \lr{VSD} در اواسط دهه ۱۹۹۰ توسعه یافت \lr{(Friedman et al., 2017, p. 64)}. \lr{VSD} رویکردی به طراحی فناوری است که ارزش‌های انسانی را در کل فرآیند طراحی در نظر می‌گیرد \lr{(Friedman et al., 2008, p. 76)}. ون دن هوون آن را به عنوان «یک ادغام پیش‌دستانه‌ی اخلاق—بارگذاری اولیه‌ی اخلاق—در طراحی، معماری، نیازمندی‌ها، مشخصات، استانداردها، پروتکل‌ها، ساختارهای انگیزشی و ترتیبات نهادی» توصیف می‌کند \lr{(Van den Hoven, 2008, p. 63)}.

\lr{VSD} در حال توسعه مداوم است و ممکن است همیشه باشد (که لزوماً چیز بدی نیست). روش‌شناسی کلی آن هنوز با چالش‌هایی روبروست (نگاه کنید به \lr{Friedman et al., 2017; Winkler \& Spiekermann, 2018})—که تعدادی از آن‌ها در زیر بحث خواهد شد. علی‌رغم چالش‌ها، به‌طور کلی، \lr{VSD} یک رویکرد نسبتاً عملی در مورد طراحی فناوریِ آگاه به ارزش است و می‌تواند برای کسانی که در قلب فرآیند طراحی هستند، ارزش قابل توجهی داشته باشد.

روش‌شناسی \lr{VSD} از جمله بر علوم اجتماعی و تحقیقات تعامل انسان و کامپیوتر تکیه دارد \lr{(Friedman et al., 2017, p. 64)}. روش‌شناسی آن مطالعات تجربی، فنی و مفهومی را ترکیب می‌کند و این‌ها را به شیوه‌ای تکرارپذیر و یکپارچه در طول فرآیند طراحی به کار می‌گیرد \lr{(Friedman et al., 2008, p. 93)}. با این ترکیب روش‌شناختی، \lr{VSD} یک موضع تعاملی اتخاذ می‌کند و از این پیش‌فرض شروع می‌کند که «انسان‌هایی که به عنوان افراد، سازمان‌ها یا جوامع عمل می‌کنند، ابزارها و فناوری‌هایی را که طراحی و پیاده‌سازی می‌کنند شکل می‌دهند؛ و در مقابل، آن ابزارها و فناوری‌ها تجربه انسانی و جامعه را شکل می‌دهند» \lr{(Friedman et al., 2017, p. 68)}.

\vspace{0.4cm}

\noindent
\fcolorbox{black}{gray!10}{%
	\begin{minipage}{\dimexpr\linewidth-2\fboxsep-2\fboxrule}
		\vspace{0.2cm}
		\textbf{مثال ۲} \\
		\textit{تصور کنید می‌خواهید نرم‌افزاری توسعه دهید که به مردم کمک کند زمان کمتری را صرف نگاه کردن به گوشی هوشمند خود کنند. با استفاده از تحلیل تجربی، می‌توانید زمینه و تجربه استفاده فعلی افراد از گوشی هوشمند را بررسی کنید و ایده‌ای از خواسته‌ها و مشکلات آن‌ها به دست آورید.}
		
		\par \vspace{0.2cm}
		
		برای بررسی این موضوع، به‌طور ایده‌آل از روش‌های تحقیق کمی و/یا کیفی از علوم اجتماعی، مانند مصاحبه، نظرسنجی و تحلیل آماری استفاده می‌کنید. علاوه بر این، می‌توانید از چنین تحلیل‌های تجربی برای آزمایش طراحی خود استفاده کنید.
		
		با این حال، این تحلیل‌ها تمام آن چیزی نیستند که مرتبط است. کاربران ممکن است همیشه ندانند چه می‌خواهند (به‌ویژه از قبل)، یا از تمام پیامدهای کاری که انجام می‌دهند و استفاده می‌کنند آگاه نباشند، و همچنین ممکن است شما داده‌های کافی برای نظارت بر تصویر بزرگ‌تر نداشته باشید. بنابراین انجام یک **تحلیل مفهومی** \lr{(conceptual analysis)} برای به دست آوردن تصویری کامل(تر) از مفاهیم و مسائل درگیر، مانند ارزش‌هایی که نقش دارند یا پیامدهای گسترده‌تر فردی و اجتماعی فناوری، مهم است.
		
		برای این تحلیل مفهومی، شما از تحقیقات نظری و فلسفی استفاده می‌کنید که به مفاهیم و مسائل اصلی که به نوعی به فناوری (طراحی‌شونده) مربوط می‌شوند، می‌پردازد. بیایید بگوییم برای این نرم‌افزار، شما مطالعاتی در مورد حساب‌های فلسفیِ عاملیت، خودمختاری، تلنگر \lr{(nudging)}، دستکاری و حریم خصوصی خواهید خواند. این می‌تواند به نوبه خود به پرسش‌های تجربی بعدی و طراحی فناورانه شما جهت دهد.
		
		بنابراین انجام یک **تحلیل فنی** \lr{(technological analysis)} از فناوری نیز مهم است. درک بهتر از فناوری می‌تواند با تحلیل مکانیسم‌ها و نتایج ملموس آن، و همچنین با نگاه کردن به فناوری‌های موجود که شباهت‌های خاصی دارند و ارزیابی تأثیر آن‌ها، حاصل شود. یافته‌های شما از مکانیسم‌های فنی می‌تواند تحلیل مفهومی و تجربی شما را بیشتر آگاه و مشخص کند، که به نوبه خود می‌تواند به شما در بهبود طراحی کمک کند. و این فرآیند رفت و برگشت ادامه می‌یابد تا زمانی که به طرحی برسید که جامع باشد و توسط تحقیقات پشتیبانی شود.
		\vspace{0.2cm}
	\end{minipage}
}
\vspace{0.4cm}

\subsubsection*{۲۱.۳.۳ ارزش‌ها}
\addcontentsline{toc}{subsubsection}{۲۱.۳.۳ ارزش‌ها}
\label{sec:21-values}

در زمینه \lr{VSD}، اصطلاح «ارزش» به «آنچه برای مردم در زندگی‌شان مهم است، با تمرکز بر اخلاق و معنویات» اشاره دارد \lr{(Friedman \& Hendry, 2019, p. 24)}. بنابراین تمرکز بر ارزش‌های اجتماعی و اخلاقی است، نه بر ارزش اقتصادی. در این زمینه، می‌توانیم به ارزش‌هایی مانند رفاه انسانی، اعتماد، حریم خصوصی، انصاف، خودمختاری، کاربردپذیری جهانی، ایمنی، سلامت و پایداری محیط زیست فکر کنیم. ارزش‌هایی که به‌طور بالقوه توسط \lr{VSD} پوشش داده می‌شوند، از آن‌هایی که می‌توانند در نظریه‌های مختلف فلسفی اخلاقی مانند وظیفه‌گرایی، پیامدگرایی و اخلاق فضیلت یافت شوند (نگاه کنید به فصل کیمولن و تیلور در این کتاب)، تا ارزش‌های شخصی مانند ترجیحات سلیقه و رنگ، و قراردادهایی مانند استانداردهای پروتکل متغیر است \lr{(Friedman et al., 2008, p. 94)}.

\lr{VSD} تمایل دارد انتخاب و ارزیابی ارزش خود را بر اساس تجربیات و نظرات ذینفعان بنا کند. بنابراین، یک عنصر کلیدی در \lr{VSD} شناسایی ذینفعان مستقیم و غیرمستقیم و ارزش‌های مربوط به آن‌هاست \lr{(Friedman et al., 2017, p. 69)}. ذینفعان مستقیم افراد، گروه‌ها یا سازمان‌هایی هستند که مستقیماً با فناوری مورد نظر تعامل دارند \lr{(Friedman et al., 2017, p. 76)}. ذینفعان غیرمستقیم شامل افراد، گروه‌ها یا سازمان‌هایی هستند که تحت تأثیر فناوری قرار می‌گیرند، اما مستقیماً با آن تعامل ندارند \lr{(Friedman et al., 2017, p. 76)}.

مثالی از روشی برای درک ارزش‌های در معرض خطرِ ذینفعان مستقیم و غیرمستقیم، انجام مصاحبه‌های نیمه‌ساختاریافته است \lr{(Friedman et al., 2008, pp. 100–101)}. با این حال، شناسایی ذینفعان می‌تواند دشوار باشد، و شکست در شناسایی گروه خاصی از ذینفعان می‌تواند منجر به طرد آن‌ها و همچنین طرد ارزش‌های خاص شود \lr{(Manders-Huits, 2011; Winkler \& Spiekermann, 2018)}. علاوه بر این، خود ذینفعان ممکن است همیشه نتوانند بر تأثیر فناوری‌های خاص نظارت کنند و تشخیص دهند که کدام یک از ارزش‌های آن‌ها در یک زمینه خاص ممکن است در خطر باشد.

در برخی موارد، مشخص می‌شود که دو یا چند ارزش متضاد درگیر هستند. این ارزش‌های متضاد لزوماً نباید از ذینفعان مختلف سرچشمه بگیرند: همان ذینفع می‌تواند ارزش‌های متعددی داشته باشد که ممکن است طراحی را به جهات مختلف بکشاند. اگر تنشی بین ارزش‌ها وجود داشته باشد، مهم است که این موضوع را در فرآیند طراحی در نظر بگیریم \lr{(Friedman et al., 2017, p. 69)}.

\vspace{0.4cm}

\noindent
\fcolorbox{black}{gray!10}{%
	\begin{minipage}{\dimexpr\linewidth-2\fboxsep-2\fboxrule}
		\vspace{0.2cm}
		\textbf{مثال ۳} \\
		\textit{یک نمونه از تنش بین ارزش‌ها، تنش احتمالی بین حریم خصوصی و امنیت ملی یا عمومی است: در حالی که حریم خصوصی عموماً از جمع‌آوری و افشای داده‌های شخصی کمتر سود می‌برد، امنیت عموماً از دسترسی به داده‌های شخصی بیشتر سود می‌برد.}
		
		\par \vspace{0.2cm}
		
		این تنش نقش محوری در معرفی اسکنرهای بدن در فرودگاه‌ها داشت. هدف اسکنر بدن افزایش ایمنی با نشان دادن بصری به کارکنان امنیتی فرودگاه است که افراد کجای بدنشان اشیاء حمل می‌کنند. برای این کار، آن‌ها سطح بدن را اسکن می‌کنند و این می‌تواند دید نسبتاً دقیقی از آنچه بدن برهنه شخص اسکن شده به نظر می‌رسد نمایش دهد. این کار عمیقاً حریم خصوصی بدنیِ افراد اسکن شده را نقض می‌کند.
		
		بسیاری از توسعه‌دهندگان اسکنر بدن این نقض حریم خصوصی را به عنوان یک قربانیِ قابل قبول به نام ایمنی بدیهی فرض کردند \lr{(Spiekermann, 2015, p. 169)}. با این حال، مشخص شد که نه عموم مردم و نه کارکنان امنیتی و اپراتورهای فرودگاه (که مجبور بودند با شکایات مشتریان برخورد کنند و از کاهش مشتریان می‌ترسیدند) از این نقض حریم خصوصی راضی نبودند.
		
		یک شرکت هر دو ارزش—ایمنی و حریم خصوصی—را جدی گرفت و سعی کرد نرم‌افزار اسکنری طراحی کند که نقض حریم خصوصی را کاهش دهد و در عین حال هدف ایمنی خود را حفظ کند. در طراحی حاصل، نمایش بدن با یک طرح کلی انتزاعی از یک بدن جایگزین شد که در آن مناطقی که شیء روی بدن قرار داشت علامت‌گذاری شده بود. با این طراحی، شرکت به هدف ایمنی خود رسید و در عین حال تدابیر حفاظتی حریم خصوصی را در نرم‌افزار ایجاد کرد. با جدی گرفتن هر دو ارزش و تعبیه آن‌ها در طراحی، شرکت موفق شد اکثریت بازار را تصاحب کند \lr{(Spiekermann, 2015, p. 169)}.
		\vspace{0.2cm}
	\end{minipage}
}
\vspace{0.4cm}

\subsubsection*{۲۱.۳.۴ ارزش‌های قانونی و طراحی}
\addcontentsline{toc}{subsubsection}{۲۱.۳.۴ ارزش‌های قانونی و طراحی}
\label{sec:21-legal-values}

یک منبع خوب برای یافتن ارزش‌هایی که طراحی نرم‌افزار حساس به ارزش باید به‌طور ایده‌آل در نظر بگیرد، قانون است. در زمینه طراحی نرم‌افزار، \lr{GDPR} به دلیل تمرکز بر پردازش داده‌ها از اهمیت ویژه‌ای برخوردار است. \lr{GDPR} مجموعه‌ای از ارزش‌ها را برای ما فراهم می‌کند که باید به نمایندگی از (حفاظت از) «موضوعات داده» \lr{(data subjects)} و کل جامعه در نظر گرفته شوند.

در زیر برخی از ارزش‌های اصلی (مشتق شده) ذکر شده‌اند که می‌توان در \lr{GDPR} یافت (شبیه ساختار تصویر):

\begin{itemize}
	\item[\textbf{--}] \textbf{خودمختاری} \lr{(Autonomy)} (نگاه کنید به، مثلاً، رضایت آگاهانه، ماده ۴(۱۱)، ماده ۷)
	\item[\textbf{--}] \textbf{حریم خصوصی} \lr{(Privacy)} (نگاه کنید به، مثلاً، کنترل بر داده‌های شخصی، مواد ۱۷ و ۲۱)
	\item[\textbf{--}] \textbf{حفاظت در برابر عدم تعادل قدرت} (نگاه کنید به، تصمیم‌گیری فردی خودکار، ماده ۲۲؛ محدودیت هدف، ماده ۵(۱)(ب)؛ به حداقل رساندن داده‌ها)
	\item[\textbf{--}] \textbf{کرامت انسانی} \lr{(Human dignity)}
	\item[\textbf{--}] \textbf{انصاف} \lr{(Fairness)} (نگاه کنید به، ماده ۵(۱)(الف))
	\item[\textbf{--}] \textbf{ایمنی/حفاظت} \lr{(Safety/protection)} (نگاه کنید به، ماده ۲۵)
	\item[\textbf{--}] \textbf{امنیت} \lr{(Security)} (نگاه کنید به، ماده ۳۲)
	\item[\textbf{--}] \textbf{احترام به حقوق و آزادی افراد}
	\item[\textbf{--}] \textbf{رفاه انسانی} \lr{(Human welfare)}
	\item[\textbf{--}] \textbf{شفافیت} \lr{(Transparency)} (نگاه کنید به، ماده ۵(۱)(الف))
	\item[\textbf{--}] \textbf{رفاه اقتصادی} \lr{(Economic prosperity)}
\end{itemize}

آنچه در مورد \lr{GDPR} در پرتو این فصل جالب است، این است که \lr{GDPR} حتی به‌طور صریح تعبیه برخی از ارزش‌های زیربنایی خود را در طراحی نرم‌افزار الزامی می‌کند. ماده ۲۵(۱) \lr{GDPR} در مورد «حفاظت از داده‌ها از طریق طراحی و به‌طور پیش‌فرض» بیان می‌کند:

\begin{quote}
	\small
	«با در نظر گرفتن وضعیت هنر [تکنولوژی روز]، هزینه اجرا و ماهیت، دامنه، زمینه و اهداف پردازش و همچنین خطراتِ با احتمال و شدتِ متغیر برای حقوق و آزادی‌های اشخاص حقیقی که ناشی از پردازش است، کنترل‌کننده باید (...) اقدامات فنی و سازمانی مناسب، مانند نام‌گذاری مستعار \lr{(pseudonymisation)} را اجرا کند که برای پیاده‌سازی اصول حفاظت از داده‌ها، مانند به حداقل رساندن داده‌ها، به شیوه‌ای مؤثر و برای ادغام تدابیر حفاظتی لازم در پردازش به منظور برآوردن الزامات این مقررات و حفاظت از حقوق موضوعات داده طراحی شده‌اند (ماده ۲۵(۱)، \lr{GDPR}).»
\end{quote}

اصول حفاظت از داده‌ها (که در فصل گلرت به‌طور مفصل تحلیل شده‌اند) در ماده ۵ \lr{GDPR} ذکر شده‌اند و بیان می‌کنند که داده‌های شخصی باید:
به‌طور قانونی، منصفانه و شفاف پردازش شوند («قانونی بودن، انصاف و شفافیت»)؛
فقط برای اهداف مشخص، صریح و مشروع جمع‌آوری شوند («محدودیت هدف»)؛
کافی، مرتبط و محدود به آنچه ضروری است باشند («به حداقل رساندن داده‌ها»)؛
دقیق و در صورت لزوم به‌روز باشند («دقت»)؛
به شکلی نگهداری شوند که شناسایی موضوعات داده را بیش از حد لازم ممکن نسازد («محدودیت ذخیره‌سازی»)؛
و به شیوه‌ای پردازش شوند که امنیت مناسب داده‌های شخصی را تضمین کند («یکپارچگی و محرمانگی»).

نقش مهم این اصول، محدود کردن داده‌های شخصی است که می‌توان جمع‌آوری و نگهداری کرد. در اینجا، ضرب‌المثل «دانش قدرت است» به ذهن می‌آید. در این زمینه، اصل محدودیت هدف، اصل به حداقل رساندن داده‌ها و اصل محدودیت ذخیره‌سازی محدودیت‌های مهمی هستند که عدم تعادل قدرتی را که ممکن است بین شهروندان و موسسات/شرکت‌ها ایجاد شود، کاهش می‌دهند.

در سوی دیگر سکه، اقداماتی در \lr{GDPR} وجود دارد که هدف آن‌ها متعادل کردن بهتر زمین بازی با اطمینان از اینکه خودِ موضوعات داده دانش کافی در مورد نحوه پردازش داده‌هایشان دارند، است. به عنوان مثال، ماده ۲۲ \lr{GDPR} به‌ویژه به تصمیم‌گیری خودکار می‌پردازد و در مواردی که تصمیم تولید شده می‌تواند تأثیر قابل توجهی بر زندگی یک فرد داشته باشد، مداخله انسانی را الزامی می‌کند. یک مثال از تأثیر قابل توجه، رد خودکار درخواست اعتبار آنلاین بدون مداخله انسانی است (مقدمه ۷۱). تصمیم‌گیری خودکار «باید تابع تدابیر حفاظتی مناسب باشد، که باید شامل اطلاعات خاص به موضوع داده و حق دریافت مداخله انسانی، بیان دیدگاه خود، دریافت توضیح در مورد تصمیم اتخاذ شده و به چالش کشیدن تصمیم باشد» (مقدمه ۷۱).\footnote{با این حال، الزام حضور یک عامل انسانی در حلقه تصمیم‌گیری تضمین نمی‌کند که ارزش‌های مورد نظر محافظت می‌شوند \lr{(Binns, 2019)}. مداخله انسانی نیز مزایا و معایب خود را دارد: انسان‌ها می‌توانند عمداً و سهواً تبعیض قائل شوند. علاوه بر این، اگر یک انسان به حلقه تصمیم‌گیری اضافه شود، این خطر وجود دارد که انسان صرفاً تصمیمات گرفته شده را برای اعتبارسنجی خروجی آن‌ها و دور زدن الزامات بیشتر ماده ۲۲ \lr{GDPR} «مهر تأیید» بزند \lr{(Veale \& Edwards, 2018, p. 400)}.}

چالش طراحی نرم‌افزار به صورت حساس به ارزش و تلاش برای حساب کردن روی ارزش‌هایی مانند موارد فوق، چگونگی تعبیه ملموس این ارزش‌ها در محصول طراحی شده است. برای این کار، هیچ روش یکسانی برای همه وجود ندارد.

علاوه بر این، خودِ «چگونگی» موضوع تحقیقات در حال انجام است. در مورد طراحی نرم‌افزار به شیوه‌ای که حریم خصوصی را افزایش دهد، مقاله «تحلیل انتقادی استراتژی‌های طراحی حریم خصوصی» توسط کولسکی و همکاران \lr{(Colesky et al., 2016)} می‌تواند منبع الهام ارزشمندی باشد. محققان چندین «استراتژی طراحی حریم خصوصی» را شناسایی می‌کنند که می‌توانند برای تعبیه حریم خصوصی در طراحی استفاده شوند. یکی از تاکتیک‌های پیشنهادی **به حداقل رساندن داده‌ها** است؛ این به ماده ۵(۱)(ج) \lr{GDPR} مربوط می‌شود. این مستلزم انتخابی است که در آن داده‌هایی که مورد نیاز نیستند حذف، جدا یا نابود می‌شوند.

استراتژی‌های دیگر شامل محدودیت دسترسی و تفکیک داده‌ها هستند \lr{(Colesky et al., 2016)}. با ایزوله کردن مجموعه‌های داده، یا با توزیع آن‌ها در مکان‌های مختلف، خطر ترکیب داده‌ها و ارائه دیدگاه دقیق‌تر در مورد یک فرد خاص کاهش می‌یابد. تاکتیک دیگری که آن‌ها ذکر می‌کنند **انتزاع** \lr{(abstraction)} است. اگر داده‌ها در سطح کلی‌تری خلاصه یا گروه‌بندی شوند، تمرکز داده‌ها از افراد خاص به سطح عمومی‌تر تغییر می‌کند. مطالعاتی مانند آنچه توسط کولسکی و همکاران انجام شده است، می‌تواند منبع الهام برای طراحان باشد تا طرح‌هایی ارائه دهند که ارزش‌های مورد نظرشان را محقق سازد.

% =================================================================
% بخش نتیجه‌گیری
% =================================================================

\raggedbottom

\vspace{0.5cm}

\noindent
\fcolorbox{black}{gray!15}{%
	\begin{minipage}{\dimexpr\linewidth-2\fboxsep-2\fboxrule}
		\vspace{0.4cm}
		\begin{center}
			\textbf{\Large نتیجه‌گیری}
		\end{center}
		\vspace{0.3cm}
		
		فناوری خنثی نیست. می‌تواند بر ادراک مردم، آنچه می‌دانند، آنچه می‌توانند انجام دهند، نحوه تعامل آن‌ها با جهان، و شیوه‌ای که جامعه، دولت‌ها، شرکت‌ها و دیگران با آن‌ها تعامل دارند، تأثیر بگذارد و آن را شکل دهد. بنابراین استفاده نوآورانه از فناوری می‌تواند بسیار ارزشمند، اما همچنین بسیار مشکل‌ساز باشد. از این رو، فناوری الهام‌بخشِ هم آرمان‌شهرها \lr{(utopias)} و هم ویران‌شهرها \lr{(dystopias)} بوده است.
		
		طراحی حساس به ارزش رویکردی است که هدف آن مقابله با عدم خنثی بودن فناوری به شیوه‌ای سودمند است. این رویکرد بر هدف‌گذاری فعالانه برای گنجاندن ارزش‌های خاص در طراحی فناوری تمرکز دارد. در حالی که \lr{VSD} بدون مشکلات و چالش‌ها نیست، شروعی امیدوارکننده برای طراحی فناوری است که هدف آن قرار گرفتن در سمت آرمان‌شهریِ امور است تا سمت ویران‌شهری. در برخی موارد، قانون حتی تعبیه ارزش‌های خاصی را در طراحی نرم‌افزار الزامی می‌کند: ماده ۲۵ \lr{GDPR} خواستار اجرای «حریم خصوصی در طراحی» و «حریم خصوصی به‌طور پیش‌فرض» است.
		
		طراحی حساس به ارزش کار آسانی نیست، به‌ویژه به این دلیل که اغلب دشوار، اگر نه غیرممکن، است که تأثیر و استفاده از یک فناوری جدید را به‌طور کامل پیش‌بینی کنیم. با این حال، این نباید ما را از تلاش باز دارد. در اینجا نقش محوری برای طراحان وجود دارد. با آگاهی از دیدگاه‌ها و فرضیاتی که لزوماً در طراحی سیستم ساخته شده‌اند، آن‌ها می‌توانند سعی کنند این کار را به شیوه‌ای آگاهانه و حساس به ارزش انجام دهند.
		
		به منظور شروع طراحی حساس به ارزش، در نظر داشتن **قواعد سرانگشتی** زیر می‌تواند کمک‌کننده باشد:
		
		\begin{enumerate}
			\item \textbf{از عدم خنثی بودنِ ذاتیِ آنچه طراحی می‌کنید آگاه باشید:} فکر کنید که فناوری چه چیزی را به وضعیت فعلی اضافه می‌کند، چه چیزی را از آن می‌گیرد، یا چه چیزی را تغییر می‌دهد.
			\item \textbf{شناسایی کنید که کدام ارزش‌ها را می‌خواهید با طراحی خود تایید کنید} (مثلاً ممکن است بخواهید نرم‌افزاری طراحی کنید تا کارایی را ارتقا دهید و در عین حال از حریم خصوصی محافظت کنید).
			\item \textbf{تأثیر طراحی را ارزیابی کنید:} آیا فناوری به گروه خاصی از مردم سود می‌رساند یا تأثیر منفی می‌گذارد؟ و این افراد چه کسانی هستند و پیامدهای آن برای آن‌ها چیست؟
			\item \textbf{ردیابی کنید که آیا ممکن است تعصبات خاصی را به‌طور غیرضروری یا نامطلوب در طراحی درج کنید} (مثلاً آیا کاربری که در ذهن دارید نماینده کل گروه کاربران است، یا ناخودآگاه نرم‌افزار را به گونه‌ای طراحی می‌کنید که فقط به زیرمجموعه خاصی از کاربران سود می‌رساند؟).
			\item \textbf{سعی کنید ببینید آیا می‌توانید طراحی را تنظیم کنید تا از شر سوگیری ناخواسته یا تأثیر منفی خلاص شوید}، در حالی که ارزش‌هایی را که می‌خواهید تایید کنید ارتقا می‌دهید (یعنی: آزمایش، ارزیابی، تنظیم).
		\end{enumerate}
		
		چگونگی تحقق (بهترین) طراحی حساس به ارزش (و به‌ویژه حریم خصوصی در طراحی و به‌طور پیش‌فرض، با توجه به اینکه این‌ها توسط قانون الزامی شده‌اند) هنوز—و با فناوری‌های جدید همیشه خواهد بود—موضوع کاوش و آزمایش است. با این حال، اولین قدم آگاهی از عدم خنثی بودن فناوری، و تمایل به تفکر در مورد این است که کدام ارزش‌ها به‌طور ایده‌آل باید در طراحی آن محافظت شوند و چگونه می‌توان به این امر دست یافت. امیدواریم این فصل به خوانندگان در برداشتن این قدم اول کمک کند.
		\vspace{0.4cm}
	\end{minipage}
}

% =================================================================
% بخش پرسش و پاسخ
% =================================================================

\raggedbottom 

\vspace{0.8cm}

\noindent
\textbf{\large سوال:} \\
\textbf{طراحان چگونه می‌توانند بر عدم خنثی بودن نرم‌افزار تأثیر بگذارند؟}

\vspace{0.3cm}
\noindent
\textbf{\large پاسخ:} \\
طراحان با طراحی نرم‌افزار برای انجام وظایف خاص و کمک به کاربران برای رسیدن به اهداف معین، عمداً بر جهت‌مندیِ غیرخنثیِ فناوری تأثیر می‌گذارند. آن‌ها تعیین می‌کنند که کاربر با برنامه چه کاری می‌تواند و چه کاری نمی‌تواند انجام دهد و بدین ترتیب بر عدم خنثی بودن در سطح «عمل کاربر» تأثیر می‌گذارند. این موضوع در کد برنامه «تثبیت» \lr{(set in stone)} شده است. علاوه بر این، طراحان تعیین می‌کنند که کاربر هنگام تعامل با برنامه (در ترکیب با ویژگی‌های سخت‌افزار مورد استفاده) چه چیزی را درک می‌کند، و بدین ترتیب تجربه و تفسیر کاربر از فناوری را هدایت می‌کنند.

در اینجا تصویری که طراح از کاربر در ذهن دارد نقش مهمی ایفا می‌کند. یک مثال، طراحی وب‌سایتی با متن کم و تصاویر زیاد است. کاربرانی با اختلال بینایی که برای وب‌گردی خود به برنامه‌ای وابسته هستند که متن را با صدای بلند می‌خواند، برای پیمایش در چنین وب‌سایتی با مشکل مواجه خواهند شد.

علاوه بر این، طراحان می‌توانند با تعبیه فرضیات خود در طراحی، به‌طور ناخودآگاه بر عدم خنثی بودن نرم‌افزار تأثیر بگذارند. یک مثال، فرم آنلاین درخواست کارت اعتباری است که فرد را ملزم می‌کند مستقیماً با تلفن هوشمند از کارت شناسایی عکس بگیرد و اجازه آپلود فایل را نمی‌دهد. این فرض را ایجاد می‌کند که همه کاربران اینترنت دارای تلفن هوشمند هستند.

علاوه بر این، با استفاده از دفترچه‌های راهنمای کاربر و بازاریابی، می‌توان دیدگاه‌های خاصی در مورد فناوری و استفاده از آن را تحت تأثیر قرار داد و ترویج کرد؛ برای مثال با پیشنهاد اینکه استفاده از یک نرم‌افزار خاص می‌تواند سلامت، موقعیت اجتماعی، دوستی و کارایی را بهبود بخشد، خطاهای انسانی را کاهش دهد و غیره.

\vspace{0.8cm}
\hrule
\vspace{0.8cm}

\noindent
\textbf{\large سوال:} \\
\textbf{مزایای طراحی حساس به ارزش چیست؟}

\vspace{0.3cm}
\noindent
\textbf{\large پاسخ:} \\
اول و مهم‌تر از همه، درگیر شدن در طراحی حساس به ارزش به طراحان کمک می‌کند تا با بارگذاریِ اولیه‌ی ارزش‌های اخلاقی در طراحیِ نرم‌افزار (در رابط کاربری، معماری، استانداردها، مشخصات، ساختار انگیزشی، جاسازی نهادی، تنظیمات پیش‌فرض، نیازهای کاربر و غیره) به شیوه‌ای اخلاقاً مسئولانه نوآوری کنند.

علاوه بر این، در نظر گرفتن منافع ذینفعان مختلفِ مستقیم و غیرمستقیم و تلاش برای حساب کردن روی ارزش‌های آن‌ها در طراحی تا حد امکان، به ایجاد حمایت اجتماعی بیشتر برای استفاده از فناوری کمک خواهد کرد. افراد خوشحال‌تر عموماً به معنای کاربران بیشتر و تعامل کاربر بیشتر است.

همچنین، از آنجا که تامل دقیق در مورد تأثیر بالقوه نرم‌افزار بخش ضروریِ طراحیِ حساس به ارزش است، طراحان (یا شرکت سفارش‌دهنده) کمتر احتمال دارد که با پیامدهای پیش‌بینی‌نشده‌ی فناوری غافلگیر شوند.

\vspace{0.8cm}
\hrule
\vspace{0.8cm}

\noindent
\textbf{\large سوال:} \\
\textbf{متن ماده ۲۵(۱) \lr{GDPR} را که در متن بالا ذکر شد دوباره بخوانید. چه چیزی برای تحقق صحیح «حریم خصوصی در طراحی» مهم است؟}

\vspace{0.3cm}
\noindent
\textbf{\large پاسخ:} \\
ماده ۲۵(۱) \lr{GDPR} خواستار اجرای بی‌چون‌ و چرایِ حریم خصوصی در طراحی نیست. در عوض، خواستار ایجاد تعادلی ظریف بین گزینه‌های فنی موجود، منافع افراد درگیر، اهداف پردازش داده‌ها، و زمینه، خطرات و تأثیر آن بر مردم است.

بنابراین تنها حریم خصوصی نیست که باید به عنوان یک ارزش در طراحی در نظر گرفته شود: ارزش‌های دیگر، مانند ایمنی، رفاه انسانی و رفاه اقتصادی نیز باید در نظر گرفته شوند و با حریم خصوصی متعادل شوند.

به زبان ساده، حریم خصوصی در طراحی به معنای پیشگیری فعالانه از هرگونه نقض احتمالی حریم خصوصی است که برای تحقق یک هدف خاص کاملاً ضروری نیست. همان‌طور که مورد اسکنرهای بدن که در بخش ۲۱.۳ بحث شد نشان می‌دهد، یک تعادل دقیق می‌تواند منجر به طرحی شود که قادر است تا حد قابل توجهی به ارزش‌های متضاد احترام بگذارد: اسکنرهای بدنِ «آدمک‌های خطی» در حالی که هدف ایمنی خود را حفظ می‌کنند، نقض حریم خصوصی افراد اسکن شده را نیز به‌طور قابل توجهی کاهش می‌دهند. با این تعادل، این اسکنرهای بدن نمونه خوبی از حریم خصوصی در طراحی هستند.

% =================================================================
% بخش منابع
% =================================================================

\raggedbottom

\vspace{1cm}

\section*{منابع}
\addcontentsline{toc}{section}{منابع}
\label{sec:21-references-main}

\begin{latin}
	\begin{itemize}
		\setlength\itemsep{0.5em}
		
		\item[] Acquisti, A., Brandimarte, L., \& Loewenstein, G. (2015). Privacy and human behavior in the age of information. \textit{Science}, 347(6221), 509–514.
		
		\item[] Binns, R. (2019). Human judgement in algorithmic loops; individual justice and automated decisionmaking. \textit{Individual Justice and Automated Decision-Making} (September 11, 2019).
		
		\item[] Brey, P., \& Søraker, J. H. (2009). Philosophy of computing and information technology. In: \textit{Philosophy of technology and engineering sciences}, Elsevier, pp. 1341–1407.
		
		\item[] Brouwer, E., et al. (2011). Legality and data protection law: The forgotten purpose of purpose limitation.
		
		\item[] Chabot, P. (2013). \textit{The philosophy of Simondon: Between technology and individuation}. A\&C Black.
		
		\item[] Colesky, M., Hoepman, J. H., \& Hillen, C. (2016). A critical analysis of privacy design strategies. In: \textit{2016 IEEE Security and Privacy Workshops (SPW)}, IEEE, pp 33–40.
		
		\item[] De Mul, J., \& van den Berg, B. (2011). Remote control: Human autonomy in the age of computermediated agency. In: \textit{Law, human agency and autonomic computing}, Routledge, pp. 62–79.
		
		\item[] Feenberg, A. (2002). \textit{Transforming technology: A critical theory revisited}. Oxford University Press.
		
		\item[] Fogg, B. J. (1999). Persuasive technologies. \textit{Communications of the ACM}, 42(5), 27–29.
		
		\item[] Friedman, B., \& Hendry, D. G. (2019). \textit{Value sensitive design: Shaping technology with moral imagination}. MIT Press.
		
		\item[] Friedman, B., Hendry, D. G., Borning, A., et al. (2017). A survey of value sensitive design methods. \textit{Foundations and Trends® in Human–Computer Interaction}, 11(2), 63–125.
		
		\item[] Friedman B, Kahn PH, Borning A (2008). Value sensitive design and information systems. In: \textit{The handbook of information and computer ethics}, pp. 69–101.
		
		\item[] Friedman, B., \& Nissenbaum, H. (1996). Bias in computer systems. \textit{ACM Transactions on Information Systems (TOIS)}, 14(3), 330–347.
		
		\item[] Gandy, O. H. (2010). Engaging rational discrimination: Exploring reasons for placing regulatory constraints on decision support systems. \textit{Ethics and Information Technology}, 12(1), 29–42.
		
		\item[] Giritli Nygren, K. (2009). The rhetoric of e-government management and the reality of e-government work: The Swedish action plan for e-government considered. \textit{International Journal of Public Information Systems}, 2, 135–146.
		
		\item[] Goldman, E. (2008). Search engine bias and the demise of search engine utopianism. In: \textit{Web Search}, Springer, pp. 121–133.
		
		\item[] Grimmelmann, J. (2010). Some skepticism about search neutrality. \textit{The next digital decade: Essays on the future of the Internet}, p. 435.
		
		\item[] Harjumaa, M., \& Oinas-Kukkonen, H. (2007). Persuasion theories and it design. In: \textit{International Conference on Persuasive Technology}, Springer, pp. 311–314.
		
		\item[] Heidegger, M. (1954). The question concerning technology. translated by William Lovitt in \textit{the question concerning technology and other essays}. 1977.
		
		\item[] Hildebrandt, M. (2015). \textit{Smart Technologies and the End (s) of Law: Novel Entanglements of Law and Technology}. Edward Elgar Publishing.
		
		\item[] Ihde, D. (1983). \textit{Existential technics}. SUNY Press.
		
		\item[] Kiran, A. H., \& Verbeek, P. P. (2010). Trusting our selves to technology. \textit{Knowledge, Technology \& Policy}, 23(3–4), 409–427.
		
		\item[] Kitchin, R. (2017). Thinking critically about and researching algorithms. \textit{Information, Communication \& Society}, 20(1), 14–29.
		
		\item[] Latour, B. (1993). \textit{We have never been modern}. Harvard University Press.
		
		\item[] Lazzarato, M., \& Jordan, J. D. (2014). \textit{Signs and machines: Capitalism and the production of subjectivity}. Semiotext (e) Los Angeles.
		
		\item[] Lessig, L. (2006). \textit{Code Version 2.0}. Basic Books.
		
		\item[] Manders-Huits, N. (2011). What values in design? the challenge of incorporating moral values into design. \textit{Science and Engineering Ethics}, 17(2), 271–287.
		
		\item[] Mittelstadt, B. D., Allo, P., Taddeo, M., Wachter, S., \& Floridi, L. (2016). The ethics of algorithms: Mapping the debate. \textit{Big Data \& Society}, 3(2), 2053951716679679.
		
		\item[] Noorman, M. (2020). Computing and moral responsibility. In E. N. Zalta (Ed.), \textit{The Stanford Encyclopedia of Philosophy}, Spring 2020th Edition. Metaphysics Research Lab, Stanford University.
		
		\item[] Pariser, E. (2011). \textit{The filter bubble: What the Internet is hiding from you}. Penguin UK.
		
		\item[] Skitka, L. J., Mosier, K., \& Burdick, M. D. (2000). Accountability and automation bias. \textit{International Journal of Human-Computer Studies}, 52(4), 701–717.
		
		\item[] Skitka, L. J., Mosier, K. L., \& Burdick, M. (1999). Does automation bias decision-making? \textit{International Journal of Human-Computer Studies}, 51(5), 991–1006.
		
		\item[] Sparrow, B., Liu, J., \& Wegner, D. M. (2011). Google effects on memory: Cognitive consequences of having information at our fingertips. \textit{Science}, 333(6043), 776–778.
		
		\item[] Sparrow, B. C., Chapman, P., \& Gould, J. (2005). Social cognition in the internet age: Same as it ever was? pp. 273–292.
		
		\item[] Spiekermann, S. (2015). \textit{Ethical IT innovation: A value-based system design approach}. CRC Press.
		
		\item[] Stiegler, B. (1998). \textit{Technics and time: The fault of Epimetheus} (Vol. 1). Stanford University Press.
		
		\item[] Stiegler, B. (2012). Relational ecology and the digital pharmakon. \textit{Culture Machine}, 13.
		
		\item[] Thaler, R. H., \& Sunstein, C. R. (2009). \textit{Nudge: Improving decisions about health, wealth, and happiness}. Penguin.
		
		\item[] van den Berg, B., \& Leenes, R. E. (2013). Abort, retry, fail: Scoping techno-regulation and other techno-effects. In: \textit{Human law and computer law: Comparative perspectives}, Springer, pp. 67–87.
		
		\item[] Van den Hoven, J. (2008). Moral methodology and information technology. \textit{The handbook of information and computer ethics}, p. 49.
		
		\item[] van den Hoven, J., Vermaas, P. E., \& van de Poel, I. (2015). Design for values: An introduction. In: \textit{Handbook of ethics, values, and technological design: Sources, theory, values and application domains} pp. 1–7.
		
		\item[] Van Dijck, J. (2013). \textit{The culture of connectivity: A critical history of social media}. Oxford University Press.
		
		\item[] Veale, M., \& Edwards, L. (2018). Clarity, surprises, and further questions in the article 29 working party draft guidance on automated decision-making and profiling. \textit{Computer Law \& Security Review}, 34(2), 398–404.
		
		\item[] Verbeek, P. P. (2005). \textit{What things do: Philosophical reflections on technology, agency, and design}. Penn State Press.
		
		\item[] Verbeek, P. P. (2011). \textit{Moralizing technology: Understanding and designing the morality of things}. University of Chicago Press.
		
		\item[] Wihlborg, E., Larsson, H., \& Hedström, K. (2016). “The computer says no!”—A case study on automated decision-making in public authorities. In: \textit{2016 49th Hawaii International Conference on System Sciences (HICSS)}, IEEE, pp. 2903–2912.
		
		\item[] Winkler, T., \& Spiekermann, S. (2018). Twenty years of value sensitive design: A review of methodological practices in VSD projects. \textit{Ethics and Information Technology}. pp. 1–5.
		
		\item[] Wittkower, D. (2014). Facebook and dramauthentic identity: A post-goffmanian theory of identity performance on SNS. \textit{First Monday}, 19(4).
		
	\end{itemize}
\end{latin}
% =================================================================
% شروع فصل ۲۲: علم داده برای کارآفرینی (فایل chapter4_22.tex)
% =================================================================

\raggedbottom % حذف فاصله‌های اضافی عمودی

% -----------------------------------------------------------------
% تنظیمات شماره‌گذاری برای فصل ۲۲
% -----------------------------------------------------------------
\setcounter{section}{22}   % تنظیم شماره فصل روی ۲۲
\setcounter{subsection}{0} % ریست کردن زیربخش‌ها
\renewcommand{\thesection}{22}
\renewcommand{\thesubsection}{22.\arabic{subsection}}

% -----------------------------------------------------------------
% بلوک عنوان فصل و نویسندگان (شبیه تصویر ارسالی)
% -----------------------------------------------------------------
\noindent
\fcolorbox{black}{gray!15}{% کادر خاکستری برای عنوان
	\begin{minipage}{\dimexpr\linewidth-2\fboxsep-2\fboxrule}
		\vspace{0.5cm}
		\begin{center}
			% عنوان اصلی
			\textbf{\huge علم داده برای کارآفرینی: راه پیش رو}
			
			\vspace{0.5cm}
			
			% نام نویسندگان
			\large
			\textit{ویلم-جان ون دن هوول، ورنر لیبرگتس و آریان ون دن بورن} \\
			\lr{\textit{Willem-Jan van den Heuvel, Werner Liebregts and Arjan van den Born}}
		\end{center}
		\vspace{0.5cm}
	\end{minipage}
}

\vspace{0.8cm}

% -----------------------------------------------------------------
% فهرست مطالب فصل (Contents)
% -----------------------------------------------------------------
\noindent
\textbf{\Large فهرست مطالب}

\vspace{0.3cm}

\noindent
\begin{minipage}{\linewidth}
	\setlength{\parskip}{0.3em} % فاصله بین خطوط فهرست
	
	\textbf{۲۲.۱} \hspace{0.3cm} \textbf{مقدمه} 
	
	\textbf{۲۲.۲} \hspace{0.3cm} \textbf{راه پیش رو} 
	
	\hspace{1cm} ۲۲.۲.۱ \hspace{0.2cm} نرم‌افزار هوش مصنوعی \lr{(AI Software)}
	
	\hspace{1cm} ۲۲.۲.۲ \hspace{0.2cm} عملیات یادگیری ماشین \lr{(MLOps)}
	
	\hspace{1cm} ۲۲.۲.۳ \hspace{0.2cm} رایانش لبه‌ای \lr{(Edge Computing)}
	
	\hspace{1cm} ۲۲.۲.۴ \hspace{0.2cm} دوقلوهای دیجیتال \lr{(Digital Twins)}
	
	\hspace{1cm} ۲۲.۲.۵ \hspace{0.2cm} آزمایش در مقیاس بزرگ \lr{(Large-Scale Experimentation)}
	
	\hspace{1cm} ۲۲.۲.۶ \hspace{0.2cm} فرصت‌های کلان‌داده و هوش مصنوعی \lr{(Big Data and AI Opportunities)}
	
	\hspace{1cm} ۲۲.۲.۷ \hspace{0.2cm} مقررات دولتی \lr{(Government Regulation)}
	
	\vspace{0.3cm}
	\textbf{منابع}
\end{minipage}

\vspace{1cm}
% اینجا فضای خالی برای شروع متن اصلی (بخش ۲۲.۱) در مراحل بعدی است


% =================================================================
% بخش اهداف یادگیری (Learning Objectives)
% =================================================================

\vspace{0.5cm}

\noindent
\fcolorbox{black}{gray!15}{% کادر خاکستری
	\begin{minipage}{\dimexpr\linewidth-2\fboxsep-2\fboxrule}
		\vspace{0.3cm}
		% تیتر
		\textbf{\large اهداف یادگیری}
		
		\vspace{0.2cm}
		پس از خواندن این فصل، شما قادر خواهید بود تا:
		
		\begin{itemize}
			\setlength\itemsep{0.5em} % تنظیم فاصله بین آیتم‌ها
			
			\item[\textbf{--}] شناسایی و تبیین چند تحول مهمِ در حال انجام که بر کارآفرینی داده و در نتیجه بر پژوهش‌های کارآفرینی داده تأثیر می‌گذارند (و ادامه خواهند داد).
			
			\item[\textbf{--}] ترسیم چگونگی تکامل احتمالی حوزه عملی کارآفرینی داده در سال‌های پیش رو.
			
			\item[\textbf{--}] مشخص کردن تعدادی از مسیرهای امیدوارکننده برای پژوهش‌های آتی در محل تلاقی رشته‌های علم داده و کارآفرینی.
		\end{itemize}
		\vspace{0.2cm}
	\end{minipage}
}

\vspace{0.5cm}


% =================================================================
% بخش ۲۲.۱: مقدمه
% =================================================================

\vspace{0.6cm}

% تیتر بخش ۲۲.۱ (دستی)
\subsection*{۲۲.۱ مقدمه}
\addcontentsline{toc}{subsection}{۲۲.۱ مقدمه}
\label{sec:22-introduction}

این کتاب به دو حوزه‌ای پرداخته است که—تا همین اواخر—در انزوای کامل نسبت به هم قرار داشتند، یعنی علم داده و کارآفرینی. همان‌طور که در این کتاب بررسی کردیم، هر دو رشته یکدیگر را در رشته نوظهورِ کارآفرینیِ داده‌محور، مبتنی بر داده، یا کارآفرینی علم داده، که اغلب به اختصار «کارآفرینی داده» \lr{(data entrepreneurship)} نامیده می‌شود، پیدا می‌کنند.

کارآفرینی داده نیازمند دانش (حداقل پایه‌ای) در حوزه‌های مهندسی داده و تحلیل داده، و به نوبه خود، دانشِ آنچه ما «داده و جامعه» نامیده‌ایم (یعنی بافت تجاری و اجتماعی، مانند قوانین حاکم و دیدگاه‌های عموماً پذیرفته‌شده در مورد رفتار اخلاقی در قبال داده‌ها) است. از این رو، چهار بخش این کتاب هر کدام موضوعات مرتبط متعددی را در حوزه‌های مربوطه خود پوشش دادند.

اکنون که درک عمیق‌تری از دانشِ روز در تمام این حوزه‌ها به دست آورده‌ایم، وقت آن است که نگاهی به آینده (نزدیک) بیندازیم. ما تحولات مهم مختلفی را در حال ظهور می‌بینیم که—دیر یا زود—بر کارآفرینی داده تأثیر خواهند گذاشت و روش‌های جدید و وسوسه‌انگیزی را برای ایجاد ارزش تجاری بیشتر با بهره‌برداری از فرصت‌هایی که علم داده به ارمغان می‌آورد، فراهم می‌کنند: به اختصار (بسیار) کوتاه، «علم داده برای کارآفرینی».

سوال محوری این فصل، زیرعنوان این کتاب را بیشتر توضیح می‌دهد: کارآفرینان چگونه می‌توانند از کلان‌داده \lr{(big data)} و هوش مصنوعی \lr{(AI)} برای خلق ارزش جدید بهره‌برداری کنند؟ این تحولات همچنین راه‌های کاملاً جدیدی را برای پژوهش‌های آتی در محل تلاقی علم داده و کارآفرینی می‌گشایند. بنابراین، ما همچنین به اختصار در مورد پیامدهای آن‌ها برای پژوهش توسط محققان کارآفرینی بحث می‌کنیم.

در این فصل، ما ابتدا مروری بر آنچه معتقدیم مهم‌ترین تحولات هستند ارائه می‌دهیم و به اختصار در مورد پیامدها و تبعات آن‌ها از دیدگاه کارآفرینانه و علمی بحث می‌کنیم. پس از آن، ما این فصل خاص و کل کتاب را به پایان می‌رسانیم.




% =================================================================
% بخش ۲۲.۲: راه پیش رو
% =================================================================

\vspace{0.6cm}

% تیتر بخش ۲۲.۲ (دستی)
\subsection*{۲۲.۲ راه پیش رو}
\addcontentsline{toc}{subsection}{۲۲.۲ راه پیش رو}
\label{sec:22-road-ahead}

این بخش مسیری از فرصت‌ها و چالش‌ها را می‌گشاید که معتقدیم به شدت بر گفتمان نوظهورِ علم داده و کارآفرینی تأثیر خواهند گذاشت. برخی از این تحولات مربوط به فناوری‌های جدیدی هستند که افراد و شرکت‌ها را قادر می‌سازند تا پیشنهادها و ترکیبات محصول-بازار جدیدی را توسعه دهند (نوآوری‌های محصول و خدمت)، و سایر تحولات فناوری افراد و شرکت‌ها را قادر می‌سازند تا تحویل مؤثرتر یا کارآمدتری را دنبال کنند (نوآوری فرآیند). چنین شرکت‌هایی می‌توانند هم جدید و هم از قبل موجود باشند.

پنج تا ده سال پیش، زمانی که حوزه میان‌رشته‌ای کارآفرینی داده شروع به ظهور کرد، فناوری‌های مذکور وجود نداشتند یا صرفاً نقاطی کم‌اهمیت در دستور کار بودند. در حال حاضر، می‌بینیم که این فناوری‌ها بالغ‌تر شده‌اند و نقش فزاینده‌ای برای شرکت‌هایی که به دنبال مزیت رقابتی هستند، ایفا می‌کنند.

برجسته‌ترین توسعه فناورانه در چارچوبِ به‌اصطلاح «انقلاب صنعتی چهارم» (یا صنعت ۴.۰) مربوط به **هوش مصنوعی** \lr{(AI)} است. هوش مصنوعی به‌طور انکارناپذیری ویژگی‌های یک فناوریِ تحول‌آفرین با هدف عمومی \lr{(general-purpose technology)} را داراست \lr{(Brynjolfsson \& McAfee, 2014; Cockburn et al., 2018)}. و بنابراین، همان‌طور که چالمرز و همکاران \lr{(Chalmers et al., 2020)} بیان می‌کنند: «هوش مصنوعی ... پیامدهای عمیقی برای چگونگی توسعه، طراحی و مقیاس‌دهی سازمان‌ها توسط کارآفرینان دارد» (ص ۱۵). علاوه بر این تحولات فناورانه، ما شاهد تغییرات مهم اجتماعی و اقتصادی در چشم‌انداز خود نیز هستیم. به روندهای کلان مانند جهانی‌سازی و بین‌المللی‌سازی فکر کنید.

همان‌طور که کسب‌وکارها از نظر پذیرش و استفاده از داده‌ها بالغ‌تر شده‌اند و الگوریتم‌های جدید و محصولات و خدمات داده‌محور توسعه داده‌اند، چشم‌انداز رقابتی نیز متراکم‌تر شده است. دوران کشف و اکتشاف اولیه، و در نتیجه رقابت محدود، به سرعت در حال پایان است. اکثر شرکت‌ها و صنایع خدمات خود را به خدمات دیجیتال تبدیل کرده‌اند. در این فرآیند، راهکارهای سهل‌الوصول (اصطلاحاً میوه‌های پایین‌دست) قبلاً توسعه یافته و اتخاذ شده‌اند، و تنها تعداد محدودی از بازارها استثناهای قابل توجه هستند. خدمات دیجیتال مبتنی بر الگوریتم‌ها نیز بیشتر به یک کالای عمومی \lr{(commodity)} تبدیل شده‌اند.

در نهایت، بحث‌ها در مورد تنظیم‌گریِ پلتفرم‌ها و/یا الگوریتم‌ها تقریباً وجود نداشت، در حالی که امروزه تنظیم‌گری پلتفرم‌ها \lr{(Newman, 2019)} و الگوریتم‌ها \lr{(Parikh et al., 2019)} هر دو زمینه‌های مهم پژوهشی هستند. علاوه بر این، دولت‌ها به‌طور فزاینده‌ای قوانین جدید و اغلب سخت‌گیرانه‌تری را برای تنظیم بازارهای دیجیتال و خدمات اتخاذ می‌کنند \lr{(e.g., European Commission, 2020; U.S. House of Representatives, 2020)}.

در باقیمانده این بخش، ما چند پیشرفت و روند اخیر در علم داده را (بدون ترتیب خاصی) برجسته خواهیم کرد، که—به اعتقاد ما—به شدت بر کارآفرینی داده در سال‌های آینده، و در نتیجه بر پژوهش‌های کارآفرینی داده تأثیر خواهند گذاشت. این یک لیست جامع نیست، اما چند تحول عمده با تأثیر مورد انتظارِ قابل توجه را برجسته می‌کند.

% =================================================================
% بخش ۲۲.۲.۱: نرم‌افزار هوش مصنوعی
% =================================================================

\vspace{0.4cm}

% تیتر زیربخش ۲۲.۲.۱ (دستی)
\subsubsection*{۲۲.۲.۱ نرم‌افزار هوش مصنوعی}
\addcontentsline{toc}{subsubsection}{۲۲.۲.۱ نرم‌افزار هوش مصنوعی}
\label{sec:22-ai-software}

پس از چند دهه رشد به عنوان یک رشته علمی و عملی، هوش مصنوعی اکنون با هزاران کاربرد در تجارت و جامعه به سرعت در حال بلوغ است. با بهره‌گیری از پتانسیل هوش مصنوعی، نسل جدیدی از برنامه‌های نرم‌افزاری ظهور کرده است که اغلب به عنوان **نرم‌افزار هوش مصنوعی** \lr{(AI software)} نامیده می‌شود.

در واقع، نرم‌افزار هوش مصنوعی دیگر خود را به برنامه‌های «اسباب‌بازی» \lr{(toy applications)} نسبتاً آزمایشی و غیرمقیایس‌پذیر که فاقد هرگونه ارزش تجاری هستند، محدود نمی‌کند. در اینجا، بهترین‌هایِ دو دنیایی که تا همین اواخر جداگانه عمل می‌کردند، به هم پل زده شده‌اند؛ یعنی دنیای هوش مصنوعی و دنیای مهندسی نرم‌افزار.

هوش مصنوعی تکنیک‌ها و ابزارهای قابل توجهی را برای کاوشِ راهکارهای بهینه در موارد بسیار بدون ساختار، پیچیده، مبهم، غیرقابل پیش‌بینی و/یا ناقص به ارمغان آورده است. از سوی دیگر، مهندسی نرم‌افزار ارزش خود را در تبدیل فضاهای راهکارِ به‌خوبی درک‌شده، نسبتاً پایدار و به‌وضوح مرزبندی‌شده به کد اثبات کرده است \lr{(Ford, 1987)}.

این آخرین روند هوش مصنوعی، جامعه مهندسی نرم‌افزار را بر آن داشته است تا به‌طور فزاینده‌ای فناوری‌ها و پلتفرم‌های هوش مصنوعی (مانند پلتفرم هوش مصنوعی گوگل، تنسورفلو \lr{TensorFlow}، واتسون استودیو آی‌بی‌ام و آژور مایکروسافت) را تزریق کنند و سری جدیدی از مدل‌ها و شیوه‌های مهندسی نرم‌افزار را توسعه دهند که تولید خودکار کد، تست و یکپارچه‌سازی مداوم، و طراحی نرم‌افزار را تقویت می‌کند. بنابراین، فرصت‌های تجاری و پژوهشی هیجان‌انگیز جدیدی در این حوزه نوظهورِ نرم‌افزار هوش مصنوعی یافت می‌شود.



% =================================================================
% بخش‌های ۲۲.۲.۲ تا ۲۲.۲.۵ (ادامه بخش راه پیش رو)
% =================================================================

\raggedbottom % حذف فاصله‌های اضافی عمودی

% -----------------------------------------------------------------
% 22.2.2 MLOps
% -----------------------------------------------------------------
\subsubsection*{۲۲.۲.۲ عملیات یادگیری ماشین \lr{(MLOps)}}
\addcontentsline{toc}{subsubsection}{۲۲.۲.۲ عملیات یادگیری ماشین \lr{(MLOps)}}
\label{sec:22-mlops}

توسعه مهم دیگری که مشاهده می‌کنیم، تبدیل هوش مصنوعی و یادگیری ماشین به یک رشته مهندسی و بهبود همکاری و هماهنگی بین متخصصان مهندسی داده (شامل برنامه‌نویسان و کارکنان نگهداری نرم‌افزار)، دانشمندان داده (شامل متخصصان یادگیری ماشین) و متخصصان دامنه \lr{(domain experts)} است. این توسعه به زیبایی در پذیرش نسل جدیدی از «عملیات یادگیری ماشین» \lr{(MLOps)} که منضبط، تکرارپذیر و شفاف هستند، منعکس شده است.

\lr{MLOps} همراه با تکنیک‌های خودکار برای پیاده‌سازی خطوط لوله \lr{(pipelines)} یادگیری ماشین با توسعه نرم‌افزار، و فرهنگی است که از تیم‌های مدل‌سازی که از نزدیک با هم کار می‌کنند، حمایت می‌کند.

\lr{MLOps} عمدتاً از فلسفه «دِواپس» \lr{(DevOps)} \lr{(Ebert et al., 2016)} و شیوه‌های مرتبط با آن که جریان کاری توسعه نرم‌افزار و فرآیندهای تحویل را ساده و به‌طور تنگاتنگی یکپارچه می‌کنند، الهام گرفته شده است. مانند \lr{DevOps}، \lr{MLOps} نیز چرخه «یکپارچه‌سازی مداوم» \lr{(continuous integration)} و «تست مداوم» \lr{(continuous testing)} را برای تولید و استقرار ریز-انتشارها \lr{(micro-releases)} و نسخه‌های جدیدِ آماده‌ی تولید از برنامه‌های کاربردی هوشمند سازمانی اتخاذ می‌کند.

این امر مستلزم تغییر فرهنگ بین مهندسان داده، تحلیلگران داده، مهندسان استقرار و سیستم، و متخصصان دامنه است، که با مدیریت وابستگیِ بهبودیافته—و در نتیجه شفافیت—بین توسعه مدل، آموزش، اعتبارسنجی و استقرار همراه است. به این ترتیب، \lr{MLOps} به وضوح نیازمند سیاست‌های پیچیده‌ای مبتنی بر معیارهای عملکرد و تله‌متری \lr{(telemetry)}، مانند امتیازات \lr{F1}، دقت \lr{(accuracy)} و کیفیت نرم‌افزار است \lr{(Nogueira et al., 2018)}.

با توجه به اینکه مرزهای دقیق بین \lr{MLOps} و \lr{DevOps} مبهم است، یک سناریوی کاربردیِ بنیادین از \lr{MLOps} را می‌توان در خدمات وب آمازون \lr{(AWS)} یافت که از یک جریان کاری یکپارچه یادگیری ماشین برای ساخت، تست و یکپارچه‌سازی پشتیبانی می‌کند و تحویل مداوم را با کنترل منبع \lr{(source control)} و خدمات نظارتی ارائه می‌دهد.

% -----------------------------------------------------------------
% 22.2.3 Edge Computing
% -----------------------------------------------------------------
\subsubsection*{۲۲.۲.۳ رایانش لبه‌ای \lr{(Edge Computing)}}
\addcontentsline{toc}{subsubsection}{۲۲.۲.۳ رایانش لبه‌ای \lr{(Edge Computing)}}
\label{sec:22-edge-computing}

رایانش لبه‌ای یک پارادایم رایانشی جدید است که امکان پردازش و تحلیل بسیار توزیع‌شده‌ی حجم عظیمی از داده‌ها را در لبه‌های شبکه، و در نزدیک‌ترین مکان به محل مورد نیاز، فراهم می‌کند.



به این ترتیب، پردازش از ابر \lr{(cloud)} به لبه‌های شبکه منتقل می‌شود و پردازش، ذخیره‌سازی و تحلیلِ بسیار غیرمتمرکز را فراهم می‌کند. این نشان می‌دهد که رایانش لبه‌ای مدلی از رایانش توزیع‌شده را به جای رایانش متمرکز (که در مدل‌های رایانش ابری مرسوم وجود دارد) در آغوش می‌گیرد \lr{(Khan et al., 2019)}.

مزایای بالقوه شامل تأخیر \lr{(latency)} کمتر که باعث آزادسازی پهنای باند می‌شود، وابستگی کمتر به شبکه، و نزدیکی به کاربر است، که البته به قیمت کاهش قابلیت اطمینان و ظرفیت پردازشِ کمترِ ارائه شده توسط دستگاه‌های لبه تمام می‌شود \lr{(Bagchi et al., 2019)}. این امر نیازمند مکانیسم‌های امنیتی مقیاس‌پذیر و مستحکمی است که باید روی دستگاه‌های لبه توزیع شوند.

سیستم‌های رایانش لبه‌ای معمولاً تحت مالکیت ارائه‌دهندگان خدمات مختلف هستند و ممکن است تحت مفاد مدل‌های تجاری گوناگون عمل کنند. هر کسب‌وکاری طبق استراتژی‌های تجاری و سیاست‌های مدیریتی متفاوتی اداره می‌شود، در حالی که از قوانین و مقررات متفاوتی بر اساس سازمانِ عملیاتی خود پیروی می‌کند \lr{(Khan et al., 2019)}. به همین ترتیب، دستگاه‌های لبه توسط فروشندگان مختلف توسعه می‌یابند و رابط‌های خاص خود را دارند که بر عملکرد تأثیر می‌گذارد و هزینه‌های بالایی را به همراه دارد. به منظور غلبه بر مسائل فوق، یک مدل تجاری مدیریت و استقرارِ مشترک برای اطمینان از عملکرد بالا و ارائه خدمات کم‌هزینه به کاربران نهایی حیاتی است.

% -----------------------------------------------------------------
% 22.2.4 Digital Twins
% -----------------------------------------------------------------
\subsubsection*{۲۲.۲.۴ دوقلوهای دیجیتال \lr{(Digital Twins)}}
\addcontentsline{toc}{subsubsection}{۲۲.۲.۴ دوقلوهای دیجیتال \lr{(Digital Twins)}}
\label{sec:22-digital-twins}

در حالی که ناسا \lr{(NASA)} برای اولین بار از دهه ۱۹۶۰ با مفاهیم به‌اصطلاح دوقلوی دیجیتال برای شبیه‌سازی و تحلیلِ (مثلاً) شرایط زندگی در سفینه‌های فضایی مانند آپولو ۱۳ تمرین کرد، خودِ این اصطلاح توسط مایکل گریوز \lr{(Michael Grieves)} در سال ۲۰۰۲ در زمینه معرفی یک موسسه جدید مدیریت چرخه عمر محصول معرفی شد \lr{(Grieves, 2005)}.

پذیرش سریع فناوری دوقلوی دیجیتال با بلوغ فناوری‌های توانمندساز مختلف، از جمله یادگیری ماشین، همجوشی داده‌ها \lr{(data fusion)}، ارتباطات داده، اینترنت اشیاء \lr{(IoT)}، واقعیت افزوده، واقعیت مجازی و تحلیل کلان‌داده‌ها تقویت شده است.

در اصل، دوقلوهای دیجیتال را می‌توان به عنوان کپی‌های دیجیتالی از اشیاء فیزیکی (زنده یا غیرزنده) تعریف کرد \lr{(Shafto et al., 2012)}. آن‌ها اساساً فراتر از بازنمایی‌های دیجیتال موجود، مانند مدل‌های \lr{CAD} می‌روند و از سیستم‌های سایبری-فیزیکی در تمام چرخه عمر، یعنی از طراحی تا تولید و تا اجرا و مدیریت واقعی، پشتیبانی می‌کنند.



به این ترتیب، دوقلوهای دیجیتال از هوش مصنوعی و یادگیری ماشین بهره‌برداری می‌کنند تا داده‌های عملیاتی و بی‌درنگ \lr{(real-time)} به دست آمده از اشیاء فیزیکی و مجازیِ مجهز به دستگاه‌های اینترنت اشیاء را تجسم کنند و در نتیجه، تصمیم‌گیری انسانی را تقویت نمایند. این امر اطلاعات مربوط به اشیاء (مانند ساختمان‌ها و خطوط تولید) یا مفاهیم (مانند برنامه‌ریزی تولید) را برای ارتباط روان‌تر و شهودی‌تر بین ذینفعان مجاز، به راحتی در دسترس قرار می‌دهد.

در اینجا، می‌توان به داده‌های تاریخی، گزارش‌های وضعیت و فراداده‌های بافتی (مانند گزارش‌های آب‌وهوا) فکر کرد. با افزودن قابلیت‌های مبتنی بر هوش مصنوعی، دوقلوهای دیجیتال حتی می‌توانند موقعیت‌های مختلف را شبیه‌سازی کرده و درباره آن‌ها استدلال کنند و برای مثال سناریوهای «چه می‌شود اگر» \lr{(what-if)} را با بهره‌گیری از قابلیت‌های تشخیصی، پیش‌بینی و بهینه‌سازی اجرا کنند.

% -----------------------------------------------------------------
% 22.2.5 Large-Scale Experimentation
% -----------------------------------------------------------------
\subsubsection*{۲۲.۲.۵ آزمایش در مقیاس بزرگ}
\addcontentsline{toc}{subsubsection}{۲۲.۲.۵ آزمایش در مقیاس بزرگ}
\label{sec:22-large-scale-experimentation}

یک نقد مکرر بر علم داده این است که به جای روابط علی و معلولی \lr{(causal relationships)} و «خلاف‌واقع‌ها» \lr{(counterfactuals)}، بر همبستگی‌ها \lr{(correlations)} و تداعی‌ها تمرکز دارد. در اینجا، استدلال به این صورت است که اکثر همبستگی‌ها حتی با وجود پایگاه‌های داده بزرگ، بنا به تعریف «کاذب» \lr{(spurious)} هستند \lr{(Calude \& Longo, 2017)}. در حالی که این نقد تا حدودی معتبر است، اما این نکته را نادیده می‌گیرد که تحلیل مجموعه داده‌های بزرگ لزوماً به معنای تمرکز بر همبستگی‌ها نیست. مجموعه داده‌های بزرگ همچنین می‌توانند در کشف علیت و بسطِ خلاف‌واقع‌ها مفید باشند.

امروزه، اکثر شرکت‌های دیجیتال، از جمله \lr{Airbnb}، آمازون، \lr{Booking.com}، \lr{eBay}، فیس‌بوک، گوگل، لینکدین، مایکروسافت، نتفلیکس، توییتر و اوبر، آزمایش‌های کنترل‌شده تصادفی آنلاین را در مقیاسی (بسیار) بزرگ اجرا می‌کنند \lr{(Kohavi et al., 2020)}. این امر آن‌ها را قادر می‌سازد تا از داده‌های (کلان) برای یافتن عوامل علیِ زیربنایی استفاده کنند. معمولاً، شرکت‌های بزرگتر روزانه صدها تا هزاران مورد از این آزمایش‌های کنترل‌شده را، گاهی بر روی میلیون‌ها کاربر، اجرا می‌کنند.

در حالی که به‌اصطلاح «کارآزمایی‌های تصادفی کنترل‌شده» \lr{(RCTs)} در پزشکی اغلب به دلیل گران و پیچیده بودن مورد انتقاد قرار می‌گیرند، در محیط‌های دیجیتال، هزینه نهایی چنین آزمایش‌هایی بسیار کم است و ارزش افزوده کشف روابط علی را نباید دست‌کم گرفت. اگر این کار به درستی انجام شود، پذیرش آزمایش در مقیاس بزرگ مستقیماً منجر به نوآوری (تدریجی) و افزایش درآمد می‌شود. در نتیجه، با رشد سالانه تعداد چنین آزمایش‌هایی، شرکت‌های دیجیتال دو، سه یا حتی چهار برابر شده‌اند.

خوشبختانه، در حالی که شرکت‌های بزرگ ممکن است از نظر اندازه و هزینه‌های آزمایش (به دلیل صرفه به مقیاس) مزیت داشته باشند، \lr{RCT}ها به عنوان چنین روشی می‌توانند برای شرکت‌های کوچک و متوسط \lr{(SMEs)} نیز ارزشمند باشند. زیرساخت اجرای چنین آزمایش‌های در مقیاس بزرگی به‌طور فزاینده‌ای برای انواع شرکت‌ها در دسترس قرار گرفته است \lr{(Fabijan et al., 2018; Tang et al., 2010)}.

برای مثال، \lr{Google Optimize} یک ابزار تستِ تقسیم‌شده \lr{(split-testing)} آنلاین است که به وب‌سایت‌ها متصل می‌شود و بدین ترتیب \lr{SME}ها را قادر می‌سازد تا با روش‌های مختلفِ ارائه محتوا آزمایش کنند. مثال دیگر پلتفرم آزمایش آی‌بی‌ام است که هدف آن عملیات هوش مصنوعی است \lr{(Rausch et al., 2020)}. این تحولات با توجه به ارزش بالقوه کشف روابط علی، افزایش مداوم داده‌ها (مثلاً به دلیل ظهور اینترنت اشیاء، همچنین نگاه کنید به \lr{Attaran (2017)})، و کاهش مداوم هزینه‌های انجام آزمایش، احتمالاً ادامه خواهند یافت. مورد آخر تا حدی به دلیل ظهور پلتفرم‌های آزمایش در صنایع و حوزه‌های مختلف است.



% =================================================================
% بخش‌های ۲۲.۲.۶ و ۲۲.۲.۷ (ادامه بخش راه پیش رو)
% =================================================================

\raggedbottom % حذف فاصله‌های اضافی عمودی

% -----------------------------------------------------------------
% 22.2.6 Big Data and AI Opportunities
% -----------------------------------------------------------------
\subsubsection*{۲۲.۲.۶ فرصت‌های کلان‌داده و هوش مصنوعی}
\addcontentsline{toc}{subsubsection}{۲۲.۲.۶ فرصت‌های کلان‌داده و هوش مصنوعی}
\label{sec:22-big-data-ai-opportunities}

پنج تا ده سال پیش، مفهوم کلان‌داده \lr{(big data)} تازه در حال بلوغ بود \lr{(Provost \& Fawcett, 2013)}، و راهکارهای هوش مصنوعی عمدتاً توسط شرکت‌های بزرگ فناوری با دسترسی به انبوهی از داده‌ها، مقادیر هنگفت بودجه و کارکنان بااستعدادشان استفاده می‌شد.

در این دوره، همه شرکت‌های بزرگ و همچنین بسیاری از شرکت‌های نسبتاً بزرگ در میان گروه \lr{SME}ها (شرکت‌های کوچک و متوسط) قبلاً در حال آزمایش با علم داده بودند. معمولاً، این شرکت‌ها با راه‌اندازی یک تیم پروژه، یا حتی یک آزمایشگاه داده با متخصصان شروع کردند تا ببینند علم داده چه چیزی می‌تواند به آن‌ها ارائه دهد. گاهی اوقات، این شرکت‌ها از شرکت‌های مشاوره خارجی برای کمک به توسعه قابلیت‌های داده خود استفاده می‌کردند. بدون استثناء، این شرکت‌ها کشف کردند که علم داده به هیچ وجه آسان نیست. موانع فرهنگی، فنی، مدیریتی و سازمانی زیادی برای غلبه بر آن‌ها وجود دارد.

با این حال، شرکت‌هایی که استقامت کردند اغلب در داده‌ها ارزش یافتند. بینش‌های به دست آمده از داده‌ها حداقل برای تصمیم‌گیری داخلی مفید بود. شرکت‌های دیگر حتی قادر به توسعه محصولات و خدمات (دیجیتال) جدید مبتنی بر داده بودند. در حالی که کاوش و توسعه محصولات و خدمات دیجیتال دشوار است، و کسب درآمد از داده‌ها اغلب حتی دشوارتر است \lr{(Bataineh et al., 2020; Wixom \& Ross, 2017)}، در صورتی که این شرکت‌ها قادر به غلبه بر همه موانع بودند، فرصت‌های فراوانی را با رقابت نسبتاً محدود پیدا کردند \lr{(Zuboff, 2019)}. در این تنظیمات، شرکت‌ها اغلب قادر به بهره‌برداری از «مزیت پیشرو بودن» \lr{(first-mover advantage)} خود بودند \lr{(Varadarajan et al., 2008)}.

بعدها، شرکت‌های بسیار بیشتری توانستند بر موانع اولیه غلبه کنند و قابلیت‌های علم داده خود را بر این اساس بسازند \lr{(Davenport \& Ronanki, 2018; Fountaine et al., 2019)}. بنابراین، امروزه شانس ایجاد مزیت رقابتی پیرامون یک قابلیت علم داده منحصر به فرد اندک است. این احتمالاً نیاز به دسترسی به مجموعه داده‌های منحصر به فرد و حفاظت شده، استفاده از فناوری‌های هوش مصنوعیِ روز، و/یا تعهد کارکنان فوق‌العاده بااستعداد دارد.



به نظر می‌رسد شرکت‌های دارای قابلیت‌های «تحلیل کلان‌داده» \lr{(BDA)} عملکرد کلی بهتری دارند، اما اندازه اثرات به شدت به گرایش کارآفرینانه شرکت، صنعتی که در آن فعالیت می‌کند و پویایی محیطی بستگی دارد \lr{(Dubey et al., 2020; Müller et al., 2018; Wamba et al., 2017)}. در هر صورت، کاربرد علم داده با ترکیب مجموعه داده‌های معمولی و همه‌جا حاضر و استفاده از آمار استاندارد دیگر کافی نیست. این امروزه «بلیط ورود به بازی» است، اما دیگر به شرکت‌ها مزیت رقابتی نخواهد داد.

% -----------------------------------------------------------------
% 22.2.7 Government Regulation
% -----------------------------------------------------------------
\subsubsection*{۲۲.۲.۷ مقررات دولتی}
\addcontentsline{toc}{subsubsection}{۲۲.۲.۷ مقررات دولتی}
\label{sec:22-government-regulation}

همان‌طور که در بالا ذکر شد، دشت‌های رقابتیِ باز و وسیعِ دهه‌های اول این هزاره به پایان رسیده است. این نه تنها در مورد سطح رقابت صدق می‌کند، بلکه در مورد حضور آژانس‌های دولتی نیز صادق است. تاریخ نشان می‌دهد که مداخله و مقررات دولتی همیشه هنگام معرفی فناوری جدید عقب‌تر است \lr{(Wiener, 2004)}.

علم داده و هوش مصنوعی به هیچ وجه از این قاعده مستثنی نیستند. حتی می‌توان گفت که پیچیدگی و تازگی هوش مصنوعی در ترکیب با عدم شفافیتِ تأثیر اجتماعی خدمات دیجیتال، باعث شده است که دولت‌ها در سراسر جهان رویکرد «صبر و مشاهده» را اتخاذ کنند. با این حال، به نظر می‌رسد که این روزها دیگر به پایان رسیده است. انتقاد از فناوری دیجیتال و شرکت‌های دیجیتال در همه جای جهان رو به افزایش است: از چین تا اروپا و از آفریقا تا ایالات متحده.

این انتقاد در امتداد خطوط متعددی شکل گرفته است. اول و مهم‌تر از همه، انتقاداتی بر ویژگی‌های «برنده همه‌چیز را می‌برد» \lr{(winner-takes-all)} در بازارهای دیجیتال و رفتار غول‌های بزرگ فناوری، مانند اپل، گوگل، فیس‌بوک و مایکروسافت وجود دارد، زیرا آن‌ها از قدرت بازار خود برای محدود کردن رقابت استفاده (سوءاستفاده؟) می‌کنند. همان‌طور که گفته شد، تنظیم‌کنندگان بازار به‌طور فزاینده‌ای در حال توسعه و تصویب قوانین جدید ضد انحصار هستند. در زمان نگارش این متن، فهرستی از این قوانین جدید هنوز در حال بحث بود، اما دولت‌ها به آرامی اما مطمئن قدرت بازارِ غول‌های فناوری \lr{(Big Tech)} را هدف قرار می‌دهند.

در دموکراسی‌های غربی، دولت‌ها نگران رقابت و رفاه مصرف‌کننده هستند، در حالی که در دولت‌های اقتدارگراتر، دولت‌ها نگران قدرت شرکت‌های بزرگ فناوری در برابر دولت هستند.

با این حال، قدرت بازار به هیچ وجه تنها دلیلی نیست که دولت‌ها مایل به تنظیم شرکت‌های داده‌محور هستند. حریم خصوصی سال‌هاست که استدلال مهمی بوده و منجر به معرفی چارچوب \lr{GDPR} در اتحادیه اروپا شده است. در ایالات متحده، انتظار می‌رود ریاست‌جمهوری جدید سیاست‌های جدیدی را در زمینه جنبه‌هایی مانند قانون فدرال حریم خصوصی، انتقال بین‌المللی داده‌ها و بی‌طرفی شبکه ارائه دهد. اخیراً، مسائل اخلاقی و اجتماعی بیشتر، مانند تبعیض در الگوریتم‌ها و حباب‌های فیلترِ نوظهور در جامعه، به نقاط اصلی مورد توجه تبدیل شده‌اند.

در مجموع، روزهای دسترسی نامحدود به داده‌ها و ارائه بدون محدودیت و نظارتِ خدمات دیجیتال به پایان رسیده است. بی‌اعتمادی جهانی به غول‌های فناوری سر به فلک کشیده است و دولت‌ها با تقاضاها، قوانین، چارچوب‌ها و نهادهای دولتیِ روزافزون برای نظارت بر بازار دیجیتال خواهند آمد. برای رقابت در این بازار، آگاهی از این قوانین نوظهور و ایجاد روابط قابل اعتماد و بلندمدت با سازمان‌های دولتی در همه سطوح حیاتی است.




% =================================================================
% بخش نتیجه‌گیری (قسمت اول) - صفحه جاری
% =================================================================

\raggedbottom 

\vspace{0.8cm}

\noindent
\fcolorbox{black}{gray!15}{% کادر خاکستری قسمت اول
	\begin{minipage}{\dimexpr\linewidth-2\fboxsep-2\fboxrule}
		\vspace{0.4cm}
		\begin{center}
			\textbf{\Large نتیجه‌گیری}
		\end{center}
		\vspace{0.3cm}
		
		کارآفرینی داده ماندگار است. با رشد سالانه درک شده داده‌ها به میزان بیش از ۴۰ درصد، اهمیت و فراگیری داده‌ها در سال‌های آینده تنها افزایش خواهد یافت. داده‌های اینترنت اشیاء، داده‌های حسگرها، داده‌های ژنومی، داده‌های سلامت شخصی، داده‌های صوتی، داده‌های ویدئویی و بسیاری از انواع (جدید) دیگر داده‌ها به رشد انفجاری خود ادامه خواهند داد. بنابراین، کارآفرینی داده، که به طور خلاصه به عنوان اکتشاف و بهره‌برداری از فرصت‌ها با استفاده از علم داده تعریف می‌شود، بدون شک از نظر اهمیت نیز رشد خواهد کرد.
		
		با این حال، به نظر می‌رسد روزهای بهره‌برداری از «میوه‌های پایین‌دست» (فرصت‌های سهل‌الوصول) به پایان رسیده است. شناسایی و کسب درآمد از فرصت‌های جدید با علم داده به تدریج نیازمند دسترسی به مجموعه داده‌های منحصر به فرد، استفاده از فناوری‌های نوظهور، و/یا مشارکت افراد فوق‌العاده بااستعداد است. امروزه، صرفاً کاوش در یک مجموعه داده و ارائه آمار توصیفی به عنوان بینش، برای به دست آوردن و حفظ (موقتی) مزیت رقابتی بسیار ناکافی است.
		
		خوشبختانه برای شرکت‌های دیجیتالی که در خط مقدم علم داده هستند و آن‌هایی که عقب مانده‌اند، فناوری‌ها هنوز به سرعت در حال تکامل هستند. علاوه بر این، چارچوب‌های فناوری جدید، مانند نرم‌افزار هوش مصنوعی و \lr{MLOps}، به شرکت‌ها اجازه می‌دهند تا قابلیت‌های هوش مصنوعی خود را مقیاس‌دهی کنند. در نهایت، قوانین و مقررات ناشی از دولت (یا تغییرات در آن‌ها) در رابطه با پذیرش و استفاده از علم داده، باری بر دوش برخی شرکت‌ها خواهد بود، اما در عین حال، ممکن است فرصت‌های جدیدی را برای آن دسته از شرکت‌هایی فراهم کند که می‌توانند به راحتی با چنین قوانین (جدیدی) سازگار شوند و شاید حتی آن‌ها را شکل دهند.
		
		بدیهی است که همه این تحولات با پیامدهای عملی شدید برای کارآفرینان داده و توسعه‌دهندگان کسب‌وکار داده‌محور (یا کارآفرینان سازمانیِ داده)، بر حوزه پژوهش کارآفرینی داده نیز تأثیر خواهد گذاشت. این حوزه پژوهشی هنوز در دوران نوزادی خود است. با این حال، چند موضوع، هرچند بسیار اخیراً، توجه بیشتری را به خود جلب کرده‌اند. این موضوعات شامل فرصت‌هایی است که فناوری‌های دیجیتال مانند هوش مصنوعی برای کارآفرینی به ارمغان می‌آورند \lr{(e.g., Ransbotham et al., 2017; Townsend \& Hunt, 2019; Von Briel et al., 2018; Von Krogh, 2018)}، تأثیر هوش مصنوعی بر مدیران و تصمیم‌گیری آن‌ها \lr{(e.g., Huang et al., 2019; Raisch \& Krakowski, 2021; Shrestha et al., 2019)}، و رابطه بین قابلیت‌های تحلیل کلان‌داده \lr{(BDA)} و عملکرد شرکت \lr{(e.g., Dubey et al., 2020; Müller et al., 2018; Wamba et al., 2017)}.
		\vspace{0.2cm}
	\end{minipage}
}

% =================================================================
% شکستن صفحه برای بخش دوم (قسمت بیرون زده)
% =================================================================
\newpage 

% =================================================================
% بخش نتیجه‌گیری (قسمت دوم) - صفحه جدید
% =================================================================
\noindent
\fcolorbox{black}{gray!15}{% کادر خاکستری قسمت دوم (ادامه متن)
	\begin{minipage}{\dimexpr\linewidth-2\fboxsep-2\fboxrule}
		\vspace{0.3cm}
		با این وجود، به‌طور کلی، ما هنوز فاقد درک عمیقی از چگونگی، زمان و چراییِ استفاده کارآفرینان و کارآفرینان سازمانی از علم داده برای خلق ارزش جدید (یا عدم خلق آن) هستیم. موضوعات پژوهشی خاصی که نیاز به توجه بیشتر دارند عبارتند از:
		(۱) آگاهی و شایستگی‌های کارآفرینانه و مدیریتی در رابطه با تحولات فناورانه جدید؛
		(۲) موانع پذیرش و استفاده از علم داده در میان شرکت‌های تمام رده‌های سنی و اندازه؛
		(۳) تعیین‌کننده‌ها و پیامدهای داشتن سطوح مختلفِ به‌اصطلاح «بلوغ داده» در شرکت‌ها؛
		(۴) مزایا و معایب اشتراک‌گذاری داده (باز) برای کارآفرینی و نوآوری؛
		و (۵) تأثیر قوانین و مقررات جدید بر اکتشاف و بهره‌برداری از داده‌ها توسط شرکت‌ها.
		نمونه‌های مرتبط دیگر توسط چالمرز و همکاران \lr{(Chalmers et al., 2020)} به تفصیل بحث شده‌اند.
		
		بدین‌وسیله ما خواستار نظریه‌پردازی دقیق و بررسی تجربی گسترده در مورد هر یک از موضوعات فوق هستیم. بنا به تعریف، محققان باید در پژوهش‌های چندرشته‌ای شرکت کنند و بین رشته‌های علم داده و کارآفرینی پل بزنند. عصر جدیدی آغاز شده است \lr{(Obschonka \& Audretsch, 2020)}، و بیایید به جامعه‌ای کارآفرینانه کمک کنیم که در آن کلان‌داده و هوش مصنوعی توسط کارآفرینان به سازنده‌ترین روش‌ها مورد بهره‌برداری قرار می‌گیرند.
		\vspace{0.3cm}
	\end{minipage}
}

\vspace{1cm} % فضای خالی بین دو کادر طبق تصویر

% =================================================================
% بخش نکات بحث (Discussion Points)
% =================================================================
\noindent
\fcolorbox{black}{gray!15}{% کادر خاکستری برای نکات بحث
	\begin{minipage}{\dimexpr\linewidth-2\fboxsep-2\fboxrule}
		\vspace{0.3cm}
		\textbf{\large نکات بحث}
		
		\begin{enumerate}
			\setlength\itemsep{0.8em} % فاصله بین آیتم‌ها
			
			\item در این فصل، ما پیشرفت‌ها و روندهای اخیر مختلفی را برجسته کردیم که—به اعتقاد ما—بر کارآفرینی داده تأثیر خواهند گذاشت (و ادامه خواهند داد). فکر می‌کنید کدام یک از آن‌ها به احتمال زیاد قوی‌ترین تأثیر را خواهد داشت و چرا؟
			
			\item پیشنهاد شده است که اجرای قوانین و مقررات جدید (برای مثال، برای کاهش قدرت بازار شرکت‌های بزرگ فناوری) همچنین می‌تواند فرصت‌های کارآفرینانه جدیدی را برای برخی فراهم کند. حداقل یک نمونه از چنین فرصتی را که ممکن است به عنوان پیامد تصویب قوانین جدید ضد انحصار ایجاد شود، نام ببرید و توضیح دهید.
			
			\item یکی از مسیرهای امیدوارکننده برای پژوهش کارآفرینی داده مربوط به موانع پذیرش و استفاده از علم داده است که توسط شرکت‌ها درک می‌شود. تمام موانعی را که می‌توانید به آن‌ها فکر کنید فهرست کنید، و بحث کنید که چگونه هر یک از این موانع می‌توانند کاهش یابند یا حتی کاملاً از بین بروند.
		\end{enumerate}
		\vspace{0.3cm}
	\end{minipage}
}


% =================================================================
% بخش پیام‌های کلیدی (Take-Home Messages)
% =================================================================

\raggedbottom % حذف فاصله‌های اضافی عمودی

\vspace{0.8cm}

\noindent
\fcolorbox{black}{gray!15}{% کادر خاکستری
	\begin{minipage}{\dimexpr\linewidth-2\fboxsep-2\fboxrule}
		\vspace{0.3cm}
		% تیتر
		\textbf{\large پیام‌های کلیدی}
		
		\begin{itemize}
			\setlength\itemsep{0.5em} % تنظیم فاصله بین آیتم‌ها
			
			\item[\textbf{--}] از زمان پیدایش آن در ۵ تا ۱۰ سال پیش، کارآفرینی داده بسیار سریع تکثیر شده و همچنان در تمام عناصر زندگی روزمره و جامعه ما اهمیت بیشتری پیدا می‌کند.
			
			\item[\textbf{--}] فرصت‌ها و چالش‌های جدید را باید در فناوری‌های نوظهور، مانند نرم‌افزار هوش مصنوعی، عملیات یادگیری ماشین \lr{(MLOps)}، رایانش لبه‌ای و دوقلوهای دیجیتال یافت.
			
			\item[\textbf{--}] با رواج سریع علم داده، صرفاً به‌کارگیری آن دیگر به شرکت‌ها مزیت رقابتی نمی‌دهد، بلکه در عوض نیاز به مجموعه داده‌های منحصر به فرد، فناوری‌های پیشرفته و/یا استعدادهای استثنایی است.
			
			\item[\textbf{--}] قوانین و مقررات جدیدی برای برخورد بهتر با قدرت روزافزون شرکت‌های بزرگ فناوری، و مسائلی مانند حریم خصوصی داده‌ها و توضیح‌پذیری هوش مصنوعی مورد نیاز است.
			
			\item[\textbf{--}] پژوهش کارآفرینی داده هنوز در دوران نوزادی خود است، بنابراین تحقیقات بسیار بیشتری برای درک بهتر اینکه چگونه کارآفرینان می‌توانند از کلان‌داده و هوش مصنوعی برای خلق ارزش جدید بهره‌برداری کنند، مورد نیاز است.
		\end{itemize}
		\vspace{0.3cm}
	\end{minipage}
}

% =================================================================
% بخش منابع (References)
% =================================================================

\vspace{1cm}

% تیتر منابع (به فارسی)
% =================================================================
% بخش منابع فصل ۲۲ (اصلاح شده: تمام & ها به \& تبدیل شدند)
% =================================================================

\vspace{1cm}

% تیتر منابع
% =================================================================
% بخش منابع فصل ۲۲ (اصلاح نهایی: رفع خطای &)
% =================================================================

\vspace{1cm}

% تیتر منابع
\section*{منابع}
\addcontentsline{toc}{section}{منابع}

% شروع محیط لاتین
\begin{latin}
	\begin{itemize}
		\setlength\itemsep{0.5em} 
		
		\item[] Attaran, M. (2017). The internet of things: Limitless opportunities for business and society. \textit{Journal of Strategic Innovation and Sustainability}, 12(1), 10–29.
		
		\item[] Bagchi, S., Siddiqui, M. B., Wood, P., \& Zhang, H. (2019). Dependability in edge computing. \textit{Communications of the ACM}, 63(1), 58–66.
		
		\item[] Bataineh, A. S., Mizouni, R., Bentahar, J., \& El Barachi, M. (2020). Toward monetizing personal data: A two-sided market analysis. \textit{Future Generation Computer Systems}, 111, 435–459.
		
		\item[] Brynjolfsson, E., \& McAfee, A. (2014). \textit{The second machine age: Work, progress, and prosperity in a time of brilliant technologies}. WW Norton \& Company.
		
		\item[] Calude, C. S., \& Longo, G. (2017). The deluge of spurious correlations in big data. \textit{Foundations of Science}, 22(3), 595–612.
		
		\item[] Chalmers, D., MacKenzie, N. G., \& Carter, S. (2020). Artificial intelligence and entrepreneurship: Implications for venture creation in the fourth industrial revolution. \textit{Entrepreneurship Theory and Practice}, 45, 1–26.
		
		\item[] Cockburn, I. M., Henderson, R., \& Stern, S. (2018). \textit{The impact of artificial intelligence on innovation}. National Bureau of Economic Research.
		
		\item[] Davenport, T. H., \& Ronanki, R. (2018). Artificial intelligence for the real world. \textit{Harvard Business Review}, 96(1), 108–116.
		
		\item[] Dubey, R., Gunasekeran, A., Childe, S. J., Bryde, D. J., Giannakis, M., Foropon, C., Roubaud, D., \& Hazen, B. T. (2020). Big data analytics and artificial intelligence pathway to operational performance under the effects of entrepreneurial orientation and environmental dynamism: A study of manufacturing organizations. \textit{International Journal of Production Economics}, 226, 107599.
		
		\item[] Ebert, C., Gallardo, G., Hernantes, J., \& Serrano, N. (2016). DevOps. \textit{IEEE Software}, 33(3), 94–100.
		
		\item[] European Commission. (2020). Proposal for a Regulation of the European Parliament and of the Council on a Single Market for Digital Services (Digital Services Act) and amending Directive 2000/31/EC. COM(2020) 825 final. Brussels: European Commission.
		
		\item[] Fabijan, A., Dmitriev, P., McFarland, C., Vermeer, L., Holmström Olsson, H., \& Bosch, J. (2018). Experimentation growth: Evolving trustworthy A/B testing capabilities in online software companies. \textit{Journal of Software: Evolution and Process}, 30(12), e2113.
		
		\item[] Ford, L. (1987). Artificial intelligence and software engineering: A tutorial introduction to their relationship. \textit{Artificial Intelligence Review}, 1, 255–273.
		
		\item[] Fountaine, T., McCarthy, B., \& Saleh, T. (2019). Building the AI-powered organization. \textit{Harvard Business Review}, 97(4), 62–73.
		
		\item[] Grieves, M. W. (2005). Product lifecycle management: The new paradigm for enterprises. \textit{International Journal of Product Development}, 2(1–2), 71–84.
		
		\item[] Huang, M., Rust, R., \& Maksimovic, V. (2019). The feeling economy: Managing in the next generation of artificial intelligence (AI). \textit{California Management Review}, 61(4), 43–65.
		
		\item[] Khan, W. Z., Ahmed, E., Hakak, S., Yaqoob, I., \& Ahmed, A. (2019). Edge computing: A survey. \textit{Future Generation Computer Systems}, 97, 219–235.
		
		\item[] Kohavi, R., Tang, D., Xu, Y., Hemkens, L. G., \& Ioannidis, J. P. (2020). Online randomized controlled experiments at scale: Lessons and extensions to medicine. \textit{Trials}, 21(1), 1–9.
		
		\item[] Müller, O., Fay, M., \& Vom Brocke, J. (2018). The effect of big data and analytics on firm performance: An econometric analysis considering industry characteristics. \textit{Journal of Management Information Systems}, 35(2), 488–509.
		
		\item[] Newman, J. M. (2019). Antitrust in digital markets. \textit{Vanderbilt Law Review}, 72(5), 1497–1561.
		
		\item[] Nogueira, A.F., Ribeiro, J.C., Zenha-Rela, M.A., \& Craske, A. (2018). Improving La Redoute’s CI/CD pipeline and DevOps processes by applying machine learning techniques. In: \textit{Proceedings of the 11th International Conference on the Quality of Information and Communications Technology (QUATIC)}, pp. 282–286.
		
		\item[] Obschonka, M., \& Audretsch, D. B. (2020). Artificial intelligence and big data in entrepreneurship: A new era has begun. \textit{Small Business Economics}, 55, 529–539.
		
		\item[] Parikh, R. B., Obermeyer, Z., \& Navathe, A. S. (2019). Regulation of predictive analytics in medicine. \textit{Science}, 363(6429), 810–812.
		
		\item[] Provost, F., \& Fawcett, T. (2013). Data science and its relationship to big data and data-driven decision making. \textit{Big Data}, 1(1), 51–59.
		
		\item[] Raisch, S., \& Krakowski, S. (2021). Artificial intelligence and management: The automationaugmentation paradox. \textit{Academy of Management Review}, 46(1), 192–210.
		
		\item[] Ransbotham, S., Kiron, D., Gerbert, P., \& Reeves, M. (2017). Reshaping business with artificial intelligence: Closing the gap between ambition and action. \textit{MIT Sloan Management Review}, 59(1).
		
		\item[] Rausch, T., Hummer, W., \& Muthusamy, V. (2020). An experimentation and analytics framework for large-scale {AI} operations platforms. In: \textit{2020 {USENIX} Conference on Operational Machine Learning (OpML20)}.
		
		\item[] Shafto, M., Conroy, M., Doyle, R., Glaessgen, E., Kemp, C., LeMoigne, J., \& Wang, L. (2012). Modeling, simulation, information technology and processing roadmap. \textit{National Aeornautics and Space Administration}, 32, 1–38.
		
		\item[] Shrestha, Y. R., Ben-Menahem, S. M., \& Von Krogh, G. (2019). Organizational decision-making structures in the age of artificial intelligence. \textit{California Management Review}, 61(4), 66–83.
		
		\item[] Tang, D., Agarwal, A., O’Brien, D., \& Meyer, M. (2010). Overlapping experiment infrastructure: More, better, faster experimentation. In: \textit{Proceedings of the 16th ACM SIGKDD International Conference on Knowledge Discovery and Data Mining}, pp. 17–26.
		
		\item[] Townsend, D. M., \& Hunt, R. A. (2019). Entrepreneurial action, creativity, and judgment in the age of artificial intelligence. \textit{Journal of Business Venturing Insights}, 11, e00126.
		
		\item[] U.S. House of Representatives. (2020). \textit{Investigation of competition in digital markets: Majority staff reports and recommendations}. U.S. House of Representatives.
		
		\item[] Varadarajan, R., Yadav, M. S., \& Shankar, V. (2008). First-mover advantage in an internet-enabled market environment: Conceptual framework and propositions. \textit{Journal of the Academy of Marketing Science}, 36(3), 293–308.
		
		\item[] Von Briel, F., Davidsson, P., \& Recker, J. (2018). Digital technologies as external enablers of new venture creation in the IT hardware sector. \textit{Entrepreneurship Theory and Practice}, 42(1), 47–69.
		
		\item[] Von Krogh, G. (2018). Artificial intelligence in organizations: New opportunities for phenomenonbased theorizing. \textit{Academy of Management Discoveries}, 4(4), 404–409.
		
		\item[] Wamba, S. F., Gunasekaran, A., Akter, S., Ren, S. J. F., Dubey, R., \& Childe, S. J. (2017). Big data analytics and firm performance: Effects of dynamic capabilities. \textit{Journal of Business Research}, 70, 356–365.
		
		\item[] Wiener, J. B. (2004). The regulation of technology, and the technology of regulation. \textit{Technology in Society}, 26(2–3), 483–500.
		
		\item[] Wixom, B. H., \& Ross, J. W. (2017). How to monetize your data. \textit{MIT Sloan Management Review}, 58(3), 10–13.
		
		\item[] Zuboff, S. (2019). \textit{The age of surveillance capitalism: The fight for a human future at the new frontier of power}. Profile Books.
		
	\end{itemize}
\end{latin}



